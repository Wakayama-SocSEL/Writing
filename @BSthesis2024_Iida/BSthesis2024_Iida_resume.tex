%
% 卒論レジュメフォーマット Ver.2.0 pLaTeX版
%
\documentclass[twocolumn]{jarticle} % 2段組のスタイルを用いている
\usepackage{indentfirst}
\usepackage{url}	% \url{}コマンド用.URLを表示する際に便利
\usepackage{otf}
\usepackage{xcolor}
\usepackage{array}
\usepackage{multirow}
\usepackage[dvipdfmx]{graphicx}
\usepackage{wuse_resume}
\usepackage{url}	% \url{}コマンド用.URLを表示する際に便利
%\usepackage[dvipdfmx]{graphicx}  % ←graphicx.styを用いてEPSを取り込む場合有効にする
			% 他のパッケージ・スタイルを使う場合には適宜追加

%%%%%%%%%%%%%%%%%%%%%%%%%%%%%%%%%%%%%%%%%%%%%%%%%%%%%%%%%%%%%%%%%%%%%%%%

%%
%% タイトル,学生番号,氏名などを設定する
%%

\title{JavaScriptライブラリにおける後方互換性の\\損失の影響を受ける記述パターン生成手法}
\研究室{ソーシャルソフトウェア工学}
\学生番号{60266016}
\氏名{飯田 智輝}

\概要{
本研究では,ライブラリの後方互換性の損失がクライアントに与える影響を検出する手法を提案する.ソフトウェア開発では,開発効率を向上させるため,再利用しやすい形式に機能がまとめられたライブラリが利用される.ライブラリ開発者がライブラリの品質を維持するために機能追加や修正などを行い,バージョンを更新する中で,既存機能の変更や削除によって後方互換性を損失することがある.後方互換性の損失はライブラリを利用するクライアントソフトウェア(以降,クライアント)の振る舞いの阻害につながるが,ライブラリ開発者がクライアントを実行することなく影響を特定することは容易ではない.従来研究では,クライアントのテストをバージョン更新前後で実行し,後方互換性の損失を検出する手法が提案されている.しかし,クライアントでエラーが発生する実装方法やクライアントに与える影響の度合いは分析されていない.本研究では,バージョン更新後にテストが失敗したクライアントから後方互換性の損失の影響を受ける記述パターンを自動生成し,テストが成功しているクライアントに適用して後方互換性の損失の影響を特定する.
}

\キーワード{JavaScript}
\キーワード{ライブラリ}
\キーワード{後方互換性の損失}

%%%%%%%%%%%%%%%%%%%%%%%%%%%%%%%%%%%%%%%%%%%%%%%%%%%%%%%%%%%%%%%%%%%%%%%%

%% 以下の3行は変更しない

\begin{document}
\maketitle
\thispagestyle{empty} % タイトルを出力したページにもページ番号を付けない

%%%%%%%%%%%%%%%%%%%%%%%%%%%%%%%%%%%%%%%%%%%%%%%%%%%%%%%%%%%%%%%%%%%%%%%%

%%
%% 本文 - ここから
%%

\section{はじめに}

ソフトウェア開発では,開発効率が向上させるため,特定の機能が使いやすい形にまとめられたライブラリが利用される.ライブラリ開発者は,機能追加やバグ修正のために頻繁にソースコードを更新する.ライブラリ更新には,既存機能の削除や変更のようなクライアントに影響を与える変更が含まれる場合,バージョンの更新前後でクライアントの振る舞いが変化し,実行時エラーを発生させることがある.このように,更新後のライブラリがクライアントの動作に影響を与えることを後方互換性を損失するという.後方互換性の損失はライブラリを利用するクライアントの振る舞いの阻害につながるため,クライアントにライブラリの変更内容を伝えることが求められるが,機能の動作変更内容が文書化されていないこともある\cite{UnderstandingWild}.また,ライブラリ開発者がクライアントソフトウェアを実行することなくクライアントへの影響を特定することは容易ではない.

従来研究では,ライブラリのバージョン更新前後でクライアントテストを実行した結果から後方互換性の損失を検出する研究が行われている\cite{mujahid}.ただし,後方互換性の損失によりクライアントでエラーが発生する実装方法やクライアントに与える影響の度合いを分析していない.

本研究では,従来研究において松田ら\cite{matsuda}が分析対象とした2,111組のライブラリのバージョンから8組を分析対象とする.本研究では,更新後にテストが失敗したクライアントからライブラリと関数の呼び出し文を抽出と正規表現への変換により記述パターンを自動生成する.さらに,テストが成功したクライアントのうち,後方互換性の損失の影響を受けているが,テストで気付いていないクライアントを特定する.
\section{分析手法}
\noindent\textbf{手順1.テストが失敗したクライアントからライブラリの呼び出し文,関数呼び出し文を抽出}

ライブラリは,ライブラリの呼び出し文と関数呼び出し文を用いてクライアントで使用される.そこで,ライブラリの呼び出し文は,正規表現を用いて\texttt{import}文または\texttt{require}文を持ち,ライブラリ名を定義しているという条件を満たす行を抽出する.次に関数として使用される部分をノードごとに判別して取得するため,ソースコードを抽象構文木に変換し,変数名をもとに関数呼び出し文を抽出する.

\noindent\textbf{手順2.ライブラリ呼び出し文,関数呼び出し文による記述パターンの自動生成}

ライブラリと関数呼び出し文は,各クライアントが命名した変数名や引数の数のみを考慮して抽象化したのち,正規表現の記述パターンを生成する.ライブラリの呼び出し文とライブラリの関数呼び出し文を紐づけるため,変数名を抽象化した場合に\texttt{variable1}のようなIDを付与し,同一クライアント内で統一する.生成した記述パターン候補において包含関係のあるパターン対が存在する場合に自動的にパターンの集約を行い,より広く適用可能なパターンを残すが,パターンの集約によって誤検出や取りこぼしが発生するリスクはない.

\noindent\textbf{手順3.クライアントテストが成功する中で後方互換性の損失による影響を受けるクライアントの検出}

記述パターンを用いて,テストが成功しているクライアントで後方互換性の損失の影響を受けるクライアントと同じ実装をしているクライアントを検出する.具体的に,手順1のようにテストが成功したクライアントからライブラリの呼び出し文と関数呼び出し文を抽出する.次に記述パターンを適用し,ライブラリの呼び出し文が一致した場合,変数名の部分を取得し,対応する関数呼び出し文の同じ変数名の部分を動的に置換する.続いて関数呼び出し文もライブラリ呼び出し文と同様に検出を行い,クライアントの実装方法が記述パターンに含まれるライブラリ呼び出し文と関数呼び出し文の両方が完全一致するクライアントを特定する.

\section{分析結果}

\begin{table}[h]
     \caption{記述パターンを取得できたクライアント数}
     \label{table:RQ1}
    \centering
     \scalebox{0.55}{
         \begin{tabular}{l|r|r|r}
         \hline 
         \begin{tabular}{c} ライブラリ名@旧バー\\ジョン{...}新バージョン \end{tabular}
         & \begin{tabular}{c} テスト失敗の\\クライアント数 \end{tabular}
         & \begin{tabular}{c} 記述パターン候補 \end{tabular}
         & \begin{tabular}{c}集約後の記述\\パターン種類数\end{tabular}
         \\ \hline \hline
         uuid@7.0.3...8.0.0-beta.0 & 119 & 100 & 9\\
         uuid@3.4.0...7.0.0-beta.0 & 55 & 43  & 11\\
         globby@8.0.0...8.0.1 & 42 & 31 & 10\\
         meow@3.6.0...4.0.0  & 26 & 17 & 5\\
         globby@6.1.0...7.0.0 & 24 & 19 & 7\\
         pump@1.0.3...2.0.0 & 19 & 17 & 2\\
         globby@7.1.1...8.0.0 & 16 & 15 & 5\\
         vinyl@1.2.0...2.0.0  & 11 & 10 & 2\\
         \hline
         \end{tabular}
         }
\end{table}

\begin{table}[h]
    \caption{記述パターンを用いて検出したクライアント数}
    \label{table:RQ2}
    \centering
     \scalebox{0.60}{
        \begin{tabular}{l|r|r|r}
        \hline
        \begin{tabular}{c}ライブラリ名@旧バー\\ジョン{...}新バージョン\end{tabular} & \begin{tabular}[c]{@{}c@{}}テスト失敗の\\クライアント数\end{tabular} & \begin{tabular}[c]{@{}c@{}}テスト成功の\\クライアント数\end{tabular} & \begin{tabular}[c]{@{}c@{}}検出したテスト\\成功数(割合)\end{tabular}\\ \hline \hline
            uuid@7.0.3...8.0.0-beta.0 & 119 & 314 & 38 (12\%)\\
            uuid@3.4.0...7.0.0-beta.0 & 55 & 435 & 117 (27\%)\\
            globby@8.0.0...8.0.1 & 42 & 86 & 61 (71\%)\\
            meow@3.6.0...4.0.0 & 26 & 128 & 77(60\%) \\
            globby@6.1.0...7.0.0 & 24 & 113 & 81 (72\%)\\
            pump@1.0.3...2.0.0 & 19 & 83 & 46(55\%) \\
            globby@7.1.1...8.0.0 & 16 & 127 & 94(74\%) \\
            vinyl@1.2.0...2.0.0 & 11 & 58 & 42(72\%) \\ \hline
        \end{tabular}
    }
\end{table}

表\ref{table:RQ1}にテストに失敗したクライアントから生成した記述パターンの種類数を示す.本研究で分析対象とした8組のライブラリのバージョンにおいて,2種類から11種類の記述パターンを生成できた.生成した記述パターンがライブラリの公開する文書に記載されているかどうかを調査した.その結果,ライブラリで文書化されている記述パターンを正しく生成でき,周知されていない記述パターンの生成も可能であることを確認した.

表\ref{table:RQ2}は,左からライブラリ名とバージョン,更新後にテストを失敗したクライアント数,更新後にテストが成功したクライアント数,更新後のテスト成功の中から記述パターンにより検出されたクライアント数とテスト成功に占める割合を表している.各ライブラリで更新後にテストが成功しているクライアントのうち,本研究で生成した記述パターンを用いて検出した(テストに失敗しているクライアントと同じ呼び出し文を使用している)クライアントは,それぞれ12\%から74\%も存在していることがわかった.これらのクライアントは,更新後にテストを実行してもテストが成功するため,後方互換性の損失の影響を判断できないが,実際には影響を受けている可能性が高い.検出したクライアントがテスト不十分であることを確認するため,一部のクライアントのテストを目視調査した.その結果,クライアントにテストが存在しない, standardやeslintのような静的テストのみを設定,テストがライブラリ使用箇所を通過していない,テストがライブラリの箇所を通過しているが,実行結果の戻り値まで検証していないという要因を確認できた.そのため,本手法が提案する記述パターンを用いることでクライアントのテスト結果のみで判断するよりも後方互換性の損失の影響を受ける可能性の高いクライアントを正確に把握することが期待できる.

\section{おわりに}
本研究では,バージョン更新に伴いテストが失敗したクライアントから記述パターンを生成し,テストが成功したクライアントから記述パターンを含むクライアントの検出を行った.本研究で提案した記述パターンは,ライブラリ開発者がバージョン更新に伴いクライアントがどのような実装方法で後方互換性の損失の影響を受ける実装方法の把握に役立つと考えられる.ライブラリ開発者が後方互換性の損失する実装方法の例を文書に記載することで,クライアントがテストから後方互換性の損失に気づくことができない場合でも注意を促せるようになることを期待する.


%%
%% 本文 - ここまで
%%

%%%%%%%%%%%%%%%%%%%%%%%%%%%%%%%%%%%%%%%%%%%%%%%%%%%%%%%%%%%%%%%%%%%%%%%%

%%
%% 参考文献

\bibliographystyle{junsrt}
\bibliography{@BSthesis2024_Iida/BSthesis2024_Iida}


%%%%%%%%%%%%%%%%%%%%%%%%%%%%%%%%%%%%%%%%%%%%%%%%%%%%%%%%%%%%%%%%%%%%%%%%

\end{document}
