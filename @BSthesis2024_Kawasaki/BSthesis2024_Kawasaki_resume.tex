%
% 卒論レジュメフォーマット Ver.2.0 pLaTeX版
%

\documentclass[twocolumn]{jarticle} % 2段組のスタイルを用いている

\usepackage{wuse_resume}
\usepackage{url}	% \url{}コマンド用.URLを表示する際に便利
\usepackage{otf}
\usepackage{xcolor}
\usepackage[dvipdfmx]{graphicx}
\usepackage{enumitem}
\usepackage{float} % 図表の強制的な配置設定が可能な[H]を使用可能にする
            
\newcommand{\RQOne}{レビューコメントの中から修正要求を抽出することは可能か}
\newcommand{\RQTwo}{レビューコメントの中から修正確認を抽出することは可能か}
\newcommand{\RQThree}{コードレビュー票単位と修正要求単位に基づくタスクの完了状況評価結果は異なるか}

%%%%%%%%%%%%%%%%%%%%%%%%%%%%%%%%%%%%%%%%%%%%%%%%%%%%%%%%%%%%%%%%%%%%%%%%

%%
%% タイトル,学生番号,氏名などを設定する
%%

\タイトル{修正要求・確認コメントの自動抽出による\\
コードレビューのタスク完了状況追跡手法}
\研究室{ソーシャルソフトウェア工学}
\学生番号{60266083}
\氏名{川\UTF{FA11} 晴斗}

\概要{%
コードレビュー対象のソースコードが改善を要する場合,検証者は実装者に修正を要求する.ソフトウェア開発においてプロジェクト管理者は,コードレビューのタスクの完了状況を確認しながら次のリリースに導入可能なソースコードを決定する.しかし,大規模なプロジェクトでは,膨大なコードレビューが並行で進められ,個々のタスクの完了状況を確認することは容易ではない.本研究では,レビューコメントからソースコードの修正を求めるコメント,およびソースコードの修正が確認されたことを示すコメントをそれぞれ自動抽出し,コードレビューのタスクの完了状況を追跡する手法を提案する.
}

\キーワード{タスク管理}
\キーワード{オープンソースソフトウェア}
\キーワード{コードレビュー}
\キーワード{テキスト分類}

%%%%%%%%%%%%%%%%%%%%%%%%%%%%%%%%%%%%%%%%%%%%%%%%%%%%%%%%%%%%%%%%%%%%%%%%

%% 以下の3行は変更しない

\begin{document}
\maketitle
\thispagestyle{empty} % タイトルを出力したページにもページ番号を付けない

%%%%%%%%%%%%%%%%%%%%%%%%%%%%%%%%%%%%%%%%%%%%%%%%%%%%%%%%%%%%%%%%%%%%%%%%

%%
%% 本文 - ここから
%%

%%%%%%%%%%%%%%%%%%%%%%%%%%%%%%%%%%%%%%%%%%%%%%%%%%%%%%%%%%%%%%%%%%%%%%%%

%%%%%%%%%%%%%%%%%%%%%%%%%%%%%%%%%%%%%%%%%%
% 1章 はじめに
\section{はじめに}\label{sec:intro}
%%%%%%%%%%%%%%%%%%%%%%%%%%%%%%%%%%%%%%%%%%

オープンソースソフトウェア(OSS)は,ソースコードが公開され,定められたライセンスの下で自由に利用・修正・再配布が可能なソフトウェアである\cite{oss}.OSS開発では,開発者(実装者)が変更したソースコードを提出し,別の開発者(検証者)によってコードレビューが実施される.コードレビューでは,コードの欠陥や可読性の確認が行われ,必要に応じて修正が要求される.

コードレビューはプロジェクトのリリースに直接影響を与える重要な要素であり,各タスクの進捗を計画し完了状況を追跡する必要がある\cite{review_time}.ソフトウェア開発において,期限までに完了すべきコードレビュー票の件数は把握できているが,個々のコードレビューのタスクの完了状況は十分に追跡できていない.

本研究では,3つのResearch Questions(RQ)を設定し,レビューコメントからタスクの完了状況を追跡する手法の確立を目指す.

\begin{itemize}
\item RQ1:\RQOne
\item RQ2:\RQTwo
\item RQ3:\RQThree
\end{itemize}

%%%%%%%%%%%%%%%%%%%%%%%%%%%%%%%%%%%%%%%%%%
% 2章 データセット
\section{データセット}\label{sec:rq1}
%%%%%%%%%%%%%%%%%%%%%%%%%%%%%%%%%%%%%%%%%%

本研究では,OpenStackのコアコンポーネントであるNova,Neutron,Cinder,Keystone,Swift,Glanceと大規模プロジェクトのHorizonの合計7つのプロジェクトのうち,プロジェクト立ち上げ時から2022年9月までにリポジトリに導入された89,552件のコードレビュー票と,そのコードレビュー票に投稿された4,420,333件のレビューコメントを分析対象とする.

%%%%%%%%%%%%%%%%%%%%%%%%%%%%%%%%%%%%%%%%%%
% 3章 RQ1
\section{RQ1:\RQOne}\label{sec:rq1}
%%%%%%%%%%%%%%%%%%%%%%%%%%%%%%%%%%%%%%%%%%

\subsection{手法}
本研究で分析対象とするから信頼区間95\%,許容誤差5\%となる383件のコードレビュー票を無作為に選択し,そのコードレビュー票に投稿されているレビューコメント12,086件のレビューコメントに対して,修正要求か否かを目視調査する.また,目視調査で得られたデータセットを用いて,自然言語処理モデルのBERTにファインチューニングを行い,修正要求の抽出精度を5分割検証を用いて評価する.また,評価指標には5回の評価の適合率,再現率,F値の中央値を用いる.

\subsection{結果}
目視調査にて,12,086件のレビューコメントから874件の修正要求を抽出した.また,表\ref{table:request_score}は構築したモデルの評価結果を示す.全て評価指標が高いことから,BERTを用いた手法は高い精度でレビューコメントから修正要求を抽出できることを明らかにした.

%-----------------------
\begin{table}[t]
\centering
  \caption{修正要求の予測精度}
  \label{table:request_score}
  \scalebox{0.86}{   
  \begin{tabular}{l|r|r|r}  \hline \hline
    \multicolumn{1}{c|}{スコア(件数)} & \multicolumn{1}{c|}{適合率} & \multicolumn{1}{c|}{再現率} & \multicolumn{1}{c}{F値}\\ \hline
    全レビューコメント(12,086件) & 0.84 & 0.78 & 0.81\\ \hline    
  \end{tabular}
  }
\end{table}
%-----------------------

%%%%%%%%%%%%%%%%%%%%%%%%%%%%%%%%%%%%%%%%%%
% 4章 RQ2
\section{RQ2:\RQTwo}\label{sec:rq2}
%%%%%%%%%%%%%%%%%%%%%%%%%%%%%%%%%%%%%%%%%%

\subsection{手法}
RQ1と同様の無作為に選択したデータセットに対して,修正確認か否かを目視調査する.また,目視調査で得られた知見をもとにパターンマッチを行い,修正確認の抽出精度を評価する.評価指標には,適合率,再現率,F値を用いる.

\subsection{結果}
目視調査にて,12,086件のレビューコメントから1,874件の修正確認を抽出した.また,目視調査にて,\textit{``LGTM''},\textit{``Looks good''},\textit{``Looks ok''}の語彙のみ,「修正確認として分類された回数が10回以上」かつ,「その語彙を含むコメントは90\%以上の割合で修正確認と分類される」ということが確認できた.そのため,本研究で対象とするOpenStackで用いられる承認を示す検証評価ラベル,または\textit{``LGTM''},\textit{``Looks good''},\textit{``Looks ok''}のいずれかが含まれるコメントを修正確認として判断するパターンマッチを行った.表\ref{table:achieve_score}はパターンマッチの評価結果を示す.全ての評価指標が高いことから,パターンマッチを用いた手法は高い精度でレビューコメントから修正確認を抽出できることを明らかにした.

%-----------------------
\begin{table}[t]
\centering
  \caption{修正確認コメントの予測精度}
  \label{table:achieve_score}
  \scalebox{0.86}{   
  \begin{tabular}{l|r|r|r}  \hline \hline
    \multicolumn{1}{c|}{スコア(件数)} & \multicolumn{1}{c|}{適合率} & \multicolumn{1}{c|}{再現率} & \multicolumn{1}{c}{F値}\\ \hline
    全レビューコメント(12,086件) & 0.94 & 0.97 & 0.96\\ \hline    
  \end{tabular}
  }
\end{table}
%-----------------------

%%%%%%%%%%%%%%%%%%%%%%%%%%%%%%%%%%%%%%%%%%
% 5章 RQ3
\section{RQ3:\RQThree}\label{sec:rq3}
%%%%%%%%%%%%%%%%%%%%%%%%%%%%%%%%%%%%%%%%%%

\subsection{手法}
本研究で対象としたデータセット全てを対象に,RQ1とRQ2の手法を用いて修正要求と修正確認を抽出する.抽出した修正要求と修正確認を「修正確認の投稿以前の修正要求は全て検証されたものとみなす」に基づいて紐付けを行う.

次に,従来のコードレビュー票に基づく手法と,本研究の修正要求に基づく,コードレビューのタスクの完了状況追跡の評価結果の違いを検証する.

\subsection{結果}
定義に基づいて紐付けを行うと,4,420,333件のレビューコメントから抽出した239,286件の修正要求のうち,86.6\%を紐づけることができた.

また,コードレビュー票の状態として投稿はされているが,導入はされていないものを「未完了のコードレビュー票」,修正要求は投稿されているが,修正確認と紐づいていない修正要求を「未完了の修正要求」としたとき,本研究で対象とする期間において,1日ごとの未完了のコードレビュー票と未完了の修正要求の件数の分布を調査した.調査の結果,全てのプロジェクトでMann-WhitneyのU検定において,有意水準1\%を下回っており,両者の分布間に統計的に有意な差が認められた.

表\ref{table:close_time}は,クローズ時間までに投稿された修正要求が0から4件であったコードレビュー票について,クローズまでの所要時間の中央値が大きくなることが確認できる.

これらの知見から,修正要求単位でのタスクを抽出を行うことで,従来では困難であった個々のコードレビュー票ごとのタスクの重みの比較やより詳細なタスクの完了状況の追跡が可能になることが分かる.

%-----------------------
\begin{table}[t]
\centering
  \caption{修正要求の件数ごとのコードレビュー票のクローズ時間}
  \label{table:close_time}
  \scalebox{1.00}{   
  \begin{tabular}{r|r}  \hline \hline
    \multicolumn{1}{c|}{修正要求の件数} & \multicolumn{1}{c}{クローズ時間の中央値(日)}\\ \hline
    0 & 3.13\\
    1 & 6.71\\
    2 & 9.09\\
    3 & 12.36\\
    4 & 14.97\\ \hline
  \end{tabular}
  }
\end{table}
%-----------------------

%%%%%%%%%%%%%%%%%%%%%%%%%%%%%%%%%%%%%%%%%%
% 6章 おわりに
\section{おわりに}\label{sec:conclusion}
%%%%%%%%%%%%%%%%%%%%%%%%%%%%%%%%%%%%%%%%%%

本研究では,コードレビュー票に投稿された修正要求および修正確認に基づいたコードレビューのタスクの完了状況の追跡手法を提案し,3つのRQを検証した.

RQ1,およびRQ2ではBERTとパターンマッチを用いてレビューコメントから修正要求と修正確認を高い精度で抽出できることを確認した.

また,RQ3では,定義した紐付け手法を用いて修正要求と修正確認を紐づけることが可能であるかを検証した.その結果,高い割合で修正要求と修正確認を紐づけることが可能であることを確認した.また,分析対象区間内の修正要求の完了度合いの分布は,従来のコードレビュー票の分布や推移と異なることを確認した.

本研究で提案した修正要求単位でのコードレビューのタスクの完了状況追跡手法により,個々のコードレビュー票のタスクの完了状況を詳細に把握することが可能となる.この手法により,ソフトウェアの開発プロセスの効率化に貢献することを期待する.

%%%%%%%%%%%%%%%%%%%%%%%%%%%%%%%%%%%%%%%%%%%%%%%%%%%%%%%%%%%%%%%%%%%%%%%%

%%
%% 本文 - ここまで
%%

%%%%%%%%%%%%%%%%%%%%%%%%%%%%%%%%%%%%%%%%%%%%%%%%%%%%%%%%%%%%%%%%%%%%%%%%

%%
%% 参考文献
%%

\bibliographystyle{junsrt}
\bibliography{@BSthesis2024_Kawasaki/BSthesis2024_Kawasaki}

%%%%%%%%%%%%%%%%%%%%%%%%%%%%%%%%%%%%%%%%%%%%%%%%%%%%%%%%%%%%%%%%%%%%%%%%

\end{document}