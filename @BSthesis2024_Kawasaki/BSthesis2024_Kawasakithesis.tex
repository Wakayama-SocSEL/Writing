\documentclass[11pt]{jreport}
\usepackage{wuse_thesis}
\usepackage{indentfirst}
\usepackage{url}	% \url{}コマンド用.URLを表示する際に便利
\usepackage{otf}
%\usepackage{graphicx}  % ←graphicx.styを用いてEPSを取り込む場合有効にする
			% 他のパッケージ・スタイルを使う場合には適宜追加
            
\newcommand{\RQOne}{レビューコメントにおいて修正要求を分類することは可能か}
\newcommand{\RQTwo}{レビューコメントにおいて修正確認を分類することは可能か}
\newcommand{\RQThree}{レビュー票単位と修正要求単位に基づく進捗状況評価結果は異なるか}

\newcommand{\todo}[1]{\colorbox{yellow}{{\bf TODO}:}{\color{red} {\textbf{[#1]}}}}
\newcommand{\change}[1]{\colorbox{green}{{\bf CHANGE}:}{\color{blue} {\textbf{[#1]}}}}
\newcommand{\new}[1]{\colorbox{cyan}{{\bf NEW}:}{\color{black} {\textbf{[#1]}}}}
%%%%%%%%%%%%%%%%%%%%%%%%%%%%%%%%%%%%%%%%%%%%%%%%%%%%%%%%%%%%%%%%%%%%%%%%

%%
%% 主に表紙を作成するための情報
%%

%%  タイトル(修論の場合は英語表記も指定)
\title{修正要求確認コメントの自動抽出による\\
コードレビューのタスク完了状況追跡手法に向けて}
%\etitle{Test\\Test\\Test}

%%  著者名(修論の場合は英語表記も指定)
\author{川\UTF{FA11} 晴斗}
%\eauthor{Akinori Ihara}

%% 卒業論文・修士論文(以下のどちらかを選択)
\bachelar	% 卒業論文(4年生用)
%\master  	% 修士論文(M2用)

%%  学科・クラスタ
\department{システム工}
%\department{デザイン情報}
%\department{デザイン科学}

%%  学生番号
\studentid{60266083}

%%  卒業年度
\gyear{2024}		% 提出年が2022年なら,2021年度

%%  論文提出日
\date{2025年2月12日}	% 修士の場合は月(2021年2月)までとし,英語表記も指定
%\edate{February 2021}	% 修士の場合,こちら(英語表記)も有効化

%%%%%%%%%%%%%%%%%%%%%%%%%%%%%%%%%%%%%%%%%%%%%%%%%%%%%%%%%%%%%%%%%%%%%%%%

\begin{document}

\maketitle

%%
%%  概要
%%
\begin{abstract}
本稿では,
「和歌山大学システム工学部卒業論文/
  大学院システム工学研究科修士論文用スタイルファイル」
を用いて卒業論文を作成する方法を解説する.
本稿自身,
「和歌山大学システム工学部卒業論文/
  大学院システム工学研究科修士論文用スタイルファイル」
を用いて記述されており,例によってその使い方を示している.
「和歌山大学システム工学部卒業論文/
  大学院システム工学研究科修士論文用スタイルファイル」
では,タイトルページ,概要,目次,参考文献などの書式を設定している.

「和歌山大学システム工学部卒業論文/
  大学院システム工学研究科修士論文用スタイルファイル」
は,
\begin{quote}
  \begin{description}
    \item[\tt wuse\_thesis.sty:] 卒業/修士論文用スタイルファイル
    \item[\tt thesis\_sample.tex:] スタイルファイル利用例
  \end{description}
\end{quote}
からなる.

なお,この卒業論文用スタイルファイル(p\LaTeX 版)に関する質問は,
メールにて
\begin{quote}
ihara@wakayama-u.ac.jp
\end{quote}
まで.

\end{abstract}

%%  目次
\tableofcontents

%%  図目次 (図目次をいれたければ以下のコメントをはずす)
%\listoffigures

%%  表目次 (表目次をいれたければ以下のコメントをはずす)
%\listoftables

\newpage
\pagenumbering{arabic}	% 以降のページ番号を算用数字に

%%%%%%%%%%%%%%%%%%%%%%%%%%%%%%%%%%%%%%%%%%%%%%%%%%%%%%%%%%%%%%%%%%%%%%%%

%%
%%  本文はここから
%%

%%%%%%%%%%%%%%%%%%%%%%%%%%%%%%%%%%%%%%%%%%
\chapter{ケーススタディ}
\section{概要}
コードレビュー票には多数のコメントが記録されている.実装者が投稿するコメントには,作成したソースコードの補足,検証者からの質問に対する回答などがある.検証者が投稿するコメントには,実装者への修正要求,提出されたパッチへの問い,修正要求が解決したことの確認などがある.RQ1では,コードレビュー票に記録されたコメントの中から,検証者がコードレビューによって更なる修正を依頼するコメント(修正要求コメント)か否か,ソースコードが修正されたことを検証者が確認したことを伝えるコメント(確認コメント)か否か,をそれぞれ分類できるか否かを目視調査する.

\section{データセット}

%%%%%%%%%%%%%%%%%%%%%%%%%%%%%%%%%%%%%%%%%%

%%%%%%%%%%%%%%%%%%%%%%%%%%%%%%%%%%%%%%%%%%
\chapter{RQ1:\RQOne}\label{sec:RQ1}
%%%%%%%%%%%%%%%%%%%%%%%%%%%%%%%%%%%%%%%%%%



\chapter{タイトルページ,概要,目次}

{\bf 「和歌山大学システム工学部卒業論文/
大学院システム工学研究科修士論文用スタイルファイル」}\cite{wusethesis}
では,専用のタイトルページを出力する.
記述すべき項目は,
\begin{itemize}
  \item タイトル
  \item 著者名
  \item 学士(4年)/修士(M2)の設定
  \item 学科名/クラスタ名
  \item 学生番号
  \item 卒業年度
  \item 論文提出日
\end{itemize}
である.
これらのデータは,\verb|\maketitle|によってタイトルページに出力される.
また,概要の部分において,論文の内容をまとめる.その内容は論文の2ペー
ジ目(タイトルページの次)に出力される.
このソースでは,目次(\verb|\tableofcontents|)を出力している.
他に,図目次(\verb|\listoffigures|),表目次(\verb|\listoftables|)を
出力することもできるので,必要ならばそれぞれのコメントをはずす.
図目次,表目次については,第\ref{chap:fig-tab-exp}章において説明する.

\section{タイトル}
\subsection{title}
論文のタイトルを記述する.

\section{著者}
\subsection{author}
著者名を記述する.

\subsection{bachelar/master}
卒業論文の場合には,\verb|\master|をコメントアウトし,
\verb|\bachelar|を設定する.
修士論文の場合には,\verb|\bachelar|をコメントアウトし,
\verb|\master|を設定する.

\subsection{department}
所属を記述する.
システム工学科所属(学部生)の場合には``システム工'',
デザイン科学クラスタ所属(大学院生)の場合には``デザイン科学''と記述する.

\subsection{studentid}
学生番号を記述する.

\subsection{gyear}
卒業年度を記述する.

\section{提出日}
\subsection{date}
論文提出日を記述する.


\chapter{図,表,数式}\label{chap:fig-tab-exp}

論文では,図,表,数式などを効果的に使用する.

\section{図}

{\tt figure}環境を利用することによって図にキャプション
(\verb|\caption|)を付けることができる.図に付けられたキャプションは
\verb|\listoffigures|によって図目次として出力される.図には章ごとに通
し番号が付けられ,キャプションに\verb|\label|を設定しておくと,
``図\ref{fig:sample}''のように\verb|\ref|によって図を番号で参照するこ
とができる.図\ref{fig:sample}に{\tt figure}環境を用いた記述例を示す.

\begin{figure}
  \centering
    ここで図を取り込む.
    % 試しに,tiger.psが自分のマシンのどこに格納されているかを調べて
    % 以下の命令を有効にしてみて下さい.
    % ただし,同時に\begin{document}より前にある\usepackage{graphicx}
    % も有効にする必要があります.
    %\includegraphics[width=5cm,clip]{/usr/local/share/ghostscript/7.07/examples/tiger.ps}
  \caption{図の例}
  \label{fig:sample}
\end{figure}

また,{\tt graphicx.sty}などのスタイルファイルを利用することによって
EPS形式やPDF形式の図を文章の中に取り込むことができる.
この場合,\verb|\begin{document}|の前に\verb|\usepackage{graphicx}|を
追加する.

なお,図表の配置は基本的には\LaTeX{}が決めるので,思った位置に入らない
からといって無理に場所を指定するのはよくない.
どうしても位置を固定したい場合には,すべての文章が書きあがった後に指定
するとよい\footnote{そうしないと文章を書き換えるたびに,位置がずれる可能性がある}.

\section{表}

{\tt table}環境を利用することによって図と同じように,キャプションをつ
けたり,ラベルにより参照したりすることができる.また
\verb|\listoftables|によって表目次として出力される.
表\ref{tab:sample}に{\tt table}環境で作成した表を示す.

\begin{table}
  \caption{表の例}
  \label{tab:sample}
  \centering
  \begin{tabular}{|c|c|c|}
    \hline
    8 & 3 & 4\\
    \hline
    1 & 5 & 9 \\
    \hline
    6 & 7 & 2 \\
    \hline
  \end{tabular}
\end{table}

\section{数式}

\TeX では数式のための機能が豊富である.
{\tt equation}環境などを利用することによって数式に番号を付けることがで
きる.図や表と同じくラベルを付けておけば,``式\ref{exp:sample}''のよう
に数式を番号で参照することができる.

\begin{equation}
  y = ax^2 + bx + c \label{exp:sample}
\end{equation}

\chapter{参考文献}

文献を参照する場合には,論文の最後に参考文献として列挙するとともに,
\verb|\cite|を使って,例えば,
\begin{quote}
  文献\cite{latex}によれば…
\end{quote}
や,
\begin{quote}
  …である\cite{latex2e}.
\end{quote}
のように参照する.

文献の列挙には,{\tt thebibliography}環境などを用いる\footnote{使い方
は,この資料のソースを参照.}.

%%%%%%%%%%%%%%%%%%%%%%%%%%%%%%%%%%%%%%%%%%%%%%%%%%%%%%%%%%%%%%%%%%%%%%%%

%%
%% 謝辞
%%
%% \begin{acknowledgements}
%% 感謝します.
%% \end{acknowledgements}

%%%%%%%%%%%%%%%%%%%%%%%%%%%%%%%%%%%%%%%%%%%%%%%%%%%%%%%%%%%%%%%%%%%%%%%%

%%
%% 参考文献
%%
\begin{thebibliography}{99}

\bibitem{wusethesis}
  伊原彰紀,
  卒業論文スタイルファイル(和歌山大学システム工学部用),\\
  \url{https://github.com/fukuyasu/wuse_thesis}.

\bibitem{tex}
  Knuth, D.,
  Remarks to Celebrate the Publication of Computers \& Typesetting,
  TUGboat, Vol.7, No.2, pp.95--98, 1986.

\bibitem{latex}
  Lamport, L.,
  文書処理システム\LaTeXe{},
  ピアソン・エデュケーション,1999,
  \newblock{}阿瀬はる美 訳.

\bibitem{latex_j}
  奥村晴彦,\LaTeX{}入門 ---美文書作成のポイント---,技術評論社,1993.

\bibitem{latex2e}
  奥村晴彦,黒木裕介,[改定第6版] \LaTeXe~美文書作成入門,技術評論社,2013.

\bibitem{latexcomp}
  Goossens, M., Mittelbach, F. and Samarin, A.,
  The \LaTeX{}コンパニオン,アスキー出版局,1998,
  \newblock{}アスキー書籍編集部 監訳.

\bibitem{texwiki}
  \LaTeX 入門 --- \TeX{} Wiki,\\
  \url{https://texwiki.texjp.org/?LaTeX%E5%85%A5%E9%96%80},
  2021年12月3日閲覧.
\end{thebibliography}

%%%%%%%%%%%%%%%%%%%%%%%%%%%%%%%%%%%%%%%%%%%%%%%%%%%%%%%%%%%%%%%%%%%%%%%%

%%
%% 付録
%%
% \appendix
% 
% \chapter{サンプルプログラム}
% 
% プログラムリストや実行結果など,本論を補足する上で必要と思われるものが
% あれば付録として付ける.
% 
% {
% \footnotesize
% \begin{verbatim}
% #include <stdio.h>
% int main(void)
% {
%     printf("Hello, World!\n");
%     return 0;
% }
% \end{verbatim}
% }

%%%%%%%%%%%%%%%%%%%%%%%%%%%%%%%%%%%%%%%%%%%%%%%%%%%%%%%%%%%%%%%%%%%%%%%%

\end{document}
