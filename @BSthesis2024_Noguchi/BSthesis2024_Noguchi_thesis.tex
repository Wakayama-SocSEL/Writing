\documentclass[11pt]{jreport}
\usepackage{wuse_thesis}
\usepackage{indentfirst}
\usepackage{url}	% \url{}コマンド用.URLを表示する際に便利
\usepackage{xcolor} 	% 色を扱うためのパッケージ
\usepackage{listings}
\usepackage[dvipdfmx]{graphicx}
\usepackage{amsmath}
\usepackage{multirow}
% \usepackage{graphicx}  % ←graphicx.styを用いてEPSを取り込む場合有効にする
			% 他のパッケージ・スタイルを使う場合には適宜追加

% コード表示のカスタマイズ

\lstset{
  basicstyle={\ttfamily},
  identifierstyle={\small},
  commentstyle={\smallitshape},
  keywordstyle={\small\bfseries},
  ndkeywordstyle={\small},
  stringstyle={\small\ttfamily},
  frame={tb},
  breaklines=true,
  columns=[l]{fullflexible},
  numbers=left,
  xrightmargin=0zw,
  xleftmargin=3zw,
  numberstyle={\scriptsize},
  stepnumber=1,
  numbersep=1zw,
  lineskip=-0.5ex,
  escapeinside={(*@}{@*)},
}
\renewcommand{\lstlistingname}{Program}
% ハイライト用の色定義(赤と緑)
\definecolor{highlightred}{rgb}{1.0, 0.8, 0.8} % 淡い赤
\definecolor{highlightgreen}{rgb}{0.8, 1.0, 0.8} % 淡い緑

\newcommand{\RQone}{コーディングパターンとして抽出すべき基準はどこなのか}
\newcommand{\RQtwo}{抽出したコーディングパターンはプロジェクトでどの程度使用されているのか}

\newcommand{\todo}[1]{\colorbox{yellow}{{\bf TODO}:}{\color{red} {\textbf{[#1]}}}}
\newcommand{\change}[1]{\colorbox{green}{{\bf CHANGE}:}{\color{blue} {\textbf{[#1]}}}}
\newcommand{\new}[1]{\colorbox{cyan}{{\bf NEW}:}{\color{black} {\textbf{[#1]}}}}

%%%%%%%%%%%%%%%%%%%%%%%%%%%%%%%%%%%%%%%%%%%%%%%%%%%%%%%%%%%%%%%%%%%%%%%%

%%
%% 主に表紙を作成するための情報
%%

%%  タイトル(修論の場合は英語表記も指定)
\title{プロジェクト特有のコーディングパターン収集方法の定量的評価}

%%  著者名(修論の場合は英語表記も指定)
\author{野口 朋弥}

%% 卒業論文
\bachelar	% 卒業論文(4年生用)

%%  学科・クラスタ
\department{システム工}

%%  学生番号
\studentid{60266227}

%%  卒業年度
\gyear{2024}		% 提出年が2022年なら,2021年度

%%  論文提出日∂
\date{2025年2月12日}	% 修士の場合は月(2021年2月)までとし,英語表記も指定
%\edate{February 2021}	% 修士の場合,こちら(英語表記)も有効化

%%%%%%%%%%%%%%%%%%%%%%%%%%%%%%%%%%%%%%%%%%%%%%%%%%%%%%%%%%%%%%%%%%%%%%%%

\begin{document}

\maketitle

%%
%%  概要
%%
\begin{abstract}
\todo{hoge}

\end{abstract}

%%  目次
\tableofcontents

%%  図目次 (図目次をいれたければ以下のコメントをはずす)
%\listoffigures

%%  表目次 (表目次をいれたければ以下のコメントをはずす)
%\listoftables

\newpage
\pagenumbering{arabic}	% 以降のページ番号を算用数字に

%%%%%%%%%%%%%%%%%%%%%%%%%%%%%%%%%%%%%%%%%%%%%%%%%%%%%%%%%%%%%%%%%%%%%%%%

%%
%%  本文はここから
%%
%%%%%%%%%%%%%%%%%%%%%%%%%%%%%%%%%%%%%%%%%%
\chapter{はじめに}
%%%%%%%%%%%%%%%%%%%%%%%%%%%%%%%%%%%%%%%%%%
昨今のソフトウェア開発では,開発効率の向上を目指した深層学習や正規表現などを利用したソースコードの自動生成,自動修正手法が多数提案されている\cite{deep_learnnig1}\cite{deep_learning2}\cite{deep_learnnig3}\cite{deep_learnning4}.
これらの手法の多くは,GitHub\footnote{GitHub:~\url{https://github.com}},Gerrit\footnote{Gerrit:~\url{https://www.gerritcodereview.com}}などでオープンソースソフトウェアとして公開されている大規模なプロジェクトな開発履歴を学習データとして利用している.GitHubには,数百万を超えるプロジェクトが公開されており,多様なプログラミング言語やコーディングスタイルで実装されたソースコードが蓄積されている.蓄積されたソースコードから,関数やクラスの実装方法やAPIの使用方法,エラーハンドリングなどの特徴を深層学習や正規表現によって形式化することで,ソースコードの自動生成,自動修正手法を実現している.また,開発履歴には再利用を容易にする実装パターンも多く含み,生成したソースコードの保守性や品質を担保するのに役立ち,長期的な保守や拡張を実現する.
%さらに,ソースコードの自動生成手法は,デザインパターンと呼ばれるソフトウェア設計の知識も学習に利用している.デザインパターンは,特定の問題に対する再利用可能な設計パターンを提供し,生成したソースコードの保守性や品質を担保するのに役立っている.
%これにより,自動生成されたコードが単に動作するだけでなく,長期的なメンテナンスや拡張が容易なものとなることが期待される.
しかし,個々のプロジェクトでは,ドメイン固有の要件が存在する.例えば,金融システムや医療システムといった重要インフラでは,セキュリティ要件や法的規制が厳格であり,独自のアーキテクチャ設計が不可欠となっている\cite{finacial}\cite{financial2}\cite{medical}.システムの利用場所の違いに限らず,コーディング規約は組織によって異なるため,汎用的な生成手法では,ドメイン固有の要件を満たすソースコードを適切に生成できないことも多い.

従来研究では,個別プロジェクトに特化した実装方法を再利用可能な形に抽象化したもの(以降コーディングパターン)を抽出する手法が提案されている.しかし,実装方法全てをコーディングパターンとして抽出すると膨大になってしまい,その中にはプロジェクトの実装方法として現在は利用されていないコーディングパターンも多数含まれる.本研究では,個別のプロジェクトに特化したコーディングパターンを抽出する条件を提案し,ソースコードの自動生成,自動修正手法に有用なパターンの収集手法を開発し,定量的に評価する.具体的には,2つのResearch Question(RQ)に回答する.
\begin{itemize}
    \item RQ1: \RQone
    \item RQ2: \RQtwo
\end{itemize}

RQ1では,プロジェクトから生成された実装方法から,コーディングパターンとして含めるべき基準を分析し,
RQ2では,検出したコーディングパターンが出現する期間,その後も使用されているのかなどを分析することで,有用性の評価を行う.

以降,本論文では,\ref{sec:related}章でコーディングパターンと,従来のデザインパターンの違いと,本研究の位置付けを述べ,\ref{sec:pattern}章では,プロジェクトの開発履歴から,変更差分を取得し,コーディングパターンになり得る候補を生成する手法について説明する.\ref{sec:dataset}章では本研究で使用するデータセット,\ref{sec:pattern}章では抽出したコーディンパターン候補を述べ,\ref{sec:filter}章でRQ1,\ref{sec:time}章でRQ2のそれぞれの動機,手法,結果,考察について述べ,6章で本研究の妥当性について述べ,7章でまとめる.
% 多人数でソフトウェアを開発する場合,開発チーム全体であらかじめルールを共有することが重要である.共通のルールの導入により,開発効率の向上,ソフトウェア品質の改善,保守性の向上などが期待できる.\todo{引用}
% 開発チームで共有するルールの1例として,デザインパターンがある.
% デザインパターンとは,過去のソフトウェア設計者が発見し,編み出してきた設計のノウハウを体系的に蓄積し,再利用しやすい形式でまとめたものである.
% デザインパターンを活用することで,開発プロセスを効率化するとともに,設計の一貫性を保つことができる.\todo{引用}

% デザインパターンはプロジェクト全般で活用可能な汎用性を重視して設計されている.しかし,デザインパターンを採用するだけでは,個別のプロジェクトや開発チームごとの特性や要件を十分に反映することは難しい.
%%%%%%%%%%%%%%%%%%%%%%%%%%%%%%%%%%%%%%%%%%
\chapter{コーディングパターン}\label{sec:related}
%%%%%%%%%%%%%%%%%%%%%%%%%%%%%%%%%%%%%%%%%%
\section{デザインパターン\todo{消すかもなのでまだ見ない}}
多人数でソフトウェアを開発する場合,ソフトウェアの開発効率,品質,保守性の向上はプロジェクト成功の鍵となる.これらの特性を高めるため,ソースコードの記述方法や構造を統一する基準としてデザインパターンが用いられる.
デザインパターンは,ソフトウェア設計において繰り返し発生する問題に対する再利用可能な解法を,パターン名と設計原則に基づいて体系化したものである.特定のプログラミング言語やフレームワークに依存せず,様々な開発場面で適用可能である\todo{引用}

代表的なデザインパターンとして,GoF(Gang of Four)が提唱した23種類のパターンが知られる\cite{GoF}.
これらは次の3つのパターンに分類できる.
\begin{description}
    \item[\textbf{生成}] オブジェクトの生成プロセスを抽象化し,システムが特定のクラスに依存しない柔軟な設計を実現するためのパターン(例: Factory Method, Singleton).
    \item[\textbf{構造}] クラスやオブジェクトの構成や関係を整理し,柔軟なシステム構造を構築するためのパターン(例: Adapter, Composite).
    \item[\textbf{振る舞い}] オブジェクト間の相互作用やアルゴリズムのカプセル化により,責任の分配や通信プロセスを最適化するためのパターン(例: Observer, Strategy).
\end{description}
% \begin{description}
%     \item[\textbf{可読性}] 他の開発者が理解しやすいソースコードを記述すること.変数名や関数名を役割が直感的にわかるように命名することが推奨される.また,適切なコメントやインデントを使用することで,可読性を向上させる.
%     \item[\textbf{保守性}] ソースコードを継続的に管理する過程で,簡単に変更や修正を行えるかを指す.ソースコードの重複を避け,関数やモジュールを適切に分割することで,他の部分への影響を最小限に抑える.
%     \item[\textbf{パフォーマンス}] 適切なアルゴリズム,データ構造を選択することで,処理速度やメモリ使用量を最適化する.
% \end{description}
\section{コーディングパターン}
コーディングパターンは,各プロジェクトや開発チームの修正方法を再利用可能な形で抽象化し,ソースコードを形式化したものである.
デザインパターン\cite{GoF}が設計レベルの汎用性を重視するのに対し,コーディングパターンは実装レベルでプロジェクトの特性を捉えることで,開発効率とソースコードの品質向上を目指す.

従来研究では,コーディングパターンの自動適用・検証技術が多数提案されている.例えば,コードレビューで頻繁に指摘されるパターン違反を抽出するため,開発履歴から頻出する修正パターンを機械学習で抽出し,開発者に提示する手法がある\cite{AutomaticPatch}\cite{findBugs}\cite{Relntancer}\cite{don'tDIY}.

Zhangらは,PythonとJavaの構文を体系的に比較することで,9種類のPython独自のコーディングパターン(リスト内包表記など)を特定した.開発履歴の解析を通じて,頻出するPython独自のコーディングパターンに逸脱するソースコードを抽出する.
これにより,コードレビューの効率化とPythonらしいソースコードの品質向上に成功している.

Uedaらが開発したツールDevReplay\cite{devreplay}は,抽象構文木(AST)の差分解析に基づき,トークン単位で修正方法を収集する.その後,収集した修正方法を抽象化することにより,コーディングパターンを特定した.これにより,従来の静的解析ツールでは検出困難であったプロジェクト固有の特性に依存するコーディングパターン(例: プロジェクト独自の依存関係に関するもの)を抽出可能であった.

\subsection{動機}
Uedaらは,単一のプロジェクトの開発履歴から頻出する修正方法を抽象化することで,コーディングパターンを特定した.しかし,特定したコーディングパターンがプロジェクトの修正方法を反映しているかを十分に検証されていない.その結果,従来手法では,不要なパターンが大量に生成される,または必要なパターンを取りこぼしている課題がある.
本研究ではこれらの課題を解決するために,個別プロジェクトに特化したコーディングパターンを抽出する手法を提案するとともに,取得した修正方法から,有用なコーディングパターンを選別する評価基準の確立を目指す.

% \section{従来研究}
% 従来研究では,チームで定めたコーディングスタイルに準拠していないパッチを自動修正するために,ソフトウェアの開発履歴で頻出するソースコードの修正方法をパターンとして収集し,頻繁にコードレビューで指摘を受けるソースコードの修正方法を開発者に提示する手法が提案されている.\cite{AutomaticPatch}\cite{findBugs}\cite{Relntancer}\cite{don'tDIY}

% Uedaらは,頻繁に修正される修正方法を検出するためのツールDevReplayを開発している.当該手法は,開発履歴から頻出する修正方法をトークン単位でパターンとして収集することで,頻繁にコードレビューで指摘を受けるソースコードの修正方法を開発者に提示する.\cite{devreplay}
% 当該手法により,従来の静的解析ツールでは検出することが困難だったパターンも検出可能であった.
% しかし,単一プロジェクトの開発履歴が少ない場合,コーディングパターンを特定するのが困難になる.
% そのため,本研究では,複数のプロジェクトから開発履歴を収集し,パターンを生成することで,各プロジェクトにおけるコーディングパターンを特定するためのデータを補完することを目的とする.これにより,単一プロジェクトでは捉えきれないパターンを網羅し,より多様なコーディングパターンの特定を可能にする.

%%%%%%%%%%%%%%%%%%%%%%%%%%%%%%%%%%%%%%%%%%
\chapter{コーディングパターン候補の作成}\label{sec:pattern}
%%%%%%%%%%%%%%%%%%%%%%%%%%%%%%%%%%%%%%%%%%
%-------------------
\begin{figure}[h]
\includegraphics[width=0.95\linewidth]{@BSthesis2024_Noguchi/Noguchi_fig/3_chapter.pdf}
\centering
\caption{コーディングパターン候補の生成手法の概略図}
\label{fig:create_pattern}
\end{figure}
%-------------------

\section{概要}
図\ref{fig:create_pattern}を用いて,本研究ではPython言語を対象にコーディングパターン候補を生成する手法を示す.本研究では,従来研究を踏襲し修正方法を表したコーディングパターンを抽出するために,コーディングパターンになり得る候補(コーディングパターン候補)を生成する.候補の中から有用なコーディングパターンを抽出する方法は\ref{sec:filter}章RQ1の手法で述べる.

図\ref{fig:create_pattern}で示す変更は,Python言語において加算を行う際に,変数\texttt{`a'}を2回記述する必要がなくなり,ソースコードをシンプルかつ明確に記述できるようにする変更である.
本研究では,このような変更チャンクから,次のようなコーディングパターンを抽出する方法を示す.

\colorbox{lightgray!50}{\texttt{VAR\_1}} \colorbox{lightgray!50}{\texttt{- =}} \colorbox{lightgray!50}{\texttt{- VAR\_1}} \colorbox{lightgray!50}{\texttt{- +}} \colorbox{lightgray!50}{\texttt{+ +=}} \colorbox{lightgray!50}{\texttt{VAR\_2}}

このコーディングパターンは変数`\texttt{a}'と`\texttt{b}'を用いた加算操作を形式化したパターンである.具体的には,変更チャンク内で削除が行われた`\texttt{=}',`\texttt{a}',`\texttt{+}'のトークンに接頭辞`\texttt{-}'を付与し,追加が行われた`\texttt{+=}'のトークンに接頭辞`\texttt{+}'を付与している.その後,変数`\texttt{a}'と`\texttt{b}'をそれぞれ`\texttt{VAR\_1}',`\texttt{VAR\_2}'に抽象化する.これらの変更により,変数名に依存しない形で変更チャンクを表現できる.

コーディングパターン候補は次の2つのステップで生成する.
\begin{description}
    \item[前処理] 変更履歴から,修正前後のソースコードチャンクを抽出し,比較可能な形に整える.
    \item[コーディングパターン候補の生成] 修正前後のチャンクを比較することで,修正方法から想定されるコーディングパターン候補を生成する.
\end{description}
% 本研究では,RQ1で単一のプロジェクトで生成されるパターン数の遷移を検証する.
% RQ2では,単一のプロジェクトに生成されるパターンと,複数プロジェクトのパターンがどれだけ有用なパターンであるかを検証する.
% RQ3では,従来手法ではパターンの生成が難しかったプロジェクトに対して,提案手法である複数プロジェクトからパターンを生成することで,プロジェクトにとって意味のあるパターンは増加したのかを検証する.

% 本研究では,以下の3つの手法でパターンを作成する.
% \begin{itemize}
%     \item 単一のプロジェクトからパターンを生成(従来手法)
%     \item データセットとして選択したプロジェクト全てからパターンを生成(提案手法)
%     \item データセットの中からコーディングパターンが似ていると思われるプロジェクトを選択し,パターンを生成(提案手法)
% \end{itemize}

% 提案手法において,複数プロジェクトからパターンを生成する理由は,開発履歴から作成されたパターンを適用できないソースコードの実装方法が存在する場合,実装方法として正しいかどうかをパターンから判断できない.
% そのため,複数のプロジェクトからパターンを生成し,パターン数を増やすことで,実装方法の正当性をパターンに基づいて判断できるケースを増加させることを目指す.

% 次に,コーディングパターンが似ていると思われるプロジェクトを選定して統合する理由について説明する.すべてのプロジェクトの開発履歴を用いる手法はパターン数を増加させることで,パターンが適用可能な実装方法を広げる効果が期待できる.しかし,この方法では生成されたパターンが特定のプロジェクトのコーディングパターンに沿った修正を保証するものではない.そのため,コーディングパターンが類似していると考えられるプロジェクトの開発履歴に限定して統合を行うことで,パターンの適用範囲を拡大し,正当性を向上させることを目指す.


\section{前処理}\label{subsec: pre_process}
本手法では,コードレビューによって修正された変更を分析対象とする.コードレビューでは開発者が変更したソースコードを複数人の別の開発者(検証者)が確認し,バグの修正や,組織の方針に従った実装になるよう指摘をしている.したがって,本手法では組織の方針を捉えるためにコードレビューで変更されたソースコードを対象とする.

コードレビューが依頼されたソースコード'\texttt{Patch\textsubscript{1}}'と,検証者による指摘と開発者によって修正が繰り返され,最終的にプロジェクトに採択されたソースコード'\texttt{Patch\textsubscript{n}}'までの間で,ソースコードが追加された時点でのパッチと,最終編集時点のパッチの差分をファイル単位で取得する.
具体的には,'\texttt{Patch\textsubscript{1}}'と'\texttt{Patch\textsubscript{n}}'でファイルAに対してソースコードの修正が行われ,'\texttt{Patch\textsubscript{2}}'と'\texttt{Patch\textsubscript{n-1}}'でファイルBに対してソースコードの修正が行われた場合,ファイルAについては,ファイルが編集された最初のパッチである'\texttt{Patch\textsubscript{1}}'と'\texttt{Patch\textsubscript{n}}'の差分を取得し,ファイルBについては,'\texttt{Patch\textsubscript{2}}'と'\texttt{Patch\textsubscript{n-1}}'の差分を取得する.このとき,'\texttt{Patch\textsubscript{1}}'から'\texttt{Patch\textsubscript{n}}'の期間内で修正が行われなかったソースコードは,分析対象外とする.

次に,ソースコードに追加されたソースコードのチャンクから,変更された部分(以降,変更チャンク)を特定する.変更チャンクは,ファイル単位で差分を取得した際に得られる追加行と削除行から構成される.このとき,文字の置換なども追加と削除の組み合わせで表現する.

変更チャンクには,コメントの変更なども含まれるため,プロジェクトのコーディングパターンを反映した変更チャンクを収集するために前処理を行う.
前処理は次の3つを行う.
\begin{description}
    \item[コード注釈の削除:]ソースコードに含まれるコメント行(`\texttt{\#}'から始まる行)と関数やクラスの説明に利用される\texttt{docstring}を削除
    \item[関数名の抽象化:]ソースコードに含まれる関数名(Pythonコードにおいて`\texttt{def}'によって定義されたもの)を`\texttt{FUNCTION}'に抽象化
    \item[識別子・数値・文字列の抽象化:]変数名を`\texttt{VAR}',数値を`\texttt{NUMBER}',文字列を`\texttt{STRING}'に抽象化
    \item[変更チャンクの分割:]差分解析ツールであるGumTree\cite{gumtree}を用いて,取得した変更チャンクをより小さな変更チャンクに変換
\end{description}

\section{パターン候補の生成方法}\label{generate}
パターンの生成には系列パターンマイニング手法の1つであるPrefixSpan(Prefix-projected Sequential Pattern Mining)\cite{prefixspan}を用いる.PrefixSpanは効率的な系列パターンマイニング手法であり,順序を持つ系列データから頻出パターンを抽出する手法である.この手法を採用する理由は,変更チャンクに含まれるトークンの順序が,コード修正のパターンを表現する上で重要であるためである.
具体的なパターンの収集方法は,次の3ステップで構成される.
\begin{enumerate}
    \item 変更チャンクをトークン単位に分割し,変更前と変更後の差分情報を符号化する.変更前後のソースコードを比較した際に,変更により削除されたトークンに接頭辞`\textbf{-}',追加されたトークンには接頭辞`\textbf{+}'を付加する.この符号化により,変更操作のタイプ(追加・削除)を保持したトークン系列を生成することができる.
    \item トークン化された変更チャンクを,PrefixSpanの入力形式に適合するように変換する.変更前後のトークン列の順序関係を正確に表現するため,抽象構文木(AST)の構造に基づく差分解析手法を採用し,トークン間の意味的関連性を保持した系列表現を構築する.
    \item PrefixSpanアルゴリズムを適用し,頻出するトークン系列を抽出する.抽出されたパターン候補からは, 系列長が2未満のパターン,系列長が15以上のパターン,変更操作を示す符号(`\textbf{+}'/`\textbf{-}')を1つも含まないパターンを除外する.これにより,コード変更の文脈を反映した有意義なコーディングパターンのみを選定する.
\end{enumerate}
ここまでの手順で,コードレビューで行われた修正の特徴を抽象化したコーディングパターン候補を取得する.

% \section{パターンのフィルタリング方法}\label{filter}
% パターンの生成をしただけでは,冗長なパターンも含まれる.本研究では,3つのフィルタリングによって,冗長なパターンを削除する.
% \begin{itemize}
%     \item 変更を示唆しないパターンを削除する.変更を示唆しないパターンとは,パターンを構成するトークンに変更前と変更後のトークンが含まれないパターンである.
%     \item 重複するパターンを削除する.重複するパターンとは,パターンを構成するトークンが長いパターンの一部になっているパターンのことである.具体的には,図\ref{fig:create_pattern}に存在するパターン(\colorbox{lightgray}{\texttt{VAR\_1}} \colorbox{lightgray}{\texttt{=}})は,パターン(\colorbox{lightgray}{\texttt{VAR\_1}} \colorbox{lightgray}{\texttt{=}} \colorbox{lightgray}{\texttt{VAR\_1}})の一部とみなすことができるため,前者のパターンを重複したパターンとして削除する.
%     \item パターンを作成した開発履歴のデータセットにおいて,1度しか作成されなかったパターンを削除する.
% \end{itemize}
% 最後に各パターンの信頼度を式\ref{eq:confidence}を用いて算出する.
% 信頼度とは,パターンが適用可能なパッチに対して,パターンを適用することで正しい修正を行った割合を示す.

% %-------------------
% \begin{equation}\label{eq:confidence}
% \text{信頼度} = \frac{\text{パターンによって正しく変更されたパッチ数}}{\text{パターンが適用可能なパッチ数}}
% \end{equation}
% %-------------------
% 信頼度により,データセットにおいて各パターンによる修正の正確性を定量的に評価する.


%%%%%%%%%%%%%%%%%%%%%%%%%%%%%%%%%%%%%%%%%%
\chapter{データセット}\label{sec:dataset}
%%%%%%%%%%%%%%%%%%%%%%%%%%%%%%%%%%%%%%%%%%
\section{分析対象プロジェクト}
本研究では,ケーススタディとして大規模なオープンソースプロジェクトであるOpenStackのコンポーネントの中で最大規模のNovaを対象とした.OpenStackでは,レビュー管理システムとして,Gerritを使用しており,各パッチに対するレビューアのコメント投稿数が多いため,レビューアの修正意図を反映したソースコードの変更差分が取得できると考えられる.分析対象とする2013年1月1日から2025年1月1日までの9年間にmainブランチにマージされたコミットから,Python言語で記述されたファイルの変更差分を取得する.
本研究ではケーススタディとして,大規模オープンソースプロジェクトの実態を明らかにするために,OpenStackの中核コンポーネントであるNovaを分析対象とした.Novaが分析対象として適切と判断した根拠は次の2点に基づく
\begin{description}
    \item[規模的妥当性:]Open Infrastructure Fundationの技術レポート\cite{openstack}によると,Novaは2023年時点で約120万行のPythonコードを有しており,OpenStackプロジェクト群において最大の開発規模を有している.
    \item[開発プロセスの透明性:]OpenStackコミュニティが採用するGerritは,活発にコメントが投稿されているため,開発者の意図がソースコードに反映されている可能性が高い.
\end{description}
データの収集方法は,mainブランチにマージされたパッチの履歴を抽出し,次の基準で取得する.
\begin{itemize}
    \item Pythonファイル(.py拡張子)に限定
    \item 変更が提案されたパッチと,mainにマージされたパッチに限定
\end{itemize}
本研究では,これらの基準で取得したパッチのファイル単位での変更差分について分析を行う.

\begin{table}[h]
    \centering
    \caption{データセットから抽出したコーディングパターン候補}
    \scalebox{1.1}{
    
        \label{table:not_filter_pattern}
        \begin{tabular}{r|c|r}
            \hline \hline
                順位 & パターン & 出現回数\\
            \hline
                 1 & \colorbox{lightgray!50}{\texttt{+ ,}} \colorbox{lightgray!50}{\texttt{)}} & 83,570 \\
                 2 & \colorbox{lightgray!50}{\texttt{(}} \colorbox{lightgray!50}{\texttt{+ ,}} & 65,253 \\
                 3 & \colorbox{lightgray!50}{\texttt{.}} \colorbox{lightgray!50}{\texttt{+ .}}     & 
                 53,093 \\
             \hline
        \end{tabular}
    }
\end{table}

\section{データセットから抽出したコーディングパターン候補}\label{subsec:difficult}
Novaプロジェクトを対象に,\ref{sec:pattern}章に説明した手法を用いてコーディングパターン候補を生成した結果を表\ref{table:not_filter_pattern}に示す.
出現頻度の高いパターンとして (\colorbox{lightgray!50}{\texttt{+ ,}} \colorbox{lightgray!50}{\texttt{)}}) のような形式が確認された.このような記号のみや括弧が閉じられていない不正な構造のパターンが,約2,790万件の候補のうち,約1,680万件(60.0\%)を占めた.

この傾向は,PrefixSpanのアルゴリズム特性に起因する.PrefixSpanのアルゴリズムは入力系列から全ての頻出部分系列を列挙するため,ソースコードとして文法的に不正なパターンであっても,出現頻度が閾値を超えていれば候補として抽出される.
例えば,`\textbf{,}'や`\textbf{.}'といったトークンは,関数呼び出し(例: \textbf{func(arg1, arg2)})やメソッドチェーン(例: \textbf{obj.method1().method2})など異なる文脈で高頻度に出現するため,偶然的に頻出してしまう.

したがって,頻出するパターンの抽出だけではソースコードの文法的妥当性を保証したパターンを抽出することは困難である.
従来手法では,不適切なパターンを除外するための条件が提案されているが,その条件の妥当性に関する十分な議論はされていない.本来プロジェクトのコーディングパターンとして抽出すべきパターンが見逃される可能性がある.そのため,本研究では,\ref{sec:filter}章にて生成したコーディングパターン候補に対して,新たな選別条件を追加することで,より適切なプロジェクト固有のコーディングパターンの抽出を試みる.


%%%%%%%%%%%%%%%%%%%%%%%%%%%%%%%%%%%%%%%%%%
\chapter{\RQone}\label{sec:filter}
%%%%%%%%%%%%%%%%%%%%%%%%%%%%%%%%%%%%%%%%%%
%-------------------
\begin{table*}[t]
    \centering
        \caption{条件ごとにパターンを抽出した結果}
        \label{table:created_pattern}
        \begin{tabular}{r|rrr|r|l}
            \hline \hline
                \multirow{2}{*}{id} & \multicolumn{3}{c|}{順位} & \multirow{2}{*}{\begin{tabular}{c} 出現\\回数 \end{tabular}} & \multirow{2}{*}{説明} \\ \cline{2-4}
                
                & 条件なし & 従来 & 提案 & \\ 
            \hline
                 1 & 4068      & 1   & 1  & 2,504  & 画像メタデータをオブジェクトとして渡すように変更\\
                 2 & 26,738    & -   & 2  & 1,067  & デコレータの名前・引数の変更\\
                 3 & 54,081    & -   & 3  & 975    & デコレータへの引数の追加\\
                 4 & 23,974    & 2   & -  & 1,311  & ログメッセージのローカライズ\\
                 5 & 75,393    & 3   & 4  & 756    & DB操作のリトライ処理をライブラリ化\\
                 6 & 1,201,923 & -   & -  & 110    & uuidを呼び出す処理の関数化\\
             \hline
        \end{tabular}
\end{table*}
%-------------------
%-------------------
\begin{table*}[t]
    \centering
    \caption{抽出したパターンの詳細}
    \label{table:description}
    \begin{tabular}{r|p{6cm}|p{5cm}|p{5cm}}
        \hline \hline
        id & パターン & 変更前ソースコード例 & 変更後ソースコード例 \\ \hline
        1 & 
        \colorbox{lightgray!50}{\texttt{VAR\_1}} \colorbox{lightgray!50}{\texttt{=}} \colorbox{lightgray!50}{\texttt{- \{}} \colorbox{lightgray!50}{\texttt{+ objects}} \colorbox{lightgray!50}{\texttt{+ .}} \newline
        \colorbox{lightgray!50}{\texttt{+ ImageMeta}} \colorbox{lightgray!50}{\texttt{+ ,}} \colorbox{lightgray!50}{\texttt{+ from\_dict}} \colorbox{lightgray!50}{\texttt{+ (}} \newline
        \colorbox{lightgray!50}{\texttt{+ self}} \colorbox{lightgray!50}{\texttt{+ .}} \colorbox{lightgray!50}{\texttt{+ test\_VAR\_1}} \colorbox{lightgray!50}{\texttt{+ )}}
        & 
        \texttt{image\_meta = \{\}} 
        & 
        \texttt{image\_meta = \newline
        objects.ImageMeta.from\_dict \newline
        (self.test\_image\_meta)} \\
        \hline
        2 &
        \colorbox{lightgray!50}{\texttt{@}} \colorbox{lightgray!50}{\texttt{- exception}} \colorbox{lightgray!50}{\texttt{- .}} \colorbox{lightgray!50}{\texttt{wrap\_exception}}
        \newline
        \colorbox{lightgray!50}{\texttt{- notifier}} \colorbox{lightgray!50}{\texttt{- =}} \colorbox{lightgray!50}{\texttt{- notifier}} \colorbox{lightgray!50}{\texttt{- ,}}
        \newline
        \colorbox{lightgray!50}{\texttt{- publisher\_id}} \colorbox{lightgray!50}{\texttt{- =}} 
        \colorbox{lightgray!50}{\texttt{- publisher\_id}} \colorbox{lightgray!50}{\texttt{- (}} \colorbox{lightgray!50}{\texttt{- )}} \colorbox{lightgray!50}{\texttt{- )}}
        &
        \texttt{@exception.wrap\_exception
        \newline
        (notifier=notifier, publisher\_id=publisher\_id())}
        &
        \texttt{@wrap\_exception()}\\
        \hline
        3 &
        \colorbox{lightgray!50}{\texttt{@}} \colorbox{lightgray!50}{\texttt{wrap\_instance\_event}} \colorbox{lightgray!50}{\texttt{+ (}} 
        \newline
        \colorbox{lightgray!50}{\texttt{+ prefix}} \colorbox{lightgray!50}{\texttt{+ =}} \colorbox{lightgray!50}{\texttt{+ compute}} \colorbox{lightgray!50}{\texttt{+ )}}
        &
        \texttt{@wrap\_instance\_event}
        &
        \texttt{@wrap\_instance\_event\newline
        (prefix = 'compute')}\\
        \hline
        4 &
        \colorbox{lightgray!50}{\texttt{VAR\_1}} \colorbox{lightgray!50}{\texttt{.}} \colorbox{lightgray!50}{\texttt{VAR\_2}} \colorbox{lightgray!50}{\texttt{(}}
        \colorbox{lightgray!50}{\texttt{- VAR\_3}} \colorbox{lightgray!50}{\texttt{+ VAR\_4}}
        \colorbox{lightgray!50}{\texttt{(}}
        &
        \texttt{LOG.info(\_(}
        &
        \texttt{LOG.info(\_LI(}\\
        \hline
        5 &
        \colorbox{lightgray!50}{\texttt{@}} \colorbox{lightgray!50}{\texttt{- \_retry\_on\_deadlock}} \colorbox{lightgray!50}{\texttt{+ oslo\_db\_api}}
        \newline
        \colorbox{lightgray!50}{\texttt{+ .}} \colorbox{lightgray!50}{\texttt{+ wrap\_db\_retry}} \colorbox{lightgray!50}{\texttt{+ (}}
        \newline
        \colorbox{lightgray!50}{\texttt{+ max\_retries}} \colorbox{lightgray!50}{\texttt{+ =}} \colorbox{lightgray!50}{\texttt{+ 5}} \colorbox{lightgray!50}{\texttt{+ ,}} 
        \newline
        \colorbox{lightgray!50}{\texttt{+ retry\_on\_deadlock}} \colorbox{lightgray!50}{\texttt{+ =}} \colorbox{lightgray!50}{\texttt{+ True}} \colorbox{lightgray!50}{\texttt{+ )}}
        &
        \texttt{@\_retry\_on\_deadlock}
        &
        \texttt{@oslo\_db\_api.wrap\_db\_retry(
        \newline
        max\_retries=5, retry\_on\_deadlock=True
        \newline
        )}\\
        \hline
        6 &
        \colorbox{lightgray!50}{\texttt{- str}} \colorbox{lightgray!50}{\texttt{- (}} \colorbox{lightgray!50}{\texttt{- uuid}} \colorbox{lightgray!50}{\texttt{+ uuidutils}} \colorbox{lightgray!50}{\texttt{.}} \colorbox{lightgray!50}{\texttt{- uuid4}} \colorbox{lightgray!50}{\texttt{+ generate\_uuid}} \colorbox{lightgray!50}{\texttt{= (}} 
        \colorbox{lightgray!50}{\texttt{= )}} \colorbox{lightgray!50}{\texttt{- )}}
        &
        \texttt{str(uuid.uuid4())}
        &
        \texttt{str(uuidutils.generate\_uuid())}\\
        \hline
    \end{tabular}
\end{table*}
% %-------------------
%フィルタリングに関するRQ
\section{動機}
\ref{subsec:difficult}節の分析結果から,コーディングパターン候補は膨大に生成されることが明らかになった.従来研究では,不適切なパターンを除外するための条件が提案されているが,その条件の妥当性に関する十分な検討がなされていない.その結果,プロジェクト固有のコーディングパターンが適切に抽出されない可能性がある.そのため,本研究では,プロジェクト特有のコーディングパターンを捉えるために,\ref{subsec:difficult}節の分析により判明した出現頻度の多かった不正な構造のパターンを排除し,従来手法より多くのプロジェクト固有のコーディングパターンを収集することを目指す.

\section{手法}
生成した候補からコーディングパターンとして抽出するために,従来研究で用いられていた条件を次に示す.

\textbf{従来研究の条件}
\begin{itemize}
    \item 変更を示さないパターンを削除する.具体的には,パターンの接頭辞に`\textbf{-}'と`\textbf{+}'が付加されたトークンの両方が含まれていないパターンを削除する.
    \item 重複した意味のパターンを削除する.具体的には,(\colorbox{lightgray!50}{\texttt{a}} \colorbox{lightgray!50}{\texttt{b}} \colorbox{lightgray!50}{\texttt{c}})というパターンと,(\colorbox{lightgray!50}{\texttt{a}} \colorbox{lightgray!50}{\texttt{b}})というパターンがあった場合,後者は前者の一部とみなすことができるため,後者のパターンは削除する.
\end{itemize}

さらに,本研究では従来研究のパターンを次の3つの条件に変更する.

\textbf{提案手法の条件}
\begin{itemize}
    \item[条件1: ] 接頭辞に`\textbf{-}'と`\textbf{+}'が付加されたトークンが1つもないパターンを削除する.
    \item[条件2: ] 重複した意味のパターンを削除する.(従来研究と同じ)
    \item[条件3: ] 記号のみで構成されたパターンを削除する.
    \item[条件4: ] 括弧が閉じられていないパターンを削除する.
\end{itemize}

これらの条件を使用し,従来研究と本研究で抽出されたパターンを比較する.

\section{結果・考察}
表\ref{table:created_pattern}は,条件を適用しない場合,従来手法,および提案手法それぞれで抽出されたコーディングパターンの順位と出現回数を示している.また,表\ref{table:description}では,各パターンの具体的な内容と,そのパターンが適用される際の変更前後のソースコード例を示している.なお,表\ref{table:created_pattern}の順位において`\texttt{-}'は,その手法ではパターンとして抽出されなかったことを表している.約2,790万件のコーディングパターン候補を条件によって絞り込むと,従来手法では4,575件,提案手法では7,727件に集約することができた.従来研究であっても十分にコーディングパターン候補から絞り込むことができており,提案手法の方が絞り込みできていないようにも見える.具体的な内容について順に考察する.

条件1により,従来手法では変更を示さないパターンを除外し,提案手法では,変更を示すトークンが含まれないパターンを除外したことで,パターンとして取得する範囲が広くなったため,従来手法よりも3,000件ほど多くなった.取得する範囲が広くなったことにより,関数への引数追加/削除のようなパターンを取得することが可能になった.

しかし,抽出したパターンは,本研究の\ref{subsec: pre_process}節の前処理の影響により,引数の追加/削除という操作の事実は把握できるものの,具体的なコーディングパターンの抽出が困難となる課題が確認された.この結果から,関数の引数変更に特化したパターン抽出を目的とする場合,関数名の抽象化は適切ではないと考えられる.関数名の抽象化を行わないことで,関数の引数を変更したという汎用的なパターンから,特定の関数に関する引数の追加をパターンとして抽出できるようになると考える.

条件2は,従来研究で用いられていた重複した意味のパターンを削除するという条件と同一のものである,この条件を適用することで,抽出したコーディングパターン候補の中から,似た意味のパターンを削除し,冗長なパターンを大幅に削除できた.

しかし,重複パターンを包括的に削除する手法には根本的な課題が存在する.具体例として,(\texttt{a = b + c})から(\texttt{a = b - c})への変更を上げる.この変更の本質は,`\texttt{+}'から`\texttt{-}'への置換であるが,行単位の差分解析では変更前後トークン列が(\colorbox{lightgray}{\texttt{a}} \colorbox{lightgray}{\texttt{=}} \colorbox{lightgray}{\texttt{b}} \colorbox{lightgray}{\texttt{- +}} \colorbox{lightgray}{\texttt{+ -}} \colorbox{lightgray}{\texttt{c}})という冗長な形式に集約されてしまう.重複削除処理により,本来抽出すべき演算子変更の核心パターン(\colorbox{lightgray}{\texttt{- +}} \colorbox{lightgray}{\texttt{+ -}})が失われ,冗長な行単位パターンだけが残存するという問題が発生する.

これを解決するための今後の課題として,抽象構文木(AST)を用いた差分解析手法の導入が考えられる.具体的には,AST同士を比較することで部分木の変更を検出し,その変更内容から適切なトークン列を生成する.この手法により,行単位の解析では捉えきれなかった演算子の変更や構文レベルのパターンも正確に抽出できる可能性がある.結果として,より意味的に妥当なコーディングパターンの抽出が期待され,冗長なパターンが削減されるだけでなく,プロジェクト固有のコーディングパターンの特定精度が向上する可能性がある.

また,条件3,4を利用することで,候補を生成する段階で出現頻度が高かった(\colorbox{lightgray}{\texttt{+,}} \colorbox{lightgray}{\texttt{)}})のようなソースコードの文法的に誤ったパターンを除外することができた.従来手法で抽出されたパターンには,頻出パターンの中で条件3,4を追加することで絞り込めるものは少なかったが,発生頻度が低いパターンには条件に合致するパターンも見られた.従って,これらの条件は頻出したパターン候補だけを対象にするのではなく,すべての抽出されたパターンを分析する場合において有用であると言える.

%%%%%%%%%%%%%%%%%%%%%%%%%%%%%%%%%%%%%%%%%%
\chapter{\RQtwo}\label{sec:time}
%%%%%%%%%%%%%%%%%%%%%%%%%%%%%%%%%%%%%%%%%%
%時系列で何個かのパターン見た時の評価
\section{動機}
\ref{sec:filter}章RQ1の分析結果から,パターン候補に条件を設定することで,頻出するプロジェクトのコーディングパターンを反映したソースコードを特定した.しかし,Uedaらの従来研究\cite{devreplay}において,抽出されたコーディングパターンは時間経過に伴い使用頻度が変動することが明らかになっている.
このことから,抽出されたパターンには,以下のような特徴が存在すると考えられる.
\begin{itemize}
    \item コーディングパターンによる修正が行われる前のソースコード(以降,変更前パターン)が標準的な実装として採用されるケース
    \item 現在もコーディングパターンによる修正が行われたソースコード(以降,変更後パターン)が継続的に使用されているケース
    \item 過去には頻繁に使用されていた変更後パターンが,現在では使用されなくなったケース
\end{itemize}
本研究では,RQ1において抽出したコーディングパターンの生存期間を分析することで,長期間にわたって使用されるコーディングパターンと,比較的短期間で使用されなくなるコーディングパターンの特徴を明らかにする.この分析を通じて,プロジェクトにおけるコーディングパターンの持続性を評価し,効果的なコーディングパターンの収集手法の確立を目指す.

\section{手法}
本研究では,\ref{sec:filter}章にて抽出したコーディングパターンid\_1からid\_6を対象として時系列分析を行う.分析は次の手順にて行う.
\begin{itemize}
    \item NovaプロジェクトのGitリポジトリから,2013年から2025年までの各年1月1日時点にリポジトリに含まれるソースコードを取得する.
    \item 各スナップショットに対して,変更前パターンと変更後パターンが出現した回数を記録する.
\end{itemize}

\section{結果・考察}
図\ref{table:id_1}から図\ref{table:id_6}までは,id\_1からid\_6までの6つのパターンに対して時系列分析を行った結果を示す.横軸は取得したスナップショットの年数,縦軸は変更前/変更後パターンが出現した回数を示す.図中の黒色の棒グラフは,変更前パターンが抽出された数,白色の棒グラフは変更後パターンが出現した数を示す.id\_1は変更前/変更後パターンの両方が全期間で使用され,id\_2,id\_3,id\_5,id\_6のパターンはある時点を境に,変更後パターンしか使用されなくなり,id\_4はある地点以降は変更前/変更後パターンが両方使用されなくなった.
6つのパターンから,次の3つの特徴的な傾向を考察する
\begin{description}
    \item[持続的共存型パターン(id\_1)]: 分析期間を通じて,変更前/変更後パターンが継続的に共存している.
    \item[移行完了型パターン(id\_2,id\_3,id\_5,id\_6)]: 特定の時点を境に,変更前パターンが完全に変更後パターンに置き換えられた.
    \item[消失型パターン(id\_4)]: ある時点で,変更前/変更後パターンが出現しなくなった.
\end{description}
\begin{figure}[h]
    \centering
        \includegraphics[width=0.9\linewidth]{@BSthesis2024_Noguchi/Noguchi_fig/id_1.png}
        \vspace{-4mm}
        \caption{id\_1のパターン}
        \label{table:id_1}
        
        \includegraphics[width=0.9\linewidth]{@BSthesis2024_Noguchi/Noguchi_fig/id_2.png}
        \vspace{-4mm}
        \caption{id\_2のパターン}
        \label{table:id_2}
        
        \includegraphics[width=0.9\linewidth]{@BSthesis2024_Noguchi/Noguchi_fig/id_3.png}
        \vspace{-4mm}
        \caption{id\_3のパターン}
        \label{table:id_3}
        
        \includegraphics[width=0.9\linewidth]{@BSthesis2024_Noguchi/Noguchi_fig/id_4.png}
        \vspace{-4mm}
        \caption{id\_4のパターン}
        \label{table:id_4}
        
        \includegraphics[width=0.9\linewidth]{@BSthesis2024_Noguchi/Noguchi_fig/id_5.png}
        \vspace{-4mm}
        \caption{id\_5のパターン}
        \label{table:id_5}
        
        \includegraphics[width=0.9\linewidth]{@BSthesis2024_Noguchi/Noguchi_fig/id_6.png}
        \vspace{-4mm}
        \caption{id\_6のパターン}
        \label{table:id_6}
\end{figure}


\noindent\textbf{[持続共存型パターン]} 

当該パターンは,変更前/変更後パターンが継続的に共存している.id\_1のパターンは,テストの信頼性向上を目的として,テストデータをダミーデータからより具体的なオブジェクトに変更するものである.このパターンは,テストの品質向上を意図した変更であるにもかかわらず,変更前のパターンも引き続き使用されている.そのため,変更前パターンが引き続き使用されている理由について,分析を行う.
このパターンはテストの信頼性向上を目的とした変更であることから,変更前パターンの出現箇所をテストコードと非テストコードに分類して分析を行った.その結果を図\ref{figure:test_or_not}に示す.
横軸は取得したスナップショットの年数,縦軸はテスト/テスト以外のコードで変更前パターンが出現した回数を示す.黒色の棒グラフがテストコード以外で出現した回数,白色の棒グラフがテストコードで出現した回数である.
\begin{figure}[h]
    \includegraphics[width=0.9\linewidth]{@BSthesis2024_Noguchi/Noguchi_fig/test_or_not_test.png}
    \vspace{-4mm}
    \caption{id\_1の変更後パターンの内訳}
    \label{figure:test_or_not}
\end{figure}

図\ref{figure:test_or_not}の黒色の棒グラフを参照すると,テストコード以外で変更前パターンが出現した回数は,2019年から4件であり,変更前パターンが出現した箇所を分析すると同様の箇所であった.2019年以降で使用されたソースコードの内訳は,適用後のオブジェクトを定義した箇所,保守がなされていない箇所,後々のために残さざるを得なった箇所,推奨されない関数内で使われている箇所の4つである.
テストコードで変更前パターンが出現した箇所では,\texttt{image\_meta}がオブジェクト型として入力されることを期待しているが,単体テストでは\texttt{image\_meta}が空でも動作するため,空の辞書で定義しているものであった.そのため,持続共存型のパターンは,保守がされていないパターン,または限定的に使用できるパターンで,今回の場合,テスト内では,使用するべきパターンである.

\noindent\textbf{[移行完了型パターン] }

当該パターンに該当するid\_2,id\_5,id\_6は,ある時点を境に,変更前パターンから完全に変更後パターンに置き換えられている.これらのパターンは,従来の記述方法を関数化することで再利用性を高めている.これらのパターンは,再利用性を重視したソースコードの実装方法がプロジェクト内で広く採用された結果であると考えられる.関数化によってコードの重複を減らし,保守性を向上させる効果があったため,これらのパターンは継続的に使用するべきパターンであるため,コード自動修正技術における多様なプログラミング言語を学習した深層学習では,このようなパターンの自動修正は難しいと考える.

\noindent\textbf{[消失型パターン] }

当該パターンに該当するid\_4は,図\ref{table:id_4}より,2021年を境に変更前/変更後パターンの両方が出現しなくなっている.このパターンは,ログメッセージを異なる言語で翻訳して出力するための変更である.変更前パターンで使用される\texttt{\_}は,一般的なローカライズ関数で,変更後パターンでは,ログ専用のローカライズ関数である\texttt{\_LI}などが用いられていた.ローカライズを明確にすることで,メッセージ管理や保守性を向上させるための修正方法である.

しかし,このパターンは\texttt{Python}の標準モジュールである\texttt{logging}を用いて実装されたソースコードに基づいている.\texttt{logging}モジュールは,\texttt{Python3.2}以降翻訳関数を使用せず,引数を追加することでログメッセージの翻訳をより効率的に行えるようになった.そのため,id\_4のパターンは2021年を境に使われなくなったと考えられる.従って,消失型パターンは,プログラミング言語と密接に関連しているため,結果的に不要になった,または新たな実装方法へと置き換えられた可能性が高いと考えられる.

%%%%%%%%%%%%%%%%%%%%%%%%%%%%%%%%%%%%%%%%%%
\chapter{妥当性の脅威}
%%%%%%%%%%%%%%%%%%%%%%%%%%%%%%%%%%%%%%%%%%
\section{内的妥当性}
コーディングパターン候補を生成する際に使用した手法について議論する.本研究で使用したPrefixSpanアルゴリズムは,頻出系列パターンを効率的に抽出する手法であり,特に大規模なデータセットに対して,高い性能を有する.今回の研究対象である12年間にわたる大規模なデータセットに対して,頻出するコードパターンを探索する上で適切な選択であるといえる.

さらに,本研究では,パターンとして採用するトークン長を最大15に制限した.この制限は,PrefixSpanによる探索時間の膨張を防止するための措置である.また,対象とした変更差分が行単位の比較的短い変更であることから,長いトークン系列はプロジェクト内で頻出しにくい.従って,トークン長に制限を設けることで分析の焦点がより妥当な範囲に絞られ,効率的かつ実用的なパターン抽出が可能となったと考えられる.
\section{外的妥当性}
本研究では,データセットとして,OpenStackのコンポーネントの中で,最大規模のプロジェクトであるNovaプロジェクト内のPythonファイルを対象に修正方法を収集し,分析を行った.本論文の手法は,バージョン管理ツールを利用し,変更チャンクを取得できるプロジェクトであれば,Python以外の言語を使用するプロジェクトにも同様に適用可能である.従って,本手法は他のプログラミング言語を用いたプロジェクトを対象とした場合でも,有効な分析結果が得られる可能性が高い.特に,異なるプロジェクトに適用することで,各プロジェクトに特有の実装方法や設計手法の違いに関する新たな知見が得られることが期待される.これにより,修正方法やパターンの普遍性の検証,さらなる適用領域の拡張につながる可能性がある.
%%%%%%%%%%%%%%%%%%%%%%%%%%%%%%%%%%%%%%%%%%
\chapter{おわりに}
%%%%%%%%%%%%%%%%%%%%%%%%%%%%%%%%%%%%%%%%%%
本研究では,個別のプロジェクトに特化したコーディングパターンを収集するための条件を提案することで,自動修正に有用なパターンを抽出する方法を開発し,2つのRQを検証した.その結果,変更チャンクから個別のプロジェクトにとって有用なパターンを抽出することができ,それらのパターンを時系列に沿った分析を行うことで,そのパターンが現在も使用されている有用なパターンかどうかを判別することができた.

本手法は,パターンを抽出するために使用した前処理で適切ではない部分が多いという点でまだまだ改善の必要がある.今後は,行単位の差分を取得することでは難しい修正方法をパターンとして取得するために,抽象構文木ベースで比較することで,より細かいパターンを生成することで,汎用的なプロジェクトのコーディングパターンを抽出する方法を確立を目指す.
%%%%%%%%%%%%%%%%%%%%%%%%%%%%%%%%%%%%%%%%%%%%%%%%%%%%%%%%%%%%%%%%%%%%%%%%

%%
%% 謝辞
%%
\begin{acknowledgements}
感謝します.
% %-------------------
% \begin{center}
% \includegraphics[width=1.0\linewidth]{@BSthesis2024_Noguchi/Noguchi_fig/thank_irasutoya.pdf}
% \end{center}
% %-------------------

\end{acknowledgements}

%%%%%%%%%%%%%%%%%%%%%%%%%%%%%%%%%%%%%%%%%%
%% 参考文献
%%%%%%%%%%%%%%%%%%%%%%%%%%%%%%%%%%%%%%%%%%
\bibliographystyle{junsrt}
\bibliography{@BSthesis2024_Noguchi/BSthesis2024_Noguchi_thesis}

\end{document}
