%
% 卒論レジュメフォーマット Ver.2.0 pLaTeX版
%
\documentclass[twocolumn]{jarticle} % 2段組のスタイルを用いている

\usepackage{wuse_resume}
\usepackage{url}	% \url{}コマンド用.URLを表示する際に便利
\usepackage[dvipdfmx]{graphicx}
\usepackage{booktabs}
\usepackage{tabularx}

%%%%%%%%%%%%%%%%%%%%%%%%%%%%%%%%%%%%%%%%%%%%%%%%%%%%%%%%%%%%%%%%%%%%%%%%

%%
%% タイトル,学生番号,氏名などを設定する
%%

\タイトル{逆強化学習を用いた継続的な貢献を行うLTCの予測モデル}
\研究室{ソーシャルソフトウェア工学}
\学生番号{60276189}
\氏名{橋本 一輝}

\概要{%
オープンソースソフトウェア(OSS: Open Source Software)開発において,長期貢献者(LTC: Long-Term Contributor)はプロジェクトの継続的な発展に不可欠であるが,多くの開発者はプロジェクト参加後早期に離脱し,少数のLTCに作業負担が集中しやすい.そのため,LTC候補者を早期に特定し,適切なタスク配分やサポートを行うことが重要である.しかし,単一時点以降の貢献の有無を予測する従来研究では,開発者がどの期間で貢献を行うのかが明確でない.本研究では,コードレビュー依頼の承諾を継続的な意思決定プロセスと捉え,逆強化学習(IRL: Inverse Reinforcement Learning)により報酬関数を推定し,連続する各期間のレビュー承諾を予測するモデルを提案する.ケーススタディとしてOpenStack/Novaプロジェクトを対象とし評価した結果,提案手法は多くの評価パターンでランダムフォレスト(RF: Random Forest)を上回る精度を示し,予測期間に応じて重要な特徴量が変化することを明らかにした.
}

\キーワード{オープンソースソフトウェア}
\キーワード{長期貢献者}
\キーワード{コードレビュー}
\キーワード{プロジェクトの持続可能性}
\キーワード{逆強化学習}



%%%%%%%%%%%%%%%%%%%%%%%%%%%%%%%%%%%%%%%%%%%%%%%%%%%%%%%%%%%%%%%%%%%%%%%%

%% 以下の3行は変更しない

\begin{document}
\maketitle
\thispagestyle{empty} % タイトルを出力したページにもページ番号を付けない

%%%%%%%%%%%%%%%%%%%%%%%%%%%%%%%%%%%%%%%%%%%%%%%%%%%%%%%%%%%%%%%%%%%%%%%%

%%
%% 本文 - ここから
%%

\section{はじめに}
オープンソースソフトウェア(OSS)は,世界中に分散した開発者によって開発・保守されている.多くの開発者は金銭的な利益を得ることなく,知的好奇心や技術学習を動機として貢献しているため\cite{motivation},時間的制約や興味の変化により数回の貢献後に活動を停止する開発者が多い\cite{OTC}.こうした開発者の離脱はプロジェクトの失敗の一要因となっている.

OSS開発において,長期貢献者(LTC)はソフトウェアの実装に限らず,コードレビューや新規参加者のメンターなど,プロジェクトの継続的な運用を支える重要な役割を担っている\cite{LTC}.しかし,開発者の流動性が高く離脱が頻繁に発生するため,LTCは少数であり,作業負担が集中しやすい\cite{related2}.そのため,LTCとなる開発者を早期に特定し,適切なサポートを行うことが求められている.

従来研究\cite{LTC}\cite{related1}では,LTCとなり得る開発者を早期に予測するLTC予測として,プロジェクト参加後の一定期間を基準とし,1年後や3年後以降に貢献があるか否かを予測する手法が提案されている.しかし,単一時点以降の貢献の有無を予測するため,開発者がどの期間で貢献するかを予測することはできず,適切なタスク配分が困難である.

本研究では,作業依頼に対する応答が把握しやすいコードレビューに着目する.レビュー依頼の承諾は単発の行動ではなく,過去の活動や作業負荷に応じた継続的な意思決定プロセスである.このプロセスを状態と行動の時系列的な遷移としてモデル化するために,逆強化学習(IRL)を用いて報酬関数を推定し,連続する各期間(0-3m,3-6m,6-9m,9-12m)におけるレビュー承諾の有無を予測するモデルを提案する.

\section{提案手法}
従来の教師あり学習を用いたLTC予測では,単一時点の特徴量の累積値や平均値を用いて学習を行うため,活動頻度の変化といった動的なパターンを捉えることが難しい.これに対し,逆強化学習は時系列の活動を軌跡として扱い,各状態と行動から報酬関数を推定する.そのため,高負荷なタスクが重なる,応答時間が伸びるといった活動量の変化の過程を考慮できる.

本研究では,レビュアーの活動履歴や作業負荷を捉える10次元の状態特徴量と,レビュー規模や応答速度などレビュー依頼に対する行動を表す4次元の行動特徴量を用い,時系列的な遷移から報酬関数を推定する.評価期間を3ヶ月ごとの区間(0-3m,3-6m,6-9m,9-12m)に分割し,各区間におけるレビュー承諾の有無を連続的に予測する.

\section{ケーススタディ}
\subsection{データセット}
OpenStack/Novaプロジェクトを対象とし,Gerrit\footnote{https://www.gerritcodereview.com/} REST APIからレビュー活動データを収集した.学習期間は2021年1月〜2023年1月,評価期間は2023年1月〜2024年1月とした.学習時のレビュアー数は約70名,評価時は約60名である.

\subsection{RQ1: 逆強化学習に基づく提案モデルは,コードレビューにおけるレビュー承諾をどの程度予測できるか}
3-6mモデルを代表例として,学習期間と評価期間の予測精度を比較した.両期間の各指標に大きな乖離は見られず,評価期間のAUC-ROCは0.820であった.この結果は,提案モデルが学習データに過剰に適合せず,未知の期間に対しても汎化性能を有することを示唆する.

\subsection{RQ2: 学習方法や評価期間の長さに応じて,予測モデルの精度が異なるのか}
学習期間と評価期間の組み合わせを変えた10パターンのクロス評価を実施した.RFはAUC-ROCの最大値が0.928でIRL(0.910)より高いが,IRLは評価期間6-9mを除く大部分のパターンでRFを上回った.特に評価期間9-12mではRFの精度が大きく低下する一方,IRLは相対的に高精度を維持した.また,IRLでは学習期間3-6mのモデルが評価期間に関わらず平均的に高い精度を示した.これは対象プロジェクトであるNovaが6ヶ月ごとにリリースしていることが一要因として考えられる.

\subsection{RQ3: 推定された報酬関数において,LTCの継続的なレビュー承諾に寄与する特徴量は何か}
IRLの勾配ベースの重要度分析の結果,短期予測(0-3m)では総レビュー数などの活動量が予測に大きく寄与したが,長期予測(9-12m)では協力度や応答速度などの質的な特徴量の寄与が高まった.また,レビュー規模や平均活動間隔は一貫して負の影響を示し,レビュアーへの過度な負担が離脱の要因となることが示唆された.一方,RFでは主に総レビュー依頼数が重要であるが,9-12mモデルでは総承諾率やレビュー負荷の寄与も高まった.

\section{考察}
\subsection{時系列データの考慮による影響}
評価期間ごとの承諾率の分布を確認すると,3-6mでは承諾率0--10\%の層が38.1\%と最多であるのに対し,9-12mでは70--100\%の層が55.6\%と過半数を占めるなど,期間により大きく変化していた.時系列的な変化を軌跡として学習するIRLは,こうした分布の移動を反映しやすく,複数期間で安定的に精度を確保できたと考えられる.

また,IRLとRFの予測結果を比較すると,全452件中IRLのみが正解した事例は112件(24.8\%)であった.特に,レビュー依頼が多いが承諾率が低いレビュアーに対して,RFがレビュー依頼数の多さに引きずられて承諾と誤判定する一方,IRLは承諾率の低下や活動パターンの時系列的な変化を捉えることで正しく非承諾と予測できた.

\subsection{変動型レビュアーの存在}
全評価期間に存在する39名のレビュアーを分析した結果,一貫承諾型が11名(28.2\%),一貫非承諾型が15名(38.5\%),期間によって活動の有無が変動する変動型が13名(33.3\%)であった.図\ref{fig:accuracy_by_pattern_type}は,これら3つの活動パターン分類別のIRLとRFの予測正解率を示している.一貫型のレビュアーに対しては両手法とも高い正解率を示すが,変動型のレビュアーに対しては両手法とも正解率が低下したが,IRLはRFより高い正解率を示しており,時系列学習による変動パターンの捉えやすさが示唆される.これらの結果から,従来手法では変動型レビュアーの離脱や復帰を捉えることができず,連続的な区間予測の必要性が示唆された.

\begin{figure}[ht]
    \centering
    \includegraphics[width=0.85\columnwidth]{./Hashimoto_fig/accuracy_by_pattern_type.pdf}
    \caption{活動パターン分類別のIRL/RF正解率}
    \label{fig:accuracy_by_pattern_type}
\end{figure}

\section{おわりに}
本研究では,IRLに基づくLTC予測モデルを提案し,連続する各期間におけるレビュー承諾を予測した.提案手法は多くの評価パターンでRFを上回る精度を示した.また,予測期間に応じて重要な特徴量が変化し,短期では活動量,長期では質的な要因が重視されることを明らかにした.さらに,約3分の1のレビュアーが変動型の活動パターンを持ち,連続的な区間予測の意義を示した.

%%
%% 本文 - ここまで
%%

%%%%%%%%%%%%%%%%%%%%%%%%%%%%%%%%%%%%%%%%%%%%%%%%%%%%%%%%%%%%%%%%%%%%%%%%

%%
%% 参考文献
%%

\begin{thebibliography}{9}

\bibitem{motivation}
  Wu, Y., et al.,
  Open Source Software Developer and Project Networks,
  Proc. KDD Workshop, 2007.

\bibitem{OTC}
  Lee, A., et al.,
  One-Time Contributors to FLOSS: Surveys and Data Analysis,
  Proc. OSS, 2017.

\bibitem{LTC}
  Zhou, M. and Mockus, A.,
  Who Will Stay in the FLOSS Community? Modeling Participant's Initial Behavior,
  IEEE Trans. Software Eng., Vol.41, No.1, pp.82--99, 2015.

\bibitem{related1}
  Eluri, V.R., et al.,
  Predicting Long-Term Contributors to Open Source Projects Using Machine Learning,
  Proc. MSR, pp.497--501, 2021.

\bibitem{related2}
  Bao, L., et al.,
  A Large Scale Study of Long-Time Contributor Prediction for GitHub Projects,
  IEEE Trans. Software Eng., Vol.47, No.6, pp.1277--1298, 2021.

\end{thebibliography}

%%%%%%%%%%%%%%%%%%%%%%%%%%%%%%%%%%%%%%%%%%%%%%%%%%%%%%%%%%%%%%%%%%%%%%%%

\end{document}
