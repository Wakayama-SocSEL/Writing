\documentclass[11pt]{jreport}
\usepackage{wuse_thesis}
\usepackage{indentfirst}
\usepackage{url}	% \url{}コマンド用.URLを表示する際に便利
\usepackage{xcolor}
\usepackage{listings}
\usepackage{amsmath,amssymb}
\usepackage{tabularx}
\usepackage{stfloats}
\usepackage{booktabs}
\usepackage{threeparttable}
\usepackage{caption}
%\usepackage{graphicx}  % ←graphicx.styを用いてEPSを取り込む場合有効にする
			% 他のパッケージ・スタイルを使う場合には適宜追加

\newcommand{\todo}[1]{\colorbox{yellow}{{\bf TODO}:}{\color{red} {\textbf{[#1]}}}}
\newcommand{\memo}[1]{\colorbox{magenta!30}{{\bf MEMO}:}{\color{red!50} {\textbf{[#1]}}}}
\newcommand{\ihara}[1]{\colorbox{green}{{\bf IHARA}:}{\color{blue} {\textbf{[#1]}}}}

% コード例を載せるためのあれこれ
\definecolor{lightred}{RGB}{255,230,230}
\definecolor{lightgreen}{RGB}{230,255,230}

\lstset{
    basicstyle=\small\ttfamily,
    abovecaptionskip=0pt,
    captionpos=b,
    frame=tb,
    framexleftmargin=2em,
    numbers=left,
    numberstyle={\scriptsize},
    xleftmargin=\parindent,
    escapechar=|
}

%ListingのキャプションがFigureになってしまうのをListingに直すコマンド
\usepackage{caption}
\makeatletter
\let\MYcaption\@makecaption
\makeatother
\usepackage{caption}
\makeatletter
\let\@makecaption\MYcaption
\makeatother

%%%%%%%%%%%%%%%%%%%%%%%%%%%%%%%%%%%%%%%%%%%%%%%%%%%%%%%%%%%%%%%%%%%%%%%%

%%
%% 主に表紙を作成するための情報
%%

%%  タイトル(修論の場合は英語表記も指定)
\title{卒業・修士論文用スタイルファイルを用いた\\
       p\LaTeX による卒業・修士論文の作成}
%\etitle{Test\\Test\\Test}

%%  著者名(修論の場合は英語表記も指定)
\author{野口 隼杜}
%\eauthor{Akinori Ihara}

%% 卒業論文・修士論文(以下のどちらかを選択)
\bachelar	% 卒業論文(4年生用)
%\master  	% 修士論文(M2用)

%%  学科・クラスタ
\department{システム工}
%\department{デザイン情報}
%\department{デザイン科学}

%%  学生番号
\studentid{60276185}

%%  卒業年度
\gyear{2025}		% 提出年が2022年なら,2021年度

%%  論文提出日
\date{2026年2月10日}	% 修士の場合は月(2021年2月)までとし,英語表記も指定
%\edate{February 2021}	% 修士の場合,こちら(英語表記)も有効化

%%%%%%%%%%%%%%%%%%%%%%%%%%%%%%%%%%%%%%%%%%%%%%%%%%%%%%%%%%%%%%%%%%%%%%%%

\begin{document}

\maketitle

%%
%%  概要
%%
\begin{abstract}
\todo{最後に作る}

「和歌山大学システム工学部卒業論文/
  大学院システム工学研究科修士論文用スタイルファイル」
は,
\begin{quote}
  \begin{description}
    \item[\tt wuse\_thesis.sty:] 卒業/修士論文用スタイルファイル
    \item[\tt thesis\_sample.tex:] スタイルファイル利用例
  \end{description}
\end{quote}
からなる.

なお,この卒業論文用スタイルファイル(p\LaTeX 版)に関する質問は,
メールにて
\begin{quote}
ihara@wakayama-u.ac.jp
\end{quote}
まで.

\end{abstract}

%%  目次
\tableofcontents

%%  図目次 (図目次をいれたければ以下のコメントをはずす)
%\listoffigures

%%  表目次 (表目次をいれたければ以下のコメントをはずす)
%\listoftables

\newpage
\pagenumbering{arabic}	% 以降のページ番号を算用数字に

%%%%%%%%%%%%%%%%%%%%%%%%%%%%%%%%%%%%%%%%%%%%%%%%%%%%%%%%%%%%%%%%%%%%%%%%

%%
%%  本文はここから
%%

\chapter{はじめに}

\memo{研究背景・研究課題・目的および本研究の貢献・論文構成}
 ソフトウェアの性能効率性は,ユーザ体験や運用コスト,さらにはシステム全体の品質に直結する重要な要素である\cite{performance1}\cite{performance2}\cite{negative}.
 部分的なソースコードの最適化を積み重ねることで実現することも多い.

性能効率性を向上するためのプログラム変更は,実装が進行するにつれて複雑になりやすい\cite{complicate}.
保守性や可読性など,他のプログラム品質に否定的な影響を与えることもある\cite{negative}.
実装が進むほどに改善のための変更は大変.
ソフトウェアの性能効率性を,実装途中の早期の段階で見積もり,性能低下の原因となるボトルネックを検出することは,性能効率性の改善にかかる修正工数を小さくできる.


% ソースコード「片」で統一
本研究では,ソフトウェア中に潜在する,修正することで実行速度の向上が期待されるソースコード片の検出を目的とする.具体的には,マイクロベンチマーク共有サービスでベンチマークとして比較されたソースコード片の構造的な差分から低速コードパターンを作成する.作成した低速コードパターンを利用し,静的解析エンジンであるCodeQL\footnote{\url{https://codeql.github.com/}}\cite{ql}を用いて,ソースコード中から低速なソースコード片を検出する.CodeQLは,マイクロベンチマーク共有サービスで公開されるソースコード片とは入出力が異なっていても,構造が類似するソースコードの検出が期待される.
提案手法は,動的解析に依存せずに潜在的な性能効率性に寄与するボトルネックを実装初期段階で検出できる.

続く\ref{chap:background}章では,本研究で利用するマイクロベンチマーク共有サービスにおけるマイクロベンチマークの特徴,および関連研究を紹介し,本研究の立ち位置を述べる. \ref{chap:pre-analysis}章で事前分析について示し, \ref{chap:approach}章では,本研究の提案手法を述べ,\ref{chap:evaluation}章で評価方法について述べる.\ref{chap:case-study}章においてケーススタディの結果を述べ,\ref{chap:discussion}章で考察と妥当性への脅威を示し,\ref{chap:summary}章で本研究をまとめる. 



\chapter{プログラム実行速度の改善}\label{chap:background}

\section{ソフトウェアの性能効率性と部分的な最適化}
\todo{\textbf{はじめに}との棲み分けを考える}
 ソフトウェアの性能効率性は,ユーザ体験や運用コスト,さらにはシステム全体の品質に直結する重要な要素である\cite{performance1}\cite{performance2}\cite{negative}.
 部分的なソースコードの最適化を積み重ねることで実現することも多い.

\section{マイクロベンチマーク共有サービス}

マイクロベンチマーク共有サービスは,開発者がプログラムの実行速度を比較・共有するためのオンラインサービスである.図\ref{fig:jsPerf}は,マイクロベンチマーク共有サービスの1つであるJsPerf上で実際に投稿されているマイクロベンチマークの例を示す.

マイクロベンチマーク共有サービスにおける各ベンチマークは,1つのセットアッププログラムと,1つ以上のテストプログラムから構成される.図上部に示すセットアッププログラムは,検証対象のソースコードで共通して利用される変数や関数の初期化などが行われる.一方,図下部に示すテストプログラムでは,同一の機能を異なる方法で実装したソースコード片が複数提示され,それぞれの実行速度を計測し,比較できる.
このようなマイクロベンチマークは,データ構造や制御構文の選択,メソッド呼び出しの方法などに多様性が見られることが特徴であり,実行速度の差とその要因となる実装方法を捉えることができる.

% 本研究では,このようなマイクロベンチマーク共有サービス上の実装対を,性能効率性,特に実行速度に影響を与える構造的特徴を分析および抽出するために利用する.

%----------------------
\begin{figure}[!h]
    \centering
    \includegraphics[width=1.0\linewidth]{./Noguchi_fig/jsPerf_example.pdf}
    \caption{マイクロベンチマーク共有サービスの投稿例\protect\footnotemark}
    \label{fig:jsPerf}
\end{figure}

\footnotetext{\url{https://jsperf.app/qiwudo}}
%----------------------



開発者はマイクロベンチマーク共有サービスを用いて複数の実装方法を比較することはもちろん,サービス上でマイクロベンチマークを証拠としてソースコードの改善案を作成している事例もある\cite{saiki}.
% しかし,マイクロベンチマーク共有サービスで公開されているベンチマークに基づいて,開発者が実装するソフトウェアの中から性能効率性が向上できる箇所を検出し,その修正を実現することは容易ではない.これは,実装するソフトウェアにおけるソースコード間の依存関係や構造に対する,開発者の知識や技量に依存するためである.


\section{低速な箇所の検出における難しさ}

\memo{動的検出では実装が進まないできないこと}
ソフトウェアの性能効率性を評価する方法として,プロファイラなどの動的解析ツールが広く用いられている.動的解析ツールは,実行時の関数呼び出しやリソース使用状況を精緻に観測し,性能低下の原因となる箇所を特定することができる.しかし,動的解析ツールによる性能効率性の評価は,評価対象の機能が実行できるまで実装が進んだ状態でなければ評価できないため,開発途中に性能効率性を評価することは難しい.

\memo{静的検出では「正解」がないこと}
\memo{マイクロベンチマーク(規模が小さい)ならではの難しさ}


\section{関連研究}

\subsection{実行速度向上を目的としたリファクタリングの調査}

Selakovicら\cite{jsRefac}は,JavaScriptを利用しているプロジェクトにおいて,開発者が高速化のために行ったリファクタリングを調査している.調査の結果,開発者は10行程度の小さい範囲の修正によって高速化への対処を行なっていることを明らかにした.この結果は,マイクロベンチマーク共有サービスで頻繁に比較される短いソースコード片の修正が,ソフトウェア高速化につながっていることを示し,本研究の動機づけとなっている.また,\cite{jsRefac}では,JavaScriptプロジェクトの解析によって10件の頻出する高速化修正パターンを作成している.この改良パターンを用いた自動修正は一定の高速化効果を示しているが,該当箇所の特定における制約などから,広範な適用には至っていない.

\subsection{静的解析と動的解析を用いた性能ボトルネックの検出}

Turcotte ら\cite{DrAsync}は,JavaScript 言語における非同期処理に注目した性能アンチパターンを定義し,静的解析エンジンであるCodeQL\cite{ql}を利用したアンチパターンの検出と,動的解析を利用したパフォーマンスの監視を組み合わせ,修正可能な性能アンチパターンの検出を行った.本研究の静的解析による低速コードパターンの検出はこれに着想を得ている.\todo{このキモ文章は直す}


\subsection{本研究の位置付け}

\todo{SIGSEから卒論用に表現を直す.ゴールっぽくする}
本研究では,マイクロベンチマーク共有サービスで公開される実装対における低速コードから,構造的差分に基づいて低速コードパターンを抽出し,静的解析を用いた性能ボトルネックの早期検出を行うことを目指す.本論文は,その第一歩として,ベンチマークのソースコード片のうち,繰り返し処理にターゲットを絞り,ベンチマークに基づいて修正することで実行速度の向上が期待されるソースコード片の検出を目的とする.


\chapter{事前分析:マイクロベンチマークにおける低速コードの特徴}\label{chap:pre-analysis}

\section{データセット}

本研究では,大森ら\cite{omori}が作成したデータセットを分析対象とする.当該データセットは,マイクロベンチマーク共有サービスJsPerfで評価されたベンチマークの中で,実行速度に有意差があり,外的振る舞いが等しいことが検証された実装対29,809件を含む.以降,分析対象とするベンチマークで比較されたソースコード片の対をマイクロベンチマーク実装対,各実装対において,実行速度が遅いコードを低速コード,実行時間が速いコードを高速コードとする.

\section{抽象構文木に基づく実装方法の分析}

マイクロベンチマーク実装対を目視調査した結果,実装対の両方もしくは片方に繰り返し処理を含むものが多く存在することを確認した.特に,巨大な配列や長大な繰り返し処理を用いることで性能差が顕著になる実装対や,\texttt{for}と\texttt{for-of}のように,繰り返し処理の実装方法の違いによって実行時間に差が生じる実装対など,繰り返し処理に関する複数の特徴を確認した.
このような観察結果から,本研究ではマイクロベンチマーク実装対に対して繰り返し処理構造に着目した特徴分析を行う.対象とする繰り返し処理は,JavaScriptにおける \texttt{for},\texttt{for-of},\texttt{for-in},\texttt{while},および \texttt{do-while}の構文を対象とする.
以後,マイクロベンチマーク実装対のもつ特徴について,実施した分析内容とともに例を示す.

マイクロベンチマーク実装対の特徴分析には,各実装対のソースコードをGumTree\cite{gumtree}により抽象構文木へと変換し,実装対の抽象構文木間の差分解析を行った.差分解析の結果に対して,次の2点を確認し,繰り返し処理に関連する差分要素として収集する.\\
\noindent(1) 差分が直接繰り返し処理構造を含むか\\
\noindent(2) 差分要素の構造的な親要素に繰り返し処理を含むか\\
抽出された箇所に応じて目視で確認を行い,実装対の構造的特徴および性能差との関係を整理した.

\section{繰り返し処理を含む低速コードの主要パターン}

分析の結果,マイクロベンチマーク実装対において,実行速度に差が生じる繰り返し処理の使用パターンを複数確認した.具体例として,Listing~\ref{diff-loop},Listing~\ref{diff-inloop},Listing~\ref{diff-method}に,特徴となる部分について示す.

Listing~\ref{diff-loop}は,それぞれ\texttt{for-in}文を用いて配列に含まれる要素にアクセスする実装と,同様の処理を\texttt{for}文で実装したソースコード片である.ここで示すパターンは実行時間の差が,繰り返し処理の実装方法の違いに起因するパターンである.
%----------------------------------
\begin{lstlisting}[caption=Pairs with loop differences, label=diff-loop, captionpos=t, columns=flexible]
// slow
for (key in VAR_1) {
    if (!VAR_1.hasOwnProperty(key)) continue;
    VAR_2 = VAR_1[key];
}

// fast
for (var VAR_8=0; VAR_8<VAR_1.length; ++VAR_8) {
    VAR_2 = VAR_1[VAR_8];
}
\end{lstlisting}
%----------------------------------

Listing~\ref{diff-inloop}は,それぞれ\texttt{for}文内で \texttt{concat}メソッド,\texttt{push}メソッドを用いている実装対である.これは,繰り返し処理の実装方法は一致しているが,繰り返し内部で実行する処理が異なるパターンである.
%----------------------------------
\begin{lstlisting}[caption=Pairs with differences within the loop, label=diff-inloop, captionpos=t, columns=flexible]
// slow
for (var VAR_2=0; VAR_2<5000; VAR_2++)
    VAR_1 = VAR_1.concat([\"1\", \"2\"]);

// fast
for (var VAR_2=0; VAR_2<5000; VAR_2++)
    VAR_1.push(\"1\", \"2\");
\end{lstlisting}
%----------------------------------

Listing~\ref{diff-method}は,繰り返し処理を\texttt{forEach}メソッドで実装したものと\texttt{for-of}文で実装したものである.これは,メソッドによる処理とそれに代替する繰り返し処理について比較したパターンである.
%----------------------------------
\begin{lstlisting}[caption=Pairs of Method and alternative loop, label=diff-method, captionpos=t, columns=flexible]
// slow
var VAR_5 = new Set(VAR_2);
VAR_5.forEach(VAR_6 => {});

// fast
for (let VAR_7 of VAR_2) {}
\end{lstlisting}
%----------------------------------

これらの結果から,マイクロベンチマーク実装対には繰り返し処理の実装および繰り返し内部の操作内容の違いが性能差の要因の1つとして存在することを確認した.本研究では,このような繰り返し処理に関する構造の差分に基づいて低速コードの特徴を捉え,次章で述べる低速コードパターンの抽出および静的検出に利用する.


\chapter{構造的差分に基づく低速コードパターンの検出手法}

\section{}

\section{}

\section{}

\section{}


\chapter{類似度に基づく検出結果の順位付け}

\section{}

\section{}

\section{}


\chapter{ケーススタディ}

\section{}

\section{}

\section{}

\section{}


\chapter{考察}

\section{}

\section{}



\chapter{妥当性の脅威}

\section{}

\section{}


\chapter{おわりに}

%%%%%%%%%%%%%%%%%%%%%%%%%%%%%%%%%%%%%%%%%%%%%%%%%%%%%%%%%%%%%%%%%%%%%%%%

%%
%% 謝辞
%%
%% \begin{acknowledgements}
%% 感謝します.
%% \end{acknowledgements}

%%%%%%%%%%%%%%%%%%%%%%%%%%%%%%%%%%%%%%%%%%%%%%%%%%%%%%%%%%%%%%%%%%%%%%%%

%%
%% 参考文献
%%
\begin{thebibliography}{99}

\bibitem{wusethesis}
  伊原彰紀,
  卒業論文スタイルファイル(和歌山大学システム工学部用),\\
  \url{https://github.com/fukuyasu/wuse_thesis}.

\bibitem{tex}
  Knuth, D.,
  Remarks to Celebrate the Publication of Computers \& Typesetting,
  TUGboat, Vol.7, No.2, pp.95--98, 1986.

\bibitem{latex}
  Lamport, L.,
  文書処理システム\LaTeXe{},
  ピアソン・エデュケーション,1999,
  \newblock{}阿瀬はる美 訳.

\bibitem{latex_j}
  奥村晴彦,\LaTeX{}入門 ---美文書作成のポイント---,技術評論社,1993.

\bibitem{latex2e}
  奥村晴彦,黒木裕介,[改定第6版] \LaTeXe~美文書作成入門,技術評論社,2013.

\bibitem{latexcomp}
  Goossens, M., Mittelbach, F. and Samarin, A.,
  The \LaTeX{}コンパニオン,アスキー出版局,1998,
  \newblock{}アスキー書籍編集部 監訳.

\bibitem{texwiki}
  \LaTeX 入門 --- \TeX{} Wiki,\\
  \url{https://texwiki.texjp.org/?LaTeX%E5%85%A5%E9%96%80},
  2021年12月3日閲覧.
\end{thebibliography}

%%%%%%%%%%%%%%%%%%%%%%%%%%%%%%%%%%%%%%%%%%%%%%%%%%%%%%%%%%%%%%%%%%%%%%%%

%%
%% 付録
%%
% \appendix
% 
% \chapter{サンプルプログラム}
% 
% プログラムリストや実行結果など,本論を補足する上で必要と思われるものが
% あれば付録として付ける.
% 
% {
% \footnotesize
% \begin{verbatim}
% #include <stdio.h>
% int main(void)
% {
%     printf("Hello, World!\n");
%     return 0;
% }
% \end{verbatim}
% }

%%%%%%%%%%%%%%%%%%%%%%%%%%%%%%%%%%%%%%%%%%%%%%%%%%%%%%%%%%%%%%%%%%%%%%%%

\end{document}
