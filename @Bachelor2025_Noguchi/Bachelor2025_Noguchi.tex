\documentclass[11pt]{jreport}
\usepackage{wuse_thesis}
\usepackage{indentfirst}
\usepackage{url}	% \url{}コマンド用.URLを表示する際に便利
\usepackage{xcolor}
%\usepackage{graphicx}  % ←graphicx.styを用いてEPSを取り込む場合有効にする
			% 他のパッケージ・スタイルを使う場合には適宜追加

\newcommand{\todo}[1]{\colorbox{yellow}{{\bf TODO}:}{\color{red} {\textbf{[#1]}}}}
\newcommand{\memo}[1]{\colorbox{magenta!30}{{\bf MEMO}:}{\color{red!50} {\textbf{[#1]}}}}
\newcommand{\ihara}[1]{\colorbox{green}{{\bf IHARA}:}{\color{blue} {\textbf{[#1]}}}}

%%%%%%%%%%%%%%%%%%%%%%%%%%%%%%%%%%%%%%%%%%%%%%%%%%%%%%%%%%%%%%%%%%%%%%%%

%%
%% 主に表紙を作成するための情報
%%

%%  タイトル(修論の場合は英語表記も指定)
\title{卒業・修士論文用スタイルファイルを用いた\\
       p\LaTeX による卒業・修士論文の作成}
%\etitle{Test\\Test\\Test}

%%  著者名(修論の場合は英語表記も指定)
\author{野口 隼杜}
%\eauthor{Akinori Ihara}

%% 卒業論文・修士論文(以下のどちらかを選択)
\bachelar	% 卒業論文(4年生用)
%\master  	% 修士論文(M2用)

%%  学科・クラスタ
\department{システム工}
%\department{デザイン情報}
%\department{デザイン科学}

%%  学生番号
\studentid{60276185}

%%  卒業年度
\gyear{2025}		% 提出年が2022年なら,2021年度

%%  論文提出日
\date{2026年2月10日}	% 修士の場合は月(2021年2月)までとし,英語表記も指定
%\edate{February 2021}	% 修士の場合,こちら(英語表記)も有効化

%%%%%%%%%%%%%%%%%%%%%%%%%%%%%%%%%%%%%%%%%%%%%%%%%%%%%%%%%%%%%%%%%%%%%%%%

\begin{document}

\maketitle

%%
%%  概要
%%
\begin{abstract}
\todo{最後に作る}

「和歌山大学システム工学部卒業論文/
  大学院システム工学研究科修士論文用スタイルファイル」
は,
\begin{quote}
  \begin{description}
    \item[\tt wuse\_thesis.sty:] 卒業/修士論文用スタイルファイル
    \item[\tt thesis\_sample.tex:] スタイルファイル利用例
  \end{description}
\end{quote}
からなる.

なお,この卒業論文用スタイルファイル(p\LaTeX 版)に関する質問は,
メールにて
\begin{quote}
ihara@wakayama-u.ac.jp
\end{quote}
まで.

\end{abstract}

%%  目次
\tableofcontents

%%  図目次 (図目次をいれたければ以下のコメントをはずす)
%\listoffigures

%%  表目次 (表目次をいれたければ以下のコメントをはずす)
%\listoftables

\newpage
\pagenumbering{arabic}	% 以降のページ番号を算用数字に

%%%%%%%%%%%%%%%%%%%%%%%%%%%%%%%%%%%%%%%%%%%%%%%%%%%%%%%%%%%%%%%%%%%%%%%%

%%
%%  本文はここから
%%

\chapter{はじめに}

\memo{研究背景・研究課題・目的および本研究の貢献・論文構成}
 ソフトウェアの性能効率性は,ユーザ体験や運用コスト,さらにはシステム全体の品質に直結する重要な要素である\cite{performance1}\cite{performance2}\cite{negative}.
 部分的なソースコードの最適化を積み重ねることで実現することも多い.

性能効率性を向上するためのプログラム変更は,実装が進行するにつれて複雑になりやすい\cite{complicate}.また,保守性や可読性など,他のプログラム品質に否定的な影響を与えることもある\cite{negative}.したがって,性能効率性を向上するソースコードへ書き換えるためには,実装が進むほどに開発者に対して広範な知識や経験を要するようになることが考えられる.そのため,ソフトウェアの性能効率性を,実装途中の早期の段階で見積もり,性能低下の原因となるボトルネックを検出することは,性能効率性の改善にかかる修正工数を小さくできる.

ソフトウェアの性能効率性を評価する方法として,プロファイラなどの動的解析ツールが広く用いられている.動的解析ツールは,実行時の関数呼び出しやリソース使用状況を精緻に観測し,性能低下の原因となる箇所を特定することができる.しかし,動的解析ツールによる性能効率性の評価は,評価対象の機能が実行できるまで実装が進んだ状態でなければ評価できないため,開発途中に性能効率性を評価することは難しい.

実装途中においても,部分的なソースコードの性能を定量的に評価する手法として,マイクロベンチマークを使用する方法がある.マイクロベンチマークは機能的に等価な複数の異なるソースコード片に対して実行時間を測定することで,ソースコード間の性能効率性を比較できる.マイクロベンチマークを共有するサービスには,JavaScriptを対象としたJsPerf\footnote{JsPerf: \url{https://jsperf.app/}}やMeasureThat.net\footnote{MeasureThat.net: \url{https://measurethat.net/}}がある.これらのサービスでは,ブラウザ上でマイクロベンチマークの実行環境を提供しており,JavaScriptのソースコード片の実行速度の測定および比較ができる.また,評価されたソースコード片,および測定結果はサービス上で公開されている.

開発者はマイクロベンチマーク共有サービスを用いて複数の実装方法を比較することはもちろん,サービス上でマイクロベンチマークを証拠としてソースコードの改善案を作成している事例もある\cite{saiki}.
しかし,マイクロベンチマーク共有サービスで公開されているベンチマークに基づいて,開発者が実装するソフトウェアの中から性能効率性が向上できる箇所を検出し,その修正を実現することは容易ではない.これは,実装するソフトウェアにおけるソースコード間の依存関係や構造に対する,開発者の知識や技量に依存するためである.

% ソースコード「片」で統一
本研究では,ソフトウェア中に潜在する,修正することで実行速度の向上が期待されるソースコード片の検出を目的とする.具体的には,マイクロベンチマーク共有サービスでベンチマークとして比較されたソースコード片の構造的な差分から低速コードパターンを作成する.作成した低速コードパターンを利用し,静的解析エンジンであるCodeQL\footnote{\url{https://codeql.github.com/}}\cite{ql}を用いて,ソースコード中から低速なソースコード片を検出する.CodeQLは,マイクロベンチマーク共有サービスで公開されるソースコード片とは入出力が異なっていても,構造が類似するソースコードの検出が期待される.
提案手法は,動的解析に依存せずに潜在的な性能効率性に寄与するボトルネックを実装初期段階で検出できる.

続く\ref{chap:background}章では,本研究で利用するマイクロベンチマーク共有サービスにおけるマイクロベンチマークの特徴,および関連研究を紹介し,本研究の立ち位置を述べる. \ref{chap:pre-analysis}章で事前分析について示し, \ref{chap:approach}章では,本研究の提案手法を述べ,\ref{chap:evaluation}章で評価方法について述べる.\ref{chap:case-study}章においてケーススタディの結果を述べ,\ref{chap:discussion}章で考察と妥当性への脅威を示し,\ref{chap:summary}章で本研究をまとめる. 



\chapter{プログラム実行速度の改善}\label{chap:background}

論文では,図,表,数式などを効果的に使用する.

\section{図}

{\tt figure}環境を利用することによって図にキャプション
(\verb|\caption|)を付けることができる.図に付けられたキャプションは
\verb|\listoffigures|によって図目次として出力される.図には章ごとに通
し番号が付けられ,キャプションに\verb|\label|を設定しておくと,
``図\ref{fig:sample}''のように\verb|\ref|によって図を番号で参照するこ
とができる.図\ref{fig:sample}に{\tt figure}環境を用いた記述例を示す.

\begin{figure}
  \centering
    ここで図を取り込む.
    % 試しに,tiger.psが自分のマシンのどこに格納されているかを調べて
    % 以下の命令を有効にしてみて下さい.
    % ただし,同時に\begin{document}より前にある\usepackage{graphicx}
    % も有効にする必要があります.
    %\includegraphics[width=5cm,clip]{/usr/local/share/ghostscript/7.07/examples/tiger.ps}
  \caption{図の例}
  \label{fig:sample}
\end{figure}

また,{\tt graphicx.sty}などのスタイルファイルを利用することによって
EPS形式やPDF形式の図を文章の中に取り込むことができる.
この場合,\verb|\begin{document}|の前に\verb|\usepackage{graphicx}|を
追加する.

なお,図表の配置は基本的には\LaTeX{}が決めるので,思った位置に入らない
からといって無理に場所を指定するのはよくない.
どうしても位置を固定したい場合には,すべての文章が書きあがった後に指定
するとよい\footnote{そうしないと文章を書き換えるたびに,位置がずれる可能性がある}.

\section{表}

{\tt table}環境を利用することによって図と同じように,キャプションをつ
けたり,ラベルにより参照したりすることができる.また
\verb|\listoftables|によって表目次として出力される.
表\ref{tab:sample}に{\tt table}環境で作成した表を示す.

\begin{table}
  \caption{表の例}
  \label{tab:sample}
  \centering
  \begin{tabular}{|c|c|c|}
    \hline
    8 & 3 & 4\\
    \hline
    1 & 5 & 9 \\
    \hline
    6 & 7 & 2 \\
    \hline
  \end{tabular}
\end{table}

\section{数式}

\TeX では数式のための機能が豊富である.
{\tt equation}環境などを利用することによって数式に番号を付けることがで
きる.図や表と同じくラベルを付けておけば,``式\ref{exp:sample}''のよう
に数式を番号で参照することができる.

\begin{equation}
  y = ax^2 + bx + c \label{exp:sample}
\end{equation}

\chapter{参考文献}

文献を参照する場合には,論文の最後に参考文献として列挙するとともに,
\verb|\cite|を使って,例えば,
\begin{quote}
  文献\cite{latex}によれば…
\end{quote}
や,
\begin{quote}
  …である\cite{latex2e}.
\end{quote}
のように参照する.

文献の列挙には,{\tt thebibliography}環境などを用いる\footnote{使い方
は,この資料のソースを参照.}.

%%%%%%%%%%%%%%%%%%%%%%%%%%%%%%%%%%%%%%%%%%%%%%%%%%%%%%%%%%%%%%%%%%%%%%%%

%%
%% 謝辞
%%
%% \begin{acknowledgements}
%% 感謝します.
%% \end{acknowledgements}

%%%%%%%%%%%%%%%%%%%%%%%%%%%%%%%%%%%%%%%%%%%%%%%%%%%%%%%%%%%%%%%%%%%%%%%%

%%
%% 参考文献
%%
\begin{thebibliography}{99}

\bibitem{wusethesis}
  伊原彰紀,
  卒業論文スタイルファイル(和歌山大学システム工学部用),\\
  \url{https://github.com/fukuyasu/wuse_thesis}.

\bibitem{tex}
  Knuth, D.,
  Remarks to Celebrate the Publication of Computers \& Typesetting,
  TUGboat, Vol.7, No.2, pp.95--98, 1986.

\bibitem{latex}
  Lamport, L.,
  文書処理システム\LaTeXe{},
  ピアソン・エデュケーション,1999,
  \newblock{}阿瀬はる美 訳.

\bibitem{latex_j}
  奥村晴彦,\LaTeX{}入門 ---美文書作成のポイント---,技術評論社,1993.

\bibitem{latex2e}
  奥村晴彦,黒木裕介,[改定第6版] \LaTeXe~美文書作成入門,技術評論社,2013.

\bibitem{latexcomp}
  Goossens, M., Mittelbach, F. and Samarin, A.,
  The \LaTeX{}コンパニオン,アスキー出版局,1998,
  \newblock{}アスキー書籍編集部 監訳.

\bibitem{texwiki}
  \LaTeX 入門 --- \TeX{} Wiki,\\
  \url{https://texwiki.texjp.org/?LaTeX%E5%85%A5%E9%96%80},
  2021年12月3日閲覧.
\end{thebibliography}

%%%%%%%%%%%%%%%%%%%%%%%%%%%%%%%%%%%%%%%%%%%%%%%%%%%%%%%%%%%%%%%%%%%%%%%%

%%
%% 付録
%%
% \appendix
% 
% \chapter{サンプルプログラム}
% 
% プログラムリストや実行結果など,本論を補足する上で必要と思われるものが
% あれば付録として付ける.
% 
% {
% \footnotesize
% \begin{verbatim}
% #include <stdio.h>
% int main(void)
% {
%     printf("Hello, World!\n");
%     return 0;
% }
% \end{verbatim}
% }

%%%%%%%%%%%%%%%%%%%%%%%%%%%%%%%%%%%%%%%%%%%%%%%%%%%%%%%%%%%%%%%%%%%%%%%%

\end{document}
