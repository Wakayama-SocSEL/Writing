\documentclass[11pt]{jreport}
\usepackage{wuse_thesis}
\usepackage{indentfirst}
\usepackage{url}	% \url{}コマンド用.URLを表示する際に便利
\usepackage[dvipdfmx]{graphicx,xcolor}
\usepackage{listings}
\lstset{
  basicstyle={\ttfamily},
  identifierstyle={\small},
  commentstyle={\smallitshape},
  keywordstyle={\small\bfseries},
  ndkeywordstyle={\small},
  stringstyle={\small\ttfamily},
  frame={tb},
  breaklines=true,
  columns=[l]{fullflexible},
  numbers=left,
  xrightmargin=0zw,
  xleftmargin=3zw,
  numberstyle={\scriptsize},
  stepnumber=1,
  numbersep=1zw,
  lineskip=-0.5ex
}
\usepackage{siunitx}
\sisetup{group-separator={,}} % 3桁ごとにコンマを入れる設定
%\usepackage{graphicx}  % ←graphicx.styを用いてEPSを取り込む場合有効にする
			% 他のパッケージ・スタイルを使う場合には適宜追加
\newcommand{\todo}[1]{\colorbox{yellow}{{\bf TODO}:}{\color{red} {\textbf{[#1]}}}}
\newcommand{\memo}[1]{\colorbox{magenta}{\textbf{MEMO}}{\color{red}\textbf{[#1]}}}

%%%%%%%%%%%%%%%%%%%%%%%%%%%%%%%%%%%%%%%%%%%%%%%%%%%%%%%%%%%%%%%%%%%%%%%%

%%
%% 主に表紙を作成するための情報
%%

%%  タイトル(修論の場合は英語表記も指定)
\title{タイトル}
%\etitle{Test\\Test\\Test}

%%  著者名(修論の場合は英語表記も指定)
\author{豊嶋 浩基}
%\eauthor{Akinori Ihara}

%% 卒業論文・修士論文(以下のどちらかを選択)
\bachelar	% 卒業論文(4年生用)
%\master  	% 修士論文(M2用)

%%  学科・クラスタ
\department{システム工}
%\department{デザイン情報}
%\department{デザイン科学}

%%  学生番号
\studentid{60276157}

%%  卒業年度
\gyear{2025}		% 提出年が2022年なら,2021年度

%%  論文提出日
\date{2026年2月10日}	% 修士の場合は月(2021年2月)までとし,英語表記も指定
%\edate{February 2021}	% 修士の場合,こちら(英語表記)も有効化

%%%%%%%%%%%%%%%%%%%%%%%%%%%%%%%%%%%%%%%%%%%%%%%%%%%%%%%%%%%%%%%%%%%%%%%%

\begin{document}

\maketitle

%%
%%  概要
%%
\begin{abstract}
大規模言語モデル(LLM) はソフトウェア開発自動化に大きく貢献するが,複雑な要件を持つソフトウェア開発の実現には課題が多い.
本研究は,このような課題の解決に向けて,自然言語からソースコードを自動生成するマルチエージェント型のプラットフォームChatDevを拡張する.
具体的には,few-shot学習を用いた要件の細分化と各ステップの開発を行うLLM に要件の全体概要の学習によるソースコード自動生成並びに,生成ステップ毎の差分行数に基づく要件の統合及び,統合結果を分割例と再分割を組み合わせた反復的に進める開発プロセスを提案する.
ケーススタディとして,競技プログラミングサイトであるAtCoderの問題を題材に,この提案手法を実施した結果,few-shot学習と全体概要の共有の組み合わせによりテスト通過率が向上し,統合に基づく要件の再分割により生成されたソースコードはさらにテスト通過率が向上することを明らかにした.
\begin{quote}
  \begin{description}
    \item[\tt wuse\_thesis.sty:] 卒業/修士論文用スタイルファイル
    \item[\tt thesis\_sample.tex:] スタイルファイル利用例
  \end{description}
\end{quote}
からなる.

なお,この卒業論文用スタイルファイル(p\LaTeX 版)に関する質問は,
メールにて
\begin{quote}
ihara@wakayama-u.ac.jp
\end{quote}
まで.

\end{abstract}

%%  目次
\tableofcontents

%%  図目次 (図目次をいれたければ以下のコメントをはずす)
%\listoffigures

%%  表目次 (表目次をいれたければ以下のコメントをはずす)
%\listoftables

\newpage
\pagenumbering{arabic}	% 以降のページ番号を算用数字に

%%%%%%%%%%%%%%%%%%%%%%%%%%%%%%%%%%%%%%%%%%%%%%%%%%%%%%%%%%%%%%%%%%%%%%%%

%%
%%  本文はここから
%%

\chapter{はじめに}

大規模言語モデル(LLM)技術の急速な発展に伴い\todo{引用},LLMはこれまで人間が時間や労力をかけて取り組んできたタスクの自動化を実現し,幅広い分野において作業を効率化する技術として関心を集めている.ソフトウェア開発においても同様に,コードレビュー,リファクタリング,テストケース生成など,様々な場面において飛躍的に生産性を向上させることが確認されている.\todo{引用,引用,引用} その中でも,開発者の意図や要件をプロンプトとして提示し,ソースコードを自動生成するタスクに対して大きな期待が寄せられている.

LLMは,自然言語で記述された要求文に基づきソースコードの自動生成を実現し,昨今では複雑で,大規模なソフトウェア開発の自動化に向けた研究が進められている.\todo{引用} 規模が小さいソフトウェア開発の自動化に向けた研究が進められている\todo{引用}.規模が小さいソフトウェアの要求は実現できる一方で,ソフトウェア要件が複数内包し,それぞれが相互に依存し合う大規模なソフトウェア要件を満たすソースコード生成には課題が多い.従来研究ではソフトウェア要件が複雑になると,LLM が十分に推論せずに,短絡的なソースコードを生成することが示されている\todo{引用}.例えば,仕様の一部が欠損している場合や,要件の文脈を正しく理解できない場合には,生成されるコードの構造的なロジックの誤りが見られることがある.これは,要件文から要求を抽出し,それらを元にソースコードを生成する,という流れをLLMが実施するが,複雑な要件からソースコードを生成する場合,LLMは要件を元にソースコードの生成を行うため,要求から抽出される要件の誤りが,不完全なソースコードを生成する原因の1つであると考えられる.

LLM の理解や推論を補助するプロンプト技術として,思考の過程を明示的に示すChain of Thought (CoT) や,要求を満たす例を少数提示するfew-shot 学習が用いられてる\todo{引用}.これらは局所的な推論の補完や思考パターンの学習には有効である一方で,プログラミング言語のように記述方法や実装方法によって同一の要件に対して多様な解が存在する領域では,これらの手法のみでは十分な性能を発揮しにくい.

これに対して先行研究では,全体の流れから要件を抽出するフェーズと,それらを基にソースコードを生成するフェーズの2 つに分割する手法が取り入れられている[8].前者では,大量のデータ収集やそれに対するラベリング,再学習などを要するファインチューニングと比較して,追加学習することなくLLM の挙動を制御できるfew-shot 学習が採用されている.当該研究では,複数の各機能を並列に生成し,生成されたすべてのソースコード片の結合時に,各断片における前提が共有されず,関数やファイル単位での依存関係や,グローバル変数の扱いや,エラーハンドリングなど,ソースコード単位での整合性が破綻する課題に言及している.

本研究では,従来研究[8] の要求抽出の不安定さや,段階的開発におけるソースコードの断片化の課題解決に向けて,複雑な要件の細分化および,細分化した要件ごとの開発における全体整合を実施することでソースコード自動生成を実現する.さらに,細分化した要件をベースとして生成したソースコードを評価し,要件の統合と再分割によるソースコード自動生成を複数回繰り返すことでプロンプト作成の最適化を目指す開発プロセスを提案する.本手法により,各工程においてLLM に対して全体概要を明示的に提示し,分割した要件の統合と再分割を繰り返すことで,マルチエージェント型ソースコード自動生成の品質を向上を期待する.

論文構成は以下の通りである.2章で本研究で使用するフレームワークやキーアイディアのベースとなる関連研究について,3章ではそれらに基づいたアプローチを述べ,4章でそのアプローチに関する実験設定やRQ を列挙し,5章でそれに対する結果を提示する.6章で考察と妥当性の脅威について議論した上で,7章で結論を述べる.

\chapter{ソフトウェア開発におけるLLMの活用} \memo{関連研究っぽくする}
\section{ソフトウェア開発工程のLLMによる再現・自動化}
\subsection{工程ベースでの自動化}
LLM を活用したソフトウェア開発の全プロセスの自動化を目的としたシステムとして,MetaGPT,AutoDev などの開発が盛んに進められている\todo{引用,引用}.本システムは,ソフトウェア開発プロセスにおける各工程(要求定義,設計,実装,テスト,保守)を自然言語理解と生成能力により自動化する仕組みである.本研究では,各工程の役割を担うLLM エージェントがそれぞれ要件定義から実装までを実現するChatDev\todo{引用}を用いる.

ChatDev は,要件定義,設計,実装をウォーターフォールモデルに則って開発を進める過程で,各工程においてプログラマやプロダクトマネージャといった役割を与えられた2 つのLLM エージェントが対話しながら開発を行うプラットフォームである.各工程では,開発を担当するエージェントが,プロンプトで与えられた要件文,または前工程で生成された成果物(ソースコード)に基づき,新たな成果物を生成する.生成した成果物は別の検証を担当するエージェントと共有される.共有を受けたエージェントは,成果物を検証し,開発と検証を担当するエージェント間で合意形成を図って,成果物を完成させる.

ChatDev の枠組みは,各工程のタスクが明確である点や,フレームワークがオープンソースとして公開されており,自体の構造の確認や書き換えが可能であるため,拡張性に優れている.一方で,各工程で開発を担当するLLMに対して,同一の要件(プロンプト)が与えられるため,多数の機能を内包するような複雑・大規模な要件の場合,各フェーズでのタスクの粒度が大きくなり,不完全なソースコードが生成されやすくなってしまう.

\subsection{複雑な要求の分割に基づくコード生成}
Jiang ら[8] は,few-shot によりLLM を用いて要求文から小さな実装単位の要件に分割する「計画フェーズ」と,
分割後の要件を基に開発を行なっていく実装フェーズを組み合わせた手法を提案している.計画フェーズにおいてfew-shot 学習として分割例を入力して要求文を分割することで,分割した要件を最適な粒度に均一化する事を目指している.このように,当該研究が提案する分割後要件の粒度の最適化を図る「要件の細分化」と「細分化後要件の統合と再分割」は,LLM の推論の最適化プロセスの1つとして重要であると考えられる.さらに実装フェーズにおいては,分割後の要件文を個別に並行的に実装していくのではなく,開発順序を定めるガイドとして順次参照しつつ,全体を一体として一括で生成する構成が最も高性能であると報告されている.これは,複数の要件からソースコードに並行して生成し,最終的に結合を行う方式は,コンテキストの欠如により,インタフェースやデータ構造の不整合が発生する可能性が高くなると言及されている.

本研究では,プロンプトとその時点で作成された成果物を基に開発を行う構造へとChatDevを変化させる事で段階的開発を実施しながらも,断片化を回避する手法について調査する.
\todo{断片化をはじめとする説明不足箇所の修正}


\section{要求分割と分割粒度}
\subsection{分割粒度が生成品質に与える影響}

\subsection{LLMにとって好ましい分割粒度}

\subsection{たいせな}
\todo{ワンチャン消す}

\section{本研究の位置付け}

\chapter{段階的開発と分割粒度の再検討}
\section{アプローチ概要}

{\tt figure}環境を利用することによって図にキャプション
(\verb|\caption|)を付けることができる.図に付けられたキャプションは
\verb|\listoffigures|によって図目次として出力される.図には章ごとに通
し番号が付けられ,キャプションに\verb|\label|を設定しておくと,
``図\ref{fig:sample}''のように\verb|\ref|によって図を番号で参照するこ
とができる.図\ref{fig:sample}に{\tt figure}環境を用いた記述例を示す.

\begin{figure}
  \centering
    ここで図を取り込む.
    % 試しに,tiger.psが自分のマシンのどこに格納されているかを調べて
    % 以下の命令を有効にしてみて下さい.
    % ただし,同時に\begin{document}より前にある\usepackage{graphicx}
    % も有効にする必要があります.
    %\includegraphics[width=5cm,clip]{/usr/local/share/ghostscript/7.07/examples/tiger.ps}
  \caption{図の例}
  \label{fig:sample}
\end{figure}

また,{\tt graphicx.sty}などのスタイルファイルを利用することによって
EPS形式やPDF形式の図を文章の中に取り込むことができる.
この場合,\verb|\begin{document}|の前に\verb|\usepackage{graphicx}|を
追加する.

なお,図表の配置は基本的には\LaTeX{}が決めるので,思った位置に入らない
からといって無理に場所を指定するのはよくない.
どうしても位置を固定したい場合には,すべての文章が書きあがった後に指定
するとよい\footnote{そうしないと文章を書き換えるたびに,位置がずれる可能性がある}.

\section{表}

{\tt table}環境を利用することによって図と同じように,キャプションをつ
けたり,ラベルにより参照したりすることができる.また
\verb|\listoftables|によって表目次として出力される.
表\ref{tab:sample}に{\tt table}環境で作成した表を示す.

\begin{table}
  \caption{表の例}
  \label{tab:sample}
  \centering
  \begin{tabular}{|c|c|c|}
    \hline
    8 & 3 & 4\\
    \hline
    1 & 5 & 9 \\
    \hline
    6 & 7 & 2 \\
    \hline
  \end{tabular}
\end{table}

\section{数式}

\TeX では数式のための機能が豊富である.
{\tt equation}環境などを利用することによって数式に番号を付けることがで
きる.図や表と同じくラベルを付けておけば,``式\ref{exp:sample}''のよう
に数式を番号で参照することができる.

\begin{equation}
  y = ax^2 + bx + c \label{exp:sample}
\end{equation}

\chapter{参考文献}

文献を参照する場合には,論文の最後に参考文献として列挙するとともに,
\verb|\cite|を使って,例えば,
\begin{quote}
  文献\cite{latex}によれば…
\end{quote}
や,
\begin{quote}
  …である\cite{latex2e}.
\end{quote}
のように参照する.

文献の列挙には,{\tt thebibliography}環境などを用いる\footnote{使い方
は,この資料のソースを参照.}.

%%%%%%%%%%%%%%%%%%%%%%%%%%%%%%%%%%%%%%%%%%%%%%%%%%%%%%%%%%%%%%%%%%%%%%%%

%%
%% 謝辞
%%
%% \begin{acknowledgements}
%% 感謝します.
%% \end{acknowledgements}

%%%%%%%%%%%%%%%%%%%%%%%%%%%%%%%%%%%%%%%%%%%%%%%%%%%%%%%%%%%%%%%%%%%%%%%%

%%
%% 参考文献
%%
\begin{thebibliography}{99}

\bibitem{wusethesis}
  伊原彰紀,
  卒業論文スタイルファイル(和歌山大学システム工学部用),\\
  \url{https://github.com/fukuyasu/wuse_thesis}.

\bibitem{tex}
  Knuth, D.,
  Remarks to Celebrate the Publication of Computers \& Typesetting,
  TUGboat, Vol.7, No.2, pp.95--98, 1986.

\bibitem{latex}
  Lamport, L.,
  文書処理システム\LaTeXe{},
  ピアソン・エデュケーション,1999,
  \newblock{}阿瀬はる美 訳.

\bibitem{latex_j}
  奥村晴彦,\LaTeX{}入門 ---美文書作成のポイント---,技術評論社,1993.

\bibitem{latex2e}
  奥村晴彦,黒木裕介,[改定第6版] \LaTeXe~美文書作成入門,技術評論社,2013.

\bibitem{latexcomp}
  Goossens, M., Mittelbach, F. and Samarin, A.,
  The \LaTeX{}コンパニオン,アスキー出版局,1998,
  \newblock{}アスキー書籍編集部 監訳.

\bibitem{texwiki}
  \LaTeX 入門 --- \TeX{} Wiki,\\
  \url{https://texwiki.texjp.org/?LaTeX%E5%85%A5%E9%96%80},
  2021年12月3日閲覧.
\end{thebibliography}

%%%%%%%%%%%%%%%%%%%%%%%%%%%%%%%%%%%%%%%%%%%%%%%%%%%%%%%%%%%%%%%%%%%%%%%%

%%
%% 付録
%%
% \appendix
% 
% \chapter{サンプルプログラム}
% 
% プログラムリストや実行結果など,本論を補足する上で必要と思われるものが
% あれば付録として付ける.
% 
% {
% \footnotesize
% \begin{verbatim}
% #include <stdio.h>
% int main(void)
% {
%     printf("Hello, World!\n");
%     return 0;
% }
% \end{verbatim}
% }

%%%%%%%%%%%%%%%%%%%%%%%%%%%%%%%%%%%%%%%%%%%%%%%%%%%%%%%%%%%%%%%%%%%%%%%%

\end{document}
