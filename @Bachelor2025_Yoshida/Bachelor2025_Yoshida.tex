\documentclass[11pt]{jreport}
\usepackage{wuse_thesis}
\usepackage{indentfirst}
\usepackage{url}	% \url{}コマンド用.URLを表示する際に便利
\usepackage[dvipdfmx]{graphicx,xcolor}
\usepackage{listings}
\lstset{
  basicstyle={\ttfamily},
  identifierstyle={\small},
  commentstyle={\smallitshape},
  keywordstyle={\small\bfseries},
  ndkeywordstyle={\small},
  stringstyle={\small\ttfamily},
  frame={tb},
  breaklines=true,
  columns=[l]{fullflexible},
  numbers=left,
  xrightmargin=0zw,
  xleftmargin=3zw,
  numberstyle={\scriptsize},
  stepnumber=1,
  numbersep=1zw,
  lineskip=-0.5ex
}
\usepackage{siunitx}
\sisetup{group-separator={,}} % 3桁ごとにコンマを入れる設定
%\usepackage{graphicx}  % ←graphicx.styを用いてEPSを取り込む場合有効にする
			% 他のパッケージ・スタイルを使う場合には適宜追加
\newcommand{\todo}[1]{\colorbox{yellow}{{\bf TODO}:}{\color{red} {\textbf{[#1]}}}}
\newcommand{\memo}[1]{\colorbox{magenta}{\textbf{MEMO}}{\color{red}\textbf{[#1]}}}

%%%%%%%%%%%%%%%%%%%%%%%%%%%%%%%%%%%%%%%%%%%%%%%%%%%%%%%%%%%%%%%%%%%%%%%%

%%
%% 主に表紙を作成するための情報
%%

%%  タイトル(修論の場合は英語表記も指定)
\title{構造的類似度を用いた生存メソッドとの比較による\\
削除メソッドの要因分析}
%\etitle{Test\\Test\\Test}

%%  著者名(修論の場合は英語表記も指定)
\author{吉田 将衛}
%\eauthor{Akinori Ihara}

%% 卒業論文・修士論文(以下のどちらかを選択)
\bachelar	% 卒業論文(4年生用)
%\master  	% 修士論文(M2用)

%%  学科・クラスタ
\department{システム工}
%\department{デザイン情報}
%\department{デザイン科学}

%%  学生番号
\studentid{60276296}

%%  卒業年度
\gyear{2025}		% 提出年が2022年なら,2021年度

%%  論文提出日
\date{2026年2月10日}	% 修士の場合は月(2021年2月)までとし,英語表記も指定
%\edate{February 2021}	% 修士の場合,こちら(英語表記)も有効化

%%%%%%%%%%%%%%%%%%%%%%%%%%%%%%%%%%%%%%%%%%%%%%%%%%%%%%%%%%%%%%%%%%%%%%%%

\begin{document}

\maketitle

%%
%%  概要
%%
\begin{abstract}
ライブラリの機能追加,削除を繰り返す進化を続けているが,ソフトウェアの規模は拡大の一途をたどっている.この問題を解決するためのソフトウェア再構成として,類似する機能を持つメソッドの削除が一つの手段である.しかし,削除判断の客観的な指標が存在しないため,ソフトウェア再構成の一つとしてメソッドの削除は開発者の経験に依存せざるを得ない.
本研究では,削除されるメソッドの特徴を明らかにする.具体的には,「同じバージョンに含まれる複数のメソッド間で類似度が高いと片方が削除される」を検証する.ケーススタディとして広く利用されるライブラリを対象とした結果,類似度が高いメソッド間において片方が削除される事例を確認した.
\end{abstract}

%%  目次
\tableofcontents

%%  図目次 (図目次をいれたければ以下のコメントをはずす)
%\listoffigures

%%  表目次 (表目次をいれたければ以下のコメントをはずす)
%\listoftables

\newpage
\pagenumbering{arabic}	% 以降のページ番号を算用数字に

%%%%%%%%%%%%%%%%%%%%%%%%%%%%%%%%%%%%%%%%%%%%%%%%%%%%%%%%%%%%%%%%%%%%%%%%

%%
%%  本文はここから
%%

\chapter{はじめに}

現代のソフトウェア開発はソフトウェア部品の組み合わせで行われる\cite{liblary-reuse}.ソフトウェア部品は再利用可能な形でまとめられ,ライブラリと呼ばれる.ライブラリの利用によって,ソフトウェア開発者はソフトウェアの開発工程を大幅に削減することができる\cite{reduce-productivity}.ライブラリ開発者は,さらに上流のライブラリを使用することもあり,ソフトウェアエコシステム\cite{software-wcosystem}を形成する.

一方で,ライブラリを利用してソフトウェア開発を行うことのデメリットも存在する.ライブラリ自体も機能追加やバグ修正を行いながら成長するため,破壊的変更を引き起こす可能性がある.「ライブラリが外部へ公開しているAPIの名称変更」や「引数の変更」など,様々な要因によって,破壊的変更が引き起こされる\cite{detect-backward-compatibility}.中でも,ライブラリAPIの削除はクライアントに甚大な影響を及ぼす.ライブラリAPIが削除されると,クライアントは,代替APIへ切り替える必要がある.しかし,代替APIが存在しない場合は自前で実装するか,機能自体を廃止するかの選択を迫られる.

APIの削除という大きな変更に関しては,急に行うと,ソフトウェアエコシステム全体に負荷がかかる.したがって,多くのライブラリでは,セマンティックバージョニング\footnote{\url{https://semver.org}}という戦略を採用している.セマンティックバージョニングにしたがっているライブラリは一般的にメジャーバージョンのリリース時のみで破壊的変更を加える.この仕組みによって,クライアントはメジャーバージョンのアップデート時には,破壊的変更の対策を取ることができる.また,APIの削除は急に行われるわけではなく,削除よりも一定期間前から非推奨という警告を出し,クライアントが十分に対策を行える猶予を作った上で,削除を行うという仕組みが備わっている.

一般に,ソフトウェアには,不要と考えられるソースコードが蓄積されている.一方,開発者が全ての不要コードを全て見つけ出すことは難しい.なぜなら不要コードには,様々な種類があるからである.

一つ目は,ソースコード中から不要コードを見つけ出す手法である.例えば,プログラム中で到達することができないプログラムはデッドコードと呼ばれ,不要コードと考えられる.デッドコードに関しては,静的解析と動的解析を組み合わせることでおおよそ検出ができる.静的解析では,動的な依存関係を取得することが難しいため,見逃しが発生してしまう.また,動的解析では,エッジケースの入力や年に1回だけ実行されるようなケースを誤って陽性と判断してしまう.このように,デッドコードの検出は,まだまだ課題が残っている.

二つ目は,ソースコードの変更履歴における削除に着目する方法である.変更履歴において,削除されたコードは不要であると考える.変更履歴から削除されたコードを収集し,共通する特徴を抽出するという方法である.しかし,この分析方法では,ノイズが多く混ざってしまう.例えば,リファクタリングの一環として開発者がソースコードを移動した場合,履歴上では,削除と追加という変更として記録されてしまう.したがって,本質的な削除のみを抽出する手法が必要である.
そこで,RefactoringMiner\cite{Alikhanifard:TOSEM:2024:RefactoringMiner3.0}のようなリファクタリング検出手法によりノイズを減らす工夫がされている.しかし,RefactoringMinerは全ての変更コミットを探索するため,大規模なOSSで適用するには非常に計算コストがかかってしまう.また,リファクタリングが複数コミットに渡る場合,RefactoringMinerでも検出ができなくなる.

そこで,本研究では,バージョンごとにソフトウェア中に存在するメソッドを全て抽出し,バージョン間で同じシグネチャのメソッドが存在せず,かつ,類似するメソッドが存在しなかった場合に削除されたメソッドであると判定する.バージョンごとのコミットだけに着目することで計算コストを削減し,類似メソッドで削除を判定することで,複数コミットにわたるリファクタリングも検出できるようになる.さらに,一時的に追加されたコードクローンが削除されるケースも発生するため,コードクローンに関連するケースについても検出する.


以降,本論文では,\ref{chap:deletecode}章で不要コードの種類と不要コードが与える影響,関連研究と本研究の位置付けを述べる.\ref{chap:pre-analysis}章では,本研究の動機となる事前分析を述べ,\ref{chap:RQ1}章,\ref{chap:RQ2}章では,設定したRQにおけるそれぞれの分析手法,結果,考察を述べる.続く\ref{chap:thread-to-validity}章では,本研究の妥当性の脅威を述べる.最後に\ref{chap:conclusion}章で本論文をまとめる.

\chapter{不要コードの削除}
\label{chap:deletecode}

\section{不要コードの種類}
ソフトウェアにとっての不要コードはステークホルダやタイミングによって異なる.クライアントにとって必要な機能であってもライブラリ開発者にとっては,保持したくないソースコードもある.例えば,後方互換性を維持するためのソースコードはライブラリにとっては保持したくない不要コードであるが,古いバージョンを使用するクライアントにとっては必要である.他にも,頻繁に使用されるライブラリのAPIは必要であるが,より効率的でクリーンな実装の代替APIが登場すれば,現状のAPIは価値が落ち,不要とみなされることがある.その中でも普遍的に不要であると判断されるソースコードは存在し,活発に研究で議論されてきた.表~\ref{tab:unnecessary-code}に不要コードの分類を示す.
デッドコードや未実行コードに関しては,定義しやすいため,比較的観測しやすい.一方,不要コードは定義自体に曖昧さを伴うため,観測することは比較的難しい.
以後,単に不要コードと記述する場合,これらのコードを統括的に含んだ意味を示すものとする.

\begin{table}[ht]
    \centering
    \caption{不要コードの分類}
    \begin{tabular}{l|l|l} \hline
        種類 & 定義 & 関連研究 \\ \hline
        デッドコード & プログラム中で到達不可能なソースコード & \cite{static-analysis}\cite{dynamic-analysis}\cite{decrease-readability} \\
        未実行コード & 観測期間において,実行されないソースコード & \cite{static-analysis}\cite{increase-maintainability} \\
        不要コード & システムの目的を果たすために必要ではないソースコード & \cite{static-analysis}\cite{increase-maintainability} \\
        削除コード & 削除されたソースコード \\ \hline
    \end{tabular}
    \label{tab:unnecessary-code}
\end{table}

\section{不要コードが引き起こす問題}
不要コードがコードベースに蓄積することは様々な問題を引き起こす.Sebastianら\cite{increase-maintainability}の研究では,不要コードは保守コストの増加を引き起こすと言われている.ソースコードの修正時に,未使用のソースコードも同時に修正する必要があるため,無駄なコストを作り出してしまう.また,Simoneら\cite{decrease-readability}の研究では,コードベースを理解する際に,不要コードを読んでしまうことで,ソースコード理解の有効性や効率を下げてしまうと言われている.
したがって,不要コードは検出し,削除することで,コードベースから取り除く必要がある.

\section{関連研究}

\subsection{不要コードの検出}

観測期間において,実行されていないソースコードを検出し,自動で削除する研究が提案されている\cite{dynamic-analysis}.ソフトウェア中のさまざまなコンポーネントにログを組み込み,実環境で一定期間運用することで実行状況のログを取得する.実行状況のログの中で実行履歴がないコンポーネントを不要コードと判定する.
しかし,ソフトウェアの使用状況に基づく不要コードの検出では,偽陽性が発生する.例えば,エラー処理のソースコードや復旧用のソースコード,移行のためのソースコードは観測期間において,実行されなければ,不要コードであると検出される.このようなソースコードは実行されていなくとも,他のコンポーネントとの依存関係があるので,削除してはいけない.したがって,実行状況の監視と共に依存関係の取得も必要である.

そこで,依存関係に着目し,他との依存関係を持たないデッドコードを静的に検出する手法が提案されている\cite{detect-dead-code}.\todo{デッドコードの検出手法の説明}しかし,リフレクションのように動的に依存関係が決まる場合,偽陽性を発生させてしまう.

これらは普遍的な不要コードを検出することを目的としている.本研究では,普遍的な不要コードよりも広い概念での不要コードの特徴を調査することを目的としている.


\subsection{削除コードの実証的研究}
削除ファイルの特徴を分析した研究では,「最近変更されておらず,依存の中心にいない特徴を持つファイル」が「生存するファイル」に比べ,有意に差があるということを発見した\cite{static-analysis}.

テストコードの削除に着目した研究\cite{study-of-test-deletion}では,削除されたテストメソッドの83.2\%はテストクラスごと削除され,7\%は冗長なテストであった.さらに,削除されたテストメソッドの91.4\%はプロダクションコードの廃止による副次的な削除であることがわかった.

Ashokら\cite{why-deprecated}は,メソッドが非推奨になる理由をコミット履歴やイシュートラッカーなどを基に調査,分類した.その結果,「新機能の導入」,「機能上の欠陥」,「デザインパターン」の3種類で全体の約72\%を占めていた.

\subsection{自動削除の研究}
\todo{Meta\cite{dynamic-analysis}やUberの自動削除の研究}\memo{書くか迷っている}


\section{本研究の位置付け}
既存研究は主に,デッドコードや未実行コードといった「狭義の不要コード」の自動検出に特化してきた.しかし,静的解析によるデッドコードの検出では,動的依存関係の把握が困難であり,さらに,動的解析では,動作期間に依存するという偽陽性の問題を抱えている.これに対し,本研究では,以下の独自性を有する.

\vspace{5pt}
\noindent\emph{開発者の判断に基づく削除履歴への着目}
\vspace{5pt}

既存の不要コード検出手法での課題を回避するため,本研究では,「開発者が実際に削除した」という確定した事実をGround Truthとし,分析対象とする.これにより,狭義の不要コードだけでなく,広義の不要コードの削除実態を網羅的に調査することができる.

\vspace{5pt}
\noindent\emph{コードクローンを用いたメソッドの比較}
\vspace{5pt}

単なる削除メソッドの分析ではなく,コードクローンを持つメソッドを分析上,切り分ける.これによって,コードクローンという冗長な実装がある中で,削除されたメソッドと生存したメソッドを抽出することができる.これによって,メソッドを相対的に比較し,どちらのメソッドの品質が高いかを議論することを可能にする.

\chapter{事前分析:削除されたコードの特徴}
\label{chap:pre-analysis}

\section{概要}
本章では,削除されたメソッドの実態を目視で調査し,削除の動機を明らかにすることである.特に,削除されたメソッドが狭義の不要コードだけであるのか,それとも,広義の不要コードが原因で削除を行うのかを分析する.

\section{分析手法}
対象プロジェクトの変更履歴から削除されたメソッドを無作為に抽出し,以下の観点で目視調査を行う.

\vspace{5pt}
\noindent\emph{依存関係の有無}

削除される直前のバージョンにおいて,当該メソッドが他のプログラムから呼び出されていたかどうかを調査する.

\vspace{5pt}
\noindent\emph{削除理由の分析}

削除時のコミットメッセージや関連するIssueを確認し,削除理由を分類する.

\section{データセット}
GitHub\footnote{\url{https://github.com}}上でソースコードが公開されているpandas\footnote{\url{https://github.com/pandas-dev/pandas}}プロジェクトを使用した.プロジェクトの選定方法は以下である.
Pythonのプロジェクトであること.大規模なライブラリであり,ユーザが多数存在すること.開発年数が長期であること.メジャーリリースが2つ以上存在すること.

\section{分析結果}
v0.25.0における\texttt{pandas/io/common.py}内のメソッド\texttt{\textunderscore stringify\textunderscore path}はioに関するプライベートメソッドである.これは削除直前までは,他のファイルから呼び出されていた.しかし,0df8858752のコミットにおいて,削除されている\footnote{\url{https://github.com/pandas-dev/pandas/commit/0df8858752}}.このコミットに関しては,Pull Requestは見つけられたが,Issueは見つかっていないため,削除理由に関しては不明である.しかし,Pull Requestのタイトルで,io関連のメソッドを非公開化する趣旨の内容が記載されていた.


\section{事前分析から得られた示唆と研究課題}
事前分析から次の知見が得られた.まず,削除の動機は狭義の不要コードだけに留まらないということ.デッドコードであるから削除するといった理由だけでなく,デッドコードでなくともメソッドが削除される例が存在することがわかった.

以上の示唆に基づき,本研究では,以下のリサーチクエスチョン(Research Questions)を設定する.
\begin{itemize}
    \item RQ1: クローンセットの有無はメソッドの生存と削除にどのような影響を与えるか
    \item RQ2: メソッドの統合や分割に着目することでコードベースから完全に削除されたメソッドを特定できるか
\end{itemize}


\chapter{RQ1:クローンセットの生存と削除にはどのような特徴があるか}
\label{chap:RQ1}
% \chapter{RQ1: どのようなメソッドが削除されるか}

\section{概要}

本章では,OSSプロジェクトの進化過程におけるメソッドの削除要因を明らかにするため,ソースコードの変遷を追跡し,クローンセットとの関係性に基づいた分類を行う.手順1から3は,対象プロジェクトの特定バージョンからメソッドを抽出し,抽象構文木を用いてシグネチャとボディを取得する.手順4で,隣接するバージョン間で同一メソッドを紐づける.手順5でメソッドの削除時のクローンセットの存続状況を確認することで,削除の要因をパターン化する.

\begin{figure}[t]
    \centering
    \includegraphics[width=0.75\linewidth]{@Bachelor2025_Yoshida/Yoshiba_fig/approach.pdf}
    \caption{メソッドの分類}
    \label{fig:overview-of-approach}
\end{figure}

\section{分析手法}

\noindent\emph{1. 解析対象バージョンの選定}

対象とするOSSのGitHub\footnote{\url{https://github.com/}}リポジトリから,タグを含むコミットを全て取得する.タグを含むコミットからセマンティックバージョニングを使用しているコミットのみをフィルタリングする.
さらに,セマンティックバージョニングの末尾に「rc」や「dev」といったプレリリースが含まれるタグについては,正規表現で排除する.
また,本研究では,メジャーおよびマイナーバージョンの変遷に着目するため,パッチバージョンが「0」であるタグ(例:「v1.2.0」)のみを対象とする.
これらの操作によって,マイナーバージョン,メジャーバージョンのタグとそのコミットのみを時系列順にK個抽出する.

\noindent\emph{2. メソッドのシグネチャとボディの取得}

取得したバージョンへ順次チェックアウトを行い,当該バージョンに含まれる全てのPythonファイル(.py)を対象に抽象構文木(Abstract Syntax Tree)へパースする(以降,ASTとする).このASTからメソッドのシグネチャを取得し,一意のメソッド名を構築する(「ファイルパス,クラス名.メソッド名,引数,戻り値」という形式).
同時にメソッドのボディもASTから取得しておき,トークンで分割し,リストとして保存する.

\noindent\emph{3. メソッドのN-gram化}

バージョン $k$ ($1 \le k < K$) における $M$ 個のメソッドの集合を $V_{k} = \{c_1, c_2, \dots, c_M\}$ とする.
各メソッド $c_m \in V_k$ に対して N-gram 化を行い,抽出された N-gram の集合を $G(c_m)$ と表す.
本研究では,従来研究と同様に 5-gram を採用する.

\noindent\emph{A. バージョン間でのシグネチャの一致}

隣接するバージョン$V_{k}$と$V_{k+1}$の間でシグネチャが完全に一致するメソッドを同一と見なし,共通のIDを割り当てる.

\noindent\emph{B. バージョン間での類似度の計算}

コードクローン検出器 NIL を使用して,バージョン間でのクローンメソッドを検出する.
$V_{k}$ における各メソッド $c_i$ に対し,$V_{k+1}$ における全てのメソッド $c_j$ との類似度を計測する.
この際,類似度が閾値を超えたメソッドを候補メソッドとする.従来研究と同様に類似度の閾値は 0.7 を使用する.
$V_{k}$ における特定のメソッドに対し,$V_{k+1}$ における複数のメソッドが閾値を超えた場合は,最大の類似度をもつメソッドを候補メソッドとし,同一IDを割り当てる.

NIL による詳細な類似度計算(検証フェーズ)では,以下の式を用いる.
\begin{equation}
    \text{Sim}_{\text{verif}}(c_i, c_j) = \frac{\text{LCS}(c_i, c_j)}{\min(|c_i|, |c_j|)}
\end{equation}
ここで,
\begin{description}
    \item[$c_i, c_j$:] 比較対象となる2つのメソッドのトークン列
    \item[$|c_i|, |c_j|$:] 各トークン列の長さ
    \item[$\text{LCS}(c_i, c_j)$:] 2つのトークン列間で、順序を保ったまま一致している最長共通部分列(Longest Common Subsequence)の長さ
\end{description}

本手法の特徴は,分母に最小値($\min$)を用いている点にある.
これにより,一方がもう一方を包含している場合に類似度が高く算出される.
例えば,片方のメソッドに大量の新しいコードが追加されて全体の長さが大きく変化した場合でも,元のロジックが保存されていればクローンとして検出可能である.

しかし,LCS による類似度計算は計算コストが非常に高いため,以下の式を用いたフィルタリングフェーズを導入し,明らかに異なるメソッド対を事前に除外する.
\begin{equation}
    \text{Sim}_{\text{filt}}(c_i, c_j) = \frac{|G(c_i) \cap G(c_j)|}{\min(|G(c_i)|, |G(c_j)|)}
\end{equation}
ここで,$G(c)$ はメソッド $c$ の N-gram 集合を表し,$|G(c_i) \cap G(c_j)|$ は $c_i$ と $c_j$ で共通する N-gram の要素数を表す.
このフィルタリングにより,類似度が 0.1 未満のメソッド対に対する LCS 計算を省略し,全体の実行時間を削減している.

\noindent\emph{4. 同一メソッドの追跡}

まず,Aを用いて,同一メソッドを判定する.このとき,メソッドの中身やロジックは同一でも,リネームや,ファイルの移動を受けたメソッドは同一メソッドと判定できない.
したがって,Aによる同一メソッドの判定後に残ったメソッドについて,Bを用いることで同一メソッドを判定する.これによって,リネームやファイル移動したメソッドでも追跡することができる.
このIDの変遷を比較することで,メソッドを以下の3つに分類する.$V_{k+1}$で初めて出現したIDを持つメソッドを「追加メソッド」,$V_{k}$と$V_{k+1}$の両方に存在するIDを「生存メソッド」,$V_{k}$のみに存在するIDを「削除メソッド」とする.

\noindent\emph{5. メソッドの変遷パターンの分類}

本研究では,削除メソッドが「機能の完全な削除」によるものか,あるいは,「冗長なコードの整理」によるものかを区別するため,削除メソッドをさらに詳細に分類する.

まず,バージョン$V_{k}$内において,3-Bと同様の手法を用いてメソッド間のコードクローン検出を行う.コードクローンは通常,2つのメソッドのペアとして検出されるため,これらのペアに対して推移律(メソッドaとb,bとcがクローンであれば,aとcも同一グループとする)を適用し,クローン集合(以降,クローンセットとする)を構築する.
次に,各クローンセットに属するメソッドの$V_{k+1}$における生存状況を確認し,以下の3つに分類する.

\begin{description}
    \item[全削除: ] あるクローンセットに含まれる全てのメソッドが,$V_{k+1}$において,「削除メソッド」と判定された場合.これは,クローンセットが提供していた機能自体がプロジェクトから消滅したことを意味する.
\end{description}

\begin{description}
    \item[部分削除: ] あるクローンセットに含まれるメソッドのうち,一部が$V_{k+1}$ で「削除メソッド」となり,少なくとも1つ以上のメソッドが「生存メソッド」として残っている場合.この場合,削除されたメソッドは「部分削除」,生存したメソッドは「部分生存」と定義する.これは,重複していたコードが統合されたり,特定の箇所のみが不要になったりする「リファクタリング」に近い挙動を示していると考えられる. 
\end{description}

\begin{description}
    \item[独立削除: ] バージョン$V_{k}$において,他のどのメソッドともクローン関係にない(すなわち,どのクローンセットにも属さない)メソッドが $V_{k+1}$ で削除された場合.これは,そのメソッドが担っていた固有の機能がプロジェクトから取り除かれたことを示す. 
\end{description}


\section{分析結果}
\subsection{メソッドの生存実態}

表~\ref{tab:pattern-method}は,各バージョンにいて,追加されたメソッドの数,削除されたメソッドの数(クローンセット内のすべてのメソッドが削除,クローンセット内の一部のメソッドが削除,独立してメソッドが削除),生存したメソッドの数をそれぞれ表している.

\begin{table}[ht]
    \centering
    \caption{メソッドの変遷パターン}
    \begin{tabular}{l|rrrrrrr} \hline
    & & \multicolumn{3}{c}{削除} & \\ \cline{3-5}
    バージョン & 追加 & 全削除 & 部分削除 & 独立削除 & 生存 \\ \hline \hline
    v0.4.0 & 1303 & 0 & 0 & 0 & 0 \\
    v0.5.0 & 281 & 7 & 0 & 18 & 1272 \\
    v0.6.0 & 169 & 4 & 0 & 12 & 1529 \\
    v0.7.0 & 606 & 71 & 0 & 167 & 1457 \\
    v0.8.0 & 908 & 100 & 0 & 145 & 1809 \\
    v0.9.0 & 288 & 20 & 0 & 9 & 2670 \\
    v0.10.0 & 428 & 51 & 0 & 138 & 2750 \\
    v0.11.0 & 376 & 59 & 6 & 144 & 2949 \\
    v0.12.0 & 414 & 12 & 0 & 17 & 3269 \\
    v0.13.0 & 2380 & 63 & 0 & 57 & 3542 \\
    v0.14.0 & 806 & 51 & 5 & 113 & 5706 \\
    v0.15.0 & 909 & 71 & 1 & 67 & 6325 \\
    v0.16.0 & 455 & 36 & 5 & 32 & 7113 \\
    v0.17.0 & 1217 & 49 & 4 & 57 & 7410 \\
    v0.18.0 & 743 & 68 & 4 & 79 & 8434 \\
    v0.19.0 & 1476 & 168 & 3 & 206 & 8766 \\
    v0.20.0 & 1276 & 503 & 11 & 542 & 9150 \\
    v0.21.0 & 670 & 56 & 4 & 67 & 10255 \\
    v0.22.0 & 129 & 1 & 1 & 1 & 10876 \\
    v0.23.0 & 1725 & 478 & 14 & 430 & 10044 \\
    v0.24.0 & 2593 & 504 & 12 & 601 & 10612 \\
    v0.25.0 & 1111 & 326 & 6 & 473 & 12372 \\ \hline
    v1.0.0 & 1284 & 496 & 10 & 522 & 12430 \\
    v1.1.0 & 1407 & 196 & 12 & 280 & 13198 \\
    v1.2.0 & 1544 & 279 & 7 & 217 & 14064 \\
    v1.3.0 & 2067 & 294 & 12 & 285 & 14984 \\
    v1.4.0 & 1477 & 260 & 15 & 170 & 16568 \\
    v1.5.0 & 1423 & 104 & 7 & 87 & 17788 \\ \hline
    v2.0.0 & 1376 & 345 & 17 & 416 & 18270 \\
    v2.1.0 & 1398 & 197 & 11 & 147 & 19153 \\
    v2.2.0 & 1105 & 128 & 6 & 166 & 20123 \\
    v2.3.0 & 319 & 16 & 3 & 15 & 21045 \\ \hline
    合計 & 34968 & 5313 & 184 & 5304 & 275734 \\ \hline
    \end{tabular}
    \label{tab:pattern-method}
\end{table}

観測開始地点のバージョン0.4.0は,プロジェクトの初期なので,多くのメソッドが追加されている.メソッドは基本的には,追加が続けられており,全体の総数としては上昇する.例えば,v1.0.0のリリース時は,グループの全削除が496件,独立削除が522件起こっている.さらに,v2.0.0のメジャーバージョンのリリース時にも全削除が345件,独立削除が416件起こっている.これは,メジャーバージョンの1においては,最大の削除数となっている.
このことから,メジャーバージョンのリリース時には,多くのメソッドが削除されることがわかる.


% \subsection{メソッドの生存期間}
% \begin{figure*}[t]
%     \centering
% \includegraphics[width=1.0\textwidth]{./Yoshida_fig/state_plot.pdf}
%     \caption{メソッドの生存期間}
%     \label{fig:result}
% \end{figure*}
% メソッドごとの生存期間は図4.1に示す.\todo{この図は差し替える}

\subsection{クローンセットの部分削除}


pandasプロジェクトにおいて,v0.19.0がリリースされた時点でのメソッドの例をListing~\ref{lis:include-deprecated-API}とListing~\ref{lis:include-inplace-API}に示す.これら二つのメソッドはコードクローン検出器によると類似度\todo{〇〇\%}となっており,コードクローンタイプ3である.どちらもPanelオブジェクトを生成しているという点では類似している.


%----------------------------------
\vspace{10pt}
\begin{lstlisting}[caption=非推奨な記述方法を含むメソッド(次バージョンで削除), label=lis:include-deprecated-API, captionpos=t, columns=flexible]
def test_backwards_compat_without_term_object(self):
    with ensure_clean_store(self.path) as store:

        wp = Panel(np.random.randn(2, 5, 4), items=['Item1', 'Item2'],
                   major_axis=date_range('1/1/2000', periods=5),
                   minor_axis=['A', 'B', 'C', 'D'])
        store.append('wp', wp)
        with assert_produces_warning(expected_warning=FutureWarning,
                                     check_stacklevel=False):
            result = store.select('wp', [('major_axis>20000102'),
                                         ('minor_axis', '=', ['A', 'B'])])
        expected = wp.loc[:,
                          wp.major_axis > Timestamp('20000102'),
                          ['A', 'B']]
        assert_panel_equal(result, expected)
\end{lstlisting}
\vspace{10pt}
%----------------------------------
Listing~\ref{lis:include-deprecated-API}は,古い形式のクエリの書き方が後方互換性を維持し,かつ,正しく動作するかを検証するメソッドである.
コード内で\texttt{wp = Panel(...)}とある.ここでは,条件を\texttt{('カラム名','演算子','値')}というタプルのリストで渡している.そして,\texttt{with assert\textunderscore produces\textunderscore warning(FutureWarning)}を使って,「この記述方法が将来的に消える」という警告が出ることをテストしている.


%----------------------------------
\vspace{10pt}
\begin{lstlisting}[caption=推奨APIを含むメソッド(次バージョンで生存), label=lis:include-inplace-API, captionpos=t, columns=flexible]
def test_panel_assignment(self):
    # GH3777
    wp = Panel(
        randn(2, 5, 4), items=['Item1', 'Item2'],
        major_axis=date_range('1/1/2000', periods=5),
        minor_axis=['A', 'B', 'C', 'D'])
    wp2 = Panel(
        randn(2, 5, 4), items=['Item1', 'Item2'],
        major_axis=date_range('1/1/2000', periods=5),
        minor_axis=['A', 'B', 'C', 'D'])

    # TODO: unused?
    # expected = wp.loc[['Item1', 'Item2'], :, ['A', 'B']]

    def f():
        wp.loc[['Item1', 'Item2'], :, ['A', 'B']] = wp2.loc[
            ['Item1', 'Item2'], :, ['A', 'B']]

    self.assertRaises(NotImplementedError, f)
\end{lstlisting}
\vspace{10pt}
%----------------------------------

Listing~\ref{lis:include-inplace-API}は,まだ実装されていない操作に対して,適切にエラーを発生させているかを検証するテストメソッドである.
実行したい処理を内部関数\texttt{f()}の中で,\texttt{wp.loc[...] = wp2.loc[...]}という代入処理を記述している.そして,\texttt{self.assertRaises(NotImplementedError, f)}により関数\texttt{f()}を実行した際に,未実装のエラーが発生することをテストしている.

Listing~\ref{lis:include-deprecated-API}における\texttt{test\textunderscore backwards\textunderscore compat\textunderscore without\textunderscore term\textunderscore object}メソッドは,v0.20.0リリース時には削除された.つまり,v0.19.0の時点では必要であったメソッドがv0.20.0の時点では不要であったため,削除されたと言える.
一方,Listing~\ref{lis:include-inplace-API}における\texttt{test\textunderscore panel\textunderscore assignment}メソッドはv0.20.0の時点でも削除されずに生存していたため,v0.20.0の時点では不要ではなく,必要なメソッドであると考えられる.その後,メジャーバージョンであるv1.0.0リリース直前のv0.25.0において不要となり,削除された.

この結果から,v0.20.0では,\texttt{store.select}メソッドの古い形式の記述方法のサポートが終了したため,それを利用している側のコードが削除されたと考えられる.さらに,v0.25.0においては,\texttt{Panel}オブジェクトが削除されたため,それを呼び出しているメソッドも削除されたと考えられる.


\subsection{クローンセットの全てのメソッドが削除}
v0.18.0がリリースされた時点において,クローンセットの全てのメソッドが削除された事例をListing~\ref{lis:test-get-options-data}とListing~\ref{lis:test-get-data-with-list}に示す.

\vspace{10pt}
\begin{lstlisting}[caption=v0.18.0における取引機能のテストメソッド(次バージョンで削除), label=lis:test-get-options-data, captionpos=t, columns=flexible]
@network
def test_get_options_data(self):
    # regression test GH6105
    self.assertRaises(ValueError, self.aapl.get_options_data, month=3)
    self.assertRaises(ValueError, self.aapl.get_options_data, year=1992)

    try:
        options = self.aapl.get_options_data(expiry=self.expiry)
    except RemoteDataError as e:
        raise nose.SkipTest(e)
    self.assertTrue(len(options) > 1)
    
\end{lstlisting}
\vspace{10pt}

Listing~\ref{lis:test-get-options-data}はオプション取引データの取得機能が正しく動作するかを検証するためのメソッドである.まず,\texttt{self.assertRaises(ValueError, ...)}によって,\texttt{self.aapl.get\textunderscore options\textunderscore data}メソッドに不正な引数を指定した場合に,正しくエラーが発生することを確認している.さらに,その後の\texttt{try...except...}句内で実際にデータを取得し,\texttt{self.assertTrue}によって,取得したデータが1以上であることを確認している.

\vspace{10pt}
\begin{lstlisting}[caption=v0.18.0におけるデータ一括取得のテストメソッド(次バージョンで削除), label=lis:test-get-data-with-list, captionpos=t, columns=flexible]
@network
def test_get_data_with_list(self):
    try:
        data = self.aapl.get_call_data(expiry=self.aapl.expiry_dates)
    except RemoteDataError as e:
        raise nose.SkipTest(e)
    self.assertTrue(len(data) > 1)
    
\end{lstlisting}
\vspace{10pt}

Listing~\ref{lis:test-get-data-with-list}は,複数のリストをまとめて指定し,データを一括取得できるかを検証するテストコードである.\texttt{self.appl.get\textunderscore call\textunderscore data(expiry=self.aapl.expiry\textunderscore dates)}によって,データのリストを引数にいれ,データを取得する.そして,\texttt{try...except...}句内で実際にデータを取得し,\texttt{self.assertTrue}によって,取得したデータが1以上であることを確認している.

Listing~\ref{lis:test-get-options-data}とListing~\ref{lis:test-get-data-with-list}は互いに,クローンと判定されたメソッドであり,類似度は\todo{X}\%である.これらは,v0.19.0で削除された.これらのメソッドのパスは\texttt{/pandas/io/tests/test\textunderscore data.py}に存在している.このパスに注目すると,v0.18.0においては,存在しているが,v0.19.0においては存在していなかった.つまり,これらのクローンメソッド群はファイル自体が削除されたことによって,同時にクローンセットごと削除されたと判定されたことがわかる.

次に,クローンセットの全てが削除された別の事例をListing~\ref{lis:median}とListing~\ref{lis:f}に示す.

\vspace{10pt}
\begin{lstlisting}[caption=v0.6.0における上位メソッド(次バージョンで削除), label=lis:median, captionpos=t, columns=flexible]
def median(self, axis='major', skipna=True):
    def f(arr):
        mask = common.notnull(arr)
        if skipna:
            return _tseries.median(arr[mask])
        else:
            if not mask.all():
                return np.nan
            return _tseries.median(arr)
    return self.apply(f, axis=axis)
    
\end{lstlisting}
\vspace{10pt}

Listing~\ref{lis:median}は,内部に関数を定義し,それを\texttt{apply}メソッド用いることで,指定した軸に適用している.

\vspace{10pt}
\begin{lstlisting}[caption=v0.6.0における下位メソッド(次バージョンで削除), label=lis:f, captionpos=t, columns=flexible]
def f(arr):
    mask = common.notnull(arr)
    if skipna:
        return _tseries.median(arr[mask])
    else:
        if not mask.all():
            return np.nan
        return _tseries.median(arr)
    
\end{lstlisting}
\vspace{10pt}

Listing~\ref{lis:f}は,Listing~\ref{lis:median}の内部で定義されている関数である.つまり,関数の内部で関数が定義されている場合,それらの構造は類似しているとコードクローン検出器では,判定される.したがって,上位の関数が削除された場合,下位の関数も同時に削除される.

これらの結果より,クローンと判定されたメソッド群が同時に削除される事例は上位概念の削除による副次的作用であることがわかる.

さらに,追加分析として,ファイル削除に伴うクローンセットの削除について分析した.プロジェクト上で削除されたことのあるファイルパスを収集し,クローンセットの削除が行われたメソッドが存在しているファイルパスと対応させた.その結果,33\% \memo{多少数字は変わるかも}ほどのクローンセットごと削除されたメソッドは,上位のファイルが削除されたことによる副次的作用であることが判明した.

\section{考察}

削除されたメソッドの中でもクローンを持つメソッドとクローンを持たないメソッドでは,異なる特徴を持つことがわかる.
クローンセットの一部が削除されるという事例から,機能の移行期間では,一時的に類似機能を作成し,完全な移行が完了した時点で古いメソッドを削除するというパターンがあることがわかった.
クローンセットが同時に削除されるという事例から,削除パターンの一つとして,上位概念の削除による副次的削除が含まれることがわかる.\todo{クローンを持たずに削除されたメソッドの事例も述べるべきか?}

クローンセットの一部が削除されるということは,削除されたメソッドは,より品質の低いメソッドであり,生存したメソッドは,より品質の高いメソッドであると考えられる.このような品質の低いメソッドを学習し,同じような特徴を持つメソッドを検出できれば,ソフトウェアの品質を向上させられる可能性がある.

%%%%%%%%%%%%%%%%%%%%%%%%
\chapter{RQ2:メソッドの統合や分割はコードクローンと関連しているか}
\label{chap:RQ2}
% \chapter{RQ2: メソッドが削除される前にどのような変更が行われているか}
%%%%%%%%%%%%%%%%%%%%%%%%

\begin{figure}[t]
    \centering
    \includegraphics[width=1.0\linewidth]{@Bachelor2025_Yoshida/Yoshiba_fig/method_group.pdf}
    \caption{メソッドの変更タイプの分類}
    \label{fig:merge-clone-method}
\end{figure}

\section{概要}
第 \ref{chap:RQ1} 章で提案した手法では,バージョン間でのメソッド追跡を「シグネチャの一致」および「メソッド単体間の類似度」に基づいて行っていた.しかし,実際のソフトウェア開発では,リファクタリングに伴い,図 \ref{fig:merge-clone-method} に示すような複雑な変遷が発生する.具体的には,以下のケースが考慮されていないという課題があった.

\begin{description}
    \item[意味的な変更の見逃し: ] メソッド名が同一であっても,その内部実装が大幅に刷新され,実質的に別の機能へ置き換わっている場合.
    \item[メソッドの分割: ] 一つのメソッドが担っていた機能が分割され,複数の小さなメソッドへと分散する場合.
    \item[メソッドの統合: ] 複数のメソッドが一つにまとめられ,より大きな機能を持つメソッドへと変化する場合. 
\end{description}

これらのケースを無視すると,実際には形を変えて生存しているメソッドが「削除」と誤判定される恐れがある.そこで本章では,LCS(最長共通部分列)を用いた類似度計算の特徴を活用し,メソッドの統合および分割を検出する手法を導入する.

\section{分析手法}

\noindent\emph{4. 同一メソッドの追跡}

第~\ref{chap:RQ1} 章の手順1から3は同様に実施するが,手順4において,シグネチャの一致に頼らず,3-B で定義した類似度指標のみを用いた柔軟な追跡を行う.これにより,リネームを伴う変更も追跡対象とする.

\noindent\emph{5. メソッドの分割,統合の検出}

LCS を用いた類似度計算は,2つのトークン列間で順序を保ったまま一致している部分を抽出する.このとき,比較対象間のトークン数の差異に着目することで,以下のルールに基づき統合と分割を判定する.

\begin{description}
    \item[分割: ] バージョン$V_{k}$におけるメソッド$c_{i}$と,バージョン$V_{k+1}$におけるメソッド$c_{i+1}$の類似度が閾値を超える.かつ,トークン数において,$0.7c_{i} > c_{i+1}$を満たす場合,メソッド$c_{i}$は$c_{i+1}$を含む複数のメソッドへ分割された可能性があると判定する. 
    \item[統合: ]  バージョン$V_{k}$におけるメソッド$c_{i}$と,バージョン$V_{k+1}$におけるメソッド$c_{i+1}$の類似度が閾値を超える.かつ,トークン数において,$c_{i} < 0.7c_{i+1}$を満たす場合,メソッド$c_{i}$は他のコードを取り込み,$c_{i+1}$へと統合された可能性があると判定する.
    \item[変更: ] バージョン$V_{k}$におけるメソッド$c_{i}$と,バージョン$V_{k+1}$におけるメソッド$c_{i+1}$の類似度が閾値を超える.かつ,トークン数において上記のどちらの条件も満たさなかった場合,単純に変更されたメソッドであると判定する.
\end{description}


\section{結果}

\begin{figure}[ht]
    \centering
    \includegraphics[width=1.0\linewidth]{@IPSJ_SIGSE202603_Yoshida/Yoshida_fig/boxplots_final.pdf}
    \caption{メソッドのバージョン間での分類タイプの件数を表す箱ひげ図}
    \label{fig:method-classification}
\end{figure}

図~\ref{fig:method-classification}は左から順に,追加,削除,分割,統合,変更されたメソッドがバージョン間で何件存在するかを箱ひげ図で示している.青い箱ひげ図はクローンを持つメソッドであり,赤い箱ひげ図はクローンを持たないメソッドである.
バージョン間では当然,生存(分割,統合,変更)がもっとも多い結果となっている.生存に次いで,追加が多く,100件以下で行われることは滅多にない.つまり,バージョン間でメソッドが追加される時は100件以上の単位で大規模に行われるということがわかる.削除に関しても,100件以上と大規模に行われるものの,100件以下の少数で行われることもある.メジャーバージョンのリリース直前のマイナーバージョンの更新においては,削除の件数が劇的に増加する.一方,メジャーバージョンのリリース後や,離れた時期においては,削除の件数は小さい.

それぞれの分類タイプにおけるクローンの有無を調べた結果が表~\ref{tab:has-clone}である.分割や統合を起こしたメソッドのうち,97\%以上がクローンを持っている.つまり,分割や統合が起こるための必要条件として,クローンの有無が関係していることが考えられる.

一方,削除されたメソッドは約37\%がクローンを持つ.この理由として,クローンを持つメソッドの場合,削除時の影響範囲が拡大するため,メソッドはクローンを持たない方が削除されやすいという結果になったと考えられる.

変更において,クローンを持つ割合が約64\%ほどあるが,これらは将来的に,分割や統合に向かうと考えられる.

\begin{table}[ht]
    \centering
    \caption{メソッドの状態分類におけるクローンの有無}
    \begin{tabular}{c|ccccc}
        \hline
         & 追加 & 削除 & 分割 & 統合 & 変更 \\
         \hline
        クローンなし & 26,108 & 5,472 & 2,614 & 2,668 & 166,318 \\
        クローンあり & - & 3,228 & 71,364 & 73,656 & 302,525 \\
        \hline
    \end{tabular}
    \label{tab:has-clone}
\end{table}

\section{考察}
\memo{分割,統合されたメソッドのほとんどはクローンを持っていた.分割や統合がされた後に,メソッドのクローンは解消されたのかどうかを確認したい.}\memo{クローンが解消されていた場合→クローンが解消されてからメソッドが削除}\memo{クローンが解消されていなかった場合→変更の過程でクローンが解消orクローンを持ったままRQ1のような削除}

\memo{変更と分類されたメソッドが,最終的に,分割や統合に向かっているのか,あるいは削除に向かっているのかを確認したい.}
% 図~\ref{fig:method-classification}の分類はメソッド単位で頻度を計測したものである.削除や分割,統合といった変更はファイルレベルやフォルダレベルで行われ,結果として,頻度が増大している可能性がある.したがって,メソッドレベルより,大きなレベルで変更を捉える必要がある.

\section{メソッド追跡手法の妥当性}
本研究では,NILを用いてバージョン間の類似度を測定することで,メソッドの追加,削除,生存(分割,統合,変更)を捉えた.本手法の妥当性を検証するため,本手法で追加,削除,生存と判定されたそれぞれのメソッドについて,バージョン間でのファイルパスとメソッド名の完全一致について調査した.

追加と判定されたメソッドのファイルパスとメソッド名が前バージョンで存在していれば不正解,削除と判定されたメソッドのファイルパスとメソッド名が次バージョンで存在していれば不正解,生存と判定されたメソッドのファイルパスとメソッド名が次バージョンで存在していれば正解,というように判定した結果が表~\ref{tab:evaluation-classification}である.追加,削除,生存のどの分類においても正解率が0.79を超えている.したがって,本手法での誤検出が小さいと言える.

\begin{table}[ht]
    \centering
    \caption{メソッド追跡手法における分類ごとの正解率}
    \begin{tabular}{c|ccc}
        \hline
         & 追加 & 削除 & 生存  \\
         \hline \hline
        正解 & 24,905 & 6,757 & 414,945\\
        不正解 & 420 & 1,752 & 107,336 \\
        \hline
        正解率 & 0.98 & 0.79 & 0.79 \\
        \hline
    \end{tabular}
    \label{tab:evaluation-classification}
\end{table}


\chapter{妥当性の脅威}
\label{chap:thread-to-validity}

本章では,本研究における結果の信頼性に影響を及ぼす可能性のある要因を,内的妥当性および外的妥当性の観点から述べる.

\section{内的妥当性}

本研究で使用したコードクローンの検出ツールは閾値でコードクローンを判定しているため,閾値によっては,メソッドの分類件数が異なる可能性がある.類似度を判定するための閾値に関しては,従来研究で検証されている0.7を使用することを信頼性を担保した.

クローンセットの一部のメソッドが削除された事例に関しては,目視で数件を確認しているため,この結果が全てのメソッドに当てはまるとは限らない.クローンセットの全メソッドが削除された事例に関しても,ファイルごと削除,内部メソッドの削除という事例は,あくまで一例である.したがって,分類された全てのメソッドに対して当てはまるとは言えない点は注意すべきである.

全てのコミット間隔ではなく,マイナーバージョンの間隔に絞って分析しているため,バージョン間で生成・削除された短命メソッドについては,見逃してしまう.しかし,これら短命のメソッドは開発者が一時的に導入するメソッドであるため,ソフトウェアにとって負債となり得ない.

\section{外的妥当性}
本研究はPythonのpandasプロジェクトのみを対象としている.データサイエンス分野に特化したライブラリ特有の進化パターンが存在する可能性があり,他のドメイン(例:Webフレームワーク,組み込みシステム)や他のプログラミング言語(例:Java, C++)で開発されたプロジェクトにおいても同様の削除パターンやクローン整理の傾向が見られるかは不明である.今後はより多様なOSSプロジェクトを対象とした大規模な調査が求められる.

\chapter{おわりに}
\label{chap:conclusion}
\todo{後で書く}

%%%%%%%%%%%%%%%%%%%%%%%%%%%%%%%%%%%%%%%%%%%%%%%%%%%%%%%%%%%%%%%%%%%%%%%%

%%
%% 謝辞
%%
%% \begin{acknowledgements}
%% 感謝します.
%% \end{acknowledgements}

%%%%%%%%%%%%%%%%%%%%%%%%%%%%%%%%%%%%%%%%%%%%%%%%%%%%%%%%%%%%%%%%%%%%%%%%

%%
%% 参考文献
%%
\bibliographystyle{junsrt}
\bibliography{@Bachelor2025_Yoshida/yoshida_refs}

%%%%%%%%%%%%%%%%%%%%%%%%%%%%%%%%%%%%%%%%%%%%%%%%%%%%%%%%%%%%%%%%%%%%%%%%

%%
%% 付録
%%
% \appendix
% 
% \chapter{サンプルプログラム}
% 
% プログラムリストや実行結果など,本論を補足する上で必要と思われるものが
% あれば付録として付ける.
% 
% {
% \footnotesize
% \begin{verbatim}
% #include <stdio.h>
% int main(void)
% {
%     printf("Hello, World!\n");
%     return 0;
% }
% \end{verbatim}
% }

%%%%%%%%%%%%%%%%%%%%%%%%%%%%%%%%%%%%%%%%%%%%%%%%%%%%%%%%%%%%%%%%%%%%%%%%

\end{document}
