\documentclass[11pt]{jreport}
\usepackage{wuse_thesis}
\usepackage{indentfirst}
\usepackage{url}	% \url{}コマンド用.URLを表示する際に便利
\usepackage[dvipdfmx]{graphicx,xcolor}
%\usepackage{graphicx}  % ←graphicx.styを用いてEPSを取り込む場合有効にする
			% 他のパッケージ・スタイルを使う場合には適宜追加
\newcommand{\todo}[1]{\colorbox{yellow}{{\bf TODO}:}{\color{red} {\textbf{[#1]}}}}

%%%%%%%%%%%%%%%%%%%%%%%%%%%%%%%%%%%%%%%%%%%%%%%%%%%%%%%%%%%%%%%%%%%%%%%%

%%
%% 主に表紙を作成するための情報
%%

%%  タイトル(修論の場合は英語表記も指定)
\title{\todo{タイトルを考える}\\
       p\LaTeX による卒業・修士論文の作成}
%\etitle{Test\\Test\\Test}

%%  著者名(修論の場合は英語表記も指定)
\author{吉田 将衛}
%\eauthor{Akinori Ihara}

%% 卒業論文・修士論文(以下のどちらかを選択)
\bachelar	% 卒業論文(4年生用)
%\master  	% 修士論文(M2用)

%%  学科・クラスタ
\department{システム工}
%\department{デザイン情報}
%\department{デザイン科学}

%%  学生番号
\studentid{60276296}

%%  卒業年度
\gyear{2025}		% 提出年が2022年なら,2021年度

%%  論文提出日
\date{2026年2月10日}	% 修士の場合は月(2021年2月)までとし,英語表記も指定
%\edate{February 2021}	% 修士の場合,こちら(英語表記)も有効化

%%%%%%%%%%%%%%%%%%%%%%%%%%%%%%%%%%%%%%%%%%%%%%%%%%%%%%%%%%%%%%%%%%%%%%%%

\begin{document}

\maketitle

%%
%%  概要
%%
\begin{abstract}
\todo{概要を考える}

\end{abstract}

%%  目次
\tableofcontents

%%  図目次 (図目次をいれたければ以下のコメントをはずす)
%\listoffigures

%%  表目次 (表目次をいれたければ以下のコメントをはずす)
%\listoftables

\newpage
\pagenumbering{arabic}	% 以降のページ番号を算用数字に

%%%%%%%%%%%%%%%%%%%%%%%%%%%%%%%%%%%%%%%%%%%%%%%%%%%%%%%%%%%%%%%%%%%%%%%%

%%
%%  本文はここから
%%

\chapter{はじめに}

\todo{}


\chapter{不要コードの削除}

\section{不要コードの定義}

\subsection{到達不能コード}

\subsection{未実行コード}

\subsection{不要コード}


\section{不要コードが引き起こす問題}

\section{関連研究}

\subsection{静的解析によるデッドコードの解析}

\subsection{動的解析による未実行コードの解析}

\subsection{機能的代替機能の検出}

\section{本研究の位置付け}



\chapter{事前分析:削除されたコードの特徴}

\section{概要}

\section{分析手法}

\section{データセット}

\section{分析結果}

\section{キーアイデア}

\chapter{RQ1:類似機能メソッド群の生存と削除にはどのような特徴があるか}

\section{概要}

\section{分析手法}
\subsection{類似機能メソッドの検出}
\subsection{削除メソッドと生存メソッドの抽出}
\subsection{メソッドの変遷パターンの分類}

\section{分析結果}
\subsection{類似機能メソッドの生存実態}
\subsection{削除されたメソッドの特徴}

\section{まとめ}

\chapter{RQ2:削除されたメソッドの特徴は汎化できるか}

\section{概要}

\section{検出ルールの作成}

\section{評価方法}

\section{評価結果}
\subsection{全体の検出精度}
\subsection{ルール別の検出精度}

\section{まとめ}


\chapter{考察}

\section{生存するのに削除されると予測されたのはなぜか}

\chapter{妥当性の脅威}

\section{内的妥当性}

\section{外的妥当性}

\chapter{おわりに}

%%%%%%%%%%%%%%%%%%%%%%%%%%%%%%%%%%%%%%%%%%%%%%%%%%%%%%%%%%%%%%%%%%%%%%%%

%%
%% 謝辞
%%
%% \begin{acknowledgements}
%% 感謝します.
%% \end{acknowledgements}

%%%%%%%%%%%%%%%%%%%%%%%%%%%%%%%%%%%%%%%%%%%%%%%%%%%%%%%%%%%%%%%%%%%%%%%%

%%
%% 参考文献
%%
\begin{thebibliography}{99}

\bibitem{wusethesis}
  伊原彰紀,
  卒業論文スタイルファイル(和歌山大学システム工学部用),\\
  \url{https://github.com/fukuyasu/wuse_thesis}.

\bibitem{tex}
  Knuth, D.,
  Remarks to Celebrate the Publication of Computers \& Typesetting,
  TUGboat, Vol.7, No.2, pp.95--98, 1986.

\bibitem{latex}
  Lamport, L.,
  文書処理システム\LaTeXe{},
  ピアソン・エデュケーション,1999,
  \newblock{}阿瀬はる美 訳.

\bibitem{latex_j}
  奥村晴彦,\LaTeX{}入門 ---美文書作成のポイント---,技術評論社,1993.

\bibitem{latex2e}
  奥村晴彦,黒木裕介,[改定第6版] \LaTeXe~美文書作成入門,技術評論社,2013.

\bibitem{latexcomp}
  Goossens, M., Mittelbach, F. and Samarin, A.,
  The \LaTeX{}コンパニオン,アスキー出版局,1998,
  \newblock{}アスキー書籍編集部 監訳.

\bibitem{texwiki}
  \LaTeX 入門 --- \TeX{} Wiki,\\
  \url{https://texwiki.texjp.org/?LaTeX%E5%85%A5%E9%96%80},
  2021年12月3日閲覧.
\end{thebibliography}

%%%%%%%%%%%%%%%%%%%%%%%%%%%%%%%%%%%%%%%%%%%%%%%%%%%%%%%%%%%%%%%%%%%%%%%%

%%
%% 付録
%%
% \appendix
% 
% \chapter{サンプルプログラム}
% 
% プログラムリストや実行結果など,本論を補足する上で必要と思われるものが
% あれば付録として付ける.
% 
% {
% \footnotesize
% \begin{verbatim}
% #include <stdio.h>
% int main(void)
% {
%     printf("Hello, World!\n");
%     return 0;
% }
% \end{verbatim}
% }

%%%%%%%%%%%%%%%%%%%%%%%%%%%%%%%%%%%%%%%%%%%%%%%%%%%%%%%%%%%%%%%%%%%%%%%%

\end{document}
