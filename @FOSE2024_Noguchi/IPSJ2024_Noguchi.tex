\documentclass[uplatex,dvipdfmx,a4paper,twocolumn,base=10.5pt,jbase=10.5pt,ja=standard]{bxjsarticle}  % 環境に合わせて変更してください

\usepackage{ipsj}
\usepackage{color}
%追加パッケージ
\usepackage{enumerate}
\usepackage{url}
\usepackage{graphics}
\usepackage{caption}
\usepackage{setspace}
\usepackage{multirow}

\setstretch{0.74}
\newcommand{\todo}[1]{\colorbox{yellow}{{\bf TODO}:}{\color{red} {\textbf{[#1]}}}}
\newcommand{\done}[1]{\colorbox{yellow}{{\bf DONE}:}{\color{red} {\textbf{[#1]}}}}


\title{プロジェクト別のコーディング規約使用種別に基づく\\規約違反修正率の分析}
{Analysis of Coding Convention Violation Rate Between OSS Projects using Similar Coding Rule }
\author{和歌山大学}{野口 朋弥}{Tomoya Noguchi, Wakayama University}
\author{和歌山大学}{伊原 彰紀}{Akinori Ihara, Wakayama University}
\author{和歌山大学}{亀岡 令}{Ryo Kameoka, Wakayama University}

\begin{document}
\maketitle

%================
%1
\section{はじめに}
%================
複数の開発者が協力して同じソフトウェアを実装する場合に,開発者間でコーディングスタイルを共通化することで,ソースコードの保守性向上やプログラム品質を確保することができる.多くのソフトウェア開発では,開発者間でコーディングスタイルを共通化するためにコーディング規約を導入している.コーディング規約は,命名規則,コーディングスタイル,禁止事項などを規定したルールがまとめられている.
%開発者が,コーディング規約を遵守することでソフトウェアの一貫性や可読性を向上する.

不特定多数の開発者が流動的に参加,離脱するオープンソースソフトウェアの開発では,規約違反を自動的に検出するための静的解析ツールを開発者に共有している.静的解析ツールは,コーディング規約に違反した箇所を特定し,開発者に修正を促す役割を果たす.
大規模なソフトウェア開発では,静的解析ツールが検出する違反箇所が膨大である一方で,開発者が修正する違反箇所は一部であり,多くの違反箇所は修正されていない\cite{findBug}.したがって,開発者が優先的に修正すべき規約違反を特定することが課題となっている.
%^効果的な静的解析ツールの導入がコーディング規約の順守を促進する上で重要な要素となる.

従来研究~\cite{Junjie}では,過去の修正履歴に基づき機械学習手法を用いて優先的に修正すべき違反箇所を特定する手法を提案している.しかし,同一プロジェクトにおける規約違反の修正履歴が十分に存在しない場合,学習不足により高い精度を得ることが難しい.また,先行研究~\cite{kameoka}では,複数のプロジェクトを学習することで学習不足の解消を試みたが,プロジェクトの修正方針の類似性が低い場合に精度が低下することが明らかとなった.本研究では,プロジェクトの修正方針が類似するプロジェクトを学習データとして用いた場合の修正予測精度を評価する.

%================
% \begin{table}[t]
%     \centering
%     \caption{規約違反ID別の予測精度}
%     \label{tab:coding_ID}
%     \scalebox{0.7}[0.7]{
%      \begin{tabular}{ll|rrrr|r}
%       \hline\hline
%                 & \multicolumn{1}{|c|}{モデル} & 適合率  & 再現率  & F1値  & 正解率  & 違反数 \\ \hline
%         \multirow{4}{*}{\rotatebox{90}{R1705}} & \multicolumn{1}{|l|}{hinkle単体}     & 0.20  & \textbf{1.00} & 0.33 & 0.33 & 81    \\ 
%         \multicolumn{1}{l|}{}      & hinkle統合     & \textbf{0.50} & 0.50  & \textbf{0.50}  & \textbf{0.83} & 102   \\ \cline{2-7} 
%         \multicolumn{1}{l|}{python-sdk単体}      & python-sdk単体 & \textbf{1.00} & \textbf{1.00} & \textbf{1.00} & \textbf{1.00} & 6     \\  
%         \multicolumn{1}{l|}{}      & python-sdk統合 & 0.40 & \textbf{1.00} & 0.57 & 0.75 & 27    \\ \hline
%         \multirow{4}{*}{\rotatebox{90}{C0209}} & \multicolumn{1}{|l|}{hinkle単体}     & 0    & 0    & nan  & \textbf{0.61} & 395   \\ 
%         \multicolumn{1}{l|}{}      & hinkle統合     & \textbf{0.09} & \textbf{0.67} & \textbf{0.15} & 0.47 & 412   \\ \cline{2-7} 
%         \multicolumn{1}{l|}{}      & python-sdk単体 & \textbf{0.77} & 0.67 & \textbf{0.71} & \textbf{0.90}  & 585   \\ 
%         \multicolumn{1}{l|}{}      & python-sdk統合 & 0.35 & \textbf{0.80}  & 0.49 & 0.68 & 602   \\ \hline
%       \end{tabular}
%     }
% \end{table}

%================



%================
%2
\section{方法}
%================
%2.1
\subsection{修正方針が類似するプロジェクト特定手法}
\vspace{2mm}
%================

各プログラミング言語では標準的に利用される規約が存在し,その中には多数の規約が登録されている.ソフトウェア開発プロジェクトでは,その中の一部を規約として利用していることが多い.また,各プロジェクトで静的解析ツールによって検出された規約違反の中で開発者が頻繁に修正する規約の種類は異なる.本研究では,規約の種類別に開発者が修正した割合をプロジェクトの修正方針と捉え,各プロジェクトを対象に規約種類別の修正率を格納したベクトルを作成する.ただし,一度も修正されなかった規約,プロジェクトにおいて使用されていない規約は修正率を0\%とする.
プロジェクト間の修正方針の類似度は,両プロジェクトで作成したベクトルをコサイン類似度を用いて算出する.ベクトル化したコーディング規約ごとの修正率を用い,プロジェクト間のコサイン類似度を計算する.
%\todo{これってカンニングしてる?}

%プロジェクトの各コーディング規約の修正率を取得する,方法として,対象期間までに発生したコーディング規約違反に対して,修正が行われたか否かでコーディング規約ごとの修正率を測る.
%取得した修正率から,プロジェクト間のコサイン類似度を計算するために,プロジェクトの各コーディング規約の修正率をベクトル化する.このとき,比較対象に同一の規約が含まれない場合,修正率を0として扱い,ベクトル化する.


% 対象となるプロジェクトのコミットから,コーディング規約の修正の可否に関する情報を収集する.修正の可否は「Neglect Warning」,「Delete Warning Code」,「Fix Warning」の三種類に分類され,これらを用いてプロジェクトごとにコーディング規約の修正率を算出する.ここで,「Fix Warning」は修正が行われたことを示し,一方で「Neglect Warning」と「Delete Warning Code」は修正が行われなかったことを示す.各プロジェクトごとに修正率を算出し,これを利用してプロジェクト間の修正方法の類似性を測定するためにコサイン類似度を用いる.
%予測->説明変数,目的変数,アルゴリズム,評価方法

%================
%2.2
\vspace{2mm}
\subsection{予測方法}
\vspace{2mm}
%================
本研究では,規約違反箇所を開発者が修正を要すると判断するか否かを予測する2種類のモデルを構築する.\\
\vspace{-6mm}
\begin{itemize}
\item[(1)] 同一のプロジェクトにおける過去の規約違反箇所の修正履歴を学習データとして用いた予測モデル\\\vspace{-4mm}
\item[(2)] 修正方針が類似する2プロジェクトにおける過去の規約違反箇所の修正履歴を学習データとして用いた予測モデル.
\end{itemize}
\vspace{-2mm}

\noindent\textbf{学習データの構築}:各プロジェクトの修正履歴において,検出時期が古いデータ8割を学習データ,残りの2割を検証データとして使用する.

\noindent\textbf{説明変数の計測方法}:静的解析ツールが初めて違反を検出した時点で,説明変数としてソースコードメトリクスを計測する.具体的にはソースコードの特徴量であるコード行数,コメント行数,ネストの深さの最大値などを含む特徴量43種類,規約違反箇所のコード行数1種類,規約違反IDをOne-hotベクトル化1種類の合計45種類の特徴量を計測する.

\noindent\textbf{目的変数の計測方法}:目的変数は,分析対象期間中に修正されたか否かを計測する.ソースコードの変更により,分析期間中に規約違反が検出されなくなった場合は,違反は修正された(正例)とする.分析対象期間中に修正されない場合,またはソースコードが削除された場合は,違反を修正されなかった(負例)とする.
%最終状態において各違反が修正されたか否かを計測することによって計測する.修正されたか否かは,発生した違反が修正されず放置された状況,発生した違反に該当するコードが削除された状況,発生した違反に該当するコードが修正された状況の3つに分けることができる.このうち,前二者は発生した違反が修正されていないため負例として扱う.一方後者は発生した違反が修正されているため,正例として扱う.すべてのコーディング規約違反はこの3種類に分類され,目的変数を計測し,各違反の発生地点において説明変数を計測する.

\noindent\textbf{予測モデルの構築}:本研究では,ランダムフォレストアルゴリズムを用いて規約違反の修正予測モデルを構築する.コーディング規約違反の修正に関するデータは,違反が修正されたという正例が少なく,修正されないという負例が多い不均衡なデータであることが多い.そのため,各予測モデルを構築するためのPythonパッケージに用意されているclass weightsオプションを用い,2クラスデータに重み付けを行う.

\noindent\textbf{予測モデルの評価}:構築した各予測モデルの評価には,適合率,再現率,F1値を用いる.
%================
%2.3
\vspace{2mm}
\subsection{予測の評価手法}
\vspace{2mm}
%================
構築した各予測モデルの評価には,適合率,再現率,F1値を用いる.機械学習モデルの予測結果は真陽性,真陰性,偽陽性,偽陰性に分類でき,分類結果を用いて本研究の手法を評価する.

%================
%3
\section{ケーススタディ}
%================
\begin{table}[t]
    \centering
    \caption{規約違反ID別の予測精度(hinckleとpython-sdkを入れ替えた場合)}
    \label{tab:coding_ID}
    \scalebox{0.7}[0.7]{
     \begin{tabular}{ll|rrrr|r}
      \hline\hline
                & \multicolumn{1}{|c|}{モデル} & 適合率  & 再現率  & F1値  & 正解率  & 違反数 \\ \hline
        \multirow{4}{*}{\rotatebox{90}{R1705}} & \multicolumn{1}{|l|}{python-sdk単体}     & 0.20  & \textbf{1.00} & 0.33 & 0.33 & 81    \\ 
        \multicolumn{1}{l|}{}      & python-sdk統合     & \textbf{0.50} & 0.50  & \textbf{0.50}  & \textbf{0.83} & 102   \\ \cline{2-7} 
        \multicolumn{1}{l|}{}      & hinkle単体 & \textbf{1.00} & \textbf{1.00} & \textbf{1.00} & \textbf{1.00} & 6     \\  
        \multicolumn{1}{l|}{}      & hinkle統合 & 0.40 & \textbf{1.00} & 0.57 & 0.75 & 27    \\ \hline
        \multirow{4}{*}{\rotatebox{90}{C0209}} & \multicolumn{1}{|l|}{python-sdk単体}     & 0    & 0    & nan  & \textbf{0.61} & 395   \\ 
        \multicolumn{1}{l|}{}      & python-sdk統合     & \textbf{0.09} & \textbf{0.67} & \textbf{0.15} & 0.47 & 412   \\ \cline{2-7} 
        \multicolumn{1}{l|}{}      & hinkle単体 & \textbf{0.77} & 0.67 & \textbf{0.71} & \textbf{0.90}  & 585   \\ 
        \multicolumn{1}{l|}{}      & hinkle統合 & 0.35 & \textbf{0.80}  & 0.49 & 0.68 & 602   \\ \hline
      \end{tabular}
    }
\end{table}
%3.1
\subsection{データセット}
\vspace{2mm}
%================
本研究では,オープンソースソフトウェアライブラリ検索エンジンであるlibraries.ioで公開されているデータセットの中で,紙面の都合上,利用規約数の多いプロジェクトGPflowを対象とする.GPflowと修正率が類似するプロジェクトpython-sdkと,修正率が類似していないプロジェクトhickleを対象に,予測モデルの精度を規約ごとに比較する.



%紙面の都合上,コーディング規約違反の発生数が似ているが,コサイン類似度に差が出た「GPflowとpython-sdk」と「GPflowとhickle」の2の組み合わせを対象として調査した.「GPflow」,「python-sdk」,「hickle」それぞれで予測モデルを作成した場合と,「GPflowとpython-sdk」,「GPflowとhickle」それぞれを1つの学習データとして予測モデルを作成した場合の結果の違いを分析する.
%================
%3.2
\vspace{2mm}
\subsection{結果}
\vspace{2mm}
%================
%R1705:関数内での不要なelse節を検出(具体例:関数内で,return, continue, breakの後にelse節があるとき)
%C0209:
表\ref{tab:coding_ID}は,規約別に2種類のモデルの結果を示す.紙面の都合上,規約R1705~\footnote{R1705:関数内での不要なelse節を検出(具体例:関数内で,return, continue, breakの後にelse節がある時に生じる},規約C0209~\footnote{C0209:f文字列ではなく,古いフォーマット方法で実装されている時に起こる(具体例:文字列に対して,format()が使用されている時に生じる)}のみの結果を示す.また,表中に単体と記載しているモデルは当該プロジェクトのみでモデルを構築した結果を示す.また,統合と記載しているモデルは当該プロジェクトとGPflowの修正履歴を用いてモデルを構築した結果を示す.

%を規約ごとに分析した結果を,紙面の都合上,一部を表
%\ref{tab:coding_ID}に示す.

GPflowと修正率が類似するpython-sdkは,分析対象とした両規約においてGPflowと学習データを統合したとしてもF1値や正解率が向上することはなかったが,再現率については,プロジェクト単体モデルと同等,またはそれ以上の精度になる.したがって,学習データを統合することで修正を要する多様な規約違反の特定に貢献できると考える.

GPflowと修正率が類似していないhinkleは,規約R1705においてGPflowの学習データを統合することで再現率を除く評価指標で高い精度を得た.したがって,類似しないプロジェクトと統合する場合,再現率が低下することは今後の課題である.


%今回分析した結果,プロジェクトGPflowと修正率が類似するプロジェクトpython-sdkは精度が全体的に下がった.一方,修正率が類似していないプロジェクトhinckleは精度が上がる規約もある結果となった.また,プロジェクトGPflowとプロジェクトpython-sdkを統合して学習した場合,適合率,再現率,F1値,正解率の全てが上がったものはなく,規約違反IDC0209は適合率を大幅に下げ,再現率が上がる結果となった.理由として,プロジェクトGPflowが正例の少ないデータであるため,データがより不均衡になり,再現率は上がるが,適合率が下がる結果になったと考えられる.そのため,今後の統合先のプロジェクトを正例の多いプロジェクトにすることで全体的な精度の向上が目指せると推敲する.

%================
%4
\section{おわりに}
%================
本研究では,静的解析ツールによる違反の収集率の類似性があるプロジェクト間で学習データを統合し,予測を行うことで精度の向上を試みた..分析の結果,修正率の類似度を参考にして,プロジェクトの統合を行うことは,多様な規約違反の修正において,一部で貢献できるという結果となった.今後の方針として,統合数を増加させることによる,学習モデルサイズの拡張や,類似性の算出方法の変更を行うことで,予測精度の高いモデルの構築を目指す.
%================
%\section*{参考文献}
%================
\bibliographystyle{junsrt}
\bibliography{Noguchi}

% \begin{thebibliography}{10}
%   \bibitem{Junjie} Junjie Wang , Song Wang , Qing Wan. Is There A “Golden” Feature Set for Static Warning Identification? In Proc. the 12th ACM/IEEE International Symposium on Empirical Software Engineering and Measurement (ESEM2018)
  
% \end{thebibliography}

\end{document}
