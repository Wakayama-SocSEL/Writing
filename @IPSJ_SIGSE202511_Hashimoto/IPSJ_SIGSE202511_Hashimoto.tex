%%
%% 研究報告用スイッチ
%% [techrep]
%%
%% 欧文表記無しのスイッチ(etitle,eabstractは任意)
%% [noauthor]
%%

%\documentclass[submit,techrep]{ipsj}
\documentclass[submit,techrep,noauthor]{ipsj}



\usepackage[dvips]{graphicx}
\usepackage{latexsym}

\newcommand{\todo}[1]{\colorbox{yellow}{{\bf TODO}:}{\color{red} {\textbf{[#1]}}}}
\newcommand{\ihara}[1]{\colorbox{green}{{\bf IHARA}:}{\color{blue} {\textbf{[#1]}}}}

\def\Underline{\setbox0\hbox\bgroup\let\\\endUnderline}
\def\endUnderline{\vphantom{y}\egroup\smash{\underline{\box0}}\\}
\def\|{\verb|}
%

%\setcounter{巻数}{59}%vol59=2018
%\setcounter{号数}{10}
%\setcounter{page}{1}


\begin{document}


\title{コードレビューにおける\\長期貢献者予測に向けた学習期間の検討}

\affiliate{IPSJ}{和歌山大学\\
Wakayama University}

\author{橋本 一輝}{Hashimoto Kazuki}{IPSJ}[s276189@wakayama-u.ac.jp]
\author{伊原 彰紀}{Ihara Akinori}{IPSJ}[ihara@wakayama-u.ac.jp]


\begin{abstract}
オープンソースソフトウェアプロジェクトの継続的な開発・保守の実現には,長期的に貢献する開発者(長期貢献者)の確保が不可欠であるが,多くの開発者は数回の貢献後に活動が途絶えてしまうことが多い.従来研究では,長期貢献者を早期に特定する研究が進められており,プロジェクト参加後の数ヶ月の貢献に基づいて行われてきたが、コードレビュー作業への継続的な貢献を対象にした研究は進められていない.本研究では、コードレビュー依頼において,逆強化学習(IRL)を用いてレビュアーの貢献を学習し、継続的なタスク受け入れの判断を決定する報酬関数を推定する。
特に、レビュアーがタスクを受け入れる参加確率モデルの予測性能を検証するため、モデルの学習期間と貢献予測期間を動的に変更し、その予測精度と安定性を系統的に調査する。ケーススタディとしてコードレビュー履歴を対象に評価実験を行う.
本研究では,レビュアーの貢献が時間とともにどのように変化し、長期貢献者の予測に最適な学習期間を定量的に分析する。
\end{abstract}


%
%\begin{jkeyword}
%情報処理学会論文誌ジャーナル,\LaTeX,スタイルファイル,べからず集
%\end{jkeyword}
%
%\begin{eabstract}
%This document is a guide to prepare a draft for submitting to IPSJ
%Journal, and the final camera-ready manuscript of a paper to appear in
%IPSJ Journal, using {\LaTeX} and special style files.  Since this
%document itself is produced with the style files, it will help you to
%refer its source file which is distributed with the style files.
%\end{eabstract}
%
%\begin{ekeyword}
%IPSJ Journal, \LaTeX, style files, ``Dos and Dont's'' list
%\end{ekeyword}

\maketitle

%1
\section{はじめに}
オープンソースソフトウェア (OSS) は,インターネット,クラウドサービス,人工知能などといった現代のテクノロジー基盤において中心的な役割を担っている.\todo{SE分野ではじめの1行いらない.}そのため,OSSプロジェクトの持続的な成長と安定的な運用は極めて重要である.しかし,多くOSSプロジェクトはボランティアベースの分散型コミュニティによって維持されているという特性を持つ\todo{要引用}.この開発モデル\todo{伽藍とバザールのこと言ってる?それともソーシャルコーディングのこと言ってる?}は,イノベーション\todo{イノベーション?を促進する?}多様な参加を促進する一方で,開発者の流動性が高いという問題がある\ihara{「参加を促進する一方,流動性が高い」は美しい流れ!}.世界中の自発的な貢献者\todo{開発者or貢献者?表記揺れ?}の動機は多様であり,時間的制約や興味の変化による開発者の離脱率が高いことが課題である,貢献者の離脱によるプロジェクトの新規機能開発の停滞や活動停止につながる事例も少なくない\todo{要引用}.

プロジェクトの持続可能性を確保するうえで重要であるのは,プロジェクトに長期的に貢献する開発者(長期貢献者)の存在である.多くの新規開発者が数回の貢献後に,活動が途絶えてしまう傾向がある.そのため,将来の長期貢献者を特定することで,適切なサポートを行うことはプロジェクト管理において重要である\todo{要引用}.

これまでに,OSSプロジェクトの持続可能性を支える長期貢献者を特定するための研究は,様々な手法で進められてきた.これらの研究は主に,貢献者の過去の活動履歴,特にコードの追加・修正の量や頻度といった特徴量を抽出し,機械学習モデル(決定機\todo{$\leftarrow$木},ロジスティック回帰,サポートベクタマシンなど)を用いてLTCを分類・予測すること目的としている.これらの手法により,プロジェクト参加後,数週間から数ヶ月の初期

主にコードの追加や修正といった直接的な開発行為に焦点を当ててきた.一方で,プロジェクトの品質や健全性を支えるうえで不可欠であるコードレビュー作業への長期的貢献を対象とした研究は十分に進んでいない.特にレビュアーが継続的にタスクを受け入れるという意思決定の背後にある動機や,の貢献を予測するためにどの程度の活動期間が必要なのかという,予測の精度と安定性に関わる定量的な検討はほとんど行われていない.そのため,本研究ではコードレビュアーの長期的な貢献に着目し,レビュー活動に基づく長期貢献者の予測をに向けた調査を行う.

%3
\section{関連研究}
\label{sec:format}

以下,情報処理学会論文誌ジャーナル用スタイルファイルを用いた論文フォーマットの指針について述べるので,
これに従って原稿を用意頂きたい.
\LaTeX を用いた一般的な文章作成技術については,
\cite{okumura, companion} 等を参考にされたい.



%4
\section{長期貢献者予測モデル}
\label{config}




\begin{acknowledgment}
hogehoge
\end{acknowledgment}


\bibliographystyle{ipsjunsrt}
\bibliography{bibsample}



\end{document}
