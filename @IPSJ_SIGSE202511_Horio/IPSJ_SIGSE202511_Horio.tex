%%
%% 研究報告用スイッチ
%% [techrep]
%%
%% 欧文表記無しのスイッチ(etitle,eabstractは任意)
%% [noauthor]
%%

%\documentclass[submit,techrep]{ipsj}
\documentclass[submit,techrep,noauthor]{ipsj}


% \usepackage[dvips]{graphicx}
\usepackage[dvipdfmx, hiresbb]{graphicx}
\usepackage{latexsym}

\newcommand{\todo}[1]{\colorbox{yellow}{{\bf TODO}:}{\color{red} {\textbf{[#1]}}}}
\newcommand{\ihara}[1]{\colorbox{green}{{\bf IHARA}:}{\color{blue} {\textbf{[#1]}}}}

\def\Underline{\setbox0\hbox\bgroup\let\\\endUnderline}
\def\endUnderline{\vphantom{y}\egroup\smash{\underline{\box0}}\\}
\def\|{\verb|}
%

%\setcounter{巻数}{59}%vol59=2018
%\setcounter{号数}{10}
%\setcounter{page}{1}


\begin{document}


\title{強化学習を用いたScratchにおける\\
リミックス元作品のCTスコア予測}

% \etitle{How to Prepare Your Paper for IPSJ SIG Technical Report \\ (version 2018/10/29)}

\affiliate{IPSJ}{情報処理学会\\
IPSJ, Chiyoda, Tokyo 101--0062, Japan}


\paffiliate{JU}{情報処理大学\\
Johoshori Uniersity}

\author{情報 太郎}{Joho Taro}{IPSJ}[joho.taro@ipsj.or.jp]
\author{処理 花子}{Shori Hanako}{IPSJ}
\author{学会 次郎}{Gakkai Jiro}{IPSJ,JU}[gakkai.jiro@ipsj.or.jp]

\begin{abstract}
本研究は,プログラミング初学者向けのビジュアルプログラミング言語Scratchにおいて,リミックス元作品のコンピュテーショナル・シンキング(CT)スコアを予測する手法を提案する.
従来研究では,他者の作品を複製・編集するリミックス機能が,学習者のCTスコア向上に有効であることが示されている.また,その効果はユーザの学習経験によって異なることが示唆された.しかし,現状ではユーザの学習経験に適したリミックス元作品を見つけることが困難である.
本研究では,Scratchに参加するユーザが作品制作した過程をマルコフ決定過程として捉え,強化学習を用いてユーザが選択するリミックス元作品のCTスコアを予測するモデルを構築する.これにより,ユーザの過去作品やリミックス履歴に基づいたCTスコアの推定が可能となり,今後のリミックス元作品の選択の指針となることを目指す.
\end{abstract}


%
\begin{jkeyword}
\todo{情報処理学会論文誌ジャーナル,\LaTeX,スタイルファイル,べからず集}
\end{jkeyword}
%
%\begin{eabstract}
%This document is a guide to prepare a draft for submitting to IPSJ
%Journal, and the final camera-ready manuscript of a paper to appear in
%IPSJ Journal, using {\LaTeX} and special style files.  Since this
%document itself is produced with the style files, it will help you to
%refer its source file which is distributed with the style files.
%\end{eabstract}
%
%\begin{ekeyword}
%IPSJ Journal, \LaTeX, style files, ``Dos and Dont's'' list
%\end{ekeyword}

\maketitle

%1
\section{はじめに}

近年,日本では2020年度から小学校のプログラミング教育の必修化,2021年度から中学校の技術分野において,プログラミングに関する内容の充実,2022年度から高等学校の情報科において,共通必履修科目「情報1」の新設がされている\cite{monkashou}.しかし,指導者の情報不足や人材不足,予算不足による指導者に対する問題が複数挙げられた\cite{monkashou2}.以前からプログラミング教育に力を注いでいるアメリカでは,Scratch\footnote{https://scratch.mit.edu/}\cite{resnick2009scratch}と呼ばれるプログラミング初学者向けのビジュアルプログラミング言語を通してプログラミング教育を行っている.Scratchでは「繰り返す」などの命令処理をもつブロックを,ジグソーパズルのように組み合せることで作品を完成させる.さらに,完成した作品はオンライン上に公開することができ,公開されている他ユーザの作品は,見る,使うだけでなく,複製し,編集することも可能である.この機能はリミックスと呼ばれる.

ビジュアルプログラミングは,プログラミング初学者が取り組みやすいような環境にするとともに,コンピュテーショナル・シンキング(CT)\cite{wing2006computational}スキルの向上を目的としている.学習や操作のハードルが低いため,プログラミングの入門学習として有効である.また,バグが少なく,直観的な操作のみでプログラミングをすることができるため,ユーザが短期間で作品を完成させやすく,達成感を得やすいという特徴をもつ.近年では,情報化社会の進展に伴い,プログラマー不足が深刻な問題となっている.ビジュアルプログラミング言語の活用はその課題解決の第一歩になると考えられる.

CTスキルを自身で評価・把握することは困難である.そのため,Morenoらは作品に使用されているブロックに基づき,CTスキルを計測するDr.Scratch\cite{moreno2015dr}を開発した.Dr.Scratchでは,論理,制御フロー,同期,抽象化,データ表現,ユーザ対話性,並列処理の7概念に基づいて作品評価を行う.また,各概念を0点から3点の合計21点とし,CTスコアとして評価する.表\ref{CTscoreTable}に各概念の評価方法を示す.CTスコアを3つの区分に分類し,0点から7点をBasic,8点から14点をDeveloping,15点から21点をMasterとして評価する.

\begin{table*}[t] % table環境をtable*に変更
    \centering
    \caption{CTスコア概念\cite{hashitani2022scratch}}
    \label{CTscoreTable}
    % table*はページ全体を使うため、scaleboxを調整するか、削除を検討。
    % ページ幅全体に広げるため、0.55から0.8~0.9などに広げても良いでしょう。
    \scalebox{0.8}{ % ページ幅全体を使うため、スケールを0.8などに広げました
    \begin{tabular}{c|c|c|c|c}
    \hline
    CTスキルの概念 & 0点 & 1点 & 2点 & 3点
    \\ \hline
    抽象化 & - &
    \begin{tabular}{c}
    2つ以上の\\スクリプトを使用
    \end{tabular}
    & 定義ブロックを使用 & クローンブロックを使用
    \\ \hline
    並列 & - & 
    \begin{tabular}{c}
    緑の旗ブロックを\\2個以上使用 
    \end{tabular}
    & 
    \begin{tabular}{c}
    オブジェクトへのクリックにより\\2つ以上のスクリプトを\\同時に実行する機能を実装
    \end{tabular}
    & 
    \begin{tabular}{c}
    イベント動作により\\2つ以上のスクリプトを\\同時に実行する機能を実装
    \end{tabular}
    \\ \hline
    論理 & - & Ifブロックを使用 & If elseブロックを使用 & 論理演算ブロックを使用
    \\ \hline
    同期 & - & 待機ブロックを使用 &
    \begin{tabular}{c}
    メッセージ受信により\\プログラムを停止する機能を実装
    \end{tabular}
    &
    \begin{tabular}{c}
    指定条件を満たすまで\\プログラムを停止する処理を実装
    \end{tabular}
    \\ \hline
    フロー制御 & - &
    \begin{tabular}{c}
    2個以上の処理ブロックを\\連結して使用
    \end{tabular}
    &
    \begin{tabular}{c}
    指定回数/回数無制限の\\繰り返しブロックを使用
    \end{tabular}
    &
    \begin{tabular}{c}
    指定条件までの\\繰り返しブロックを使用
    \end{tabular}
    \\ \hline
    ユーザ対話性 & - & 緑の旗ブロックを使用 &
    \begin{tabular}{c}
    ユーザの入力を伴う\\ブロックを使用
    \end{tabular}
    &
    \begin{tabular}{c}
    マイクやビデオなどの\\作用を伴うブロックを使用
    \end{tabular}
    \\ \hline
    データ表現 & - &
    \begin{tabular}{c}
    オブジェクトの\\プロパティを編集
    \end{tabular}
    & 変数ブロックを使用 & リスト変数ブロックを使用
    \\ \hline
    \end{tabular}
    }
\end{table*}

\begin{itemize}
  \item 状態
    
    $s_t$:ある作品制作地点での状態(8次元)

    $x_t$:ある作品のCTスコア概念のベクトル(7次元)

    $m_t\in\{0,1\}$:リミックスであるか否か(1次元)

    \begin{equation}
        s_t = 
        \begin{bmatrix}
           x_t \\
           m_t 
        \end{bmatrix}
    \end{equation}

  \item 行動

    $a_t$:選択したリミックス元作品のCTスコア概念のベクトル(7次元)

    % $a_t$:t+1番目に制作する作品のCTスコア概念のベクトル(7次元)
  
  \item 報酬

    $r_t$:リミックス前後作品のCTスコアの差分のスカラー
    
    $c_t$:リミックス前作品のCTスコア

    $c'_t$:リミックス後作品のCTスコア
    
    \begin{equation}
        r_t = c'_t - c_t
    \end{equation}

    % $r_t$:前後作品のCTスコアの差分のスカラー
    
    % $c_t$:t番目の作品のCTスコア

    % $c'_t$:t-1番目の作品のCTスコア
    
    % \begin{equation}
    %     r_t = c'_t - c_t
    % \end{equation}
  
  \item 状態遷移確率:
\end{itemize}

\section{RQ1}

図\ref{heatmap}が示すように

\begin{figure*}[t]
  \centering
  \includegraphics[width=\textwidth]{@IPSJ_SIGSE202511_Horio/heatmap.pdf}
  \caption{ヒートマップ}
  \label{heatmap}
\end{figure*}


\bibliographystyle{unsrt}
\bibliography{references}

\end{document}
