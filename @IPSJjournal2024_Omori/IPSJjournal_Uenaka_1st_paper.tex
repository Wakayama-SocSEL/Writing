\documentclass[submit]{ipsj}
%\documentclass{ipsj}

%\usepackage{graphicx}
%\usepackage[dvipdfmx]{graphicx,color}
%\usepackage[dvips]{graphicx}
\usepackage[dvipdfmx]{graphicx}
\usepackage{latexsym}
\usepackage{url}
\usepackage{listings}
\usepackage{multirow}
\usepackage{xcolor}
\usepackage{colortbl}


\newcommand{\todo}[1]{\colorbox{yellow}{{\bf TODO}:}{\color{red} {\textbf{[#1]}}}}

%ここからソースコードの表示に関する設定
\definecolor{darkgray}{rgb}{.4,.4,.4}
\definecolor{purple}{rgb}{0.65, 0.12, 0.82}

\lstdefinelanguage{JavaScript}{
  keywords={typeof, new, true, false, catch, function, return, null, catch, switch, var, if, in, while, do, else, case, break},
  keywordstyle=\color{blue}\bfseries,
  ndkeywords={class, export, boolean, throw, implements, import, this},
  ndkeywordstyle=\color{darkgray}\bfseries,
  identifierstyle=\color{black},
  sensitive=false,
  comment=[l]{//},
  morecomment=[s]{/*}{*/},
  commentstyle=\color{purple}\ttfamily,
  stringstyle={\small\ttfamily},
  morestring=[b]',
  morestring=[b]"
}

\lstset{
  basicstyle={\ttfamily},
  identifierstyle={\small},
  commentstyle={\smallitshape},
  keywordstyle={\small\bfseries},
  ndkeywordstyle={\small},
  stringstyle={\small\ttfamily},
  frame={tb},
  breaklines=true,
  columns=[l]{fullflexible},
  numbers=left,
  xrightmargin=0zw,
  xleftmargin=3zw,
  numberstyle={\scriptsize},
  stepnumber=1,
  numbersep=1zw,
  lineskip=-0.5ex
}
%ここまでソースコードの表示に関する設定

\def\Underline{\setbox0\hbox\bgroup\let\\\endUnderline}
\def\endUnderline{\vphantom{y}\egroup\smash{\underline{\box0}}\\}
\def\|{\verb|}

\setcounter{巻数}{59}
\setcounter{号数}{1}
\setcounter{page}{1}


\受付{2016}{3}{4}
\再受付{2015}{7}{16}   %省略可能
\再再受付{2015}{7}{20} %省略可能
\再再受付{2015}{11}{20} %省略可能
\採録{2016}{8}{1}




\begin{document}


\title{hoge}

\etitle{hoge}

\affiliate{WA}{和歌山大学システム工学部\\
Wakayama University, Faculty of Systems Engineering, Sakaedani 930, Wakayama-city 640-8510, Japan}


\author{大森 楓己}{Mizuki Uenaka}{WA}[uenaka.mizuki@g.wakayama-u.jp]
\author{伊原 彰紀}{Akinori Ihara}{WA}[ihara@wakayama-u.ac.jp]

\begin{abstract}
hogehoge
\end{abstract}


\begin{jkeyword}
hogehoge
\end{jkeyword}

\begin{eabstract}
hogehoge
\end{eabstract}

\begin{ekeyword}
hogehoge
\end{ekeyword}

\maketitle

%%%%%%%%%%%%%%%%%%%%%%%%%%%%%%%
\section{はじめに}
%%%%%%%%%%%%%%%%%%%%%%%%%%%%%%%

hoge

%%%%%%%%%%%%%%%%%%%%%%%%%%%%%%%
\section{おわりに}
\label{sec:conclusion}
%%%%%%%%%%%%%%%%%%%%%%%%%%%%%%%

hoge


\bibliographystyle{ipsjunsrt}
\bibliography{bibfile_IPSJjournal_Uenaka}

\vspace{-4mm}

\begin{biography}
\profile{n}{大森 楓己}{2024年和歌山大学システム工学部在学中,ソフトウェア工学,特にプログラム検証の研究に従事.}
%
\profile{m}{伊原 彰紀}{2009年奈良先端科学技術大学院大学情報科学研究科博士前期課程修了.2012年同大学博士後期課程修了.2012年同大学情報科学研究科助教.2018年和歌山大学システム工学部講師.博士(工学).ソフトウェア工学,特にオープンソースソフトウェア開発・利用支援の研究に従事.電子情報通信学会,IEEE各会員.}
\end{biography}



\end{document}
