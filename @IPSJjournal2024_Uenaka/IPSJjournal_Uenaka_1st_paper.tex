\documentclass[submit]{ipsj}
%\documentclass{ipsj}

%\usepackage{graphicx}
%\usepackage[dvipdfmx]{graphicx,color}
%\usepackage[dvips]{graphicx}
\usepackage[dvipdfmx]{graphicx}
\usepackage{latexsym}
\usepackage{url}
\usepackage{listings}
\usepackage{multirow}
\usepackage{xcolor}
\usepackage{colortbl}


\newcommand{\todo}[1]{\colorbox{yellow}{{\bf TODO}:}{\color{red} {\textbf{[#1]}}}}

%ここからソースコードの表示に関する設定
\definecolor{darkgray}{rgb}{.4,.4,.4}
\definecolor{purple}{rgb}{0.65, 0.12, 0.82}

\lstdefinelanguage{JavaScript}{
  keywords={typeof, new, true, false, catch, function, return, null, catch, switch, var, if, in, while, do, else, case, break},
  keywordstyle=\color{blue}\bfseries,
  ndkeywords={class, export, boolean, throw, implements, import, this},
  ndkeywordstyle=\color{darkgray}\bfseries,
  identifierstyle=\color{black},
  sensitive=false,
  comment=[l]{//},
  morecomment=[s]{/*}{*/},
  commentstyle=\color{purple}\ttfamily,
  stringstyle={\small\ttfamily},
  morestring=[b]',
  morestring=[b]"
}

\lstset{
  basicstyle={\ttfamily},
  identifierstyle={\small},
  commentstyle={\smallitshape},
  keywordstyle={\small\bfseries},
  ndkeywordstyle={\small},
  stringstyle={\small\ttfamily},
  frame={tb},
  breaklines=true,
  columns=[l]{fullflexible},
  numbers=left,
  xrightmargin=0zw,
  xleftmargin=3zw,
  numberstyle={\scriptsize},
  stepnumber=1,
  numbersep=1zw,
  lineskip=-0.5ex
}
%ここまでソースコードの表示に関する設定

\def\Underline{\setbox0\hbox\bgroup\let\\\endUnderline}
\def\endUnderline{\vphantom{y}\egroup\smash{\underline{\box0}}\\}
\def\|{\verb|}

\setcounter{巻数}{59}
\setcounter{号数}{1}
\setcounter{page}{1}


\受付{2016}{3}{4}
\再受付{2015}{7}{16}   %省略可能
\再再受付{2015}{7}{20} %省略可能
\再再受付{2015}{11}{20} %省略可能
\採録{2016}{8}{1}




\begin{document}


\title{リリースまでの期間に応じて優先的に検証/導入\\されるコードレビューチケットの特定}

\etitle{Prioritized Identification of Reviewed or Merged\\Review Tickets for Each Period before Release}

\affiliate{WA}{和歌山大学システム工学部\\
Wakayama University, Faculty of Systems Engineering, Sakaedani 930, Wakayama-city 640-8510, Japan}


\author{上中 瑞稀}{Mizuki Uenaka}{WA}[uenaka.mizuki@g.wakayama-u.jp]
\author{伊原 彰紀}{Akinori Ihara}{WA}[ihara@wakayama-u.ac.jp]

\begin{abstract}
オープンソースソフトウェア (OSS) 開発において,変更提案されたソースコードの可読性や欠陥の有無を評価するコードレビューは,ソフトウェアの品質維持のために重要な役割を担っている.しかし,コードレビューはソフトウェア開発プロセスの一連の作業において,時間,作業量ともに高いコストを要する作業であるため,昨今ではオンラインコードレビューサービスを導入することで,コードレビュー作業の効率化を図るプロジェクトが増加している.オンラインコードレビューサービスを導入するOSS開発では,日々多くのコードレビューを依頼するコードレビューチケットが提出され,検証者は優先的にコードレビューするチケットを選択する.従来研究では,チケット報告時に得られる特徴に基づき,検証者らが優先的に検証するチケットを特定する手法を提案しており,報告時期によって優先順位の変動が小さい変更内容に関するチケットに対して有用な手法となっている.しかし,本研究の事前分析において,複数プロジェクトでリリースまでの期間に応じて優先的に検証/導入されるコードレビューチケットの特徴量に違いがあることを明らかにした.このような結果から,従来研究の予測モデルはリリースまでの期間などの開発状況に応じて優先度が日々変動するチケットの優先順位の決定には適していないと考えられる.

本論文では,従来手法と同様にチケットおよび開発者の特徴を説明変数とする予測モデルと,追加で開発状況を説明変数として学習することで優先度が日々変動するチケットに対応した予測モデルの2種類のモデルを構築する.ウィンドウサイズを2週間とするスライディングウィンドウを定義し,1日ごとに学習またはテストを行うことで,両モデルの予測性能を算出する.本研究では,OpenStackの6つのコアコンポーネントプロジェクトのチケットデータをケーススタディとして2つのリサーチクエスチョン (RQ) を検証した.RQ1では,従来手法と同様にチケットおよび開発者の特徴を説明変数とする予測モデルの予測性能がリリースまでの期間内でどの程度変化するかを分析した.結果として,従来手法の予測性能の変動幅は,レビューが開始されるチケットの予測において0.16〜0.53,マージされるチケットの予測において0.11〜0.95であり,リリースまでの期間内で大きく変化することを明らかにした.また,RQ2では,追加で開発状況を説明変数とする予測モデルの予測性能がリリースまでの期間によってどの程度変化するかを分析した.結果として,提案モデルのF値はベースラインモデルと比べて,レビューが開始されるチケットの予測で0.02〜0.09,マージされるチケットの予測で0.07〜0.14向上した.また,提案モデルで予測性能が向上した期間において,提案モデルのみで正しく判別できたチケットの特徴量重要度を分析した結果,多数のプロジェクトにおいて開発状況に関する説明変数は予測における重要度が高いことを明らかにした.

本論文で明らかにしたリリースまでの期間などの開発状況による予測性能への影響より,優先度が日々変動するチケットの優先順位の決定に将来的に寄与できると考える.
\end{abstract}


\begin{jkeyword}
コードレビュー,機械学習,ソフトウェアリポジトリマイニング,OSS開発
\end{jkeyword}

\begin{eabstract}
hogehoge
\end{eabstract}

\begin{ekeyword}
hogehoge
\end{ekeyword}

\maketitle

%%%%%%%%%%%%%%%%%%%%%%%%%%%%%%%
\section{はじめに}
%%%%%%%%%%%%%%%%%%%%%%%%%%%%%%%

hoge

%%%%%%%%%%%%%%%%%%%%%%%%%%%%%%%
\section{おわりに}
\label{sec:conclusion}
%%%%%%%%%%%%%%%%%%%%%%%%%%%%%%%

hoge


\bibliographystyle{ipsjunsrt}
\bibliography{bibfile_IPSJjournal_Uenaka}

\vspace{-4mm}

\begin{biography}
\profile{m}{伊原 彰紀}{2009年奈良先端科学技術大学院大学情報科学研究科博士前期課程修了.2012年同大学博士後期課程修了.2012年同大学情報科学研究科助教.2018年和歌山大学システム工学部講師.博士(工学).ソフトウェア工学,特にオープンソースソフトウェア開発・利用支援の研究に従事.電子情報通信学会,IEEE各会員.}
%
\profile{n}{上中 瑞稀}{2024年和歌山大学システム工学部在学中,ソフトウェア工学,特にプログラム検証の研究に従事.}
\end{biography}



\end{document}
