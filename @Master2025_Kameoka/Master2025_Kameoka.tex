%%
%% 修士論文
%%
%% USE 機能的に等しい
%% not 外的振る舞いが等しい
%%
%% USE 大規模言語モデル
%% not LLM
%%
%% USE 低下させる
%% not 下げる
%%

%%%%%%%%%%%%%%%%%%%%%%%%%%%%%%%%%%%%%%%%%%%%%%%%%%%%%%%%%%%%%%%%

%%
%% 利用するコマンドの準備
%%

%% 必須パッケージ
\documentclass[11pt]{jreport}
\usepackage{wuse_thesis}
\usepackage{indentfirst}

%% 追加パッケージ
%\usepackage{graphicx}
\usepackage{comment}
\usepackage[dvipdfmx]{graphicx}
\usepackage{url}

\usepackage{latexsym}

\usepackage{cite}
\usepackage{amsmath,amssymb,amsfonts}
\usepackage{algorithmic}
\usepackage{textcomp}
\usepackage{xcolor}
\usepackage{listings}
\usepackage{makecell}
\usepackage{booktabs}
\usepackage{longtable}
\usepackage{tabularx}
\def\BibTeX{{\rm B\kern-.05em{\sc i\kern-.025em b}\kern-.08em
    T\kern-.1667em\lower.7ex\hbox{E}\kern-.125emX}}

\newcommand{\todo}[1]{\colorbox{yellow}{{\bf TODO}:}{\color{red} {\textbf{[#1]}}}}


\lstset{
basicstyle=\small\ttfamily,
abovecaptionskip=0pt,
captionpos=b,
frame=tb,
framexleftmargin=2em,
numbers=left,
numberstyle={\scriptsize},
xleftmargin=\parindent
}

%ListingのキャプションがFigureになってしまうのをListingに直すコマンド
\usepackage{caption}
\makeatletter
\let\MYcaption\@makecaption
\makeatother
\usepackage{caption}
\makeatletter
\let\@makecaption\MYcaption
\makeatother

\def\Underline{\setbox0\hbox\bgroup\let\\\endUnderline}
\def\endUnderline{\vphantom{y}\egroup\smash{\underline{\box0}}\\}
\def\|{\verb|}
%


%%%%%%%%%%%%%%%%%%%%%%%%%%%%%%%%%%%%%%%%%%%%%%%%%%%%%%%%%%%%%%%%

%%
%% 主に表紙を作成するための情報
%%

%% タイトル(修論の場合は英語表記も指定)
\title{複数ソフトウェアの修正履歴の統合による\\コーディング規約違反の修正予測精度の評価}
\etitle{Evaluating the Prediction Accuracy of Coding Standard Violation Fixes through Integrated Software Change History Data}

%% 著者名(修論の場合は英語表記も指定)
\author{亀岡 令}
\eauthor{Ryo Kameoka}

%% 修士論文(M2用)
\master

%% 学科・クラスタ
\department{システム工}

%% 学生番号
\studentid{S2420033}

%% 卒業年度
\gyear{2025}

%% 論文提出日(修士の場合は月まで)
\date{2026年2月}
\edate{February 2026}


%%%%%%%%%%%%%%%%%%%%%%%%%%%%%%%%%%%%%%%%%%%%%%%%%%%%%%%%%%%%%%%%

%%
%% 表紙・概要・目次
%%

\begin{document}

%% 表紙
\maketitle

%% 概要
\begin{abstract}
ほげほげ\cite{HowFar}
\end{abstract}

%% 目次
\tableofcontents

%% 図目次
%\listoffigures

%% 表目次
%\listoftables

\newpage
\pagenumbering{arabic}


%%%%%%%%%%%%%%%%%%%%%%%%%%%%%%%%%%%%%%%%%%%%%%%%%%%%%%%%%%%%%%%%
\chapter{はじめに}

コーディング規約は,ソースコードの記述方法に関する禁止事項や推奨事項を定め, 可読性や保守性を高めるためのコーディングスタイルの取り決めである.
ソフトウェア開発プロジェクトは,実装に参加する開発者にコーディングスタイルの共通化や, ソースコードの最適化のためにコーディング規約の遵守を促すことで, 可読性や保守性の高いソースコードを共有することができる\cite{EffectsSAT}. 
コーディング規約は,プログラミング言語ごとに規約のテンプレートが存在し,開発者はテンプレートを参考に規約の最適化を行っている.
具体的には,開発者にとってのソースコードの理解容易性や,バグの早期発見に貢献している\cite{Beller2}\cite{Johnson}\cite{Beller}.

コーディング規約に違反しているコード断片(以降,規約違反コード)を機械的に検出するために,多くのプロジェクトは静的解析ツールを使用している.
静的解析ツールは, ソースコード中の規約違反コードを正規表現などによりソースコードを実行することなく検出できるため, 継続的インテグレーションのプロセスの1つとして使用されることも多い.
%\todo{プログラミング言語ごとに規約違反のテンプレートを提供していると書いても良さそう}
プログラミング言語別に提供される規約違反のリストには,規約違反の種類を多数定義しているため,一部の違反のみ有効にしているプロジェクトも存在する\todo{引用?}しかし,開発者が有効に設定されたすべての規約違反を修正しておらず,従来研究による実証実験によると35から91パーセントの規約違反コードは修正されていない\cite{静的解析ツールの修正率}.静的解析ツールが検出する膨大な規約違反コードの中から,修正を要する規約違反コードを分類することは開発効率の低下につながるため,修正が必要な規約違反コードを分類するための研究が行われている\cite{Nguyen}.


従来研究では,大量に検出される規約違反コードを,修正すべき違反とそうでない違反に自動分類する手法が提案されている\cite{JyuraiPre}.
また,静的解析ツールの検出結果を修正すべき順番にランキングづけをする研究も数多く行われている.

従来研究では、特定のプロジェクトにおいて,過去の規約違反コードの修正履歴を学習しモデルを構築する手法を提案している.当該手法は,学習データの総数は多くても正例が少ないプロジェクトで予測精度が低くなる課題を有する.Tabassumらは,不具合予測やソースコード自動修正の分野で,データ不足やコールドスタート問題に対応するため,異なるプロジェクトの開発履歴を利用して学習データを補う手法を提案している\cite{Tabassum}.規約違反コードの検出においても,異なるプロジェクトのデータを利用すれることでコールドスタート問題の解消が期待できる一方で,実装方針の異なるプロジェクトのデータを学習することで精度が低下する課題も考えられる.


本研究では,複数プロジェクトの開発履歴を予測モデルの学習に用いることによる規約違反コードの修正予測精度を明らかにする.提案手法として,2種類の予測モデルを構築する.1つは,複数プロジェクトの開発履歴を単純に結合して学習する手法である(全学習手法).もう1つは,%\todo{XX}
規約違反の修正傾向が類似する複数プロジェクトの開発履歴を結合して学習する手法である(選定学習).本研究の提案は、著者らが以前に提案した手法\cite{mine}\cite{mine_live}を拡張し,類似プロジェクトのみを統合する選定学習を新たに導入した点にある.評価の結果,一部のプロジェクトにおいて適合率に改善が見られた.%\todo{一部のように弱気に書くか悩みどころ.}

% \todo{データセット拡張した結果に書き換える}
従来研究ではLLMを用いて修正要否を検出すると,修正の必要がない箇所まで変更されることが指摘されている\cite{LLMで検出}.本研究の結果をLLMのファインチューニングに用いることで,修正すべき違反のみを検出し,修正することが期待される.

続く\ref{chap:background}章では, コーディング規約違反の検出と, 従来研究について述べる. \ref{chap:approach}章では本研究における実験の全体像と提案手法について述べる. \ref{chap:rq1},\ref{chap:rq2}章では評価結果を示し, \ref{chap:consideration}章では考察を述べる. \ref{chap:heuristic}章では本研究の妥当性への脅威について述べ, \ref{chap:end}章でまとめる.

\chapter{背景と関連研究}\label{chap:background}



\section{可読性を担保するコーディング規約}

ソースコードの可読性や保守性といった品質を高水準で維持することは重要である.
可読性の高いソースコードを書くことで,ソースコードの理解の促進やバグの混入の予防などの効果が期待できる.
また,品質の劣るコードには,良質なコードの約15倍の欠陥が含まれる可能性があることが報告されている\cite{コード品質の重要性}.

特に大規模なソフトウェア開発では,ソースコードを実装した後に,実装者とは異なる開発者が,新規実装されたソースコードにバグなどの問題が含まれていないかを確認するレビュー作業が発生する.
可読性の高いコードはレビュー作業の短縮にもつながる\todo{参考}.
つまり,可読性の高いソースコードを書くことは,ソフトウェア開発サイクル全体の促進につながる.

そこで,開発者はソフトウェア開発にコーディング規約を導入する.
コーディング規約は,複数人によるソフトウェア開発においてソースコードの記述方法を共通化するための指針である.
コーディング規約には,変数やクラスの命名規則,関数の長さや複雑度の上限といった可読性に関するルール,エラーの原因となる記述の検出,禁止事項,制限事項,推奨事項などが含まれる.

コーディング規約は,プログラミング言語ごとに異なる内容や基準を持ち,多くの種類の規約が存在する.
例えば,Python言語にはpycodestyle,Java言語にはCode Conventions for the Java Programming LanguageやGoogle Java Style Guide,JavaScript言語にはGoogle JavaScript Style GuideやAirbnb JavaScript Style Guideなどが代表的な例ある.

開発者はコーディング規約を遵守したコーディングを行うことによって,一定のコードの品質を保つことができる.
そのため,実装したコードはレビュアーが見る際には,可読性が担保されたコードを見ることが可能となる.


\section{静的解析ツール}

ソースコードがコーディング規約に従っているかや,違反しているコードがどこに存在するのかを特定するために静的解析ツールが利用される.
コーディング規約に従っていないコードを検出する静的解析ツールも各プログラミング言語ごとに存在する.
代表的な例としては,PythonのFlake8やJavaScriptのESLint,JavaのCheckstyleが挙げられる.
また,これらの静的解析ツールは存在する,規約の中から,どの規約への違反を検出対象とするかを任意で決定することができる.
検出する違反の種類を制御するためには,それぞれの静的解析ツールごとに定められたディレクトリに決められたファイル名のテキストファイルを配置し,設定を記述することで制御ができる.

ソフトウェア開発者は,各プロジェクトが参照する規約,違反検出時の出力形式を調整を加えながら,静的解析ツールを用いて規約違反コードを検出している.
プロジェクトの中には,継続的インテグレーション (CI) 環境に組み込むことで開発効率の向上を実現している\cite{ci/cd}.
静的解析ツールは,ソースコード中の規約違反コードを網羅的に検出するが,その数は膨大であることが多く,開発者がそのすべてを確認,修正,保守するには多大なコストと労力を要する\cite{UsingStaticAnalysisTools2}.さらに,優先的に修正すべき違反を区別するためには,経験やソースコードへの深い理解が求められる\cite{shuseisarenai}.

\section{従来研究}\label{sec:zyuuraikennkyuu}


従来研究では,静的解析ツールによる大量に検出した規約違反コードの中から,機械学習を用いて修正すべき規約違反コードを分類する手法が提案されている\cite{JyuraiPre}\cite{beizu}.
Kimらは,静的解析ツールの出力をベイジアンネットワークに基づいて解析し,修正すべき規約違反コードを分類する手法を提案している\cite{beizu}.また,検出された違反に対する修正優先度付けする研究も数多く行われている\cite{Wang}\cite{Qing}\cite{HowFar}.これらの研究では,予測対象プロジェクトにおける規約違反の修正履歴を学習データとして用い,新たに検出された違反を評価対象とする機械学習モデル(規約違反コードの修正要否判定モデル)の構築と評価が行われている.



% 従来研究では,静的解析ツールによる大量の違反検出が開発効率の低下を招くことに対処するため,機械学習を用いて修正の優先度を予測する手法が提案されている.たとえば,Ruthruffらは,優先的に修正すべき違反を特定するための機械学習モデルを構築している\cite{JyuraiPre}.また,Kimらは,静的解析ツールの出力をベイジアンネットワークに基づいて解析し,違反の修正優先度を予測する手法を提案している\cite{beizu}.
% そのほかにも,検出された違反に対する修正優先度付けや修正要否の予測に関する研究が数多く行われている\cite{Wang}\cite{Qing}\cite{HowFar}.これらの研究では,予測対象プロジェクトにおける過去の規約違反修正履歴を学習データとし,新たに検出された違反を評価対象とすることで,予測モデルの構築と評価が行われている.

従来研究では,評価対象プロジェクトの修正履歴を学習データとして用いてモデルを構築している.これは各プロジェクトでコーディングスタイルや開発体制が異なり,評価対象と同一プロジェクトの修正履歴を学習することで高い精度が得られることが示されている.
% 従来研究では,一般に,評価対象プロジェクトの修正履歴を学習データとして用いて予測モデルを構築する.プロジェクトごとにコーディングスタイルや開発体制が異なるため,対象と同一プロジェクトの履歴を学習する方が,他プロジェクトの履歴を用いるよりも高い精度が得られると考えられている.
しかし,同一プロジェクトの学習データには,出現する違反の種類や修正率に偏りが生じることがある\cite{Panichella}
%\todo{この論文では予測まで行っていない?そのため,次の文であるように予測精度が低下する可能性,という表現になっているう?}
.
この結果,修正された違反(正例)と修正されなかった違反(負例)の数に不均衡が生じ,予測精度の低下を招く可能性がある.また,予測対象プロジェクトの違反修正履歴が小規模の場合,学習データが小規模になり十分な学習ができない可能性がある.

本研究では,評価対象以外のプロジェクトの修正履歴を学習データに含めることで,規約違反コードの修正予測精度の向上を目的とする.

%予測モデルの構築において,評価対象とは異なるプロジェクトのコーディング規約違反修正履歴を学習データとして用いることが,修正要否予測精度に与える影響を明らかにすることを目的とする.
従来研究では,名倉らが複数プロジェクトのデータを用いて規約違反の発生件数の増減を予測している.一方で,個々の規約違反コードの修正有無の予測は対象としていないため,本研究とは異なるが,研究動機は類似している\cite{nagura}.
異なるプロジェクトの修正履歴を用いることで,学習データの拡張による予測精度の向上が期待されるが,異なるコーディングスタイルを有するプロジェクトの修正履歴は予測精度を低下することも考えられる.
% すべての他プロジェクトのデータが有益とは限らない.すなわち,修正傾向が異なるプロジェクトのデータは,予測に対してノイズとなる可能性がある.

そこで本研究では,評価対象プロジェクトと修正傾向が類似するプロジェクト選定し,修正履歴を統合することで学習データの拡張し,予測モデルの精度向上を目指す.

\section{大規模言語モデルを利用した関連研究}

\begin{figure}[ht]
    \centering
    \includegraphics[width=0.8\linewidth]{@Master2025_Kameoka/fig/LLM_zyuurai.pdf}
    \caption{大規模言語モデルを活用したコード品質を低下させているコードの分類と自動修正の関連研究}
    \label{fig:LLM_zyuurai}
\end{figure}

近年の大規模言語モデルをの発展に伴い,生成AIを用いた静的解析ツールの解析結果の分類を行う研究が行われている.Wadhwaらは,プログラム中に存在する,コーディング規約に違反しているコードの誤検知の分類と,修正パッチの自動作成の研究を行っている\cite{LLMで検出}.Wadhwaらの研究では,静的解析ツールを用いた検出結果を分類するという手法と,静的解析ツールを用いずに違反個所の検出から生成AIに行わせる手法を検証している.
また,真の検出箇所で判定された場合,違反コードの修正パッチの作成まで行っている点が特徴的である.

Wadhwaらの研究結果では,静的解析ツールが検出可能なコード品質を低下させているコードと,大規模言語モデルが検出可能なコード品質を低下させているコードでは.検出可能な種類が異なるという結果が報告されている.
また,大規模言語モデルに修正パッチの生成を行わせるタスクでは,どれだけプロンプトで制限を行った場合であっても,関係のないコードまで修正を行ってしまうというデメリットも報告されている.
Wadhwaらの研究では,静的解析ツールと生成AIでは,コード品質を低下させている個所の検出可能な種類が異なるため,静的解析ツールと生成AIの併用を勧めている.

本研究では,\ref{sec:zyuuraikennkyuu}節で示した従来研究と同様に,予測対象のプロジェクトの開発履歴を学習に使用するため,プロジェクトごとの修正の特徴をとらえた学習が可能である.
Wadhwaらの研究では,大規模言語モデルにソースコードと静的解析ツールの結果を渡しているのみなので,プロジェクトごとに最適化はされていない.
本研究では,開発履歴の学習を行うためプロジェクトごとに最適化を行うことができるという点で異なる.

\section{RQs}

本研究では静的解析ツールによって検出されたコーディング規約違反の修正要否を予測するにあたり,予測対象以外の複数プロジェクトの開発データを学習に使用することによる修正要否予測精度の向上を目的とする.
本分野において,先行研究は多く行われている.
しかし,一般的に予測対象のプロジェクトの過去の開発データのみを学習に使用しており,検証対象のプロジェクトには十分な開発履歴のあるプロジェクトを選択している.

そこで,本研究では予測対象以外の複数プロジェクトの学習データを機械学習モデルに学習させることによる,コーディング規約違反の修正要否予測の予測精度への影響を明らかにするために,以下のRQsを設定する.

\begin{itemize}
    \item RQ1: 複数プロジェクトを学習に用いることによって、修正予測精度は向上するか
    \item RQ2: 複数プロジェクトを学習することで予測精度が改善するプロジェクトの特徴とは何か
\end{itemize}

RQ1では,提案手法である,複数プロジェクトのデータを学習に使用することによる,コーディング規約違反の修正要否予測精度への影響を明らかにする.
また,アプローチの一つである予測対象プロジェクトと類似したプロジェクトを収集する手法の妥当性を検証する.

RQ2では,従来手法の課題として挙げた,予測対象プロジェクトの過去の開発データに正例が少ない場合に.提案手法によって予測精度が向上するのかについて分析を行う.
また,予測精度が向上しやすい,つまり提案手法が有効に働くプロジェクトに特徴があるのかについて分析を行う.

\chapter{規約違反コードの修正要否判定方法とモデルの構築方法}\label{chap:approach}


\section{目的変数の計測}

\begin{figure}[ht]
    \centering
    \includegraphics[width=1.0\linewidth]{@IPSJjournal2025_Kameoka/fig/target_valiable.pdf}
    \caption{説明変数と目的変数の計測位置}
    \label{fig:mokutekihensu}
\end{figure}

\begin{table}[ht]
    \centering
    \caption{目的変数の正例と負例の分類}
    \label{tab:pos_neg}
    \scalebox{0.85}{
    \begin{tabular}{l|l}
        \hline
        分類 & 説明 \\ \hline
        負例(ケース1) & 規約違反がそのまま残存しているコード断片 \\
        正例(ケース2) & 違反コードが削除された断片 \\
        正例(ケース3) & 規約違反が修正されたコード断片 \\ \hline
    \end{tabular}
    }
\end{table}

図\ref{fig:mokutekihensu}は,本研究で構築するモデルの説明変数および目的変数の計測位置を示す.
目的変数は,分析期間内に検出した規約違反コードが,分析期間の最終リビジョンまでに修正される(正例)か否か(負例),言い換えると静的解析ツールにより規約違反として検出されるか否かと定義する.
表\ref{tab:pos_neg}は,図\ref{fig:mokutekihensu}に示す3つのケースを例として,目的変数の分類を示す.ケース1は負例,ケース2およびケース3は正例とする.

具体的な目的変数の計測方法について図\ref{fig:mokutekihensu}を用いて説明する.
\begin{enumerate}
 \item プロジェクトの開発履歴であるGithubリポジトリを取得する
 \item 取得したリポジトリ内で,分析開始点までバージョンを戻す
 \item 静的解析ツールをすべてのPythonファイルに対して実行する
 \item コーディング規約違反を記録する(図\ref{fig:mokutekihensu}のケース1とケース2が記録される)
 \item リポジトリを次のコミットの状態にする\label{mokuteki_start}
 \item 再度静的解析ツールを実行する
 \item 新規コーディング規約違反があれば,追加記録する(図\ref{fig:mokutekihensu}のケース3が記録される)
 \item 以前記録したものが消えていれば,正例として記録する(図\ref{fig:mokutekihensu}のケース2,3が正例として記録される)\label{mokuteki_end}
 \item 分析終了地点まで\ref{mokuteki_start}から\ref{mokuteki_end}を繰り返し,残存している違反を負例とする
\end{enumerate}


\section{説明変数の定義}

本研究で使用する説明変数の一覧を表\ref{tab:variables}に示す.
説明変数は,規約違反が初めて検出されたリビジョンのソースコードから計測した特徴量28種類(行数,コメント行数,循環的複雑度など)を使用する.
加えて,規約違反IDをOne-hotベクトルとしてを加えたものを説明変数とする.
使用する説明変数は,従来研究において,静的解析ツールの誤検知分類タスクにおける重要な説明変数に関する研究が行われているため,その従来研究に基づき決定した\cite{Wang}.

従来研究では研究対象言語として,Javaを主要な開発言語としているプロジェクトを利用している.
本研究では研究対象言語にPythonを採用している.
そのため,従来研究で取得していたコードメトリクスの中で,対象言語の違いにより一部取得できないものがあったため,従来研究に基づいた説明変数ではあるが,完全一致はしていない.


\section{予測モデルの構築方法}

\begin{figure}[ht]
    \centering
    \includegraphics[width=0.8\linewidth]{@IPSJjournal2025_Kameoka/fig/syuhou.pdf}
    \caption{提案手法の全体像}
    \label{fig:Teiannsyuhou}
\end{figure}

図\ref{fig:Teiannsyuhou}は,本研究で提案する機械学習モデルの構築および評価の概略図を示す.本研究では,予測対象プロジェクトにおける規約違反コードの修正履歴に加え,他のプロジェクトの修正履歴を学習データに含めることにより,修正を要する規約違反コードを予測するモデルを提案する.
具体的には3種類のモデル(全学習モデル,選定学習モデル-欠損値補完なし,選定学習モデル-欠損値補完あり)と,比較手法として,従来研究で提案されている,予測対象プロジェクトにおける修正履歴のみを学習データとして使用する予測モデル(単一学習モデル)も構築し,4つのモデルの予測精度を比較する.


\begin{enumerate}
    \item \textbf{従来手法 - 単一学習モデル}
    \item \textbf{提案手法 - 全学習モデル}
    \item \textbf{提案手法 - 選定学習モデル(欠損値補完なし)}
    \item \textbf{提案手法 - 選定学習モデル(欠損値補完あり)}
\end{enumerate}


\subsection{単一学習}

予測対象のプロジェクトの修正履歴のみを学習データとして用いるモデル\cite{JyuraiPre}.
図\ref{fig:Teiannsyuhou}の単一学習に示すように,従来研究の手法を真似たモデルで,予測対象プロジェクトの過去の開発データのみを学習データとする
例えば,プロジェクトAの修正要否を予測するにあたり,プロジェクトAの過去の開発データのみを学習に利用する
つまり予測対象のプロジェクトごとに予測モデルが異なる.


\subsection{全学習}

本研究で評価対象とする全てのプロジェクトの修正履歴を学習データとして用いるモデル.
図\ref{fig:Teiannsyuhou}の全学習に示すように,学習データ内に存在するすべてのプロジェクトの学習データをすべて学習に使用する予測モデルである.
予測対象プロジェクトによらず,一つの予測モデルですべての修正要否を予測する.
具体的には,プロジェクトAの修正要否を行う際も,プロジェクトBの修正要否予測を行う際にも,修正要否予測に使用するモデルは同一のモデルが用いられる.


\subsection{選定学習(欠損値補完なし)}

予測対象プロジェクトの修正履歴に加え,当該プロジェクトが設定する規約,およびその修正率が類似するプロジェクトを選定し,その修正履歴も学習データとして用いるモデル.ただし,プロジェクト間で一方のみが設定する規約の場合,他方の修正率は欠損値であり,本モデルでは,そのIDは類似度算出に使わないものとする.(選定方法の詳細は\ref{subsec:選定方法}節参照)


\subsection{選定学習(欠損値補完あり)}

選定学習モデル(欠損値補完なし)と同様に予測対象プロジェクトの修正履歴に加え,選定したプロジェクトの修正履歴も学習データとして用いるモデル.
ただし,欠損値は,欠損値を含むプロジェクトにおいて他の規約における平均修正率で補完する.(選定方法の詳細は\ref{subsec:選定方法}節参照)


\subsection{学習プロジェクトの選定方法}\label{subsec:選定方法}


全学習モデル1種および選定学習モデル3種では,予測対象プロジェクトの修正履歴に加え,他のプロジェクトの修正履歴も学習データに含めることにより正例データを拡張する.全学習モデルではデータセット内の全プロジェクトを対象とし,選定学習モデルでは,予測対象プロジェクトが設定する規約,およびその修正率が類似するプロジェクトの修正履歴のみを学習データに含める.\todo{プロジェクト間の類似度にIDごとに取り組んでいる研究はある?ID別に計測することの動機をここで書きたい.}

本研究では,各プロジェクトで収集するコミット履歴のうち,古い履歴8割を学習データとし,新しい2割を検証データとする.プロジェクト選定は,学習データにおいて予測対象プロジェクトが設定する規約,およびその修正率を計測し,他のプロジェクトとの類似度を算出する.プロジェクトの選定方法は,予測対象プロジェクトと類似するコーディング規約違反を設定していること,またその修正率が類似していることとする.各規約違反における修正率の算出式は式(\ref{eq:fixrate})に示す.

\begin{equation}
\label{eq:fixrate}
\text{修正率} = \frac{\text{違反の修正された数}}{\text{違反を検出した数}}
\vspace{10pt}
\end{equation}



プロジェクト間の類似度は,各プロジェクトが設定する規約の類似度,および規約の修正率の類似度をそれぞれ算出する.
各プロジェクトが設定する規約の類似度は,本研究で対象とするFlake8で検出可能な規約違反を対象とし,プロジェクトの規約設定有無(バイナリーデータ)の類似度を判定するためジャッカード係数を用いる.
修正率の類似度は,各プロジェクトが設定する個々の規約の修正率(スカラーデータ)の類似度を判定するためユークリッド距離を用いる.

なお,規約の修正率は,一方のプロジェクトのみが規約を設定している場合,他方のプロジェクトでは欠損値となるため,次の2通りの方法で欠損値補完し,類似度を計測する.

%ユークリッド距離の計算においては,ある規約違反が一度も発生していない場合,修正率が欠損値となる.この場合に備え,以下の2通りの処理を実施する.
\begin{itemize}
    \item 欠損値を計算から除外してユークリッド距離を測定する方法(選定学習モデル(欠損値補完なし))
    \item 欠損値を当該プロジェクトの平均修正率で補完し,ユークリッド距離を測定する方法(選定学習モデル(欠損値補完あり))
\end{itemize}

欠損値の補完方法には様々な手法が提案されているが,規約違反コードの修正要否を予測するモデルにおいて,複数のプロジェクトの修正履歴を統合する従来研究が確認できないため,本研究の結果に基づき他の欠損値補完手法を検討する.

\begin{table}[h]
    \centering
    \caption{プロジェクト別規約遵守状況}
    \label{tab:example_sim_table}
    \begin{tabular}{lcccc}
        \hline
         & 規約1 & 規約2 & 規約3 & 規約4 \\
        \hline
        プロジェクトA & 0.70 & 0.25 & 0.40 & 0.20 \\
        プロジェクトB & 0.60 & - & 0.30 & 0.00 \\
        プロジェクトC & - & 0.70 & - & 0.60 \\
        \hline
    \end{tabular}
\end{table}

\begin{equation}
\label{eq:jaccard_exm}
Jac(A, B) = \frac{3}{4} = 0.75
\end{equation}

\begin{equation}
\label{eq:euc_exm}
Euc(A, B) = \sqrt{(0.7 - 0.60)^2 + (0.3 - 0.2)^2 + (0.2 - 0.3)^2}
\end{equation}

\begin{equation}
\label{eq:euc_exm2}
Euc'(A, B) = \sqrt{(0.7 - 0.60)^2 + (0.25 - 0.30)^2 + (0.3 - 0.2)^2 + (0.2 - 0.3)^2}
\vspace{10pt}
\end{equation}

具体的な類似プロジェクトの選定方法の説明を行う.
表\ref{tab:example_sim_table}に,プロジェクトごとの各規約違反に対する修正率の例を示す.
表中の「規約1」は,プロジェクトAでは規約1が0.70の割合で修正されており,プロジェクトBでは0.65の割合で修正されている.
そして,プロジェクトCでは規約1への違反が一度も発生していないため,欠損値が入っている.
プロジェクトAとプロジェクトBの類似度を計算する場合,発生しているコーディング規約違反のうち,規約2以外が共通して発生しているのでジャッカード係数は式\ref{eq:jaccard_exm}のように求められる.

選定学習手法(欠損値補完なし)の修正率のユークリッド距離は,表の格値が修正率を示しているため,式\ref{eq:euc_exm}のように求められる.
最後に,欠損値補完を行う場合のユークリッド距離を求める場合,規約2がプロジェクトBでは発生していないため,他規約の修正率の平均値で補完して計算を行う.
その結果,選定学習手法(欠損値補完あり)の類似プロジェクト分析に利用するユークリッド距離は式\ref{eq:euc_exm2}のように求められる.


実験を行う際には,表\ref{tab:example_sim_table}のような修正率の表を全プロジェクトの全コーディング規約に拡張したものを計算し,類似度の計算に利用する.
プロジェクト間でそれぞれの指標が類似するか否かは,実証データに基づき閾値を決定する.

% 修正率を基に,ジャッカード係数とユークリッド距離の2種類の類似度指標を算出し,それぞれに閾値を設定することで,類似度が高いと判断されたプロジェクトのみを学習対象として選定する.


\subsection{コーディング規約違反の修正要否予測の評価方法}

本研究では,コーディング規約違反の発生地点でのデータから,修正要否の予測を行う.
各プロジェクトの規約違反修正履歴の評価用データを用い,機械学習モデルで予測を行い,その結果の評価を行う.
評価には,適合率(式\ref{eq:precision}),再現率(式\ref{eq:recall}),F1値を利用する.
本研究において適合率が高いことは,静的解析ツールが現出した警告のうち,修正の必要のない違反を効果的に削減できていることを意味する.
また,再現率が高いことは,検出された違反のうち,真に修正すべき違反の網羅性が高いことを意味する.

\begin{equation}
\label{eq:precision}
\text{適合率} = \frac{\text{修正が必要と予想できた違反数}}{\text{修正が必要と予想した違反数}}
\vspace{10pt}
\end{equation}

\begin{equation}
\label{eq:recall}
\text{再現率} = \frac{\text{修正が必要と予想できた違反数}}{\text{修正された違反数}}
\vspace{10pt}
\end{equation}

\begin{table}[ht]
    \centering
    \small % フォントサイズを小さくして収まりを良くする
    \renewcommand{\arraystretch}{1.1} % 行間を少し広げて読みやすく
    \setlength{\tabcolsep}{4pt} % 列間の余白を微調整
    \caption{変数一覧と説明}
    \label{tab:variables}
    \begin{tabularx}{\textwidth}{llX} % X列が自動改行・幅調整される
        \toprule
        種類 & 変数名 & 説明 \\
        \midrule
        flake8情報 & Violation ID & 静的解析ツールが検出した違反の種類を示すID(One-hot) \\
        & Category & 違反のカテゴリ(One-hot) \\
        & Violation Line Number & 違反が検出された行番号 \\
        \midrule
        Git情報 & File Change Frequency & ファイルの変更頻度 \\
        & Lines Added Past 25 Revisions & 過去25リビジョンで追加された行数 \\
        & Lines Added Past 3 Months & 過去3ヶ月で追加された行数 \\
        \midrule
        コード特徴量 & File Size & ファイルサイズ(バイト数) \\
        & Total Lines & ファイルの総行数 \\
        & Code Lines & コード行数 \\
        & Comment Lines & コメント行数 \\
        & Blank Lines & 空白行数 \\
        & File Depth & ファイルの階層の深さ \\
        & Filename Length & ファイル名の長さ \\
        & Total Functions & ファイル内の関数数 \\
        & Total Classes & ファイル内のクラス数 \\
        & Total Imports & インポート文の数 \\
        & Total Variables & 変数の数 \\
        & Cyclomatic Complexity & 循環的複雑度(ファイルレベル) \\
        & Line Length & 違反行の長さ(文字数) \\
        & Line Length No Whitespace & 違反行の長さ(空白を除く) \\
        & Indent Level & 違反行のインデントレベル \\
        & Line Complexity & 違反行の複雑度 \\
        & Special Chars & 違反行の特殊文字数 \\
        & Variable Count & 違反行の変数数 \\
        & Function Calls & 違反行の関数呼び出し数 \\
        & Operators & 違反行の演算子数 \\
        & In Function & 違反が関数内にあるかどうか(0/1) \\
        & Function Params & 関数のパラメータ数 \\
        & Function Lines & 関数の行数 \\
        & Function Complexity & 関数の複雑度 \\
        & In Class & 違反がクラス内にあるかどうか(0/1) \\
        & Class Methods & クラスのメソッド数 \\
        & Class Lines & クラスの行数 \\
        \bottomrule
    \end{tabularx}
\end{table}

\section{データセットの収集}



\subsection{対象言語とその選定理由}

検証に使用するプログラミング言語には,Pythonを採用する.
Pythonは,近年,機械学習やデータ分析における需要の高まりを背景に,重要性が増しているためである.
また,Java言語やC言語といったコンパイル言語に比べて,Pythonのようなスクリプト言語は,本分野における予測研究が少ないことも理由の一つである.

\subsection{検証プロジェクトの選定方法}

\begin{figure}[ht]
    \centering
    \includegraphics[width=1.0\linewidth]{@Master2025_Kameoka/fig/all_dataset_histogram.pdf}
    \caption{フィルタリング前のプロジェクトごとの規約違反数分布}
    \label{fig:all_dataset_summary}
    
    \centering
    \includegraphics[width=1.0\linewidth]{@IPSJjournal2025_Kameoka/fig/dataset_size_histogram.pdf}
    \caption{フィルタ条件適用後のプロジェクトごとの規約違反数分布}
    \label{fig:dataset_summary}
\end{figure}

検証に使用するデータセットととして,OSSライブラリ検索サービスであるLibraries.io\footnote{\url{https://libraries.io/}}に登録されているPythonライブラリを用いる.
検証に使用するライブラリの条件としては,以下の条件を満たすものを採用した.

\begin{itemize}
    \item SourceRank上位2,000プロジェクト
    \item Githubでソースコードが公開
    \item 静的解析ツールであるFlake8の設定ファイルを保有
    \item 学習データ,評価データの双方に正例・負例が存在
    \item 規約違反データが500件以上,20,000件以下
    \item 開発リポジトリが重複しているプロジェクトを削除
\end{itemize}

結果として条件を満たした170プロジェクトのデータを検証に使用する.
コーディング規約違反のデータを取得する期間は,2024年1月1日から2024年12月31日までの1年間のコミット履歴を対象とする.

本研究において,対象とする規約違反件数を500件以上かつ20,000件以下に制限した理由は,以下の2点である.

第一に,統計的な評価の信頼性を担保するためである.データセットを学習用とテスト用に分割した際,テスト用データが不足していると,少数のデータが評価指標に過度な影響を及ぼし,結果の安定性が損なわれる懸念がある.
本研究では,データを4対1の比率で分割することを前提とし,テスト用データとして最低限100件を確保するため,全データ数が500件以上であることを条件とした.

第二に,静的解析ツールが実質的に運用されていないプロジェクトを排除するためである.規約違反数が極端に多いプロジェクトは,設定ファイルのみを保持し,実際の開発プロセスではツールの警告を無視している可能性が高い.
図\ref{fig:all_dataset_summary}に示す適用前のプロジェクト数分布を確認すると,違反数20,000件付近を境に分布の不連続性が確認できる.
そのため,解析対象としての妥当性を考慮し,区切りの良い数値である20,000件を上限として設定した.

取得したデータセットの規約違反を図\ref{fig:dataset_summary}に示す.
規約違反の発生量としては,中央値が2,160件で,500件から1,000件のプロジェクトが最も多い.
また正例数と負例数の割合が,中央値で正例数が0.34という負例数が多い不均衡データであった.


\chapter{RQ1: 複数プロジェクトを学習に用いることによって、修正予測精度は向上するか}\label{chap:rq1}



\section{概要}

RQ1では従来手法である単一学習と提案手法である全学習,選定学習(欠損値補完なし),選定学習(欠損値補完あり)との修正優先度の予測結果の評価を行う.
また,提案手法によって,従来手法の課題点としてあげていた,学習データの正例数が少ない場合において予測精度が改善するかを明らかにする.

提案手法である選定学習では,閾値を定めることで類似プロジェクトを決定している.
選定学習手法で,学習データに統合しているプロジェクトを分析することで,本研究で提案している,類似プロジェクトの決定方法の妥当性についての分析を行う.

\section{結果}

\begin{figure}[ht]
	\centering
        % \includegraphics[width=0.25\textwidth, bb=0 0 4 3]{fig/dataset_hist.pdf}
	\includegraphics[width=1\linewidth]{@IPSJjournal2025_Kameoka/fig/boxplot_general.pdf}
	\caption{4手法による予測結果の箱ひげ図}
	\label{fig:boxplot_summary}
\end{figure}

	% \centering
 %        % \includegraphics[width=0.25\textwidth, bb=0 0 4 3]{fig/dataset_hist.pdf}
	% \includegraphics[width=1\linewidth]{fig/boxplot_filtered.pdf}
	% \caption{選定学習手法(欠損値補完あり)で複数プロジェクトのデータを学習したプロジェクトに絞った予測結果の箱ひげ図}
	% \label{fig:boxplot_filtered}

%---------------------

データセット内のプロジェクト全体の規約違反の修正要否予測の結果を図\ref{fig:boxplot_summary}に示す.
図は左から各手法の予測結果の適合率,再現率,F1値を順に示し,縦軸にその値を示している.
選定学習と選定学習(欠損値補完あり)の2手法では,プロジェクト間類似度の測定の際に,ジャッカード係数とユークリッド距離のそれぞれに閾値を設けている.閾値の決定方法は,ジャッカード係数とユークリッド距離の組み合わせの全50通りの中から,学習データでの予測精度が最も良かった際の値を採用した場合の,評価データでの予測結果を採用している.

結果としては,単一学習の結果と比較して提案手法3種類では,適合率が僅かに改善し,再現率が僅かに悪化する結果となった.
各メトリクスに対して,4手法間の全ての組み合わせでWilcoxonの符合順位検定を行ったところ,全ての組み合わせにおいて,統計的有意差なしの結果となった.

データセット全体のプロジェクトを対象とした検証において,予測精度に顕著な差が見られなかった点は想定の範囲内である.
これは,本提案手法が本来,学習データの確保が困難なプロジェクトに対する精度向上を主眼として設計されたものであることに起因すると考えられる.


\subsection{選定学習手法による予測精度への影響}

\begin{figure}[ht]
    \centering
    \includegraphics[width=1\linewidth]{@IPSJjournal2025_Kameoka/fig/precision_improvement_by_positive_samples.pdf}
	\caption{選定学習(欠損値補完あり)による適合率の改善度}
	\label{fig:boxplot_precision_improvement_by_selected}
% \end{figure}

% \begin{figure}[ht]
    \centering
    \includegraphics[width=1\linewidth]{@IPSJjournal2025_Kameoka/fig/recall_improvement_by_positive_samples.pdf}
	\caption{選定学習(欠損値補完あり)による再現率の改善度}
	\label{fig:boxplot_recall_improvement_by_selected}
\end{figure}

提案手法の効果が顕著であったプロジェクトの分析結果を示す.
図\ref{fig:boxplot_precision_improvement_by_selected}は,提案手法(欠損値補完あり)による適合率の変化量を示した箱ひげ図である.
また,図\ref{fig:boxplot_recall_improvement_by_selected}には,同様の分析を再現率の観点で行った結果を示す.
いずれの図も横軸は,学習データに含まれる正例数に基づき4つの区間に分類している.

これら2つの図から,提案手法(欠損値補完あり)の導入によって,適合率が上昇する一方で再現率が低下する傾向が確認できる.
また,正例数が100件から500件の範囲において,箱ひげ図の分散が最も大きくなっている.
このことから,正例数が100〜500件のプロジェクトにおいて,提案手法が与える影響が特に大きいと言える.
同様に,欠損値補完を用いない選定学習においても,同様の傾向が見られた.


\subsection{全学習手法による予測精度への影響}

\begin{figure}[ht]
    \centering
    \includegraphics[width=1\linewidth]{@Master2025_Kameoka/fig/all_precision_improvement_by_positive_samples.pdf}
	\caption{全学習手法による適合率の改善度}
	\label{fig:boxplot_precision_improvement_by_all}
\end{figure}

\begin{figure}[ht]
    \centering
    \includegraphics[width=1\linewidth]{@Master2025_Kameoka/fig/all_recall_improvement_by_positive_samples.pdf}
	\caption{全学習手法による再現率の改善度}
	\label{fig:boxplot_recall_improvement_by_all}
\end{figure}

次に全学手法による,修正予測精度への影響について述べる.


\section{類似度の妥当性の評価}



\section{まとめ}



\chapter{RQ2: 複数プロジェクトを学習することで予測精度が改善するプロジェクトの特徴とは何か}\label{chap:rq2}



\section{概要}


\section{分析方法}



\section{分析結果}

\subsection{正例数の少ないプロジェクトでの分析結果}

\begin{table}[ht]
  \centering
  \caption{適合率が改善したプロジェクトで出現する規約違反ID(上位10件)}
  \label{tab:improve_precision_id}
  \begin{tabular}{lrr}
    \hline
    Violation ID & 出現回数 & 出現プロジェクト数 \\ \hline
    E501 & 31,124 & 35 \\
    E231 & 2,846 & 8 \\
    F401 & 1,054 & 28 \\
    F405 & 798 & 7 \\
    E225 & 532 & 5 \\
    E402 & 525 & 15 \\
    E128 & 480 & 6 \\
    E203 & 357 & 22 \\
    E265 & 300 & 10 \\
    F841 & 270 & 15 \\ \hline
  \end{tabular}
  \vspace{10pt}

  \centering
  \caption{適合率が悪化したプロジェクトで出現する規約違反ID(上位10件)}
  \label{tab:not_improve_precision_id}
  \begin{tabular}{lrr}
    \hline
    Violation ID & 出現回数 & 出現プロジェクト数 \\ \hline
    E501 & 20,509 & 20 \\
    E127 & 2,460 & 7 \\
    E128 & 1,487 & 8 \\
    F401 & 1,355 & 16 \\
    E203 & 1,284 & 11 \\
    W293 & 699 & 5 \\
    E265 & 587 & 9 \\
    E302 & 490 & 9 \\
    E252 & 429 & 2 \\
    E303 & 420 & 8 \\ \hline
  \end{tabular}
\end{table}

\begin{table}[ht]
  \centering
  \small 
  \caption{プロジェクトの適合率改善状況別の統計比較}
\label{table:pisitive_sample}
  \begin{tabular}{lrr}
    \toprule
    指標 & \makecell[r]{適合率が改善した\\プロジェクト} & \makecell[r]{適合率が悪化した\\プロジェクト} \\
    \midrule
    平均総サンプル数 & 1147.6 & 1741.2 \\
    中央値総サンプル数 & 853.0 & 1230.5 \\
    平均正例数 & 302.8 & 222.2 \\
    中央値正例数 & 293.0 & 164.5 \\
    平均負例数 & 844.8 & 1519.0 \\
    中央値負例数 & 635.0 & 928.5 \\
    \bottomrule
  \end{tabular}
\end{table}

\begin{table}[ht]
  \centering
  \small
  \caption{プロジェクトの適合率改善状況別の修正率統計}
    \label{table:pisitive_rate}
  \begin{tabular}{lrr}
    \toprule
    指標 & \makecell[r]{適合率が改善した\\プロジェクト} & \makecell[r]{適合率が悪化した\\プロジェクト} \\
    \midrule
    平均修正率 & 0.344 & 0.225 \\
    中央値修正率 & 0.350 & 0.201 \\
    標準偏差 & 0.169 & 0.157 \\
    最小値 & 0.038 & 0.022 \\
    最大値 & 0.797 & 0.573 \\
    \bottomrule
  \end{tabular}
\end{table}

適合率が上昇している要因を特定するため,提案手法による適合率の変化幅が最も大きかった,正例数が100から500の範囲にある57プロジェクトに対象を絞り,出現する規約違反IDの種類に着目した詳細な分析を行った.

まず,適合率が改善したプロジェクト群において最も頻繁に出現する規約は「E501(1行あたりの文字数制限)」であった.しかし,表\ref{tab:improve_precision_id}および表\ref{tab:not_improve_precision_id}の比較から明らかな通り,本規約は適合率が悪化したプロジェクト群においても同様に高い頻度で出現している.E501はソースコードの書式に関する極めて一般的な違反であり,特定のプロジェクト群に固有の性質を示す特徴とは言い難い.

また,その他の規約について検討したところ,複数のプロジェクト間で共通して発生している違反は極めて少数であった.具体的には,共通性が確認された規約であっても,該当するプロジェクト数は最大で2件に留まっている.以上の分析結果から,適合率が向上したプロジェクト群において,規約違反の種類に基づいた明確な共通性を見出すことは困難であった.このことは,適合率の変動が特定の規約違反の傾向に依存するのではなく,プロジェクト固有の要因や他の変数の影響を受けている可能性を示唆している.

適合率が向上した要因を詳細に検討するため,説明変数の差異に着目した分析を行った.適合率が改善したプロジェクト群と悪化したプロジェクト群を対象に,マン・ホイットニーのU検定を用いた有意差検定を実施したところ,「正例数」および「修正率」の2指標において統計的な有意差が確認された.

具体的には,適合率が改善したプロジェクト群は,悪化した群と比較して正例数が有意に多く,かつ修正率も有意に高い傾向にあることが示された.この結果は,モデルの学習において十分な正例数を確保することの重要性と,人間による修正(フィードバック)の密度が適合率の向上に直接的に寄与していることを示唆している.

一方で,データの構成比にも留意すべき点が確認された.正例数そのものの多寡だけでなく,それに対する負例数の多さも適合率に悪影響を及ぼす阻害要因となっている可能性が高い.したがって,単に正例を増やすだけでなく,正負のデータバランスを考慮したデータセットの構築が,適合率を安定的に改善させるための鍵であると考えられる.



\subsection{正例数が少ないプロジェクトで予測精度が向上しなかったプロジェクトの特徴}



\section{まとめ}



\chapter{考察}\label{chap:consideration}

\begin{figure}[ht]
	\centering
        % \includegraphics[width=0.25\textwidth, bb=0 0 4 3]{fig/dataset_hist.pdf}
	\includegraphics[width=1\linewidth]{@IPSJjournal2025_Kameoka/fig/precision_similar_projects_count_boxplot.pdf}
	\caption{適合率が変化したプロジェクト統合プロジェクト数}
	\label{fig:precision_similar_projects_count_boxplot}

    \centering
        % \includegraphics[width=0.25\textwidth, bb=0 0 4 3]{fig/dataset_hist.pdf}
	\includegraphics[width=1\linewidth]{@IPSJjournal2025_Kameoka/fig/recall_similar_projects_count_boxplot.pdf}
	\caption{再現率が変化したプロジェクト統合プロジェクト数}
	\label{fig:recall_similar_projects_count_boxplot}
\end{figure}

\section{学習データを統合することによって適合率が向上した要因}\label{section:kousatu1}

学習データを統合した結果,正例数が少ない(100件から500件程度)プロジェクトにおいて,提案する選定学習手法により適合率が改善することが確認された.
特に,正例数がある程度確保されており,かつ修正率が高いプロジェクトにおいて,本手法は効果的に作用する傾向が見られた.

図\ref{fig:precision_similar_projects_count_boxplot}は,学習データの正例数が100件から500件の範囲にあるプロジェクトを対象とし,適合率が向上した群と低下した群に分けた際の,統合プロジェクト数に関する箱ひげ図である.
同様に,図\ref{fig:recall_similar_projects_count_boxplot}には再現率に関する結果を示す.

適合率の変動について分析を行ったところ,改善群と悪化群の間で,統合したプロジェクト数にマン・ホイットニーのU検定による有意差が認められた.
一方,再現率については有意差こそ確認されなかったものの,p値は0.0994であり,コーエンのdは1.009と非常に高い効果量を示した.

以上の結果は,予測精度が低下したプロジェクトにおいては,学習に寄与する類似プロジェクトを十分に収集できなかったことが精度の改善を阻んだ要因であることを示唆している.


\section{提案手法で適合率が向上した要因}

\ref{section:kousatu1}節で述べた通り,適合率が向上したプロジェクトは,学習データの正例数が多く,かつ修正率が高いという特徴を有していた.本研究で提案した選定学習手法は,規約違反の発生傾向および規約ごとの修正率が類似するプロジェクトを収集するアルゴリズムに基づいている.

そのため,元の正例数が多く修正率が高いプロジェクトにおいては,精度が低下したプロジェクトと比較して,学習に適した正例データがより効率的に収集されたと考えられる.一般に,コーディング規約違反の修正データは,正例が全体の3割程度に留まる不均衡データである.本手法によって正例データが十分に蓄積されたことで,学習データにおけるクラス不均衡が緩和され,結果として適合率の向上に寄与したものと推察される.

\chapter{妥当性への脅威}\label{chap:heuristic}



\section{内的妥当性}

目的変数の計測において,コーディング規約に違反しているコードが修正された場合と,削除された場合を修正としてとらえ,正例として計測している.
コーディング規約違反の中には,単純に削除するのみで解消されるエラーが存在するため,削除も正例と計測している.
しかし,コーディング規約に違反しているコードが単に不要になり削除された場合も正例と計測している.
コードの移動に関してもGitHubの仕様上,削除と追加という扱いになるため,コードの移動によってコーディング規約に違反しているコードが正例として計測されている可能性がある.

本研究で用いた機械学習アルゴリズム以外にも,より正確な予測を可能にするアルゴリズムが存在する可能性がある.
しかし,本研究において検証した,予測対象プロジェクト以外のデータを学習に使用する手法は,少量の学習データを他プロジェクトのデータで補完する手法であるため,他の機械学習アルゴリズムによる予測を行う際でも効果的に働くと考えられる.
それぞれの機械学習モデルのパラメータの更新回数である,イテレーション回数を10,000に設定したが,データサイズが大きいプロジェクトではモデルが収束しないことが確認された.モデルが収束しない場面も確認されたが,本研究で主に用いた実験環境とは別の環境で,複数回実行した場合でも予測結果に変化はなかったので,モデルが収束しない問題に関しては,本研究の結果に対して大きな影響を与える可能性は低い.

\section{外的妥当性}

本研究では検証対象としてPython言語を主な開発言語とした,170件のGithubリポジトリを収集して検証を行った.
そのため,プロジェクト数をさらに拡張した場合や,対象とするプロジェクトや期間を変更した場合に予測精度が変化することが示唆される.
対象言語をPython以外の言語とした場合や,検証対象の静的解析ツールを変更した場合,予測結果が変化することが示唆される.
しかし,本研究で利用したデータセットは,データ数の平均値が4,264であり,十分なデータ数を確保できているため,データ数を拡張することによる予測精度の低下の可能性は低いと考えられる.
また,選定学習においては,より類似度の高いプロジェクトを統合できるため,更なる予測精度の向上が見込める.

\chapter{おわりに}\label{chap:end}



%%%%%%%%%%%%%%%%%%%%%%%%%%%%%%%%%%%%%%%%%%%%%%%%%%%%%%%%%%%%%%%%

%%
%% 謝辞
%%

%\begin{acknowledgements}
%感謝します.
%\end{acknowledgements}


\bibliographystyle{junsrt}
\bibliography{@Master2025_Kameoka/references}


\end{document}
