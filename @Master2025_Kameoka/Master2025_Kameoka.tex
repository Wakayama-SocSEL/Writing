%%
%% 修士論文
%%
%% USE 機能的に等しい
%% not 外的振る舞いが等しい
%%
%% USE 大規模言語モデル
%% not LLM
%%
%% USE 低下させる
%% not 下げる
%%

%%%%%%%%%%%%%%%%%%%%%%%%%%%%%%%%%%%%%%%%%%%%%%%%%%%%%%%%%%%%%%%%

%%
%% 利用するコマンドの準備
%%

%% 必須パッケージ
\documentclass[11pt]{jreport}
\usepackage{wuse_thesis}
\usepackage{indentfirst}

%% 追加パッケージ
%\usepackage{graphicx}
\usepackage{comment}
\usepackage[dvipdfmx]{graphicx}
\usepackage{url}

\usepackage{latexsym}

\usepackage{cite}
\usepackage{amsmath,amssymb,amsfonts}
\usepackage{algorithmic}
\usepackage{textcomp}
\usepackage{xcolor}
\usepackage{listings}
\usepackage{makecell}
\usepackage{booktabs}
\def\BibTeX{{\rm B\kern-.05em{\sc i\kern-.025em b}\kern-.08em
    T\kern-.1667em\lower.7ex\hbox{E}\kern-.125emX}}

\newcommand{\todo}[1]{\colorbox{yellow}{{\bf TODO}:}{\color{red} {\textbf{[#1]}}}}


\lstset{
basicstyle=\small\ttfamily,
abovecaptionskip=0pt,
captionpos=b,
frame=tb,
framexleftmargin=2em,
numbers=left,
numberstyle={\scriptsize},
xleftmargin=\parindent
}

%ListingのキャプションがFigureになってしまうのをListingに直すコマンド
\usepackage{caption}
\makeatletter
\let\MYcaption\@makecaption
\makeatother
\usepackage{caption}
\makeatletter
\let\@makecaption\MYcaption
\makeatother

\def\Underline{\setbox0\hbox\bgroup\let\\\endUnderline}
\def\endUnderline{\vphantom{y}\egroup\smash{\underline{\box0}}\\}
\def\|{\verb|}
%


%%%%%%%%%%%%%%%%%%%%%%%%%%%%%%%%%%%%%%%%%%%%%%%%%%%%%%%%%%%%%%%%

%%
%% 主に表紙を作成するための情報
%%

%% タイトル(修論の場合は英語表記も指定)
\title{複数ソフトウェアの修正履歴の統合による\\コーディング規約違反の修正予測精度の評価}
\etitle{Evaluating the Prediction Accuracy of Coding Standard Violation Fixes through Integrated Software Change History Data}

%% 著者名(修論の場合は英語表記も指定)
\author{亀岡 令}
\eauthor{Ryo Kameoka}

%% 修士論文(M2用)
\master

%% 学科・クラスタ
\department{システム工}

%% 学生番号
\studentid{S2420033}

%% 卒業年度
\gyear{2025}

%% 論文提出日(修士の場合は月まで)
\date{2026年2月}
\edate{February 2026}


%%%%%%%%%%%%%%%%%%%%%%%%%%%%%%%%%%%%%%%%%%%%%%%%%%%%%%%%%%%%%%%%

%%
%% 表紙・概要・目次
%%

\begin{document}

%% 表紙
\maketitle

%% 概要
\begin{abstract}
ほげほげ\cite{HowFar}
\end{abstract}

%% 目次
\tableofcontents

%% 図目次
%\listoffigures

%% 表目次
%\listoftables

\newpage
\pagenumbering{arabic}


%%%%%%%%%%%%%%%%%%%%%%%%%%%%%%%%%%%%%%%%%%%%%%%%%%%%%%%%%%%%%%%%
\chapter{はじめに}



\chapter{背景と関連研究}



\section{可読性を担保するコーディング規約}

ソースコードの可読性や保守性といった品質を高水準で維持することは重要である.
可読性の高いソースコードを書くことで,ソースコードの理解の促進やバグの混入の予防などの効果が期待できる.
また,品質の劣るコードには,良質なコードの約15倍の欠陥が含まれる可能性があることが報告されている\cite{コード品質の重要性}.

特に大規模なソフトウェア開発では,ソースコードを実装した後に,実装者とは異なる開発者が,新規実装されたソースコードにバグなどの問題が含まれていないかを確認するレビュー作業が発生する.
可読性の高いコードはレビュー作業の短縮にもつながる\todo{参考}.
つまり,可読性の高いソースコードを書くことは,ソフトウェア開発サイクル全体の促進につながる.

そこで,開発者はソフトウェア開発にコーディング規約を導入する.
コーディング規約は,複数人によるソフトウェア開発においてソースコードの記述方法を共通化するための指針である.
コーディング規約には,変数やクラスの命名規則,関数の長さや複雑度の上限といった可読性に関するルール,エラーの原因となる記述の検出,禁止事項,制限事項,推奨事項などが含まれる.

コーディング規約は,プログラミング言語ごとに異なる内容や基準を持ち,多くの種類の規約が存在する.
例えば,Python言語にはpycodestyle,Java言語にはCode Conventions for the Java Programming LanguageやGoogle Java Style Guide,JavaScript言語にはGoogle JavaScript Style GuideやAirbnb JavaScript Style Guideなどが代表的な例ある.

開発者はコーディング規約を遵守したコーディングを行うことによって,一定のコードの品質を保つことができる.
そのため,実装したコードはレビュアーが見る際には,可読性が担保されたコードを見ることが可能となる.


\section{静的解析ツール}

ソースコードがコーディング規約に従っているかや,違反しているコードがどこに存在するのかを特定するために静的解析ツールが利用される.
コーディング規約に従っていないコードを検出する静的解析ツールも各プログラミング言語ごとに存在する.
代表的な例としては,PythonのFlake8やJavaScriptのESLint,JavaのCheckstyleが挙げられる.
また,これらの静的解析ツールは存在する,規約の中から,どの規約への違反を検出対象とするかを任意で決定することができる.
検出する違反の種類を制御するためには,それぞれの静的解析ツールごとに定められたディレクトリに決められたファイル名のテキストファイルを

ソフトウェア開発者は,各プロジェクトが参照する規約,違反検出時の出力形式を調整を加えながら,静的解析ツールを用いて規約違反コードを検出している.
プロジェクトの中には,継続的インテグレーション (CI) 環境に組み込むことで開発効率の向上を実現している\cite{ci/cd}.
静的解析ツールは,ソースコード中の規約違反コードを網羅的に検出するが,その数は膨大であることが多く,開発者がそのすべてを確認,修正,保守するには多大なコストと労力を要する\cite{UsingStaticAnalysisTools2}.さらに,優先的に修正すべき違反を区別するためには,経験やソースコードへの深い理解が求められる\cite{shuseisarenai}.

\section{従来研究}\label{sec:zyuuraikennkyuu}


従来研究では,静的解析ツールによる大量に検出した規約違反コードの中から,機械学習を用いて修正すべき規約違反コードを分類する手法が提案されている\cite{JyuraiPre}\cite{beizu}.
Kimらは,静的解析ツールの出力をベイジアンネットワークに基づいて解析し,修正すべき規約違反コードを分類する手法を提案している\cite{beizu}.また,検出された違反に対する修正優先度付けする研究も数多く行われている\cite{Wang}\cite{Qing}\cite{HowFar}.これらの研究では,予測対象プロジェクトにおける規約違反の修正履歴を学習データとして用い,新たに検出された違反を評価対象とする機械学習モデル(規約違反コードの修正要否判定モデル)の構築と評価が行われている.



% 従来研究では,静的解析ツールによる大量の違反検出が開発効率の低下を招くことに対処するため,機械学習を用いて修正の優先度を予測する手法が提案されている.たとえば,Ruthruffらは,優先的に修正すべき違反を特定するための機械学習モデルを構築している\cite{JyuraiPre}.また,Kimらは,静的解析ツールの出力をベイジアンネットワークに基づいて解析し,違反の修正優先度を予測する手法を提案している\cite{beizu}.
% そのほかにも,検出された違反に対する修正優先度付けや修正要否の予測に関する研究が数多く行われている\cite{Wang}\cite{Qing}\cite{HowFar}.これらの研究では,予測対象プロジェクトにおける過去の規約違反修正履歴を学習データとし,新たに検出された違反を評価対象とすることで,予測モデルの構築と評価が行われている.

従来研究では,評価対象プロジェクトの修正履歴を学習データとして用いてモデルを構築している.これは各プロジェクトでコーディングスタイルや開発体制が異なり,評価対象と同一プロジェクトの修正履歴を学習することで高い精度が得られることが示されている.
% 従来研究では,一般に,評価対象プロジェクトの修正履歴を学習データとして用いて予測モデルを構築する.プロジェクトごとにコーディングスタイルや開発体制が異なるため,対象と同一プロジェクトの履歴を学習する方が,他プロジェクトの履歴を用いるよりも高い精度が得られると考えられている.
しかし,同一プロジェクトの学習データには,出現する違反の種類や修正率に偏りが生じることがある\cite{Panichella}
%\todo{この論文では予測まで行っていない?そのため,次の文であるように予測精度が低下する可能性,という表現になっているう?}
.
この結果,修正された違反(正例)と修正されなかった違反(負例)の数に不均衡が生じ,予測精度の低下を招く可能性がある.また,予測対象プロジェクトの違反修正履歴が小規模の場合,学習データが小規模になり十分な学習ができない可能性がある.

本研究では,評価対象以外のプロジェクトの修正履歴を学習データに含めることで,規約違反コードの修正予測精度の向上を目的とする.

%予測モデルの構築において,評価対象とは異なるプロジェクトのコーディング規約違反修正履歴を学習データとして用いることが,修正要否予測精度に与える影響を明らかにすることを目的とする.
従来研究では,名倉らが複数プロジェクトのデータを用いて規約違反の発生件数の増減を予測している.一方で,個々の規約違反コードの修正有無の予測は対象としていないため,本研究とは異なるが,研究動機は類似している\cite{nagura}.
異なるプロジェクトの修正履歴を用いることで,学習データの拡張による予測精度の向上が期待されるが,異なるコーディングスタイルを有するプロジェクトの修正履歴は予測精度を低下することも考えられる.
% すべての他プロジェクトのデータが有益とは限らない.すなわち,修正傾向が異なるプロジェクトのデータは,予測に対してノイズとなる可能性がある.

そこで本研究では,評価対象プロジェクトと修正傾向が類似するプロジェクト選定し,修正履歴を統合することで学習データの拡張し,予測モデルの精度向上を目指す.

\section{大規模言語モデルを利用した関連研究}

近年の大規模言語モデルをの発展に伴い,生成AIを用いた静的解析ツールの解析結果の分類を行う研究が行われている.Wadhwaらは,プログラム中に存在する,コーディング規約に違反しているコードの誤検知の分類と,修正パッチの自動作成の研究を行っている\cite{LLMで検出}.Wadhwaらの研究では,静的解析ツールを用いた検出結果を分類するという手法と,静的解析ツールを用いずに違反個所の検出から生成AIに行わせる手法を検証している.
また,真の検出箇所で判定された場合,違反コードの修正パッチの作成まで行っている点が特徴的である.

Wadhwaらの研究結果では,静的解析ツールが検出可能なコード品質を低下させているコードと,大規模言語モデルが検出可能なコード品質を低下させているコードでは.検出可能な種類が異なるという結果が報告されている.
また,大規模言語モデルに修正パッチの生成を行わせるタスクでは,どれだけプロンプトで制限を行った場合であっても,関係のないコードまで修正を行ってしまうというデメリットも報告されている.
Wadhwaらの研究では,静的解析ツールと生成AIでは,コード品質を低下させている個所の検出可能な種類が異なるため,静的解析ツールと生成AIの併用を勧めている.

本研究では,\ref{sec:zyuuraikennkyuu}節で示した従来研究と同様に,予測対象のプロジェクトの開発履歴を学習に使用するため,プロジェクトごとの修正の特徴をとらえた学習が可能である.
Wadhwaらの研究では,大規模言語モデルにソースコードと静的解析ツールの結果を渡しているのみなので,プロジェクトごとに最適化はされていない.
本研究では,開発履歴の学習を行うためプロジェクトごとに最適化を行うことができるという点で異なる.

\section{RQs}

\begin{itemize}
    \item RQ1: 複数プロジェクトを学習に用いることによって、修正予測精度は向上するか
    \item RQ2: 複数プロジェクトを学習することで予測精度が改善するプロジェクトの特徴とは何か
\end{itemize}

\chapter{規約違反コードの修正要否判定モデルの構築手法}



\section{データセットの収集}



\subsection{対象言語とその選定理由}

検証に使用するプログラミング言語には,Pythonを採用する.
Pythonは,近年,機械学習やデータ分析における需要の高まりを背景に,重要性が増しているためである.
また,Java言語やC言語といったコンパイル言語に比べて,Pythonのようなスクリプト言語は,本分野における予測研究が少ないことも理由の一つである.

\subsection{選定方法}

\begin{figure}[t]
    \centering
    \includegraphics[width=1.0\linewidth]{@Master2025_Kameoka/fig/all_dataset_histogram.pdf}
    \caption{フィルタリング前のプロジェクトごとの規約違反数分布}
    \label{fig:all_dataset_summary}
    
    \centering
    \includegraphics[width=1.0\linewidth]{@IPSJjournal2025_Kameoka/fig/dataset_size_histogram.pdf}
    \caption{フィルタ条件適用後のプロジェクトごとの規約違反数分布}
    \label{fig:dataset_summary}
\end{figure}

検証に使用するデータセットととして,OSSライブラリ検索サービスであるLibraries.io\footnote{\url{https://libraries.io/}}に登録されているPythonライブラリを用いる.
検証に使用するライブラリの条件としては,以下の条件を満たすものを採用した.

\begin{itemize}
    \item SourceRank上位2,000プロジェクト
    \item Githubでソースコードが公開
    \item 静的解析ツールであるFlake8の設定ファイルを保有
    \item 学習データ,評価データの双方に正例・負例が存在
    \item 規約違反データが500件以上,20,000件以下
    \item 開発リポジトリが重複しているプロジェクトを削除
\end{itemize}

結果として条件を満たした170プロジェクトのデータを検証に使用する.
コーディング規約違反のデータを取得する期間は,2024年1月1日から2024年12月31日までの1年間のコミット履歴を対象とする.

本研究において,対象とする規約違反件数を500件以上かつ20,000件以下に制限した理由は,以下の2点である.

第一に,統計的な評価の信頼性を担保するためである.データセットを学習用とテスト用に分割した際,テスト用データが不足していると,少数のデータが評価指標に過度な影響を及ぼし,結果の安定性が損なわれる懸念がある.
本研究では,データを4対1の比率で分割することを前提とし,テスト用データとして最低限100件を確保するため,全データ数が500件以上であることを条件とした.

第二に,静的解析ツールが実質的に運用されていないプロジェクトを排除するためである.規約違反数が極端に多いプロジェクトは,設定ファイルのみを保持し,実際の開発プロセスではツールの警告を無視している可能性が高い.
図\ref{fig:all_dataset_summary}に示す適用前のプロジェクト数分布を確認すると,違反数20,000件付近を境に分布の不連続性が確認できる.
そのため,解析対象としての妥当性を考慮し,区切りの良い数値である20,000件を上限として設定した.

取得したデータセットの規約違反を図\ref{fig:dataset_summary}に示す.
規約違反の発生量としては,中央値が2,160件で,500件から1,000件のプロジェクトが最も多い.
また正例数と負例数の割合が,中央値で正例数が0.34という負例数が多い不均衡データであった.

\section{説明変数の定義}



\section{目的変数の計測}



\section{プロジェクトのフィルタリング}



\section{モデルの構築方法}



\subsection{単一学習}



\subsection{全学習}



\subsection{選定学習}



\subsubsection{類似度の測定方法}



\subsubsection{欠損値補完を行う方法}



\chapter{RQ1: 複数プロジェクトを学習に用いることによって、修正予測精度は向上するか}



\section{概要}



\section{結果}

\begin{figure}[t]
	\centering
        % \includegraphics[width=0.25\textwidth, bb=0 0 4 3]{fig/dataset_hist.pdf}
	\includegraphics[width=1\linewidth]{@IPSJjournal2025_Kameoka/fig/boxplot_general.pdf}
	\caption{4手法による予測結果の箱ひげ図}
	\label{fig:boxplot_summary}
\end{figure}

	% \centering
 %        % \includegraphics[width=0.25\textwidth, bb=0 0 4 3]{fig/dataset_hist.pdf}
	% \includegraphics[width=1\linewidth]{fig/boxplot_filtered.pdf}
	% \caption{選定学習手法(欠損値補完あり)で複数プロジェクトのデータを学習したプロジェクトに絞った予測結果の箱ひげ図}
	% \label{fig:boxplot_filtered}

\begin{figure}[t]
    \centering
    \includegraphics[width=1\linewidth]{@IPSJjournal2025_Kameoka/fig/precision_improvement_by_positive_samples.pdf}
	\caption{選定学習(欠損地補完あり)による適合率の改善度}
	\label{fig:boxplot_precision_improvement}
\end{figure}

\begin{figure}[t]
    \includegraphics[width=1\linewidth]{@IPSJjournal2025_Kameoka/fig/recall_improvement_by_positive_samples.pdf}
	\caption{選定学習(欠損地補完あり)による再現率の改善度}
	\label{fig:boxplot_recall_improvement}
\end{figure}

%---------------------

データセット内のプロジェクト全体の規約違反の修正要否予測の結果を図\ref{fig:boxplot_summary}に示す.
図は左から各手法の予測結果の適合率,再現率,F1値を順に示し,縦軸にその値を示している.
選定学習と選定学習(欠損値補完あり)の2手法では,プロジェクト間類似度の測定の際に,ジャッカード係数とユークリッド距離のそれぞれに閾値を設けている.閾値の決定方法は,ジャッカード係数とユークリッド距離の組み合わせの全50通りの中から,学習データでの予測精度が最も良かった際の値を採用した場合の,評価データでの予測結果を採用している.

結果としては,単一学習の結果と比較して提案手法3種類では,適合率が僅かに改善し,再現率が僅かに悪化する結果となった.
各メトリクスに対して,4手法間の全ての組み合わせでWilcoxonの符合順位検定を行ったところ,全ての組み合わせにおいて,統計的有意差なしの結果となった.

データセット全体のプロジェクトを対象とした検証において,予測精度に顕著な差が見られなかった点は想定の範囲内である.
これは,本提案手法が本来,学習データの確保が困難なプロジェクトに対する精度向上を主眼として設計されたものであることに起因すると考えられる.

提案手法の効果が顕著であったプロジェクトの分析結果を示す.
図\ref{fig:boxplot_precision_improvement}は,提案手法(欠損値補完あり)による適合率の変化量を示した箱ひげ図である.
また,図\ref{fig:boxplot_recall_improvement}には,同様の分析を再現率の観点で行った結果を示す.
いずれの図も横軸は,学習データに含まれる正例数に基づき4つの区間に分類している.

これら2つの図から,提案手法(欠損値補完あり)の導入によって,適合率が上昇する一方で再現率が低下する傾向が確認できる.
また,正例数が100件から500件の範囲において,箱ひげ図の分散が最も大きくなっている.
このことから,正例数が100〜500件のプロジェクトにおいて,提案手法が与える影響が特に大きいと言える.
同様に,欠損値補完を用いない選定学習においても,同様の傾向が見られた.

\section{類似度の妥当性の評価}



\section{まとめ}



\chapter{RQ2: 複数プロジェクトを学習することで予測精度が改善するプロジェクトの特徴とは何か}



\section{分析方法}



\section{分析結果}

\subsection{正例数の少ないプロジェクトでの分析結果}

\begin{table}[htbp]
  \centering
  \caption{適合率が改善したプロジェクトで出現する規約違反ID(上位10件)}
  \begin{tabular}{lrr}
    \hline
    Violation ID & 出現回数 & 出現プロジェクト数 \\ \hline
    E501 & 31,124 & 35 \\
    E231 & 2,846 & 8 \\
    F401 & 1,054 & 28 \\
    F405 & 798 & 7 \\
    E225 & 532 & 5 \\
    E402 & 525 & 15 \\
    E128 & 480 & 6 \\
    E203 & 357 & 22 \\
    E265 & 300 & 10 \\
    F841 & 270 & 15 \\ \hline
  \end{tabular}
  \vspace{10pt}

  \centering
  \caption{適合率が悪化したプロジェクトで出現する規約違反ID(上位10件)}
  \begin{tabular}{lrr}
    \hline
    Violation ID & 出現回数 & 出現プロジェクト数 \\ \hline
    E501 & 20,509 & 20 \\
    E127 & 2,460 & 7 \\
    E128 & 1,487 & 8 \\
    F401 & 1,355 & 16 \\
    E203 & 1,284 & 11 \\
    W293 & 699 & 5 \\
    E265 & 587 & 9 \\
    E302 & 490 & 9 \\
    E252 & 429 & 2 \\
    E303 & 420 & 8 \\ \hline
  \end{tabular}
\end{table}

\begin{table}[t]
  \centering
  \small 
  \caption{プロジェクトの適合率改善状況別の統計比較}
  \label{table:pisitive_sample}
  \begin{tabular}{lrr}
    \toprule
    指標 & \makecell[r]{適合率が改善した\\プロジェクト} & \makecell[r]{適合率が悪化した\\プロジェクト} \\
    \midrule
    平均総サンプル数 & 1147.6 & 1741.2 \\
    中央値総サンプル数 & 853.0 & 1230.5 \\
    平均正例数 & 302.8 & 222.2 \\
    中央値正例数 & 293.0 & 164.5 \\
    平均負例数 & 844.8 & 1519.0 \\
    中央値負例数 & 635.0 & 928.5 \\
    \bottomrule
  \end{tabular}
  \vspace{10pt}
  
  \centering
  \small
  \caption{プロジェクトの適合率改善状況別の修正率統計}
  \label{table:pisitive_rate}
  \begin{tabular}{lrr}
    \toprule
    指標 & \makecell[r]{適合率が改善した\\プロジェクト} & \makecell[r]{適合率が悪化した\\プロジェクト} \\
    \midrule
    平均修正率 & 0.344 & 0.225 \\
    中央値修正率 & 0.350 & 0.201 \\
    標準偏差 & 0.169 & 0.157 \\
    最小値 & 0.038 & 0.022 \\
    最大値 & 0.797 & 0.573 \\
    \bottomrule
  \end{tabular}
\end{table}

適合率が上昇している要因を特定するため,提案手法による適合率の変化幅が最も大きかった,正例数が100から500の範囲にある57プロジェクトに対象を絞り,出現する規約違反IDの種類に着目した詳細な分析を行った.

まず,適合率が改善したプロジェクト群において最も頻繁に出現する規約は「E501(1行あたりの文字数制限)」であった.しかし,表1および表2の比較から明らかな通り,本規約は適合率が悪化したプロジェクト群においても同様に高い頻度で出現している.E501はソースコードの書式に関する極めて一般的な違反であり,特定のプロジェクト群に固有の性質を示す特徴とは言い難い.

また,その他の規約について検討したところ,複数のプロジェクト間で共通して発生している違反は極めて少数であった.具体的には,共通性が確認された規約であっても,該当するプロジェクト数は最大で2件に留まっている.以上の分析結果から,適合率が向上したプロジェクト群において,規約違反の種類に基づいた明確な共通性を見出すことは困難であった.このことは,適合率の変動が特定の規約違反の傾向に依存するのではなく,プロジェクト固有の要因や他の変数の影響を受けている可能性を示唆している.

適合率が向上した要因を詳細に検討するため,説明変数の差異に着目した分析を行った.適合率が改善したプロジェクト群と悪化したプロジェクト群を対象に,マン・ホイットニーのU検定を用いた有意差検定を実施したところ,「正例数」および「修正率」の2指標において統計的な有意差が確認された.

具体的には,適合率が改善したプロジェクト群は,悪化した群と比較して正例数が有意に多く,かつ修正率も有意に高い傾向にあることが示された.この結果は,モデルの学習において十分な正例数を確保することの重要性と,人間による修正(フィードバック)の密度が適合率の向上に直接的に寄与していることを示唆している.

一方で,データの構成比にも留意すべき点が確認された.正例数そのものの多寡だけでなく,それに対する負例数の多さも適合率に悪影響を及ぼす阻害要因となっている可能性が高い.したがって,単に正例を増やすだけでなく,正負のデータバランスを考慮したデータセットの構築が,適合率を安定的に改善させるための鍵であると考えられる.



\subsection{正例数が少ないプロジェクトで予測精度が向上しなかったプロジェクトの特徴}



\section{まとめ}



\chapter{考察}



\chapter{妥当性への脅威}



\section{内的妥当性}



\section{外的妥当性}



\chapter{おわりに}



%%%%%%%%%%%%%%%%%%%%%%%%%%%%%%%%%%%%%%%%%%%%%%%%%%%%%%%%%%%%%%%%

%%
%% 謝辞
%%

%\begin{acknowledgements}
%感謝します.
%\end{acknowledgements}


\bibliographystyle{junsrt}
\bibliography{@Master2025_Kameoka/references}


\end{document}
