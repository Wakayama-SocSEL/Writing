% 英文で執筆する場合はクラスファイルへのオプションを[T,E]としてください.
% If you want to write your paper in English, pass to [T,E] options to document class.
\documentclass[T,J]{fose} % 「コンピュータソフトウェア」用のクラスファイルは compsoft です.
\taikai{2023} % 固定です.出版委員長が毎年変更してAuthor Kitを配布してください.

\usepackage [dvipdfmx] {graphicx}

% ユーザが定義したマクロなどはここに置く.ただし学会誌のスタイルの
% 再定義は原則として避けること.

% 以下は説明のために使用したパッケージであるため,削除可能.
\usepackage{fancyvrb}
\usepackage{xurl}
\usepackage{cite}
\usepackage{color}
\usepackage{listings}

\renewcommand{\lstlistingname}{Program}
\newcommand{\todo}[1]{\colorbox{yellow}{{\bf TODO}:}{\color{red} {\textbf{[#1]}}}}

\begin{document}

% 論文のタイトル
\title{{\foseabbrev}論文テンプレート}
% 以下の \etitle(と\@etitle)はFOSE論文フォーマット独自のマクロです.
% FOSEに投稿した論文を発展させてコンピュータソフトウェアに投稿される場合はコメントアウトしてください.
% \setetitleは奇数ページのヘッダに表示する文字列(\etitle)を設定するためのマクロです.
% タイトルが2行に渡る場合は "\\" を 使用することで任意の位置で改行をすることができます.
\setetitle{Foundation of Software Engineering}
%\setetitle{Long Long Long Long Long Long \\ Long Long Long Long Long \\ Long Long Long Long Long Long Long Long Long Long Long Long Paper Title}

% タイトル,著者などが複数行にわたり,論文冒頭の著者名が日本語アブストと重複して描画された場合に以下のコメントアウトを外してください.
%\longtitle

% 著者
% 和文論文の場合,姓と名の間には半角スペースを入れ,
% 複数の著者の間は全角スペースで区切る
%
\author{前川 大樹 伊原 彰紀 大森 楓己
%
% ここにタイトル英訳 (英文の場合は和訳) を書く.
% 英語タイトルは論文1ページ目左下,著者らの名前・所属一覧の一番上に表示される
%
% 上記\setetitle中で改行した場合は "\etitle" を削除し,改行(\\)を入れていないタイトルを記載してください.
% \ejtitleは1ページ目左下に挿入されるタイトルとして使用されます.
% また,"\etitle"はFOSE論文フォーマット独自のマクロです.
\ejtitle{\etitle}
%
% ここに著者英文表記 および
% 所属 (和文および英文) を書く.
% 複数著者の所属はまとめてよい.
%
%\shozoku{Daiki Maekawa}{和歌山大学}
%{Wakayama University}
% 複数著者の所属は以下のようにまとめてよい.
\shozoku{Daiki Maekawa, Akinori Ihara}{和歌山大学}
{Wakayama University}
}

%
% 和文アブストラクト
% In English paper, content of Jabstract will be ignored. 
\Jabstract{%
\todo{和文アブストラクト}
}
%
% 英文アブストラクト(本サンプルの原論文にはなし)
% \Eabstract{
% \todo{英語アブストラクト}
% }
%
\maketitle \thispagestyle {empty}

\section{はじめに}
サードパーティ製ライブラリ(以降,ライブラリ)は機能追加やバグ修正により頻繁に更新されており,利用者は適宜ライブラリのバージョン更新を余儀なくされる.利用者が安全にライブラリを更新するために,パッケージマネージャはセマンティックバージョニング\footnote{\url{https://semver.org}}を採用し,バージョン間の互換性の有無を管理している.セマンティックバージョニングでは,ライブラリ更新をメジャー,マイナー,パッチのレベルに分類し,マイナーとパッチの更新では後方互換性が保たれる変更が求められる.
しかし,バージョン名の付与は開発者が手動で行うため,破壊的変更が含まれているにも関わらず,誤ってマイナーやパッチに分類されてしまうことがある.
この問題は,特にJavaScriptのような動的な言語において顕著であり,ライブラリと利用者のコードの不整合が実行時まで検出されないという課題がある.
この課題に対して,ライブラリがリリースされる前に,破壊的変更の有無を検出する研究が行われている.

松田らは,破壊的変更を含むライブラリの更新は,プログラムの更新と合わせてテストコードも修正すると考え,テストコードの変更有無による破壊的変更の検出手法を提案した.
\cite{maekawa_matsuda}
しかし,テストコードの変更には,テストの誤り修正や実行手順の修正など,ライブラリ変更とは無関係にテストコードを変更することも考えられ,そのような変更は破壊的変更を誤検出する.

本研究では,テストコードの変更内容に着目し,破壊的変更の検出に寄与するテスト変更を明らかにし,テスト変更内容に基づく破壊的変更の検出手法の有効性を評価する.具体的には,2つのリサーチクエスチョンに回答する.

\begin{itemize}
\item RQ1:破壊的変更に伴うテストコード変更とは何か?
\item RQ2:テスト変更内容に基づく破壊的変更の検出手法の有効性はどの程度か?
\end{itemize}

RQ1では,松田らのデータセットを使用して,テストコードの変更内容を細分化し,実際に破壊的変更が行われた時,どのようなテスト変更がされたかを明らかにする.
RQ2では,破壊的変更の検出に寄与するテスト変更が行われた時,実際に破壊的変更が行われたかを評価し,従来研究と検出精度がどのように変わるかを評価する.


\section{\todo{todo}}

\section{RQ1:破壊的変更に伴うテストコード変更とは何か?}
本章では,テストコードの変更内容を細分化し,実際に破壊的変更が行われた時,どのようなテスト変更がされたかを明らかにする.

\subsection{テストコードの細分化}
Program~\ref{testSample}は,JavaScriptで記述されたテストコードの例を示す.
1行目でテスト対象の関数を取り込み,3行目から7行目のtest関数によってテストを行なっている.test関数は,第1引数にテスト内容を示すラベル,第2引数に実際のテストコードを含むコールバック関数が渡される.6行目で期待される結果と実際の結果を比較し,テストの成功や失敗を判断する.

本論文では,3から7行目をテストケース,4行目を入力値,5行目を期待値,6行目をアサーションと呼ぶ.

\begin{figure}[t]
    \begin{lstlisting}[caption={[upper/lower text]%
               \begin{tabular}[t]{@{}l@{}}
                test/sum.test.js \\[1.0\normalbaselineskip]
               \end{tabular}},frame={tb},numbers=left,label=testSample,identifierstyle={\small}]
const sum = require('./sum');

test('adds 1 + 2 to equal 3',() => {
  const input = sum(1, 2)
  const expected = 3
  expect(input).toBe(expected);
});
\end{lstlisting}
\vspace{-6mm}
\end{figure}

実際のテストコード変更内容を元に,テストコードの変更内容を以下の\todo{x}つに細分化した.

\begin{itemize}
    \item テストケースの追加
    \item テストケースの削除
    \item アサーションの追加
    \item アサーションの削除
    \item 入力値の変更
    \item 期待値の変更
    \item テストケース外の変更
    \item リファクタリング
\end{itemize}

\todo{テストケース外の変更とリファクタリングの説明}

\subsection{調査方法}
調査には,松田らのデータセット\cite{maekawa_matsuda}を使用する.松田らのデータセットは,Mujahidらのデータセットを元にしている.Mujahidらのデータセットには,npm\footnote{\url{https://www.npmjs.com/}}から,GitHubリポジトリが記載されていること,package.jsonの変更履歴が2回以上あることを条件に収集されたJavaScriptライブラリ290,417件が含まれている.松田らのデータセットは,Mujahidらが選定したJavaScriptライブラリ290,417件から,さらに条件を絞って238件のライブラリが含まれている.選定条件は,テストが付属していること,ライブラリの人気度を示すnpmスコア\footnote{\url{https://npms.io}}が上位500件以内であること,各バージョンのテスト実行時の成功率が100%であることが採用されている.松田らは,238件のライブラリについて,合計2,111件のライブラリバージョンを分析し,各バージョンの実際の破壊的変更の有無と,ライブラリのテスト変更有無をデータセットにまとめている.

本調査では,松田らのデータセットに含まれるライブラリバージョン2,111件から,ライブラリテストに変更があるライブラリバージョン1,027件を抽出した.その1,027件から,95%の信頼区間でサンプリングし,ランダムに280件を目視で確認した.

\subsection{調査結果}
表\ref{table_test_patern}は,ライブラリテスト変更内容と,実際の破壊的変更有無を示す.






\section{RQ2:テスト変更内容に基づく破壊的変更の検出手法の有効性はどの程度か?}
\subsection{概要}



\section{おわりに}

%\begin{adjustvboxheight} % needed only when Appendix follows
\bibliographystyle{jssst}
\bibliography{ref}
%\end{adjustvboxheight} % needed only when Appendix follows

% 以下はbibtexを使用しない場合の例です.
% 332行目と333行目をコメントアウトしてから使用してください.
% なお,この例では年数順に文献が並んでいるので適切な並び順ではありません.
%\begin{adjustvboxheight} % needed only when Appendix follows
%\begin{thebibliography}{9}
%\bibitem{fose2021} 名倉 正剛,関澤 俊弦 編:ソフトウェア工学の基礎28,日本ソフトウェア科学会{\em FOSE2021}, 近代科学社, 2021.
%\bibitem{fose2022} 角田 雅照,まつ本 真佑 編:ソフトウェア工学の基礎29,日本ソフトウェア科学会{\em FOSE2022}, 近代科学社, 2022.
%\bibitem{fose2023} 吉田 則裕,槇原 絵里奈 編:ソフトウェア工学の基礎30,日本ソフトウェア科学会{\em FOSE2023}, 近代科学社, 2023. (to appear)
%\end{thebibliography}
%\end{adjustvboxheight} % needed only when Appendix follows

%以下は付録の例です.必要ならコメントアウトして使用してください.
%なお,その際には参考文献の前後にある adjustvboxheight 環境のコメントアウトを解除してください.
%\appendix
%\section{付録A} 
%これは付録の例です.

\end{document}

