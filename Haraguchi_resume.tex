%
% 卒論レジュメフォーマット Ver.2.0 pLaTeX版
%
\documentclass[twocolumn]{jarticle} % 2段組のスタイルを用いている

\usepackage{wuse_resume}
\usepackage{url}	% \url{}コマンド用.URLを表示する際に便利
%\usepackage[dvipdfmx]{graphicx}  % ←graphicx.styを用いてEPSを取り込む場合有効にする
			% 他のパッケージ・スタイルを使う場合には適宜追加

%%%%%%%%%%%%%%%%%%%%%%%%%%%%%%%%%%%%%%%%%%%%%%%%%%%%%%%%%%%%%%%%%%%%%%%%

%%
%% タイトル,学生番号,氏名などを設定する
%%

\タイトル{分散台帳を用いた農産物の認証とトレーサビリティ評価\\-有機農産物の認証におけるケーススタディ-}
\研究室{ソーシャルソフトウェア工学}
\学生番号{60246233}
\氏名{原口 拓也}

\概要{%
本研究では,農産物の認証とそのトレーサビリティについて,分散台帳を用いたシステム設計とその評価を提案し,有機農産物におけるケーススタディによってその課題と可能性に関して評価する.従来手法では,消費者が有機農産物の認証を確認するためには有機JASマークを確認することでしか検証できない.これでは,マークの改ざんのリスク,トレーサビリティが担保されないといった課題がある.本研究での提案手法では,分散台帳上に農産物認証を付与することによって,有機農産物認証の改ざんがされず,トレーサビリティも担保された状態で認証を付与することが可能となった.また,これにより,消費者もより安心した形で,商品を選択することができ,生産者にとっても有機農産物認証を取得するための重要なインセンティブとなった.
}

\キーワード{日本農林規格}
\キーワード{分散台帳}
\キーワード{デジタル認証}
\キーワード{分散台帳}
\キーワード{ブロックチェーン}
\キーワード{トレーサビリティ}

%%%%%%%%%%%%%%%%%%%%%%%%%%%%%%%%%%%%%%%%%%%%%%%%%%%%%%%%%%%%%%%%%%%%%%%%

%% 以下の3行は変更しない

\begin{document}
\maketitle
\thispagestyle{empty} % タイトルを出力したページにもページ番号を付けない

%%%%%%%%%%%%%%%%%%%%%%%%%%%%%%%%%%%%%%%%%%%%%%%%%%%%%%%%%%%%%%%%%%%%%%%%

%%
%% 本文 - ここから
%%

\section{はじめに}

多くの消費者は食品を選択する際,食品の安全性を重視する.農林水産省による「食品を選択する際に重視すること」~\footnote[1]{https://www.maff.go.jp/j/syokuiku/ishiki/h29/zuhyou/z2-6.html}の調査結果では,安全性を重視する消費者は55.7%となっている.

また,牛海綿状脳症(以下,BSE問題とする)や,食品偽装問題によって,食品の安全性や生産方法に関する消費者の意識は年々高まっている.
それに対応するように,日本国内では生産手法に関する取り決めとして,日本農林規格(JAS)や,米トレーサビリティ法など食品の安全性に寄与する規格やガイドラインが確立されてきた.しかし,消費者がこの認証規格の基準を知ることは困難である.また,有機農産物の認証情報の改ざんの発生が問題視されている.さらに,生産者が認証を受けるまでにかかる管理コスト,認証にかかる費用などが高く,生産者にとってはこのような高いコストを払ってまでの費用対効果があるのかが疑問視されている.このような中で,農産物認証をより信頼を持った形で消費者に伝えられるような仕組み,生産者が認証制度に加盟するためのインセンティブの設計が求められている.

本研究では,消費者がより信頼性を持った形で有機農産物の認証を確認でき,生産者が認証の加盟に対してよりインセンティブが働くために,有機農産物の認証を分散台帳技術を用いて実現し,消費者における信頼性の向上と,トレーサビリティの評価を行う.

\section{有機農産物の認証と分散台帳を用いたデジタル認証}

\noindent\textbf{有機農産物の認証: }有機農産物の認証は.環境への配慮や,持続可能性が重要視されるようになり,消費者が有機栽培に関心を持つようになった.しかし,有機栽培に関しては具体的なガイドラインが制定されていなかったため,1990年後半から有機農業に関する議論が行われ1999年に有機農産物の認証規格(有機JAS制度)が制定された.有機農産物の生産者はガイドラインに則って生産工程管理記録を記録し,有機農産物の登録認証機関に申請を行う.その後,登録認証機関が調査を行い,ガイドラインに準拠した生産工程が管理されていると判断された時登録認証機関が農林水産大臣へと申請を行い,有機農産物の認証を農林水産大臣が登録を行い,登録認証機関から生産者に対して認証が付与される.
有機農産物の認証が与えられた後は,生産者の出荷する農作物に有機JASマークを付与することができる.消費者は,この認証マークを確認することによって有機農産物の認証を受けた農産物であることを確認する.

\begin{table*}[ht]
\centering
\begin{tabular}{|l|l|l|}
\hline
\textbf{ステップ} & \textbf{従来手法} & \textbf{提案手法} \\ \hline
1 & 有機JASマークを確認 & 有機JASマークを確認 \\ \hline
2 & 商品ラベルから生産事業者を確認 & QRコードから認定情報を検証 \\ \hline
3 & 生産業者に確認 &  \\ \hline
4 & 第三者認証機関(有機栽培認証協会)に確認 &  \\ \hline
5 & 認定を受けた圃場かを検証 &  \\ \hline
\end{tabular}
\caption{消費者が認証情報を確認するまでに必要なステップ}
    \label{tab:step}
\end{table*}

\noindent\textbf{分散台帳を用いたデジタル認証: }分散台帳技術(Distributed Ledger Technology:DLT)とは,複数のコンピュータが同じデータを共有,管理しており,ネットワーク参加者全員で台帳を管理することができる技術である.ブロックチェーン技術を用いたデジタル認証は,分散型台帳技術(DLT)を基盤として,取引の透明性,不変性,およびセキュリティを実現する手法である.デジタルアイデンティティ,契約,証明書などの認証情報をブロックチェーン上に記録し,その情報の検証を可能にすることで,第三者の介入なしに信頼性の高いデジタル認証を提供する.分散台帳技術の多産業への応用事例は様々なされているが,Geらは金融,産業,社会などさまざまな産業において分散台帳技術の応用について大きな関心ごとになっていると指摘している\cite{yuukisukunai}.そして農業サプライチェーンにおける分散台帳技術の導入を促進するための要因を特定し,それらの間の因果関係を確立することを目的として挙げている.Sachinらによる研究では農産物の流通全体のサプライチェーンにおけるトレーサビリティを分散台帳技術を用いて,可視性を上げる研究がなされている\cite{nousa}.

\section{有機キウイを用いたケーススタディ}

 有機農産物の認証に関しては,有機栽培認証協会の講習会への参加,栽培の管理記録,実地圃場の調査,登録機関への申し込みと承認が必要である.ここで,栽培記録の管理記録に関しては有機栽培認証協会が出している生産工程管理記録を採用した.そして,その生産管理と圃場実地調査によって,認証された証明書をを分散台帳上に記録し,消費者に届くまでのケーススタディを行う.ケーススタディに用いるデータは株式会社ReFruitsのキウイの生産管理記録を元に擬似的に作成したものである.

この証明された情報をQRコードに読み込み,キウイの商品パッケージに読み込み,消費者は商品に取り込まれた情報が,正しく有機栽培認証を受けたものであることが確認が可能となる.

section{有機農産物の分散台帳を用いた認証の流れ}
\ref{tab:step}は有機キウイを有機農産物認証を取得したのち,消費者がその認証に対しての検証にかかるステップ数である.\ref{tab:step}からわかるように,消費者が検証するまでは5つのステップ数が必要となる.一方,本研究での提案手法では,検証にかかるまでの有機JASマークを確認,QRコードでの検証の2つのステップのみであった.

これにより,消費者が有機農産物の認証をする際.従来手法と提案手法ではステップ数が大きく異なることが明らかとなった.

\section{考察}

確実に認証の真偽を確かめるためには,消費者が小売業者,流通業者,そして生産者に直接問い合わせて,認証情報の正確性を検証する必要がある.このプロセスは手間がかかるが,分散台帳技術を利用して認証情報をQRコードに組み込むことにより,消費者はその認証が正しいものであることを瞬時に,かつ容易に確認することが可能となる.この技術により,食品の安全性と信頼性の向上が期待される.

\section{おわりに}

本研究では,農産物の認証とそのトレーサビリティについて,分散台帳を用いたシステム設計とその評価を提案し,有機農産物におけるケーススタディによってその課題と可能性に関して評価した.本研究の目的は,有機農産物の認証に関して,信頼性の観点,食品情報の透明性を向上することである.その手法として,今回は分散台帳を用いて有機農産物の認証を行った.そして,その認証システムに関して,消費者が農産物認証の検証を行う際のステップ数の比較を行った.その結果,従来手法では5つのステップ数がかかっていたのを提案手法では2つのステップ数で確認できることがわかった.また,提案手法では生産者が特筆したい栽培方法等を分散台帳上に記載することが可能となり,消費者が有機農産物認証を購入するための重要なインセンティブとなった.

%%
%% 本文 - ここまで
%%

%%%%%%%%%%%%%%%%%%%%%%%%%%%%%%%%%%%%%%%%%%%%%%%%%%%%%%%%%%%%%%%%%%%%%%%%

%%
%% 参考文献
%%

\bibliographystyle{junsrt}
\bibliography{Haraguchi}

%%%%%%%%%%%%%%%%%%%%%%%%%%%%%%%%%%%%%%%%%%%%%%%%%%%%%%%%%%%%%%%%%%%%%%%%

\end{document}
