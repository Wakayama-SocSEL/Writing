\documentclass[uplatex,dvipdfmx,a4paper,twocolumn,base=11pt,jbase=11pt,ja=standard]{bxjsarticle}  % 環境に合わせて変更してください

\usepackage{ipsj}
\usepackage{color}

%追加パッケージ
\usepackage{enumerate}
\usepackage{url}
\usepackage{graphics}
\usepackage{caption}

\newcommand{\todo}[1]{\colorbox{yellow}{{\bf TODO}:}{\color{red} {\textbf{[#1]}}}}

\title{後方互換性の損失に伴うテストコード変更内容の分類}{A classification of test code changes accompanied by a lack of backward compatibility}
\author{和歌山大学}{前川 大樹}{Daiki Maekawa, Wakayama University}
\author{和歌山大学}{伊原 彰紀}{Akinori Ihara, Wakayama University}
\author{和歌山大学}{大森 楓己}{Fuki Omori, Wakayama University}
\author{和歌山大学}{才木 一也}{Kazuya Saiki, Wakayama University}

\begin{document}
\maketitle

%================
%1
\section{はじめに}
%================
% (概要)ソフトウェアライブラリの更新において,開発者は意図せずに後方互換性を損失する変更(破壊的変更)を追加してしまうことがある.従来研究では,ライブラリに破壊的変更を加えたリリースバージョンではテストコードも修正すると考え,テストコードの変更有無による破壊的変更の検出手法を提案した.しかし,テストコードの変更はライブラリ変更に関係するとは限らないため,破壊的変更を誤検出することがある.本研究では,破壊的変更に伴うテストコード変更内容の分類に取り組み,テスト変更内容に基づく破壊的変更の検出精度を向上する.

ソフトウェア開発では,開発効率向上のためにライブラリと呼ばれるプログラムを再利用することが多い.新機能追加,バグ修正のためにライブラリの新バージョンがリリースされた場合,ライブラリ利用者は適宜ライブラリのバージョン更新を余儀なくされる.ライブラリの更新には後方互換性を損失する変更(破壊的変更)を含むことがあり,破壊的変更は利用者のプログラムの実行エラーにつながる.そのため,ライブラリ開発者は,適切なバージョン名によって破壊的変更の有無を利用者に伝える.しかし,ライブラリ開発者が破壊的変更を含むことを見逃し,意図せず利用者に被害を与えることも少なくない.

%頻繁に更新されるため,ライブラリ利用者は適宜アップデートをする必要がある.しかし,ライブラリの更新には後方互換性を損失する変更(破壊的変更)が含まれることがあり,利用者のプログラムの実行エラーにつながることがある.そのため,ライブラリ開発者はライブラリを更新する際,破壊的変更の有無を判断し,適切なバージョン名を付与することで,破壊的変更の有無をライブラリ利用者に伝達する.しかし,破壊的変更の有無の確認は手動であるため,ライブラリ開発者が破壊的変更を見逃してしまい,意図せず利用者のプログラムが破壊されてしまうことがある.

従来研究~\cite{FOSE2021_Matsuda}では,破壊的変更を含むライブラリの更新は,プログラムの更新と合わせてテストコードも修正すると考え,テストコードの変更有無による破壊的変更の検出手法を提案した.しかし,テストコードの変更には,テストの誤り修正や実行手順の修正など,ライブラリ変更とは無関係にテストコードを変更することも考えられ,そのような変更は破壊的変更を誤検出する.本研究では,従来手法の誤検出を減らすために,テストコード変更内容の分類に取り組み,破壊的変更検出に寄与するテスト変更を明らかにする.

%================
\begin{table*}[t]
  \caption{ライブラリテスト変更内容の分類}
  \label{table_test_patern}
  \centering
  \scalebox{1}{
  \begin{tabular}{l|r|r|r}
        \hline
        テスト変更内容 & \multicolumn{1}{c|}{破壊的変更あり} & \multicolumn{1}{c|}{破壊的変更不明} & \multicolumn{1}{c}{合計件数} \\ \hline\hline
        等価な変数や関数に置換 & 11件 & 1件 & 12件 \\ \hline
        テストケースの追加 & 57件 & 28件 & 85件 \\ \hline
        期待する結果の変更 & 27件 & 6件 & 33件 \\ \hline
  \end{tabular}
  }

\end{table*}
%================



%================
%2
\section{分析}

%松田らのデータセット~\cite{FOSE2021_Matsuda}を使用する.データセットには,npm\footnote{npm: \url{http://hogehoge}}から収集した,人気度が高くテストが整備されているライブラリバージョン2,111組が含まれている.このデータセットから,npmに公開され,ライブラリテストに変更があり,テスト変更のdiffサイズが1024*2バイト以下であるもの1,027組を使用する.対象データ1,027組から実際に破壊的変更が含まれるバージョンと含まれないバージョンを50組ずつランダムに抽出し計100組を分析した.

%松田データセットの選定条件
%\vspace{-2mm}
%\begin{itemize}
% Mujahidらのデータセット
% \item GitHubリポジトリが記載されていること
% \item package.jsonの変更履歴が2回以上あること
% 松田らのデータセット
% \item テストが付属していること
% \item ライブラリの人気度を示すnpmスコア~\footnote{https://npms.io}が上位500件以内であること
% \item 各バージョンのテスト実行時の成功率が100%であること
%\end{itemize}



%================
%2.1
\subsection{分析手法}
%================

本研究では,Mujahidらがnpmから収集したライブラリ~\cite{Mujahid}の中で,人気度を示すnpmスコア上位500件を対象とする.また,各バージョンのテスト実行時の成功率が100%であるライブラリのバージョン更新2,111組から,破壊的変更が含まれるバージョンと含まれないバージョンを50組ずつランダムに抽出し計100組のテスト変更内容を目視により分析する.本研究では,ライブラリ利用者のプログラムがライブラリの新バージョンの利用において,テストの実行結果が失敗であった場合に,破壊的変更が加えられたと判断する.

%
%
%100件のデータについて,2段階に分けて分析する.
%\vspace{-2mm}
%\begin{enumerate}
% \item テスト変更パターンを分類し出現回数を集計する
% \item 1.をもとに,決定木を用いて,破壊的変更に関係するテスト変更パターンについて分析する
%\end{enumerate}

テストプログラムは複数のテストケースからなるテストスイートが組み合わさって構成されており,個々のテストケースに前提条件,期待する結果などが記述されている.著者らはテストプログラムの変更内容を目視で確認し,変更理由に基づきテストの変更を分類した.その後,それぞれの変更の発生回数を集計した.

%これらをもとに,ライブラリテストの変更内容を,テストケースの追加,削除,期待する結果の変更など,計19種類に分類した.分類をもとに,100件のライブラリバージョンについて,それぞれの出現回数を集計した.


%================
%2.4
\subsection{結果}
%================

ライブラリテストの変更内容を分析した結果,テスト変更内容を19種類に分類できた.具体的な種類は紙面の都合上省略し,一部を表\ref{table_test_patern}に示す.表には,破壊的変更の有無に影響を与えるテスト変更内容を探索するために,テスト変更内容19種類を説明変数,破壊的変更の有無を目的変数とした決定木分析により,特に破壊的変更有無に影響が高いとわかった3種類を示す.

\noindent\textbf{等価な変数や関数に置換: }このテスト変更においては,1件を除いて破壊的変更を含むライブラリの更新であった.しかし,等価な変数や関数に置換するテスト変更自体は,破壊的変更に伴う変更ではないが,開発者が同時に修正することで破壊的変更に関係するテスト変更として検出する.今後は,破壊的変更とは関係ないと考えられるテスト変更は無視することで,再分析する必要があると考える.

%決定木分析のルートノードであり,破壊的変更と関係のある変更という結果となった.しかし,テスト自体の挙動は変わらない変更のため,破壊的変更とは関係のないテスト変更と言える.原因として,データ数が少ないこと,他のテスト変更との組み合わせによるものが考えられる.等価な変数や関数に置換するテスト変更と同時に発生しやすいテスト変更を調査したところ,破壊的変更ありの11件中7件,テストケースの追加と同時に行われており,破壊的変更不明の1件はテストケースの追加が行われていなかった.テストケースの追加と同時に,等価な変数や関数に置換された時,破壊的変更に関係するテスト変更であると言える.

\noindent\textbf{テストケースの追加: } テストケースの追加は,破壊的変更に伴って

%テストケースの追加は,全体数が多く,破壊的変更であるものの方が少し多いという結果となった.
%破壊的変更と関係のある場合は,テスト対象の関数の挙動が変更されたため,それに対応するテストケースを追加する必要があったと考えられる.関数の振る舞いの変更という破壊的変更に関係するテスト変更と言える.破壊的変更と関係のない場合として,テストケースが不足しており,テスト品質向上のため追加したということが考えられる.自動分類の方法として,変更前後でテストケースを記述する関数(itやtest)が追加されているかどうかで捉えることが考えられる.
    
\noindent\textbf{期待する結果の変更: }期待する結果の変更事例として,equal文やassert文など,アサーション関数の引数の変更がある.テスト対象の関数の出力内容が変更されたことで,変更前の形式で出力を検証していたテストにも変更が加えられたと考えられる.期待する結果の変更は,関数の出力内容の変更という破壊的変更によるテスト変更であるため,変更前後のASTを比較することによって引数の変更を捉え,破壊的変更の検出ができると考える.

%破壊的変更はないのに期待する結果を変更する必要がある例
%テスト環境に変更があったことが考えられる
%ネストを含むファイル構造を全て取得するライブラリfs-utils/fs-readdir-recursiveの0.0.1から0.0.2のアップデートを例に上げる.fs.readdirSyncRecursive()を検証するテストスイート内にディレクトリルートでフィルター付きで適切に動作するか確認するテストケース(should work at the root with a filter)がある.
%関数の返り値として,ファイル構造の配列が返る.
%Makefileであった部分が,LICENSEに変わっている.
%関数の挙動は変わっていないが,プロジェクトのディレクトリ構造が変わった(Makefileを削除してLICENSEを追加)ために,関数の期待する結果もそれに応じて変える必要があった.
%ディレクトリ構造の変更はテストの変更のみでは捉えられないのでテスト変更内容に基づく破壊的変更検出手法では捉えることはできない.

%https://github.com/fs-utils/fs-readdir-recursive/compare/f6399f7cc61e87271ddde0035a9ce5c64f706575...fd50a8a935509cbc55e76ccb7754197d48a17457



%================

%================
%3
\section{おわりに}
%================

本研究は,テスト変更の内容分類に取り組み,テスト変更内容と破壊的変更有無の関係性を決定木により分析した.分析の結果,ライブラリテストの変更内容として,テストケースの追加,期待する結果の変更が含まれると,破壊的変更が含まれる可能性が高いことがわかった.今後は,破壊的変更とは関係ないと考えられるテスト変更を除外し,テスト変更内容の自動分類に取り組むとともに,テスト変更内容に基づく破壊的変更の検出手法の精度向上を目指す.

\vspace{-2mm}
%================
\section*{謝辞}
%================

本研究は和歌山大学「萌芽的個別研究支援」,および柏森情報科学振興財団より助成を受けたものです.

\vspace{-2mm}
%================
%\section*{参考文献}
%================
\begin{thebibliography}{10}
  \bibitem{FOSE2021_Matsuda} 松田和輝,伊原 彰紀,才木 一也.ライブラリのテストケース変更に基づく後方互換性の実証的分析.第28回ソフトウェア工学の基礎ワークショップ論文集,pp.139-144,2021.
  \bibitem{Mujahid} S. Mujahid,R. Abdalkareem,E. Shihab,and S. McIntosh.``Using others’ tests to avoid breaking updates,'' Proc of the International Conference on Mining Software Repositories (MSR), pp.466-476, 2019.
\end{thebibliography}

\end{document}
