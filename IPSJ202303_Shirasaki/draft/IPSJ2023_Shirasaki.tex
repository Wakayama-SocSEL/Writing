\documentclass[uplatex,dvipdfmx,a4paper,twocolumn,base=11pt,jbase=11pt,ja=standard]{bxjsarticle}  % 環境に合わせて変更してください

\usepackage{ipsj}
\usepackage{color}

%追加パッケージ
\usepackage{enumerate}
\usepackage{url}
\usepackage{graphics}
\usepackage{caption}
\usepackage[dviout]{graphicx}



\newcommand{\todo}[1]{\colorbox{yellow}{{\bf TODO}:}{\color{red} {\textbf{[#1]}}}}

\title{README修正内容に関連する\\ソースコード変更コミット追跡への試み}{Toward tracking source code change commits related to README revision}
\author{和歌山大学}{白﨑 優奈}{Shirasaki Yuna, Wakayama University}
\author{和歌山大学}{伊原 彰紀}{Akinori Ihara, Wakayama University}
\author{和歌山大学}{石岡 直樹}{Naoki Ishioka, Wakayama University}

\begin{document}
\maketitle


%================
%1
\section{はじめに}
%================

%hogehoge(背景,動機とか)

多くのソフトウェア開発プロジェクトでは,ソフトウェアの機能,使用方法などを開発者や利用者に発信するために,ソフトウェアドキュメントとしてREADMEを公開している.ソフトウェア開発の中でもGitHubを利用するオープンソースソフトウェア (OSS) 開発では,開発への参加方法,各機能の概要などもREADMEに記載している.

% (プロプライエタリソフトウェアでは,定期的なリリースの際にソフトウェアとドキュメントを同時に公開する.一方で)
OSSプロジェクトは,実装中のソースコードを逐一公開しているため,ソースコードの更新に伴ってREADMEも更新することが期待される.しかし従来研究ではソースコードとREADMEが紐付けられていないため,ソースコードとREADMEがどのように共進化しているかは明らかでない.また,開発者はソースコードとREADMEを同時に更新していない場合もあるため,READMEの更新が必要となったソースコードの変更コミットを紐付けることは容易でない.

本研究では,README更新の追従性を確認するために,ソースコードとREADMEを共通して含まれる単語に基づいて関連付ける.具体的には,READMEの変更箇所とソースコードを説明する特徴的な文字列とを比較することで,READMEの変更を誘発したソースコードの変更コミットの候補を特定する.

%================
%2
\section{分析}
%================


%2.1
\subsection{データセット\todo{未修正}}
%================
%hogehoge(どんなデータセット使ったか)

%fugofugo(そこからどんなデータをどうやって集めたか)


本論文では,READMEの更新が10回以上と多く,英語で記述されている,aliyun-openapi-nodejs-sdk
((TODO:リンク))と〇〇を対象とする.







%================
%2.2
\subsection{分析手法}
%================
%hogehoge(分析を行うためにどんな準備をしたか(スニペット削除とか))

%fugofugo(なぜその準備をしたのか(コードがあったらよろしくないとか))

READMEの変更を誘発するソースコードの変更コミットを確認するために,各コミットで追加された文章をデータ整形する.\todo{各コミットで変更されたソースコードに含まれる文字列を抽出する}.本研究では,READMEが変更されたコミット,および前後のコミットにおいて変更された,\todo{JavaScript言語で記述されたソースコードのみ?}を対象とし,ソースコードファイル中から記号削除,NLTKを用いて形態素解析により名詞のみを抽出する.

%具体的には,ソースコードファイル中の
%
%\vspace{-2mm}
%\begin{enumerate}
%    \item READMEが変更されたコミットについて,各コミットで追加された文章を抽出
%    \item 正規表現を用いて,記号の削除
%    \item NLTKを用いて,説明文の分かち書き
%    \item NLTKを用いて,名詞を抽出
%    \item 全コミットを対象に,追加されたソースコードに同様の処理
%\end{enumerate}
%
%以上の方法で,READMEの変更コミットで追加された文章に含まれる名詞,全コミットで追加されたソースコードに含まれる名詞が得られる.

抽出した単語に一致単語があるかを比較分析する.\todo{上記では,READMEからの収集方法について書いていないように思いますが....}比較ペアは,README変更コミットで追加された単語と,その前後Nコミットのコード変更で追加された単語である.

%================
%2.3
 %\subsection{評価方法}
%================

%42(どんな分析の仕方をしたのか(「use」と「その他」でそれぞれ分けて考えたとか))

%================
%2.4
\subsection{結果}
%================
%hogehoge(どんな結果が得られたか,図とかを一緒に載せる)

%fugofugo(結果からの考察)
%分析を行った結果次のような結果が得られた.

図1は\todo{リポジトリ名}を対象に,図2は\todo{リポジトリ名}を対象に,2.2の手法で分析した結果である.
% \documentclass{jsarticle}
% \usepackage{bardiag}
% \begin{document}
% \renewcommand{\betweenticks}{50}
% \bardiagrambegin{9.5}{350}{1}{2}{1cm}{0.02cm}
%     % \drawlevellines
%     % \baritem{-20}{23}{blue}
%     % \baritem{-19}{5}{blue}
%     % \baritem{-18}{2}{blue}
%     % \baritem{-17}{24}{blue}
%     % \baritem{-16}{24}{blue}
%     % \baritem{-15}{27}{blue}
%     % \baritem{-14}{21}{blue}
%     % \baritem{-13}{20}{blue}
%     % \baritem{-12}{7}{blue}
%     % \baritem{-11}{2}{blue}

%     \baritem{0}{23}{blue}
%     \baritem{1}{5}{blue}
%     \baritem{2}{2}{blue}
%     \baritem{3}{24}{blue}
%     \baritem{4}{24}{blue}
%     \baritem{5}{27}{blue}
%     \baritem{6}{21}{blue}
%     \baritem{7}{20}{blue}
%     \baritem{8}{7}{blue}
%     \baritem{9}{2}{blue}
% \bardiagramend{\large README変更コミットの前後のコミット}{\large 一致単語数}
% \end{document}

以下のグラフは,README変更コミットで追加された単語とその前後20コミットで追加された単語を比較し,10以上の一致単語があるならば一致数を,一致単語が10未満なら0を加算し,README変更コミットやその前後コミットにおいて,関連単語がいくつあるのかを示している.

A(0)はREADMEが更新されたコミットで,更新されたREADMEと変更されたソースコードにおいて一致した単語の数を表している.
A(-2)はREADME更新の2つ前のコミットで,A(0)で更新されたREADMEと,このコミットで変更されたソースコードにおいて一致した単語の数を表している.
A(0)では,〇〇,〇〇などの(実際の単語でも可,品詞でも可)単語が見られ,実際にREADMEに反映された単語が多く含まれていた.
A(-2)では,〇〇,〇〇などの(実際の単語でも可,品詞でも可)単語が見られ,実際にREADMEに反映された単語が多く含まれていた.ここから,ソースコードをまとめて変更してからその変更をREADMEに反映させて開発していることがわかった.

% 図1より,README変更コミットとその前のコミットで,一致単語が多くみられた.ここから,README変更に関連のあるソースコードは,READMEとほぼ同時に変更される分量が多いことがわかった.


同様に,B(0)はREADMEが更新されたコミットで,A(-2)はREADME更新の2つ前のコミットで,READMEとソースコードにおいて一致した単語の数を表している.A(0)では,〇〇,〇〇などの(実際の単語でも可,品詞でも可)単語が見られ,実際にREADMEに反映された単語が多く含まれていた.
A(0)では,〇〇,〇〇などの(実際の単語でも可,品詞でも可)単語が見られ,実際にREADMEに反映された単語が多く含まれていた.
A(-2)では,〇〇,〇〇などの(実際の単語でも可,品詞でも可)単語が見られ,実際にREADMEに反映された単語が多く含まれていた.ここから,ソースコードをまとめて変更してからその変更をREADMEに反映させて開発していることがわかった.

以上から,\todo{2つの結果からわかることの記述}


% 以下使わない
% また図2は,README変更コミットで追加された単語とその前後20コミットで追加された単語を比較し,10以上の一致単語があるならば一致する(1),一致単語が10未満なら一致しない(0)として,各コミットの一致の有無を示したグラフである.
% 図2より,README変更前の方が,README変更後よりも密にソースコードの変更がみられる.ここから,ソースコードを変更してから,まとめてREADMEが変更されることが多いことがわかった.


%--------------------------------グラフ

%--------------------------------

%================
%3
\section{おわりに}
%================

%hogehoge(今回の研究のまとめとか)

%fugofugo(今回研究から得られたことを踏まえて今後どう発展させていくか)

%42(どんな研究をしていくつもりなのか的な)
%今回の研究のから hogehoge-fugofugoである.

本論文では,READMEがソースコードの変更を追従しているかを確認するため,各ファイルの単語を比較した.
% 分析の結果,README更新の前からREADMEが更新されるまでの間に,READMEに関連のあるソースコードが多く変更されていることがわかった.
分析の結果,ソースコードの変更を反映させるREADMEの更新は,コード変更の前にも後にもおこなわれていることがわかった.
今後は,READMEとソースコードの追従性をより詳細に分析するとともに,ソースコードの変更をREADMEに反映させるタイミングによって,READMEにどのような影響があるのかを明らかにすることを目指す.



%================
\section*{謝辞}
%================


%================
\section*{参考文献}
%================
%[1]Shohei IKEDA, Akinori IHARA, Raula Gaikovina KULA, and Kenichi MATSUMOTO : An Empirical Study ofREADMEcontents for JavaScript Packages : IEICE TRANS. INF. and SYST., VOL.E102–D, NO.2 FEBRUARY 2019

%[2]JINHAN KIM, SANGHOON LEE, and SEUNG-WON HWANG, SUNGHUN KIM : Enriching Documents with Examples: A Corpus Mining Approach : ACM Trans. Inf. Syst. 31, 1, Article 1 (January 2013)

%[3]Noela Jemutai Kipyegen and William P. K. Korir : Importance of Software Documentation : IJCSI International Journal of Computer Science Issues, Vol. 10, Issue 5, No 1, September 2013 

((TODO:従来研究のところに参考文献))


%================






\bibliographystyle{ipsjunsrt}
\bibliography{bibfile}

\end{document}
