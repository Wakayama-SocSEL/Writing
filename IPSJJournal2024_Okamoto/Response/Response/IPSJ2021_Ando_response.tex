\documentclass{jarticle} %LaTeX2e


\usepackage{color}
%\usepackage{graphicx}
\usepackage[dvipdfmx]{graphicx} % pdfを使用する
\usepackage{tabularx}
\usepackage{fancyhdr} % header, footer
\usepackage{lastpage} % 総page数
\usepackage{multirow}
\usepackage{ulem}
\usepackage{latexsym}
\usepackage{maltcol}
\usepackage{jumoline}
\usepackage{caption}
\usepackage{url}

\setlength{\UnderlineTexDepth}{3pt}

\newcommand{\todo}[1]{\colorbox{yellow}{{\bf TODO}:}{\color{red}{\textbf{[#1]}}}}

\newcommand{\ihara}[1]{\colorbox{green}{{\bf ihara}:}{\color{red}{\textbf{[#1]}}}}

\renewcommand{\headrulewidth}{0pt} % ヘッダーラインを打ち出さない

% todayコマンド 西暦表示
\renewcommand{\today}{%
  \the\year /%
   {\ifnum \month < 10  0\the\month \else \the\month \fi}/%
   {\ifnum \day < 10  0\the\day \else \the\day \fi}%
}

% header, footer周りの定義
\pagestyle{fancy}

% 回答文タイトル
\def\restitle{条件付論文に対する回答文}
\def\paptitle{コンピュテーショナル・シンキングスキルに基づく\\Scratchユーザの習熟度到達予測}

%% header
\lhead{\restitle}
\rhead{\paptitle}

%% footer
\cfoot{}
\lfoot{\date{\today}}
\lfoot{\today}
\rfoot{\thepage/\pageref{LastPage}}

%% ページの余白
\setlength{\oddsidemargin}{0pt}
\setlength{\evensidemargin}{0pt}
\setlength{\topmargin}{-20pt}
\setlength{\textwidth}{440pt}
\setlength{\textheight}{640pt}
\setlength{\headheight}{30pt}
\setlength{\headwidth}{\textwidth}

%% 回答書用sectionの再定義

% ラベル:査読者Aの方から〜対する回答
\def\section#1{ \vspace{3pc} {\large \gt #1} \vspace{1pc} \hrule }

% ラベル:条件,回答,変更
\def\subsection#1{ \vspace{1pc} {\gt #1} }

% 次の回答文へ移動
\def\nextans{ \vspace{2pc} \hrule }


% テキストカラー
%\def\red#1{\textcolor{red}{#1}}


%%%
\begin{document}

% 回答書見出し
{\Large \gt \restitle}

\vspace{3pc}

査読者各位\\


この度は私共の以下の論文に対して熱心な御査読を頂き,有益な御意見を頂きましたことに厚く御礼申し上げます.\\
\textbf{論文ID:22-L012}\\
\textbf{タイトル:『コンピュテーショナル・シンキングスキルに基づくScratchユーザの習熟度到達予測』}

貴重な御意見を反映するよう,論文の加筆・修正を行いましたので再投稿させていただきます.御手数おかけ致しますが,再度御査読の程どうぞ宜しくお願い致します.	

本回答文は,御指摘に対する「回答」と,論文の変更内容を示す「変更」の順に記載しております.一つのコメントに対して複数の変更を行った場合は,(変更1-1-a),(変更1-1-b)のように番号の後ろにアルファベットを付しております.御理解の程,どうぞ宜しくお願い致します.\\

第2査読者の方から(条件2-1)において論文中の「コンピュテーショナル・シンキング(CT)スキル」の用語が適切でない(CTという語自体がスキルという意味合いを含んでいる)と御指摘を受けました.私共でも改めて従来研究を確認したところ,私共の使用していた用語が適切ではありませんでした.貴重な御指摘を頂き,誠にありがとうございました.論文中で多用している用語であり,誤解を招く可能性があるため,以降のメタ査読者,査読者1,査読者2の方々からの御指摘に対する回答は,タイトルを含めた本論文における「CTスキル」および「CTスキルの概念」の表記を,それぞれ「CT」および「CT概念」のように変更した用語でご説明することを,回答書冒頭でご報告しておきます.本変更は論文中で多用する語のため,論文中には変更箇所を赤字で記載し,アルファベットは付しません.どうぞ宜しくお願い致します.

%--------------------------------------------------------------------------------
\section{査読委員の方(メタ査読者)から頂いたコメントに対する回答}
%--------------------------------------------------------------------------------

\subsection{(コメントM-1)}

査読者1の「コメント」に基づき,「学習者が習熟したという状態」と「DevelopingとMasterという観点の評価履歴」との関係性やこれからわかることの限界について,(必要なら従来研究の知見を踏まえ,)提案手法の有効性が毀損されないように議論がなされるとよいでしょう.

\subsection{(回答M-1)}

御指摘を頂き,誠にありがとうございます.第1査読者の方から御指摘頂いた内容に対する回答を(コメントへの回答1-1)に述べ,それに伴う変更を行いましたので御確認ください.

%--------------------------------------------------------------------------------
\newpage
\nextans
\subsection{(コメントM-2)}

査読者2の「参考意見」に基づき,文意が正確に伝わるように文章を改訂されるとよいでしょう.

\subsection{(回答M-2)}

御指摘を頂き,誠にありがとうございます.第2査読者の方から御指摘頂いた参考意見について,文章の修正を行いましたので御確認ください.

%--------------------------------------------------------------------------------
\nextans
\subsection{(コメントM-3)}

Abstract 最終文: ``an important concept'' -\textgreater ``important concepts'' ?

\subsection{(回答M-3)}

貴重な御指摘を頂き,誠にありがとうございます.御指摘の通り,変更M-3に示すようにAbstractの文章を修正しました.

\subsection{(変更M-3)Abstract 最終文}
\vspace{-0.3cm}
\begin{description}
\item 修正前\\
\phantom{ }
Furthermore, this study finds ``Synchronization'' and ``Flow control'' as an important concept of CT skill to create works with master level proficiency.
\vspace{-0.3cm}
\item 修正後\\
\phantom{ }
Furthermore, this study finds ``Synchronization'' and ``Flow control'' as \textcolor{red}{\UnderlineTex{important CT concepts}} to create works with master level proficiency.
\end{description}


%--------------------------------------------------------------------------------
\nextans
\subsection{(コメントM-4)}

P2右段 8行目: ``Scratchは,・・でき,他のユーザは・・複製し,再利用する・・が提供している.'' -\textgreater  文章に違和感がなくなるよう改訂された方がよいでしょう.例:``Scratchでは,・・でき,また,他のユーザが・・複製して再利用できる・・が提供されている.''

\subsection{(回答M-4)}

貴重な御指摘を頂き,誠にありがとうございます.御指摘の通り,違和感のある文章となっていたこと,謹んでお詫び申し上げます.変更M-4に示すように文章を修正しました.

\subsection{(変更M-4)P.2 2.1節}
\vspace{-0.3cm}
\begin{description}
\item 修正前\\
\phantom{ }
Scratchは,ユーザが制作した作品をScratchサービス上に公開することができ,他のユーザは公開作品を複製し,再利用する機能「リミックス」が提供している.
\vspace{-0.3cm}
\item 修正後\\
\phantom{ }
\textcolor{red}{\UnderlineTex{Scratchでは,ユーザが制作した作品をScratchサービス上に公開し,さらに,サービス上に公開済みの他の作品を複製して再利用できる機能「リミックス」を提供している.}}
\end{description}

%--------------------------------------------------------------------------------
\newpage
\nextans
\subsection{(コメントM-5)}

P3 3.1節 6行目: ``初めて作品制作を行ったユーザ'' -\textgreater  ``初めて作品公開を行ったユーザ'' ? ※もし本指摘が妥当であれば,他の``制作''も必要な箇所は``公開''に置き換えてください.

\subsection{(回答M-5)}

貴重な御指摘を頂き,誠にありがとうございます.御指摘の通り,本研究では,Scratchサービスに公開済みの作品のみを分析対象としているため,「初めて作品\UnderlineTex{公開}を行ったユーザ」が正しい説明になります.御指摘頂いた文章を含め,論文中において修正が必要な箇所を「制作」から「公開」の表現に変更しました.変更点は,変更M-5-a,変更M-5-bに示しています.

\subsection{(変更M-5-a)P.3 3.1節}
\vspace{-0.3cm}
\begin{description}
\item 修正前\\
\phantom{ }
本研究では,ユーザが共通の開発環境で制作された作品を比較するため,バージョン3.0をリリースした2019年1月3日から2020年1月3日までに初めて作品制作を行ったユーザを分析対象とする.
\vspace{-0.3cm}
\item 修正後\\
\phantom{ }
本研究では,ユーザが共通の開発環境で制作された作品を比較するため,バージョン3.0をリリースした2019年1月3日から2020年1月3日までに\textcolor{red}{\UnderlineTex{初めて作品公開を行ったユーザ}}を分析対象とする.
\end{description}

\subsection{(変更M-5-b)P.7 5.1節}
\vspace{-0.3cm}
\begin{description}
\item 修正前\\
\phantom{ }
モデル2の分類精度はモデル1に比べて低いため,20件の作品制作で一度もMasterに到達しなかったユーザの一部は,Masterに到達したユーザとの間で,過去に使用したCTスキルに違いがないことが示唆される.
\vspace{-0.3cm}
\item 修正後\\
\phantom{ }
モデル2の分類精度はモデル1に比べて低いため,\textcolor{red}{\UnderlineTex{公開した20件の作品の中で}}一度もMasterに到達しなかったユーザの一部は,Masterに到達したユーザとの間で,過去に使用したCT概念に違いがないことが示唆される.
\end{description}

%--------------------------------------------------------------------------------
\newpage
\nextans
\subsection{(コメントM-6)}

P4右段 2行目: ``各バイオリン図の膨らみが大きい縦軸の点数は," -\textgreater  ``各バイオリンの膨らみが最も大きい箇所は," ?

\subsection{(回答M-6)}

貴重な御指摘を頂き,誠にありがとうございます.御指摘の通り,変更M-6に示すように文章を修正しました.

\subsection{(変更M-6)P.4 3.3節}
\vspace{-0.3cm}
\begin{description}
\item 修正前\\
\phantom{ }
各バイオリン図の膨らみが大きい縦軸の点数は,1番目からN-1番目に制作した作品の中で最も獲得した回数が多い合計点数を表す.
\vspace{-0.3cm}
\item 修正後\\
\phantom{ }
\textcolor{red}{\UnderlineTex{各バイオリンの膨らみが大きい箇所}}は,1番目からN-1番目に制作した作品の中で最も獲得回数の多かったCTスコアを表す.
\end{description}

%--------------------------------------------------------------------------------
\newpage
\nextans
\subsection{(コメントM-7)}

P8 表4:モデル2 グループ6で``\{リミックス/論理/0点\}"が重複しています.もし誤植でしたら,ご修正ください.

\subsection{(回答M-7)}

貴重な御指摘を頂き,誠にありがとうございます.御指摘の通り,表の結果が一部重複したものとなっていました.謹んでお詫び申し上げます.正しくは``\{\UnderlineTex{オリジナル}/論理/0点\}''であるため,変更M-7に示すように表結果を一部修正しました.


\subsection{(変更M-7)P.8 表4 }

%--------------------------
% 修正前
%--------------------------
\setcounter{table}{3}
\begin{table}[h]
    \caption{2つのモデルにおける重要度の高い説明変数の上位10件}
    \vspace{-0.5cm}
    \begin{center}
        \scalebox{0.8}[0.8]{
            \begin{tabular}{r|rp{55mm}|rp{55mm}}
                \hline
                & \multicolumn{2}{c|}{モデル1} & \multicolumn{2}{c}{モデル2} \\
                \cline{2-5}
                グループ & \begin{tabular}{r} 重要度 \end{tabular} & \begin{tabular}{l} 説明変数 \end{tabular} & \begin{tabular}{r} 重要度 \end{tabular} & \begin{tabular}{l} 説明変数 \end{tabular} \\ \hline
                1 & \begin{tabular}{r}0.05\end{tabular} & \begin{tabular}{r} \{オリジナル/フロー制御/0点\} \end{tabular} & \begin{tabular}{r} 0.04 \end{tabular} & \begin{tabular}{l} \{オリジナル/データ表現/0点\} \end{tabular} \\ \hline
                2 & \begin{tabular}{r} 0.04 \end{tabular} & \begin{tabular}{r} \{リミックス/抽象化/0点\} \end{tabular} & \begin{tabular}{r} 0.04 \end{tabular} & \begin{tabular}{l} \{オリジナル/抽象化/0点\} \end{tabular} \\ \hline
                3 & \begin{tabular}{r} 0.04 \end{tabular} & \begin{tabular}{r} \{オリジナル/データ表現/0点\} \end{tabular} & \begin{tabular}{r} 0.03 \end{tabular} & \begin{tabular}{l} \{オリジナル/同期/2点\} \end{tabular} \\ \hline
                4 & \begin{tabular}{r} 0.04 \end{tabular} & \begin{tabular}{r} \{オリジナル/ユーザ対話性/0点\} \end{tabular} & \begin{tabular}{r} 0.03 \end{tabular} & \begin{tabular}{l} \{オリジナル/フロー制御/3点\} \end{tabular} \\ \hline
                5 & \begin{tabular}{r} 0.04 \end{tabular} & \begin{tabular}{r} \{リミックス/並列/0点\} \end{tabular} & \begin{tabular}{r} 0.03 \end{tabular} & \begin{tabular}{l} \{リミックス/同期/0点\} \end{tabular} \\ \hline
                6 & \begin{tabular}{r} 0.04 \end{tabular} & \begin{tabular}{r} \{リミックス/データ表現/0点\} \end{tabular} & \begin{tabular}{r} 0.03 \end{tabular} & \begin{tabular}{l} \{リミックス/論理/0点\} \\ \{オリジナル/フロー制御/1点\} \\ \{リミックス/論理/0点\} \end{tabular} \\ \hline
                7 & \begin{tabular}{r} 0.04 \end{tabular} & \begin{tabular}{r} \{リミックス/抽象化/2点\} \end{tabular} & \begin{tabular}{r} 0.03 \end{tabular} & \begin{tabular}{l} \{オリジナル/並列/3点\} \end{tabular} \\ \hline
                8 & \begin{tabular}{r} 0.03 \end{tabular} & \begin{tabular}{r} \{リミックス/同期/0点\} \end{tabular} & \begin{tabular}{r} 0.03 \end{tabular} & \begin{tabular}{l} \{オリジナル/同期/3点\} \end{tabular} \\ \hline
                9 & \begin{tabular}{r} 0.03 \end{tabular} & \begin{tabular}{r} \{リミックス/フロー制御/1点\} \end{tabular} & \begin{tabular}{r} 0.03 \end{tabular} & \begin{tabular}{l} \{オリジナル/並列/2点\} \\ \{オリジナル/データ表現/2点\} \end{tabular} \\ \hline
                10 & \begin{tabular}{r} 0.03 \end{tabular} & \begin{tabular}{r} \{リミックス/論理/3点\} \end{tabular} & \begin{tabular}{r} 0.03 \end{tabular} & \begin{tabular}{l} \{オリジナル/論理/2点\} \end{tabular} \\ \hline
            \end{tabular}
        }
    \end{center}
\end{table}
%--------------------------

\hfill
\vspace{-0.7cm}  % 修正前と修正後の表の間隔を狭くする

%--------------------------
% 修正後の表
%--------------------------
\setcounter{table}{3}
\begin{table}[h]
    \caption{2つのモデルにおける重要度の高い説明変数の上位10件}
    \vspace{-0.5cm}
    \begin{center}
        \scalebox{0.8}[0.8]{
            \begin{tabular}{r|rp{55mm}|rp{55mm}}
                \hline
                & \multicolumn{2}{c|}{モデル1} & \multicolumn{2}{c}{モデル2} \\
                \cline{2-5}
                グループ & \begin{tabular}{r} 重要度 \end{tabular} & \begin{tabular}{l} 説明変数 \end{tabular} & \begin{tabular}{r} 重要度 \end{tabular} & \begin{tabular}{l} 説明変数 \end{tabular} \\ \hline
                1 & \begin{tabular}{r}0.05\end{tabular} & \begin{tabular}{r} \{オリジナル/フロー制御/0点\} \end{tabular} & \begin{tabular}{r} 0.04 \end{tabular} & \begin{tabular}{l} \{オリジナル/データ表現/0点\} \end{tabular} \\ \hline
                2 & \begin{tabular}{r} 0.04 \end{tabular} & \begin{tabular}{r} \{リミックス/抽象化/0点\} \end{tabular} & \begin{tabular}{r} 0.04 \end{tabular} & \begin{tabular}{l} \{オリジナル/抽象化/0点\} \end{tabular} \\ \hline
                3 & \begin{tabular}{r} 0.04 \end{tabular} & \begin{tabular}{r} \{オリジナル/データ表現/0点\} \end{tabular} & \begin{tabular}{r} 0.03 \end{tabular} & \begin{tabular}{l} \{オリジナル/同期/2点\} \end{tabular} \\ \hline
                4 & \begin{tabular}{r} 0.04 \end{tabular} & \begin{tabular}{r} \{オリジナル/ユーザ対話性/0点\} \end{tabular} & \begin{tabular}{r} 0.03 \end{tabular} & \begin{tabular}{l} \{オリジナル/フロー制御/3点\} \end{tabular} \\ \hline
                5 & \begin{tabular}{r} 0.04 \end{tabular} & \begin{tabular}{r} \{リミックス/並列/0点\} \end{tabular} & \begin{tabular}{r} 0.03 \end{tabular} & \begin{tabular}{l} \{リミックス/同期/0点\} \end{tabular} \\ \hline
                6 & \begin{tabular}{r} 0.04 \end{tabular} & \begin{tabular}{r} \{リミックス/データ表現/0点\} \end{tabular} & \begin{tabular}{r} 0.03 \end{tabular} & \begin{tabular}{l} \{\textcolor{red}{\UnderlineTex{オリジナル/論理/0点}}\} \\ \{オリジナル/フロー制御/1点\} \\ \{リミックス/論理/0点\} \end{tabular} \\ \hline
                7 & \begin{tabular}{r} 0.04 \end{tabular} & \begin{tabular}{r} \{リミックス/抽象化/2点\} \end{tabular} & \begin{tabular}{r} 0.03 \end{tabular} & \begin{tabular}{l} \{オリジナル/並列/3点\} \end{tabular} \\ \hline
                8 & \begin{tabular}{r} 0.03 \end{tabular} & \begin{tabular}{r} \{リミックス/同期/0点\} \end{tabular} & \begin{tabular}{r} 0.03 \end{tabular} & \begin{tabular}{l} \{オリジナル/同期/3点\} \end{tabular} \\ \hline
                9 & \begin{tabular}{r} 0.03 \end{tabular} & \begin{tabular}{r} \{リミックス/フロー制御/1点\} \end{tabular} & \begin{tabular}{r} 0.03 \end{tabular} & \begin{tabular}{l} \{オリジナル/並列/2点\} \\ \{オリジナル/データ表現/2点\} \end{tabular} \\ \hline
                10 & \begin{tabular}{r} 0.03 \end{tabular} & \begin{tabular}{r} \{リミックス/論理/3点\} \end{tabular} & \begin{tabular}{r} 0.03 \end{tabular} & \begin{tabular}{l} \{オリジナル/論理/2点\} \end{tabular} \\ \hline
            \end{tabular}
        }
    \end{center}
\end{table}
%--------------------------


%--------------------------------------------------------------------------------
\newpage
\nextans
\subsection{(コメントM-8)}

参考文献[1]: 2018 -\textgreater  2008 ?

\subsection{(回答M-8)}

貴重な御指摘を頂き,誠にありがとうございます.御指摘通り,参考文献[1]の出版年に誤りがありました.謹んでお詫び申し上げます.また,同著者の別の文献を引用していたことを確認したため,変更M-8に示すように参考文献を変更しました.

\subsection{(変更M-8)参考文献[1]}
\vspace{-0.3cm}
\begin{description}
\item 修正前\\
\phantom{ }
[1] Wing, J. M.: Computational thinking and thinking about computing, Journal of Philosophical Transactions of the Royal Society, Vol.366, pp.3717-3725 (2018)
\vspace{-0.3cm}
\item 修正後\\
\phantom{ }
\textcolor{red}{\UnderlineTex{[1] Wing, J. M.: Computational thinking. Communications of the ACM, Vol.49, No.3, pp.33-35 (2006).}}
\end{description}


%--------------------------------------------------------------------------------
\nextans
\subsection{(コメントM-9)}

参考文献[2]: volume(巻)の記載が抜けているように見えます.

\subsection{(回答M-9)}

貴重な御指摘を頂き,誠にありがとうございます.御指摘の通り,参考文献[2]のvolume(巻)についての記載が欠落していました.謹んでお詫び申し上げます.変更M-9に示すように加筆しました.

\subsection{(変更M-9)参考文献[2]}
\vspace{-0.3cm}
\begin{description}
\item 修正前\\
\phantom{ }
文献[2] Moreno-Le{\'o}n, J., Robles, G. and Rom{\'a}n-Gonz{\'a}lez, M.: Dr. Scratch: Automatic analysis of scratch projects to assess and foster computational thinking, RED. Revista de Educaci{\'o}n a Distancia, No.46, pp.1-23 (2015) 
\vspace{-0.3cm}
\item 修正後\\
\phantom{ }
文献[2] Moreno-Le{\'o}n, J., Robles, G. and Rom{\'a}n-Gonz{\'a}lez, M.: Dr. Scratch: Automatic analysis of scratch projects to assess and foster computational thinking, RED. Revista de Educaci{\'o}n a Distancia, \textcolor{red}{\UnderlineTex{Vol.15}}, No.46, pp.1-23 (2015) 
\end{description}

% 論文サイト
% https://revistas.um.es/red/issue/view/14141
% https://scholar.google.co.jp/citations?view_op=view_citation&hl=ja&user=ADmHd7sAAAAJ&citation_for_view=ADmHd7sAAAAJ:IWHjjKOFINEC


%--------------------------------------------------------------------------------
\nextans
\subsection{(コメントM-10)}

参考文献: プロシーディング名について,下記URLの執筆要領をご確認の上,参考文献の書き方を統一されることをご検討ください.
\url{https://www.ipsj.or.jp/journal/submit/ronbun_j_prms.html}

\subsection{(回答M-10)}

貴重な御助言を頂き,誠にありがとうございます.御提示頂いたURLの執筆要領を参照のもと,参考文献の書き方を統一させて頂きました.



\newpage
%--------------------------------------------------------------------------------
%--------------------------------------------------------------------------------
\section{査読委員の方(査読者: 1)から頂いたコメントに対する回答}
%--------------------------------------------------------------------------------
\subsection{(条件1-1)}

背景で「Scratch において特定の習熟度に到達する作品を制作するまでに,ユーザが過去に制作した作品で使用した概念を明らかにする」と述べられていますが,例えば論理の3を取得したユーザでも,過去に論理の1点2点の概念を取得していない恐れがあるということでしょうか?

配列の後に多次元配列やリスト・マップを学ぶように,基本的に学習では簡単な要素を先に覚え,次にその応用を学ぶと思います.概念にもよりますが,仮に先に応用を覚えた場合も,それを構成する基礎要素も付随して学習することかと考えます.

Scratchにおいてユーザは対話性を最初に学び,次に論理を学ぶ傾向がある,などでしたら習熟過程として分かりますが,0点から3点まで分けて調査を行う意義が,現状の論文から読み取れませんでした.

CTスキルの習熟過程自体を分析することの意義や貢献は伝わっていますが,0〜3点で分けて調査を行う意義を,Motivating Exampleや著者らが想定するシナリオなどで明記してください.

\subsection{(回答1-1)}

貴重な御指摘を頂き,誠にありがとうございます.御指摘の通り,配列の事例であれば,応用としてリストやマップを覚えた場合に,それを構成する配列や多次元配列を基本として理解していることが多いと私共も考えます.本研究で対象とするDr.Scratchが算出するCT概念であれば,論理2点のIf elseブロックを使用していれば,論理1点のIfブロックを使用する知識を有していると考えます.一方で,CT概念「論理」の論理演算ブロックはIfブロックとは異なる論理演算(例えばNotなど)を使用している場合も3点と算出するため,同一概念の高得点を獲得しているからといって,低得点の知識を有しているとは限らず,本研究では各概念の点数を区別して調査することにしました.このように同一概念の異なる点数を獲得するために必要な知識には,包含関係を持たないこともありますが,CT概念としての難易度が高いほど点数が高く設定されています.本研究で,点数を0点から3点に区別して調査を行う意義を明確にするためにも,変更1-1-aのように点数区分を行う動機について加筆しました.

また,本研究では,同一概念の各点数の獲得有無を用いることで,特定の習熟度(DevelopingとMaster)の点数に到達するまでに習熟する知識を明らかにすることを目的としており,異なるCT概念を習熟する順序関係,同一CT概念における異なる点数の知識を習熟する順序関係は着目しておりません.1章の事前分析の説明に「習熟過程を調査」と記載したことが誤解を招く表現でした.謹んでお詫び申し上げ,変更1-1-bのように「習熟過程」の表現を使用しない文章に修正致しました.習熟過程を分析するためには,概念別の習熟の容易性や,習熟順序について,さらなる分析が必要があるため今後の課題とさせていただきます.

\newpage
\subsection{(変更1-1-a)P.3 2.2節 }
\vspace{-0.3cm}
\begin{description}
\item 修正前\\
\phantom{ }
図~1中の(2)には,作品(a)を制作するために使用しているCT概念を,抽象化で1点,並列で3点,論理で3点,同期で3点,フロー制御で3点,ユーザ対話性で2点,データ表現で2点と算出し,図~1中の(1)にCTスコア17点として習熟度Masterと提示している.
\vspace{-0.3cm}
\item 修正後\\
\phantom{ }
図~1中の(2)には,作品(a)を制作するために使用しているCT概念を,抽象化で1点,並列で3点,論理で3点,同期で3点,フロー制御で3点,ユーザ対話性で2点,データ表現で2点と算出し,図~1中の(1)にCTスコア17点として習熟度Masterと提示している.\textcolor{red}{\UnderlineTex{個々の作品では,各CT概念で0点から3点のいずれかの点数を獲得するが,Dr.Scratchが定義するCT概念の一部は,同一概念にもかかわらず使用したブロックの種類によって異なる点数を決定している.例えば,CT概念の「論理」では,If elseブロックやIfブロックを使用することで2点や1点を獲得できるが,Ifブロック以外(NotブロックやAndブロック等)の論理演算命令でも3点を獲得できる.したがって,本研究ではユーザのCTを明確に捉えるため,過去に制作した作品で使用する各CT概念の0点から3点までを区別して調査する.}}
\end{description}    

\subsection{(変更1-1-b)P.2 1章 }
\vspace{-0.3cm}
\begin{description}
\item 修正前\\
\phantom{ }
事前分析として,Scratchにおいて20件以上の作品を制作したユーザ6,323人の習熟過程を調査し,Dr.Scratchが算出するCTスキルの点数は,多数の作品制作を実施することで高くなるのか否かを明らかにする.
\vspace{-0.3cm}
\item 修正後\\
\phantom{ }
事前分析として,\textcolor{red}{\UnderlineTex{Scratchにおいてユーザ6,323人が制作した作品126,460件({$6,323人\times20件$})の特徴分析を行い,その後,}}Dr.Scratchが算出するCTスコアは多数の作品制作を実施することで高くなるのか否かを明らかにする.
\end{description}      
      

%--------------------------------------------------------------------------------
\newpage
\nextans
\subsection{(条件1-2)}

条件1-1とも関連しますが,3.2節では1回目から20回目に投稿したプログラムのDr.Scratchの点数をまとめ,フロー制御の点数の獲得は容易である一方で,論理の点数の獲得が困難であると結論付けています.

このデータは大規模かつ公平に収集されており大変貴重なデータだと考えますが,習熟過程を調査するのなら,1回目から20回目それぞれで使用された各概念をまとめるべきではないでしょうか.

フロー制御の2点は確かに多くの作品で取得されていますが,もしそれらの取得が他の概念よりもあとであった場合,点数の取得は容易であると言い切れないと考えます.すなわち,多くの作品が取得している概念が取得が容易なのではなく,早期に取得する概念が取得が容易であるとも考えられます.
図で表現することは困難かと思いますが,著者らの主張の正確性にもつながりますため,フロー制御の点数の何割が5回目の投稿以内に獲得されたなど,投稿序盤で獲得されやすい概念について追記をお願いします.

\subsection{(回答1-2)}

貴重な御指摘を頂き,誠にありがとうございます.御指摘のように,3.2節の目的が習熟過程の調査であれば,1回目から20回目それぞれで使用された各CT概念について記載することが適当と考えます.しかし,3.2節は本研究で収集した作品の特徴分析,および,公開データセット[回答書引用1]との比較による収集データの信頼性を示すことを調査目的としており,習熟過程の調査が目的ではありません.1章の事前分析の説明に「習熟過程を調査」と記載したことが誤解を招く表現であったと考えます.謹んでお詫び申し上げます.変更1-1のように「習熟過程」の表現を使用しない文章に修正致しました.また,御指摘の通り,3.2節の調査結果のみから点数獲得の難易度について主張することは適当ではないと考えたため,変更1-2に示すように,点数頻度についてのみ説明する文章に変更しました.\\

\noindent[回答書引用1] Aivaloglou, E., Hermans, F., Moreno-Le\'{o}n, J., Robles, G.: A Dataset of Scratch Programs: Scraped, Shaped and Scored, Proceedings of the 14th International Conference on Mining Software Repositories (MSR'17), pp.511-514 (2017).

\subsection{(変更1-2)P.4 3.2節 }
\vspace{-0.3cm}
\begin{description}
\item 修正前\\
\phantom{ }
図2から,ユーザは作品制作において,フロー制御の点数を獲得することは容易である一方で,論理の点数を獲得することは困難であることが考えられる.この結果は,Aivaloglouらが公開する作品と同じ結果を示した~[8].
\vspace{-0.3cm}
\item 修正後\\
\phantom{ }
図2から,\textcolor{red}{\UnderlineTex{フロー制御では高い点数の作品が多い一方で,論理では高い点数の作品が少ないことが分かる.}}この結果は,Aivaloglouらが公開する作品と同じ結果を示した~[8].
\end{description}


%--------------------------------------------------------------------------------
\newpage
\nextans
\subsection{(条件1-3)}

表4の解釈について論文中に明記した方が本研究の有用性や可読性が増すと考えます.

重要度がどのようなものであるかはわかりましたが,0点の概念の重要度が高いということは,その概念は自然と習熟できるということでしょうか,それとも反対に自然な習得が困難であるためScratch学習者は意識すべきということでしょうか.

6.2節で習熟度達成予測を行うことの貢献や使用方法について述べてありますが,5章や6章の結果をほとんど使用していないため,どのように表4の結果を利用したらよいかが不明確です.

本研究の貢献を明確にするためにも,6.2節,あるいははじめになどで,表4(本研究の結果)を利用した支援を具体的に記述してください.

\subsection{(回答1-3)}

貴重な御指摘を頂き,誠にありがとうございます.6.1節においても述べた通り,Developing以上に到達するユーザと比較して,当該習熟度に到達しないユーザの多くが,重要度の高い0点のCT概念を獲得する傾向にあることが分かりました.このことから,Developing以上に到達しないユーザにとって,当該CT概念を自然に習熟することは困難であり,ユーザが意識して学ぶ必要のあるCT概念ではないかと考えます.

また,御指摘通り,本研究の結果を利用せずに6.2節を記述していたため,変更1-3に示すように,結果を利用した具体的な支援方法について加筆しました.


\subsection{(変更1-3)P.9 6.2節}
\vspace{-0.3cm}
\begin{description}
\item 修正前\\
\phantom{ }
ただし,本研究において,特定の習熟度への到達可否は,ユーザが過去に使用した概念の影響を受けることが明らかになったため,単に習熟度が1段階上の作品を提示するだけでは,ユーザのCTスキルに合わせた作品の提示は困難であると考える.今後は,ユーザが過去に使用したCTスキル概念に基づく,ユーザのCTスキルに合わせた作品提示を検討する.
\vspace{-0.3cm}
\item 修正後\\
\phantom{ }
ただし,本研究において,特定の習熟度への到達可否は,ユーザが過去に使用した概念の影響を受けることが明らかになったため,単に習熟度が1段階上の作品を提示するだけでは,ユーザのCTに合わせた作品の提示は困難であると考える.\textcolor{red}{\UnderlineTex{本研究の結果から,Developing以上に到達していないユーザの多くは,CT概念の「抽象化」や「データ表現」などにおいて0点を獲得している.このことから,0点を獲得し続けるユーザに対しては,当該CT概念での点数を満たす作品を提示することで,Developing以上の作品を制作するための支援ができると考える.一方で,Masterに到達するユーザの多くは,CT概念の「同期」や「フロー制御」で2点以上を獲得しているため,これらの概念を使用するユーザはさらに高い習熟度を必要とする作品を制作できることが考えられる.したがって,当該CT概念で高い点数を獲得するユーザに対しては,高い習熟度の作品を推薦することで,ユーザの更なるCTの向上に繋がると考える.}}今後は,ユーザ過去に使用したCT概念に基づく,ユーザのCTに合わせた作品提示方法を検討する.

\end{description}


%--------------------------------------------------------------------------------
\newpage
\nextans
\subsection{(条件1-4)}

全体を通してオリジナルとリミックスを分けて分析されていますが,結論としてリミックスをどのように扱う,あるいはとらえたら良いのかが分かりませんでした.

どのようなレベルでどのような点数の作品をリミックスするべきか,あるいはリミックスは調査対象として外すことが出来なかったが,本研究はオリジナル作品にのみ着目するなど,リミックスのデータの扱いについて追記してください.

\subsection{(回答1-4)}

貴重な御指摘を頂き,誠にありがとうございます.御指摘の通り,本研究ではオリジナル作品とリミックス作品を区別して分析したものの,リミックス作品の捉え方についての説明が不足していました.リミックス作品は,ユーザが一から制作するオリジナル作品と異なり,他の作品を再利用して制作した作品であるため,リミックス元となる作品が使用するCT概念を,ユーザが習熟したように取り扱うことは適切でありません.したがって,本研究では予測モデル構築時に計測した目的変数は,\UnderlineTex{オリジナル作品で目標習熟度に到達したユーザのみを正例クラス}として扱いました.

一方で,従来研究~[回答書引用2]において,リミックス作品の制作による学習効果が示されているため,ユーザはリミックスを通してCT概念を習熟したことで,オリジナル作品で特定の習熟度に到達したことも考えられます.そのため,本研究では説明変数の計測にオリジナル作品だけでなくリミックス作品も対象としました.

このように,本研究ではリミックスして制作した作品に関する特徴量を,説明変数に使用していましたが,リミックスが予測精度向上を妨げている可能性もあり,御提案頂いたように調査対象から外すことも可能です.そこで,本研究では説明変数にオリジナル作品のみを使用するモデルも構築し,精度の比較を行いました.回答書表1は,説明変数からリミックス作品を除去したモデルの分類精度を示します.回答書表1から,リミックス作品を除去した場合,除去前と比較して,モデル1ではF値が0.09,モデル2ではF値が0.10低下することから,リミックス作品を調査対象に含めることが適当だと考えます.本結果は,本研究における分析手法の妥当性を示す上で重要と考え,4.1節のモデル構築の説明にリミックス作品を取り除く場合の調査についても考察で議論することを述べた上で,新たに6.3節として本結果を加筆します.変更点はそれぞれ,変更1-4-a,変更1-4-bに示しています.


%--------------------------------------
\vspace{0.5cm}
\begin{table}[h]
        \begin{center}
        \captionsetup{labelformat=empty,labelsep=none}
        \caption{回答書表1: 説明変数からリミックス作品を除去した場合の分類精度(平均値)}
        \vspace{-0.4cm}
        \caption{(*リミックス作品を除去しなかった場合の分類精度)}
	 \vspace{-0.3cm}
            \begin{tabular}{l|p{30mm}|p{30mm}|p{30mm}}
                \hline
                & \multicolumn{1}{c|}{適合率} & \multicolumn{1}{c|}{再現率} & \multicolumn{1}{c}{F値} \\ \hline
                モデル1 & 0.88 (*0.87) & 0.68 (*0.87) & 0.76 (*0.87) \\
                モデル2 & 0.55 (*0.73) & 0.44 (*0.49) & 0.48 (*0.58) \\ 
                \hline
            \end{tabular}
        \end{center}
        \label{tab:predict_result}
\end{table}
%--------------------------------------

\noindent[回答書引用2] Dasgupta, S., Hale, W., Monroy-Hern\'{a}ndez, A. and Hill, B.~M.: Remixing As a Pathway to Computational Thinking, Proceedings of the 19th Conference on Computer-Supported Cooperative Work \& Social Computing (CSCW'16), pp. 1438-1449 (2016).

\newpage
\subsection{(変更1-4-a)P.6 4.1節}
\vspace{-0.3cm}
\begin{description}
\item 修正前\\
\phantom{ }
獲得点数の計測は,リミックス作品の制作による学習効果が示されているため~[6],オリジナル作品の制作において獲得した点数と,リミックス作品の制作において獲得した点数と区別する.したがって,56次元(7(7つのCT概念){$\times$} 4(0点から3点){$\times$} 2(オリジナル作品またはリミックス作品))の説明変数を計測する.
\vspace{-0.3cm}
\item 修正後\\
\phantom{ }
獲得点数の計測は,リミックス作品の制作による学習効果が示されているため~[6],オリジナル作品の制作において獲得した点数と,リミックス作品の制作において獲得した点数と区別する.したがって,56次元(7(7つのCT概念){$\times$} 4(0点から3点){$\times$} 2(オリジナル作品またはリミックス作品))の説明変数を計測する.\textcolor{red}{\UnderlineTex{尚,説明変数の計測にリミックス作品を加えることの妥当性については6.3節で考察する.}}
\end{description}


\subsection{(変更1-4-b)P.9 6.3節 リミックス作品がモデルの精度に及ぼす影響}
\vspace{-0.3cm}
\begin{description}
\item 加筆内容\\
\phantom{ }
\textcolor{red}{\UnderlineTex{本研究で構築した習熟度到達予測モデルにおける説明変数には,リミックス作品の制作による学習効果が示されていることから~[6],オリジナル作品で獲得した点数と,リミックス作品で獲得した点数と区別して計測した.しかし特定の習熟度に到達するか否かを判定するためにリミックス作品の制作が有用であるか否かは明らかでないため,リミックス作品を分析対象外とした場合の予測モデルも構築し,精度結果を比較する.\\}}~~~\textcolor{red}{\UnderlineTex{表5は説明変数からリミックス作品を除去した場合の分類精度を示す.表5から,リミックス作品を除去した場合,除去前と比較して,モデル1ではF値が0.09,モデル2ではF値が0.10低下することが分かった.したがって,ユーザがDeveloping以上の習熟度に到達するか否かを予測するために,リミックス作品を説明変数の特徴量として使用することは予測精度向上に有効であることが示唆される.今後は,特定の習熟度への到達に寄与するリミックス作品の特徴,および,ユーザがリミックスによって習熟するCT概念を明らかにする.}}
\end{description}


%--------------------------------------
\vspace{0.5cm}
\begin{table}[h]
\color{red}
        \begin{center}
        \captionsetup{labelformat=empty,labelsep=none}
        \caption{\textcolor{red}{表5: 説明変数からリミックス作品を除去した場合の分類精度(平均値)}}
        \vspace{-0.4cm}
        \caption{\textcolor{red}{(*リミックス作品を除去しなかった場合の分類精度)}}
	 \vspace{-0.3cm}
            \begin{tabular}{l|p{30mm}|p{30mm}|p{30mm}}
                \hline
                & \multicolumn{1}{c|}{適合率} & \multicolumn{1}{c|}{再現率} & \multicolumn{1}{c}{F値} \\ \hline
                モデル1 & 0.88 (*0.87) & 0.68 (*0.87) & 0.76 (*0.87) \\
                モデル2 & 0.55 (*0.73) & 0.44 (*0.49) & 0.48 (*0.58) \\ 
                \hline
            \end{tabular}
        \end{center}
        \label{tab:predict_result}
\color{black}
\end{table}
%--------------------------------------



%--------------------------------------------------------------------------------
\newpage
\nextans
\subsection{(コメント1-1)}

著者らの主張とは異なりますが,6.1節においてDevelopingやMasterに到達した作品について,1回でもその点数に到達したらDevelopingやMasterと扱うことの妥当性に疑問が残ります.特定のタイミングにおいて8点以上を取得したとしても,その際に使用した概念が継続的に使用されない場合,習熟したとはいいがたいと考えます.また,作成するプログラムの種類によっては,そもそも特定の概念を必要としない場合もあります.

習熟したか否かについての評価は長期的なものとなり困難ですが,ユーザ(b)のように1回だけDeveloperの作品を投稿しそれ以降Basicの作品を投稿する場合でも,Developerと判断することの妥当性について,既存研究で触れられていたらその旨を,触れられていない場合考察などで明記されると,本研究の貢献につながると考えます.

Scratchに関係なく,教育系の論文や本における習熟度の評価を参考にしてもよいかと思いました.

\subsection{(コメントへの回答1-1)}
貴重な御指摘,および,御助言を頂き,誠にありがとうございます.本研究では,1回でもDevelopingやMasterの作品を制作していれば,その習熟度に到達したと判定していますが,御指摘の通り,その妥当性については今後検討していく必要があると考えます.

Scratchにおけるユーザの学習傾向を調査している従来研究では,ユーザが50回の作品制作を行っていく中で,使用するブロックの種類数の遷移を調査しています~[回答書引用3].具体的には,特定の種類のブロックを1回でも使用していれば,そのブロックについて学習したと捉えています.本研究においても,この従来研究と同様に1回でも特定の習熟度を満たす作品を制作していれば,到達したと判断しています.従来研究では,特定の習熟度に到達したと判断する基準について深い議論はありませんでしたが,本論文3章の事前分析(図5)の結果でも述べている通り,ユーザは過去に制作した作品と同一点数前後の作品を制作する傾向があるため,偶然に高い点数の作品を制作することは少ないと考えます.御指摘の通り,ユーザ(b)のような特定の習熟度に到達して以降,若干低い習熟度の作品を制作し続けるユーザも存在しますが,本論文の事前分析の結果より,本定義が予測モデルへ与える影響は小さいと考えます.今後は,概念の定着に関する分析を進め,ユーザが特定の習熟度に到達したと判断する基準について検討致します.本研究における習熟の判断基準を明確にするためにも,変更C1-1のようにモデル構築時の説明に従来研究~[回答書引用3]の内容を踏まえた説明を加筆しました.\\
%\todo{このTODOは内的妥当性に追記しても良いかも.}\\

\noindent[回答書引用3] Yang, S., Domeniconi, C., Revelle, M., Sweeney, M., Gelman, B. U., Beckley, C. and Johri, A.: Uncovering Trajectories of Informal Learning in Large Online Communities of Creators, Proceedings of the 2nd Conference on Learning @ Scale (L@S'15), pp.131-140 (2015).

\newpage
\subsection{(変更C1-1)P.6 4.1節}
\vspace{-0.3cm}
\begin{description}
\item 修正前\\
\phantom{ }
目的変数は,{$N~(2 \leq N \leq 20)$}番目に目標習熟度に到達するオリジナル作品を制作したユーザを正例クラス,それ以外のユーザを負例クラスとする.提案するモデルでは,ユーザが自身で制作したオリジナル作品が目標習熟度に到達することを予測するモデルを構築し,たとえリミックス作品で目標習熟度以上の点数を獲得していても目標習熟度に到達したと判定しない.
%Scratchにおけるユーザのプログラミング能力の成長に関する研究として,Yangらは,Scratchで少なくとも50回以上の作品制作を行ったユーザ3,852人は,繰り返し作品を制作することで,作品に使用するブロックの種類数が増加することを明らかにした~[4].
\vspace{-0.3cm}
\item 修正後\\
\phantom{ }
目的変数は,{$N~(2 \leq N \leq 20)$}番目に目標習熟度に到達するオリジナル作品を制作したユーザを正例クラス,それ以外のユーザを負例クラスとする.\textcolor{red}{\UnderlineTex{従来研究~[4]では,ユーザが特定の種類のブロックを1回でも使用すれば,そのブロックを学習したと捉えていることから,本論文でも同様に,ユーザが目標習熟度を満たす作品を1回でも制作すれば到達したと判断する.}}提案するモデルでは,ユーザが自身で制作したオリジナル作品が目標習熟度に到達することを予測するモデルを構築し,たとえリミックス作品で目標習熟度以上の点数を獲得していても目標習熟度に到達したと判定しない.
%Scratchにおけるユーザのプログラミング能力の成長に関する研究として,Yangらは,Scratchで少なくとも50回以上の作品制作を行ったユーザ3,852人\textcolor{red}{\UnderlineTex{の学習傾向を調査している.特定の種類のブロックを1回でも使用すれば,そのブロックについて学習したと捉えている.その結果,ユーザは繰り返し作品を制作することで,使用可能なブロックの種類数が増加することを明らかにした~[4].}}\todo{「その結果」のあとは必要かな?それよりも回答にも記載するように「本論文でも1回でも特定の習熟度を満たす作品を制作していれば到達したと扱う」みたいな文を論文に追記した方が良いのでは?}
\end{description}


\newpage
%--------------------------------------------------------------------------------
%--------------------------------------------------------------------------------
\section{査読委員の方(査読者: 2)から頂いたコメントに対する回答}
%--------------------------------------------------------------------------------
\subsection{(条件2-1)}

論文中で「コンピュテーショナル・シンキングスキル」や「CTスキル」という語が使われていますが,文献[1]では``computational thinking''とよばれており,``computational thinking skill''という使われ方はしていません.また,文献[1]の冒頭でCTについて述べた箇所で参照されている同著者による文献[a](本条件の末尾をご参照ください)では``Computational thinking is a fundamental skill for everyone, ...''と述べており,CTをスキルであると位置づけています.つまり,CTという語自体がスキルという意味合いを含んでおり,「CTスキル」という用法は適切ではありません.もし,本論文では「CTスキル」を別の語として定義しているのであれば,文献を参照するなどして適切に定義してください.そうでなければ,単に「CT」と記載してください.

また,1章で「CTスキル:問題を抽象化し分析することで,効率的な問題解決を可能にする考え方)」という形でCTスキルを定義しています.文献[a]によると,CTにはそのような側面(抽象化と分割統治や,近似解による効率的な問題解決など)が含まれる一方,より幅広い(冗長性や競合の回避,トレードオフなど)解釈を与えています.そのため,現状の記載はCTを限定的に解釈しているように思います.全てを列挙する必要はありませんが,CTには抽象的な分析と効率的な問題解決のための考え方が含まれている,というような,より幅広いものであることが分かるように記載してください.あるいは,上述したようにCTスキルについて,別の定義を与えてください.

\hspace{3em}

\noindent 文献[a] Wing, J. M.: Computational thinking, Communications of the ACM, Vol.49, No.3, pp.33-35, Mar. 2006.


\subsection{(回答2-1)}

貴重な御指摘を頂き,誠にありがとうございます.私共でも従来研究を確認したところ,御指摘の通り,CTという語自体にスキルという意味が含まれているにも関わらず,本論文では「CTスキル」と記載していました.定義に関して誤解を招く使い方をしていましたこと,謹んでお詫び申し上げます.本論文における「CTスキル」は,文献[a]で定義されている「CT」と同一の意味として使用しているため,\UnderlineTex{「CTスキル」から「CT」への記載に変更}しました.

加えて,御指摘の通り,1章におけるCTの説明が限定的に解釈したような記述となっているため,変更2-1-aに示すように,CTに関する説明文を広汎的な意味を含む説明に修正しました.

また,御指摘を受けて,CTの説明に関して引用している文献[1]を誤って同著者の別文献を引用していることが判明しました.謹んでお詫び申し上げます.正しくは,御指摘頂いた内容にも含まれている同著者の文献[a]であるため,変更2-1-bに示すように,論文内での引用文献を変更しました.


\subsection{(変更2-1-a)P.1 1章}
\vspace{-0.3cm}
\begin{description}
\item 修正前\\
\phantom{ }
(CTスキル:問題を抽象化し分析することで,効率的な問題解決を可能にする考え方)
\vspace{-0.3cm}
\item 修正後\\
\phantom{ }
\textcolor{red}{\UnderlineTex{(CT:CTには抽象的な分析と効率的な問題解決のための考え方が含まれている)}}
\end{description}

\newpage
\subsection{(変更2-1-b)参考文献[1]}
\vspace{-0.3cm}
\begin{description}
\item 修正前\\
\phantom{ }
\noindent[1] Wing, J. M.: Computational thinking and thinking about computing, Journal of Philosophical Transactions of the Royal Society, Vol.366, pp.3717-3725 (2018)
\vspace{-0.3cm}
\item 修正後\\
\phantom{ }
\noindent\textcolor{red}{\UnderlineTex{[1] Wing, J. M.: Computational thinking, Communications of the ACM, Vol.49, No.3, pp.33-35 (2006).}}
\end{description}


%--------------------------------------------------------------------------------
\nextans
\subsection{(条件2-2)}

条件(2-1)と関連しますが,論文中で「CTスキル7概念」とよばれている概念は,文献[2]における``CT concept"のことかと思います.よって,例えば「CTにおける7概念」や「7つのCT概念」のようにスキルという語を除いた名称に変更する必要があります.

\subsection{(回答2-2)}

貴重な御指摘を頂き,誠にありがとうございます.条件2-1の御指摘を受けて,論文中の「CTスキル」は全て「CT」に変更したため,「CTスキル7概念」および「CTスキルの概念」などの表記は,「スキル」という語を除いた名称に変更しました.本変更は,論文中で多用する語のため,論文中には変更箇所を赤字で記載し,アルファベットは付しません.どうぞ宜しくお願い致します.


%--------------------------------------------------------------------------------
\newpage
\nextans
\subsection{(条件2-3)}

本論文で構築された習熟度到達モデルが,適切に習熟度を予測しているか不明確です.Scratchのサービス上で共有されている作品の作者が,どのような習熟度であるか不明であり,利用者が試行錯誤を繰り返して習熟している,というような前提が妥当であるか判断できません.例えば,十分に習熟している利用者が,CT scoreが低くなる単純なプログラムを作成するような状況も起こり得ます.

このモデルでは,Scratchサービスで作品を制作している利用者の行動を予測していますが,それが習熟度と合致するかが不明です.例えばScratchサービスの利用者のプロファイルを(個別に完全には不可能ですが)示す,あるいは習熟度を推測可能な利用者の作品を用いて検証するなどの方法で,モデルの妥当性を明確にしてください.

\subsection{(回答2-3)}

貴重な御指摘を頂き,誠にありがとうございます.査読者1の方から頂いた(コメント1-1)と類する御意見と存じます.Scratchは決められた順序で学習を進めるサービスではなく,ユーザが作品制作において試行錯誤を繰り返し,コンピュテーショナル・シンキングを習熟する環境を提供しているサービスです.しかし,御指摘の通りユーザが実際に試行錯誤を繰り返してCT概念を習熟しているか否かを確認することはできず,ユーザの正確な習熟度についても明らかにすることはできません.そのため,十分に習熟しているユーザであっても,CTスコアの低い作品を制作し続ける場合,習熟度到達予測モデルでは適切に予測できないことも考えられます.

モデルが予測する習熟度が妥当であるか否かを明確にする方法として,御指摘頂いたようにユーザのプロファイルから習熟度を確認する方法が挙げられます.具体的には,Scratchサービス上におけるユーザのアカウントページ(\url{https://scratch.mit.edu/users/(ユーザ名)/})からユーザの紹介文を確認することができます.しかし,Scratchにはアカウントページの紹介文を記載するためのガイドラインは存在しないため,ユーザによって紹介文の内容は様々であり,紹介文からユーザの習熟度を推測することは困難であると考えます.実際,私共でユーザのアカウントページに記載される紹介文を目視で確認しましたが,ユーザの習熟度を推定可能な情報を得ることはできませんでした.

ユーザのプロファイルからモデルが予測する習熟度の妥当性を確認することは難しいため,本研究では別の方法として,習熟度を推定可能なユーザの作品を用いてモデルの妥当性を明確にします.具体的には,Developingの作品を制作したユーザを一定のCT概念を有するユーザとして捉え,そのユーザがMasterに到達する作品を制作するか否かを予測する追加実験を行いました.Developingに到達したユーザのうち,Masterに到達したユーザを正例クラス,到達しなかったユーザを負例クラスとするモデルを構築し,分類精度を算出しました.その結果,適合率は0.77,再現率は0.55,F値は0.64であり,全ての評価指標が本論文で構築したモデル2(Developingに到達していないユーザも含めてMasterに到達するか否かを予測したモデル,論文中の表3を参照)よりも高い精度であることを確認しました.言い換えると,一度でもDevelopingに到達するユーザは到達していないユーザに比べて,Masterに到達するユーザを予測するために有効な特徴であると考えます.

Developingに到達するユーザが必ずしも一定のCT概念を有するユーザとは限りませんが,Developingに到達していないユーザに比べると多様な命令処理を使用しているため,Masterに到達するユーザをより高い精度で予測できたと考えます.提案モデルが予測する習熟度の妥当性を明確にするためには,習熟度が明確なユーザを対象に被験者実験を行うなど,更なる実証実験が必要であり,今後の課題とさせて頂きます.


%--------------------------------------------------------------------------------
\newpage
\nextans
\subsection{(コメント2-1)}

以下に示す不明確な表現について,意図を適切に示すよう修正することをお勧めします.

\vspace{0.2cm}
\noindent・p.4 左段 下から3行目「初めてCTスキルの合計点数を獲得するまでに」の意味を読み取ることができませんでした.全ての任意の作品について,CTスキルの合計点数(0\verb|〜|21点)が与えられるはずですので,「初めて」という表現がどういう状況を示しているかが不明確です.意図するところを適切に示す表現に修正するか,この部分を削除することをご検討ください.

\vspace{0.2cm}
\noindent・p.4 右段 下から8行目と図5の縦軸ラベル 「1番目からN-1番目に制作した作品のCTスキルの合計点数」について,N-1個の作品のCTスキル得点を合計したもののようにも読み取れます.例えば,「CTスキルの合計点数」を文献[2]の``CT score''に準じて「CTスコア」や「CT得点」のように表記すると,このような誤解が生じにくくなるかと思います.

\subsection{(コメントへの回答2-1)}

貴重な御指摘を頂き,および,御助言を頂き,誠にありがとうございます.御指摘の通り,曖昧な表現がありました.謹んでお詫び申し上げます.御指摘頂いた2点について,順に回答させて頂きます.

1点目に関して,御指摘頂いたP.4で説明する図5は,ユーザがN番目に制作した合計{$S~(0 \leq S \leq 21)$}点の作品を制作するまでに,過去に制作した作品(1番目からN-1番目に制作した作品)のCTスキルの合計点数別の作品数の分布をバイオリン図で示しています.ユーザが20番目の作品までに,同じCTスキルの合計点数の作品を複数回制作した場合は,初めて当該点数を獲得するまでに制作した作品の特徴の理解を目的とするため,初回のみの作品をバイオリン図の分布にカウントし,2回目以降はカウントしていません.論文中の説明が不十分であったため,P.4 左段 下から3行目の「初めてCTスキルの合計点数を獲得する」は変更C2-1-aに示すように修正しました.また,御指摘を受けて,P.5 図5 タイトルに関しても同様に曖昧な表現となっていたため,「オリジナル作品でCTスキルの合計点数を獲得する」は変更C2-1-bに示すように修正しました.

2点目に関して,御指摘の通り,「1番目からN-1番目に制作したCTスキルの合計点数」は誤解を招く表現となっていました.御指摘頂いたように,文献[2]の``CT score''に準じることで,誤解を防ぐだけでなく,論文の可読性向上にも繋がると考えたため,論文中での合計点数に関する表記を\UnderlineTex{「CTスコア」に変更および追記}しました.本変更は論文中で多用する語のため,論文中には変更箇所を赤字で記載し,アルファベットは付しません.どうぞ宜しくお願い致します.


\subsection{(変更C2-1-a)P.4 3.3章}
\vspace{-0.3cm}
\begin{description}
\item 修正前\\
\phantom{ }
図5は,ユーザがN番目に制作したオリジナル作品において,初めてCTスキルの合計点数を獲得するまでに,過去に制作した作品(1番目からN-1番目に制作した作品)のCTスキルの合計点数別作品数の分布をバイオリン図で示す.
\vspace{-0.3cm}
\item 修正後\\
\phantom{ }
図5は,\textcolor{red}{\UnderlineTex{ユーザがN番目に制作したCTスコア{$S~(0 \leq S \leq 21)$}点のオリジナル作品を制作するまでに,過去に制作した作品(1番目からN-1番目に制作した作品)のCTスコア別作品数}}の分布をバイオリン図で示す.
\end{description}

\newpage
\subsection{(変更C2-1-b)P.5 図5 タイトル}
\vspace{-0.3cm}
\begin{description}
\item 修正前\\
\phantom{ }
オリジナル作品(N番目に制作した作品)でCTスキルの合計点数を獲得するまでに,過去に制作した作品(1番目からN-1番目に制作した作品)のCTスキルの合計点数別作品数の分布
\vspace{-0.3cm}
\item 修正後\\
\phantom{ }
\textcolor{red}{\UnderlineTex{ユーザがN番目に制作したCTスコア{$S~(0 \leq S \leq 21)$}点のオリジナル作品を制作するまでに,過去に制作した作品(1番目からN-1番目に制作した作品)のCTスコア別作品数の分布}}
\end{description}



	
%--------------------------------------------------------------------------------


\end{document}
