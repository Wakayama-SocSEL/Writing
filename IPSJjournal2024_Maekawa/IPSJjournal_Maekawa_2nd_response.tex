\documentclass{jarticle} %LaTeX2e


\usepackage{color}
%\usepackage{graphicx}
\usepackage[dvipdfmx]{graphicx} % pdfを使用する
\usepackage{tabularx}
\usepackage{fancyhdr} % header, footer
\usepackage{lastpage} % 総page数
\usepackage{multirow}
\usepackage{ulem}
\usepackage{latexsym}
\usepackage{maltcol}
\usepackage{jumoline}
\usepackage{caption}
\usepackage{url}

\setlength{\UnderlineTexDepth}{3pt}

\newcommand{\todo}[1]{\colorbox{yellow}{{\bf TODO}:}{\color{red}{\textbf{[#1]}}}}

\newcommand{\ihara}[1]{\colorbox{green}{{\bf ihara}:}{\color{red}{\textbf{[#1]}}}}

\renewcommand{\headrulewidth}{0pt} % ヘッダーラインを打ち出さない

% todayコマンド 西暦表示
\renewcommand{\today}{%
  \the\year /%
   {\ifnum \month < 10  0\the\month \else \the\month \fi}/%
   {\ifnum \day < 10  0\the\day \else \the\day \fi}%
}

% header, footer周りの定義
\pagestyle{fancy}

% 回答文タイトル
\def\restitle{条件付論文に対する回答文}
\def\paptitle{JavaScriptライブラリのテストコード変更に基づく後方互換性を損失するバージョン検出}

%% header
\lhead{\restitle}
\rhead{\paptitle}

%% footer
\cfoot{}
\lfoot{\date{\today}}
\lfoot{\today}
\rfoot{\thepage/\pageref{LastPage}}

%% ページの余白
\setlength{\oddsidemargin}{0pt}
\setlength{\evensidemargin}{0pt}
\setlength{\topmargin}{-20pt}
\setlength{\textwidth}{440pt}
\setlength{\textheight}{640pt}
\setlength{\headheight}{30pt}
\setlength{\headwidth}{\textwidth}

%% 回答書用sectionの再定義

% ラベル:査読者Aの方から〜対する回答
\def\section#1{ \vspace{3pc} {\large \gt #1} \vspace{1pc} \hrule }

% ラベル:条件,回答,変更
\def\subsection#1{ \vspace{1pc} {\gt #1} }

% 次の回答文へ移動
\def\nextans{ \vspace{2pc} \hrule }


% テキストカラー
%\def\red#1{\textcolor{red}{#1}}


%%%
\begin{document}

% 回答書見出し
{\Large \gt \restitle}

\vspace{3pc}

査読者各位\\


この度は私共の以下の論文に対して熱心な御査読を頂き,有益な御意見を頂きましたことに厚く御礼申し上げます.\\
\textbf{論文ID:24-A081}\\
\textbf{タイトル:『JavaScriptライブラリのテストコード変更に基づく後方互換性を損失するバージョン検出』}

貴重な御意見を反映するよう,論文の加筆・修正を行いましたので再投稿させていただきます.御手数おかけ致しますが,再度御査読の程どうぞ宜しくお願い致します.	

本回答文は,御指摘に対する「回答」と,論文の変更内容を示す「変更」の順に記載しております.一つのコメントに対して複数の変更を行った場合は,(変更1-1-a),(変更1-1-b)のように番号の後ろにアルファベットを付しております.御理解の程,どうぞ宜しくお願い致します.\\

%第2査読者の方から(条件2-1)において論文中の「コンピュテーショナル・シンキング(CT)スキル」の用語が適切でない(CTという語自体がスキルという意味合いを含んでいる)と御指摘を受けました.私共でも改めて従来研究を確認したところ,私共の使用していた用語が適切ではありませんでした.貴重な御指摘を頂き,誠にありがとうございました.論文中で多用している用語であり,誤解を招く可能性があるため,以降のメタ査読者,査読者1,査読者2の方々からの御指摘に対する回答は,タイトルを含めた本論文における「CTスキル」および「CTスキルの概念」の表記を,それぞれ「CT」および「CT概念」のように変更した用語でご説明することを,回答書冒頭でご報告しておきます.本変更は論文中で多用する語のため,論文中には変更箇所を赤字で記載し,アルファベットは付しません.どうぞ宜しくお願い致します.

%--------------------------------------------------------------------------------
\section{査読委員の方(メタ査読者)から頂いたコメントに対する回答}
%--------------------------------------------------------------------------------

\subsection{(コメントM-1)}

本論文は、テストケースの変更有無を判断基準として、JavaScriptライブラリのバージョンアップに伴う後方互換性の発生有無を推定することを目的とした研究です。独自のアプローチを持ち、多くのライブラリバージョンを対象として多角的な分析が行なわれており、結果も有効であると言えます。

しかしながら、査読者1が指摘しているように、本論文の記述には複数の不明瞭な点や不整合が見られます。
また、査読者2が指摘しているように、掲載されている事例の見直しが必要かと思われます。

これらを含め、両査読者からのコメントへの対応を採録の条件とします。


\subsection{(回答M-1)}

%御指摘を頂き,誠にありがとうございます.第1査読者の方から御指摘頂いた内容に対する回答を(コメントへの回答1-1)に述べ,それに伴う変更を行いましたので御確認ください.

\newpage
%--------------------------------------------------------------------------------
%--------------------------------------------------------------------------------
\section{査読委員の方(査読者: 1)から頂いたコメントに対する回答}
%--------------------------------------------------------------------------------
\subsection{(コメント1-1)}

ライブラリの後方互換性問題について、テストコードの静的解析のみで一定の再現率を得たという点で、非常に興味深い手法と考えます。ベースとなる技術は過度に複雑でなく、追加の解析による精度向上や他言語への応用も期待できます。
また、JavaScriptのエコシステムはWeb等で広く利用されており、その観点からも価値のある研究といえます。

手法の流れや実験結果については大きく問題は認められませんが、構成や用語、記述の問題でいくらか説明に不整合が見られます。
記述を整理して本論文の分析手法および結果を正確に把握できるよう改善することを採録の条件として挙げます。

\subsection{(回答1-1)}

貴重な御指摘を頂き,誠にありがとうございます.\todo{HOGEHOGE}


      

%--------------------------------------------------------------------------------
\newpage
\nextans
\subsection{(条件1-1)}

条件1-1とも関連しますが,3.2節では1回目から20回目に投稿したプログラムのDr.Scratchの点数をまとめ,フロー制御の点数の獲得は容易である一方で,論理の点数の獲得が困難であると結論付けています.

このデータは大規模かつ公平に収集されており大変貴重なデータだと考えますが,習熟過程を調査するのなら,1回目から20回目それぞれで使用された各概念をまとめるべきではないでしょうか.

フロー制御の2点は確かに多くの作品で取得されていますが,もしそれらの取得が他の概念よりもあとであった場合,点数の取得は容易であると言い切れないと考えます.すなわち,多くの作品が取得している概念が取得が容易なのではなく,早期に取得する概念が取得が容易であるとも考えられます.
図で表現することは困難かと思いますが,著者らの主張の正確性にもつながりますため,フロー制御の点数の何割が5回目の投稿以内に獲得されたなど,投稿序盤で獲得されやすい概念について追記をお願いします.

\subsection{(回答1-1)}

貴重な御指摘を頂き,誠にありがとうございます.



\subsection{(変更1-2)P.4 3.2節 }
\vspace{-0.3cm}
\begin{description}
\item 修正前\\
\phantom{ }
\todo{hoge}
\vspace{-0.3cm}
\item 修正後\\
\phantom{ }
\todo{hoge}
\end{description}


%--------------------------------------------------------------------------------
\newpage
\nextans
\subsection{(条件1-2)}

以下『ライブラリはL、ライブラリバージョンはL(x)とL(x+1)の組』の理解が正しいと仮定してコメントを続けます。
分析対象の1955組は「ライブラリバージョンとクライアントの組み合わせ」とされていますが、これは5.2節で抽出した22271組の「ライブラリバージョンとクライアントの組み合わせ」の部分集合であるように見えます。
しかしながら、5.3節を読みなおすとこれは誤解であり、正確には以下の解釈となる認識です。
・22271組の「ライブラリバージョンとクライアントの組み合わせ」は、238種類のライブラリL(ライブラリバージョンではなく)に対してそれを利用するクライアントが存在し、その存在するL×Cの組み合わせの総数である
・1955組の「ライブラリバージョンとクライアントの組み合わせ」は、5.3節の手順(3)で「クライアントテストの実行結果を統合する」ため、あるライブラリバージョンに対してクライアント(の集合)がただ1つとなり、分析対象は各ライブラリバージョンに1:1対応するクライアント集合が存在する1955のライブラリバージョンである
この解釈に基づくと、これらは別個の概念であり双方に「ライブラリバージョンとクライアントの組み合わせ」という語を用いることは適切ではないように思います。
また、ここでは1955組は「組み合わせ」と表現されており実際に<L(X),L(X+1),C>の3つ組ではあります。
しかし、ライブラリバージョンについてユニークであるという点がはっきりしないと、その後の適合率および再現率のもととなる件数が「バージョン間での後方互換性の有無」に対するものという解釈ができなくなる(同じライブラリバージョン間について別のCを用いた複数回の検出結果が数値に含まれうるように見える)ため、分析結果の理解に支障が出ると考えます。
「ライブラリバージョンとクライアントの組み合わせ」という表現が22271組と1955組双方に使われていることについて、これが適切か上記をふまえ見直したうえ誤解のないよう記述を修正してください。


\subsection{(回答1-2)}

貴重な御指摘を頂き,誠にありがとうございます.\todo{hoge}


\subsection{(変更1-3)P.9 6.2節}
\vspace{-0.3cm}
\begin{description}
\item 修正前\\
\phantom{ }
\todo{hoge}
\vspace{-0.3cm}
\item 修正後\\
\phantom{ }
\todo{hoge}
\end{description}


%--------------------------------------------------------------------------------
\newpage
\nextans
\subsection{(条件1-3)}

仮説に対応する分析は確かに2つなのですが、章構成が「5章:分析1」「6章:~の分析」「7章:分析2」となっており掲載されている分析は3つです。
1章では6章も分析2に含めていたかと思いますので、適切な説明になるよう各章タイトルおよび1章の記述をあらためてください。

\subsection{(回答1-3)}

貴重な御指摘を頂き,誠にありがとうございます.\todo{hoge}



\newpage
\subsection{(変更1-3)P.6 4.1節}
\vspace{-0.3cm}
\begin{description}
\item 修正前\\
\phantom{ }
\todo{hoge}
\vspace{-0.3cm}
\item 修正後\\
\phantom{ }
\todo{hoge}
\end{description}



%--------------------------------------------------------------------------------
\newpage
\nextans
\subsection{(条件1-4)}

6.2節から分析2の手法へのつながりを追う中で、いくつか不整合が見られるため、なぜ分析2の手法に至ったのかが正確に把握できませんでした。
・6.2節ではテストコードの変更の内訳としてテストフィクスチャの変更を挙げていますが、6.2.3節では議論されておらず7.2節でも出現しません。
・アサーションの追加は6.2節では「テストコードの変更」に分類されていますが、6.2.1節や7.2節では「テストコード追加」として扱われています。
これらの点をみなおし、6.2節での議論が分析2の手法につながるように説明を加えてください。

\subsection{(回答1-4)}

貴重な御指摘を頂き,誠にありがとうございます.\todo{hoge}



\newpage
\subsection{(変更1-4)P.6 4.1節}
\vspace{-0.3cm}
\begin{description}
\item 修正前\\
\phantom{ }
\todo{hoge}
\vspace{-0.3cm}
\item 修正後\\
\phantom{ }
\todo{hoge}
\end{description}


%--------------------------------------------------------------------------------
\newpage
\nextans
\subsection{(条件1-5)}

文章の意味が読み取れない部分があるので修正してください。
・概要:「分析対象とするライブラリバージョン更新で後方互換性を損失している割合よりも高い精度で検出することができた」
-> 何と何を比べているのかが読み取れません。
・3章:ライブラリの後方互換性を判断する手段としてライブラリに付属しているテストへの変更を用いることができるが,頻繁に再利用されるライブラリに対して後方互換性を判断できるのかは明らかでない.
-> 「頻繁に再利用される」こととの関係が不明です。(頻繁に再利用されないものに対しては後方互換性を判断できるのが明らかだ、という意味ではないはずです)

\subsection{(回答1-5)}

貴重な御指摘を頂き,誠にありがとうございます.\todo{hoge}



\newpage
\subsection{(変更1-5)\todo{hoge}節}
\vspace{-0.3cm}
\begin{description}
\item 修正前\\
\phantom{ }
\todo{hoge}
\vspace{-0.3cm}
\item 修正後\\
\phantom{ }
\todo{hoge}
\end{description}






%--------------------------------------------------------------------------------
\newpage
\nextans
\subsection{(コメント1-1)}

文章の意味が読み取りづらい部分があるので修正を検討してください。
・7.3節:「そのうち114件(適合率:約17\%)を正しく判定することができ,5章は905件中140件(約15\%)であり本手法により改善することができた.」
・7.3節:「また,本手法は,後方互換性の損失があった223件中114件(再現率:約51\%)を正しく検出でき,5章に示す223件中140件(約63\%)・7.3節 より低下し,後方互換性を損失したライブラリバージョンの検出においては今後の課題である.」
・7.4節では「手法」「分析」の語が混じっています。整理すると理解がスムーズになると思います。
読んだ限りでは、1段落目と2段落目は最初の分析結果の件数を除き「分析手法」自体の性質について論じており、3段落目はクライアントを用いた「分析結果」に対する考察と思われます。

\subsection{(コメントへの回答1-1)}
貴重な御指摘,および,御助言を頂き,誠にありがとうございます.

\newpage
\subsection{(変更C1-1)P.6 4.1節}
\vspace{-0.3cm}
\begin{description}
\item 修正前\\
\phantom{ }
\todo{hoge}
\vspace{-0.3cm}
\item 修正後\\
\phantom{ }
\todo{hoge}
\end{description}


\newpage
%--------------------------------------------------------------------------------
%--------------------------------------------------------------------------------
\section{査読委員の方(査読者: 2)から頂いたコメントに対する回答}
%--------------------------------------------------------------------------------
\subsection{(条件2-1)}

著者らは,テストコードの変更に基づき,後方互換性を損失するJavascriptライブラリのバージョンを検出する手法を提案しています.実験を通して,提案手法はランダム判別手法と比べて,高い精度で検出を行えることがわかっています.

本論文は,後方互換性を損失するJavascriptライブラリのバージョン検出を行うために,テストコードの変更に着目するというアイデアは興味深いです.しかし,図1において,後方互換性を損失する事例であるのか判断できませんでした.もし,後方互換性を損失する事例であるのであればそのことを明示してください.もし,そうでないなら,後方互換性を損失する事例を掲載してください.また,単に後方互換性を損失する事例ではなく,既存手法と比べて提案手法が優れているものを掲載する必要があると考えます.


\subsection{(回答2-1)}

貴重な御指摘を頂き,誠にありがとうございます.


\subsection{(変更2-1)\todo{hoge}}
\vspace{-0.3cm}
\begin{description}
\item 修正前\\
\phantom{ }
\todo{hoge}
\vspace{-0.3cm}
\item 修正後\\
\phantom{ }
\todo{hoge}
\end{description}


%--------------------------------------------------------------------------------
\newpage
\nextans
\subsection{(コメント2-1)}

- 著者らは,FOSE等で関連する論文を発表しているようです.二重投稿にはあたらないとおもいますが,本論文で引用すべきでしょう.

松田和輝 伊原彰紀 才木一也, ライブラリのテストケース変更に基づく後方互換性の実証的分析, FOSE 2021.

- 4.1.1節のテストケースの収集方法について,既存研究を参考にしているのであれば,その既存研究を明示した方が説得力が上がると考えます.

- 4.1.2節2段落目は,図を用いるなどして例を示さないと,理解に時間がかかります.

- 3章:「後方互換性の損失に伴わないテストケース変更は」文意不明です.

\subsection{(コメントへの回答2-1)}

貴重な御指摘を頂き,および,御助言を頂き,誠にありがとうございます.

\subsection{(変更C2-1)\todo{hoge}}
\vspace{-0.3cm}
\begin{description}
\item 修正前\\
\phantom{ }
\todo{hoge}
\vspace{-0.3cm}
\item 修正後\\
\phantom{ }
\todo{hoge}
\end{description}



	
%--------------------------------------------------------------------------------


\end{document}
