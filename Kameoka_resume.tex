%
% 卒論レジュメフォーマット Ver.2.0 pLaTeX版
%
\documentclass[twocolumn]{jarticle} % 2段組のスタイルを用いている

\usepackage{wuse_resume}
\usepackage{url}	% \url{}コマンド用.URLを表示する際に便利
\usepackage[dvipdfmx]{graphicx}  % ←graphicx.styを用いてEPSを取り込む場合有効にする
\usepackage{multirow}
			% 他のパッケージ・スタイルを使う場合には適宜追加

%%%%%%%%%%%%%%%%%%%%%%%%%%%%%%%%%%%%%%%%%%%%%%%%%%%%%%%%%%%%%%%%%%%%%%%%

%%
%% タイトル,学生番号,氏名などを設定する
%%

\タイトル{複数プロジェクト規約修正履歴を用いた\\コーディング規約違反の修正予測精度の評価}
\研究室{ソーシャルソフトウェア工学}
\学生番号{60256065}
\氏名{亀岡 令}

\概要{%
本研究では,コーディング規約に違反しているソースコードの修正予測を行う際に,予測対象以外の複数のプロジェクトの開発履歴を学習することによる予測精度への影響を調査する.
ソフトウェア開発では,プロジェクトにコーディング規約が頻繁に導入され,開発者はコーディング規約を遵守して開発を行う.自動的に検出されたコーディング規約違反は修正されないまま放置されることが多い.
従来研究では,自動検出されたコーディング規約違反を,優先して修正すべき違反か否かの2クラスに自動分類する機械学習モデルを構築する手法を提案している.しかし,予測対象プロジェクトの過去の開発データを学習しているため,プロジェクトによっては学習不足によって予測精度が低下する場合がある.
本研究では,予測対象プロジェクト以外の開発履歴も学習データに使用することによって学習データを補完し,予測精度への影響の評価を行った.また,提案手法によって再現率とF1値が共に向上したプロジェクトにおいて予測結果の差異について分析を行った.
}

\キーワード{コーディング規約}
\キーワード{機械学習}
\キーワード{GitHub}
\キーワード{静的解析ツール}
\キーワード{Python}

%%%%%%%%%%%%%%%%%%%%%%%%%%%%%%%%%%%%%%%%%%%%%%%%%%%%%%%%%%%%%%%%%%%%%%%%

%% 以下の3行は変更しない

\begin{document}
\maketitle
\thispagestyle{empty} % タイトルを出力したページにもページ番号を付けない

%%%%%%%%%%%%%%%%%%%%%%%%%%%%%%%%%%%%%%%%%%%%%%%%%%%%%%%%%%%%%%%%%%%%%%%%

%%
%% 本文 - ここから
%%

\section{はじめに}

複数人で実装するソフトウェア開発では,コーディングスタイルを共通化するため,変数名の命名規則やコメント文の書き方などに関するコーディング規約に従って開発している.
コーディング規約を導入することによって,ソフトウェア開発におけるソースコードの理解の促進,バグの早期発見などに効果があることが明らかにされている\cite{Beller2}
コーディング規約に従って実装しているか否かの判定には,ソースコードを実行することなく,ソースコード中に含まれるコーディング規約の違反箇所を検出する静的解析ツールが使用されている.
しかし,多くのプロジェクトでは,規約違反の指摘漏れを抑えるために規約違反の判定基準を厳しく設定しており,静的解析ツールは大量の規約違反を出力するため,静的解析ツールの誤検出を防ぐための研究が数多く発表されている\cite{Nguyen}.
従来研究では,機械学習を用いて,大量に検出されたコーディング規約違反の中から優先して修正すべき違反を特定する手法を提案しているが,
予測するプロジェクトと同じプロジェクトを学習データに用いるため,データセットサイズや修正される違反の割合が少ない場合に十分な学習ができずに予測精度が下がってしまうことが考えられる\cite{JyuraiPre}.
ただし,同一プロジェクトのデータは,プロジェクトの実装方針が同一のため,異なるプロジェクトの開発履歴を用いるよりも高い精度が得られる.

\begin{table*}[ht]
    \centering
    \caption{各手法による評価指標で最も高い値で予測したプロジェクト数の一覧}
    \label{tab:gen}
    \vspace{3mm}
    \scalebox{0.75}{
        \begin{tabular}{l|p{7em}p{7em}p{7em}|p{7em}p{7em}p{7em}|p{7em}p{7em}p{7em}}
            \hline
            \multirow{2}{*}{手法}&\multicolumn{3}{c|}{{ロジスティック回帰}} & \multicolumn{3}{c|}{RandomForest} & \multicolumn{3}{c}{SVM}\\ \cline{2-10}
             & \multicolumn{1}{r}{{適合率}} & \multicolumn{1}{r}{{再現率}} & \multicolumn{1}{r|}{{F1値}} & \multicolumn{1}{r}{{適合率}} & \multicolumn{1}{r}{{再現率}} & \multicolumn{1}{r|}{{F1値}} & \multicolumn{1}{r}{{適合率}} & \multicolumn{1}{r}{{再現率}} & \multicolumn{1}{r}{{F1値}}\\ \hline
            従来手法 & \multicolumn{1}{r}{{48}} & \multicolumn{1}{r}{{27}} & \multicolumn{1}{r|}{{38}} & \multicolumn{1}{r}{{28}} & \multicolumn{1}{r}{{43}} & \multicolumn{1}{r|}{{37}} & \multicolumn{1}{r}{{36}} & \multicolumn{1}{r}{{24}} & \multicolumn{1}{r}{{28}} \\
            % 提案手法(クラスタリングなし) & 12 & 40 & 18 & 27 & 35 & 23 & 17 & 25 & 18 \\
            提案手法(クラスタリングなし) & \multicolumn{1}{r}{{12}} & \multicolumn{1}{r}{{40}} & \multicolumn{1}{r|}{{18}} & \multicolumn{1}{r}{{27}} & \multicolumn{1}{r}{{35}} & \multicolumn{1}{r|}{{23}} & \multicolumn{1}{r}{{17}} & \multicolumn{1}{r}{{25}} & \multicolumn{1}{r}{{18}} \\
            % 提案手法(クラスタリングあり) & 20 & 22 & 19  & 28 & 24 & 16 & 24 & 36 & 26 \\ \hline
            提案手法(クラスタリングあり) & \multicolumn{1}{r}{{20}} & \multicolumn{1}{r}{{22}} & \multicolumn{1}{r|}{{19}}  & \multicolumn{1}{r}{{28}} & \multicolumn{1}{r}{{24}} & \multicolumn{1}{r|}{{16}} & \multicolumn{1}{r}{{24}} & \multicolumn{1}{r}{{36}} & \multicolumn{1}{r}{{26}} \\ \hline
            % 重複数 & 12 & 21 & 7 & 15 & 34 & 8 & 9 & 17 & 4 \\ \hline
            重複数 & \multicolumn{1}{r}{{12}} & \multicolumn{1}{r}{{21}} & \multicolumn{1}{r|}{{7}} & \multicolumn{1}{r}{{15}} & \multicolumn{1}{r}{{34}} & \multicolumn{1}{r|}{{8}} & \multicolumn{1}{r}{{9}} & \multicolumn{1}{r}{{17}} & \multicolumn{1}{r}{{4}} \\ \hline
        \end{tabular}
    }
    \vspace{-3mm}
\end{table*}

% \begin{table*}[ht]
%     \centering
%     \caption{ロジスティック回帰モデルでの予測結果(総違反数上位5件を掲載)}
%     \label{tab:logistic}
%     \vspace{3mm}
%     \scalebox{0.75}{
%         \begin{tabular}{l||p{4em}|p{4em}|p{4em}||p{4em}|p{4em}|p{4em}||p{4em}|p{4em}|p{4em}}
%             \hline
%             \multirow{3}{*}{プロジェクト名}&\multicolumn{3}{c||}{\multirow{2}{*}{従来手法}} & \multicolumn{3}{c||}{提案手法} & \multicolumn{3}{c}{提案手法}\\
%             &\multicolumn{3}{c||}{} & \multicolumn{3}{c||}{(クラスタリングなし)} & \multicolumn{3}{c}{(クラスタリングあり)}\\ \cline{2-10}
%             & 適合率 & 再現率 & F1値 & 適合率 & 再現率 & F1値 & 適合率 & 再現率 & F1値 \\ \hline
%             % & \multicolumn{9}{c}{ロジスティック回帰} \\ \hline
%             sockeye & \multicolumn{1}{r|}{\textbf{0.53}} & \multicolumn{1}{r|}{\textbf{0.81}} & \multicolumn{1}{r||}{\textbf{0.64}} & \multicolumn{1}{r|}{0.35} & \multicolumn{1}{r|}{0.56} & \multicolumn{1}{r||}{0.43} & \multicolumn{1}{r|}{0.49} & \multicolumn{1}{r|}{0.67} & \multicolumn{1}{r}{0.57} \\
%             coretools & \multicolumn{1}{r|}{\textbf{0.04}} & \multicolumn{1}{r|}{0.63} & \multicolumn{1}{r||}{\textbf{0.07}} & \multicolumn{1}{r|}{0.02} & \multicolumn{1}{r|}{\textbf{0.88}} & \multicolumn{1}{r||}{0.04} & \multicolumn{1}{r|}{0.03} & \multicolumn{1}{r|}{0.83} & \multicolumn{1}{r}{0.05} \\
%             howdoi & \multicolumn{1}{r|}{\textbf{0.64}} & \multicolumn{1}{r|}{\textbf{0.99}} & \multicolumn{1}{r||}{\textbf{0.78}} & \multicolumn{1}{r|}{0.12} & \multicolumn{1}{r|}{0.33} & \multicolumn{1}{r||}{0.17} & \multicolumn{1}{r|}{0.22} & \multicolumn{1}{r|}{0.25} & \multicolumn{1}{r}{0.23} \\
%             schema\_salad & \multicolumn{1}{r|}{\textbf{0.55}} & \multicolumn{1}{r|}{0.27} & \multicolumn{1}{r||}{0.37} & \multicolumn{1}{r|}{0.51} & \multicolumn{1}{r|}{\textbf{0.64}} & \multicolumn{1}{r||}{\textbf{0.56}} & \multicolumn{1}{r|}{\textbf{0.55}} & \multicolumn{1}{r|}{0.54} & \multicolumn{1}{r}{0.54} \\
%             serverless-application-model & \multicolumn{1}{r|}{\textbf{0.76}} & \multicolumn{1}{r|}{0.45} & \multicolumn{1}{r||}{0.57} & \multicolumn{1}{r|}{0.55} & \multicolumn{1}{r|}{\textbf{0.83}} & \multicolumn{1}{r||}{0.66} & \multicolumn{1}{r|}{0.64} & \multicolumn{1}{r|}{\textbf{0.83}} & \multicolumn{1}{r}{\textbf{0.72}} \\ \hline
%         \end{tabular}
%     }
% \end{table*}

% \begin{table*}[ht]
%     \centering
%     \caption{RandomForestモデルでの予測結果(総違反数上位5件を掲載)}
%     \label{tab:RandomForest}
%     \vspace{3mm}
%     \scalebox{0.75}{
%         \begin{tabular}{l||p{4em}|p{4em}|p{4em}||p{4em}|p{4em}|p{4em}||p{4em}|p{4em}|p{4em}}
%             \hline
%             \multirow{3}{*}{プロジェクト名}&\multicolumn{3}{c||}{\multirow{2}{*}{従来手法}} & \multicolumn{3}{c||}{提案手法} & \multicolumn{3}{c}{提案手法}\\
%             &\multicolumn{3}{c||}{} & \multicolumn{3}{c||}{(クラスタリングなし)} & \multicolumn{3}{c}{(クラスタリングあり)}\\ \cline{2-10}
%             & 適合率 & 再現率 & F1値 & 適合率 & 再現率 & F1値 & 適合率 & 再現率 & F1値 \\ \hline
%             sockeye & \multicolumn{1}{r|}{0.75} & \multicolumn{1}{r|}{\textbf{0.83}} & \multicolumn{1}{r||}{\textbf{0.79}} & \multicolumn{1}{r|}{\textbf{0.76}} & \multicolumn{1}{r|}{0.79} & \multicolumn{1}{r||}{0.78} & \multicolumn{1}{r|}{\textbf{0.76}} & \multicolumn{1}{r|}{0.80} & \multicolumn{1}{r}{0.78} \\
%             coretools & \multicolumn{1}{r|}{\textbf{0.14}} & \multicolumn{1}{r|}{0.24} & \multicolumn{1}{r||}{\textbf{0.18}} & \multicolumn{1}{r|}{0.04} & \multicolumn{1}{r|}{\textbf{0.41}} & \multicolumn{1}{r||}{0.08} & \multicolumn{1}{r|}{0.04} & \multicolumn{1}{r|}{0.37} & \multicolumn{1}{r}{0.07} \\
%             howdoi & \multicolumn{1}{r|}{\textbf{0.78}} & \multicolumn{1}{r|}{\textbf{0.99}} & \multicolumn{1}{r||}{\textbf{0.87}} & \multicolumn{1}{r|}{0.07} & \multicolumn{1}{r|}{\textbf{0.99}} & \multicolumn{1}{r||}{0.13} & \multicolumn{1}{r|}{0.07} & \multicolumn{1}{r|}{\textbf{0.99}} & \multicolumn{1}{r}{0.13} \\
%             schema\_salad & \multicolumn{1}{r|}{\textbf{0.66}} & \multicolumn{1}{r|}{\textbf{0.58}} & \multicolumn{1}{r||}{\textbf{0.62}} & \multicolumn{1}{r|}{0.58} & \multicolumn{1}{r|}{0.31} & \multicolumn{1}{r||}{0.41} & \multicolumn{1}{r|}{0.46} & \multicolumn{1}{r|}{0.21} & \multicolumn{1}{r}{0.29} \\
%             serverless-application-model & \multicolumn{1}{r|}{\textbf{0.72}} & \multicolumn{1}{r|}{\textbf{0.27}} & \multicolumn{1}{r||}{\textbf{0.39}} & \multicolumn{1}{r|}{0.70} & \multicolumn{1}{r|}{0.23} & \multicolumn{1}{r||}{0.34} & \multicolumn{1}{r|}{0.63} & \multicolumn{1}{r|}{0.24} & \multicolumn{1}{r}{0.35} \\ \hline
%         \end{tabular}
%     }
% \end{table*}

本研究では,複数プロジェクトの開発履歴を用いて,複数プロジェクトにおける規約に違反したソースコードの特徴に基づき規約違反の修正予測をするモデルを構築し,予測精度を分析する.検証に使用するデータセットはPython言語で記述されたオープンソースソフトウェア68プロジェクトを対象に予測モデルを構築し予測性能を評価した.


\section{修正を要する規約違反ソースコードの特定手法}

\noindent\textbf{学習データと評価データの収集: }本研究では,複数のプロジェクトの開発履歴を統合し,類似する学習データを分類して,それぞれで機械学習モデルを構築することで予測精度の向上を図る.各プロジェクトの分析対象期間に発生するコミットのうち,前半8割を学習データ,後半2割を評価データとする.ここで時系列順に並べている理由は,学習時に本来得られない未来のデータを学習させないために時系列の古いものを学習データとしている.そのため,モデルの評価時に,交差検証は行なわない.時系列を考慮した時系列十分割交差検証という手法も存在するが,データサイズが小さいプロジェクトでは,学習データサイズが小さくなりすぎることが危惧されるため,時系列十分割交差検証も行わない.

\noindent\textbf{説明変数の計測方法: }本研究では,分析対象期間中の各コミットで変更されたソースコード全てに静的解析ツールを実行し,特定のソースコード断片において初めて規約違反が検出されたコミットを,修正を要するか否かを判定する予測時点とし,説明変数を計測する.
説明変数は,ソースコードの特徴量であるコード行数,コメント行数,循環的複雑度など含むソースコードに関する特徴量43種類に,コーディング規約の違反検出時に得られる違規約違反IDをOne-hotベクトル化した1種類の,合計44種類の特徴量を使用する.目的変数は,分析対象期間中に修正されるか否かとする.

\noindent\textbf{規約違反しているコード断片のクラスタリング: }本研究では,複数プロジェクトのデータを結合し,モデルを作成した.また,説明変数の特徴量が類似する規約違反しているコード断片をクラスタリングした後に,各クラスタに分類されたコード断片の学習データを用いてモデルを構築する.クラスタリングには,クラスタ間の類似度を確認するため階層的クラスタリングを用い,クラスタ間の距離はユークリッド距離,クラスタの連結法にはWard法を用い,クラスタ数を10に設定した.

\noindent\textbf{機械学習モデルの構築と評価: }予測モデルの構築は,ロジスティック回帰,RandomForest,SVMモデルを使用する.評価指標には,適合率,再現率,F1値を使用する.

\vspace{-2mm}

\section{結果}

本研究ではPythonで開発されているOSSライブラリで,Pythonの静的解析ツールであるpylintを用いて開発されている68プロジェクトを対象とする.分析対象は2018年12月から1,000日間のコミット履歴とする.表\ref{tab:gen}に検出されたコーディング規約違反の修正予測の結果を示す.実験の結果,RandomForestモデルが他のモデルと比較して最も高い予測精度を示したことが確認された.SVMモデルは最もモデル構築にかかる時間が長く,最も予測精度が低いため修正要否予測モデルには適していない.ロジスティック回帰モデルでは,多くのプロジェクトで提案手法によって従来手法より再現率が向上するプロジェクトが存在することが確認された.

ロジスティック回帰モデルで予測精度が大きく変化した4プロジェクトの追加分析を行った.分析の結果から,予測精度が向上したプロジェクトでは,従来手法では予測できていなかったFalse Negativeに位置していたデータを提案手法によってTrue Positive(TP)に分類することが可能となった.しかし,TPの増加とともにFalse Positive(FP)が増加することも確認した.再現率が向上したことによってF1値が向上しているプロジェクトは,FPが増加したことによって低下した予測精度より,TPが増加したことによって向上する予測精度が上回った場合に提案手法によってコーディング規約の修正予測精度が向上する.予測精度が低下したプロジェクトは,予測精度が向上した場合とは反対に,TPの増加よりFPの増加が上回る場合や,TPが増加しない場合に提案手法によって予測精度が低下することを確認した.


\section{考察}

実験結果から提案手法を用いることによって,一部プロジェクトで従来手法では予測ができなかった正例を正しく予想できるようになったことが確認できた.理由として,複数プロジェクトを学習し,単一プロジェクトでは十分に学習できなかったコーディング規約について学習ができたためであると考えられる.

\section{おわりに}

本研究では,コーディング規約の修正予測において,複数プロジェクトの開発データを学習に用いることによる修正予測精度への影響の分析を行った.
結果として,提案手法にって従来手法より予測精度が向上するプロジェクトが68プロジェクトの半数程度存在することが明らかになった.
提案手法によって予測精度が向上するプロジェクトでは,従来手法より優れた違反修正提案が可能であり,開発におけるコーディング規約違反修正のための人的コスト削減に役立つと考えられる.


%%
%% 本文 - ここまで
%%

%%%%%%%%%%%%%%%%%%%%%%%%%%%%%%%%%%%%%%%%%%%%%%%%%%%%%%%%%%%%%%%%%%%%%%%%

%%
%% 参考文献
%%

\bibliographystyle{junsrt}
\bibliography{Kameoka}

%%%%%%%%%%%%%%%%%%%%%%%%%%%%%%%%%%%%%%%%%%%%%%%%%%%%%%%%%%%%%%%%%%%%%%%%

\end{document}
